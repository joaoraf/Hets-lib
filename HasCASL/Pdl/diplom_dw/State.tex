%
\begin{isabellebody}%
\def\isabellecontext{State}%
%
\isamarkupheader{A Simple Reference Monad with \texttt{while} and \texttt{if}%
}
\isamarkuptrue%
\isacommand{theory}\ State\ {\isacharequal}\ PDL\ {\isacharplus}\ MonEq{\isacharcolon}\isamarkupfalse%
%
\label{sec:state-thy}
%
\begin{isamarkuptext}%
Read/write operations on references of arbitrary type, and a while loop.
  \label{isa:ref-spec}%
\end{isamarkuptext}%
\isamarkuptrue%
\isacommand{typedecl}\ {\isacharprime}a\ ref\isanewline
\isanewline
\isamarkupfalse%
\isacommand{consts}\isanewline
\ \ newRef\ \ \ \ \ {\isacharcolon}{\isacharcolon}\ {\isachardoublequote}{\isacharprime}a\ {\isasymRightarrow}\ {\isacharprime}a\ ref\ T{\isachardoublequote}\isanewline
\ \ readRef\ \ \ \ {\isacharcolon}{\isacharcolon}\ {\isachardoublequote}{\isacharprime}a\ ref\ {\isasymRightarrow}\ {\isacharprime}a\ T{\isachardoublequote}\ \ \ \ \ \ \ \ \ \ \ \ \ \ \ \isanewline
\ \ writeRef\ \ \ {\isacharcolon}{\isacharcolon}\ {\isachardoublequote}{\isacharprime}a\ ref\ {\isasymRightarrow}\ {\isacharprime}a\ {\isasymRightarrow}\ unit\ T{\isachardoublequote}\ \ \ \ \ \ \ \ \ \ \ {\isacharparenleft}{\isachardoublequote}{\isacharparenleft}{\isacharunderscore}\ {\isacharcolon}{\isacharequal}\ {\isacharunderscore}{\isacharparenright}{\isachardoublequote}\ {\isacharbrackleft}{\isadigit{1}}{\isadigit{0}}{\isadigit{0}}{\isacharcomma}\ {\isadigit{1}}{\isadigit{0}}{\isacharbrackright}\ {\isadigit{1}}{\isadigit{0}}{\isacharparenright}\isanewline
\ \ monWhile\ \ \ {\isacharcolon}{\isacharcolon}\ {\isachardoublequote}bool\ D\ {\isasymRightarrow}\ unit\ T\ {\isasymRightarrow}\ unit\ T{\isachardoublequote}\ \ \ \ \ \ \ {\isacharparenleft}{\isachardoublequote}WHILE\ {\isacharparenleft}{\isadigit{4}}{\isacharunderscore}{\isacharparenright}\ {\isacharslash}DO\ {\isacharparenleft}{\isadigit{4}}{\isacharunderscore}{\isacharparenright}\ {\isacharslash}END{\isachardoublequote}{\isacharparenright}\isamarkupfalse%
%
\begin{isamarkuptext}%
To make the dsef operation of reading a reference more readable (pun unintended),
  we introduce syntactical sugar: \isa{{\isacharasterisk}r} stands for \isa{{\isasymUp}\ readRef\ r}.%
\end{isamarkuptext}%
\isamarkuptrue%
\isacommand{syntax}\ \isanewline
\ \ {\isachardoublequote}{\isacharunderscore}readRefD{\isachardoublequote}\ \ {\isacharcolon}{\isacharcolon}\ {\isachardoublequote}{\isacharprime}a\ ref\ {\isasymRightarrow}\ {\isacharprime}a\ D{\isachardoublequote}\ \ \ \ \ \ \ \ \ \ \ \ \ \ \ \ {\isacharparenleft}{\isachardoublequote}{\isacharasterisk}{\isacharunderscore}{\isachardoublequote}\ {\isacharbrackleft}{\isadigit{1}}{\isadigit{0}}{\isadigit{0}}{\isacharbrackright}\ {\isadigit{1}}{\isadigit{0}}{\isadigit{0}}{\isacharparenright}\isanewline
\isanewline
\isamarkupfalse%
\isacommand{translations}\isanewline
\ \ {\isachardoublequote}{\isacharunderscore}readRefD\ r{\isachardoublequote}\ \ \ \ \ \ \ \ \ {\isasymrightleftharpoons}\ \ \ \ {\isachardoublequote}{\isasymUp}\ {\isacharparenleft}readRef\ r{\isacharparenright}{\isachardoublequote}\isamarkupfalse%
%
\begin{isamarkuptext}%
This definition is rather useless as it stands, since one actually wants
  \isa{oldref\ r} to be a formula in \isa{bool\ D}. The quantifier is necessary to
  avoid introducing a fresh variable \isa{a} on the right hand side of the
  definition.
  
  The idea is appealing however, since it would provide a statement about the
  existence of \isa{r} as a reference.%
\end{isamarkuptext}%
\isamarkuptrue%
\isacommand{constdefs}\isanewline
\ \ oldref\ \ \ \ \ {\isacharcolon}{\isacharcolon}\ {\isachardoublequote}{\isacharprime}a\ ref\ {\isasymRightarrow}\ bool{\isachardoublequote}\isanewline
\ \ {\isachardoublequote}oldref\ r\ \ {\isasymequiv}\ \ {\isasymforall}a{\isachardot}\ {\isasymturnstile}\ {\isacharbrackleft}{\isacharhash}\ s{\isasymleftarrow}newRef\ a{\isacharbrackright}{\isacharparenleft}Ret\ {\isacharparenleft}{\isasymnot}{\isacharparenleft}r{\isacharequal}s{\isacharparenright}{\isacharparenright}{\isacharparenright}{\isachardoublequote}\isamarkupfalse%
%
\begin{isamarkuptext}%
The basic axioms of a simple while language with references. In the following we will not
  make use of operation \isa{newRef} and hence neither of its axioms.%
\end{isamarkuptext}%
\isamarkuptrue%
\isacommand{axioms}\isanewline
dsef{\isacharunderscore}read{\isacharcolon}\ \ \ \ \ {\isachardoublequote}dsef\ {\isacharparenleft}readRef\ r{\isacharparenright}{\isachardoublequote}\isanewline
read{\isacharunderscore}write{\isacharcolon}\ \ \ \ {\isachardoublequote}{\isasymturnstile}\ {\isacharbrackleft}{\isacharhash}\ r\ {\isacharcolon}{\isacharequal}\ x{\isacharbrackright}{\isacharparenleft}{\isasymlambda}uu{\isachardot}\ {\isacharasterisk}r\ {\isacharequal}\isactrlsub D\ Ret\ x{\isacharparenright}{\isachardoublequote}\isanewline
read{\isacharunderscore}write{\isacharunderscore}other{\isacharunderscore}gen{\isacharcolon}\ {\isachardoublequote}{\isasymturnstile}\ {\isasymUp}\ {\isacharparenleft}do\ {\isacharbraceleft}u{\isasymleftarrow}readRef\ r{\isacharsemicolon}\ ret\ {\isacharparenleft}f\ u{\isacharparenright}{\isacharbraceright}{\isacharparenright}\ {\isasymlongrightarrow}\isactrlsub D\ \isanewline
\ \ \ \ \ \ \ \ \ \ \ \ \ \ \ \ \ \ \ \ \ \ \ \ \ \ \ \ {\isacharbrackleft}{\isacharhash}\ s\ {\isacharcolon}{\isacharequal}\ y{\isacharbrackright}{\isacharparenleft}{\isasymlambda}uu{\isachardot}\ Ret\ {\isacharparenleft}r{\isasymnoteq}s{\isacharparenright}\ {\isasymlongrightarrow}\isactrlsub D\ {\isasymUp}\ {\isacharparenleft}do\ {\isacharbraceleft}u{\isasymleftarrow}readRef\ r{\isacharsemicolon}\ ret\ {\isacharparenleft}f\ u{\isacharparenright}{\isacharbraceright}{\isacharparenright}{\isacharparenright}{\isachardoublequote}\isanewline
while{\isacharunderscore}par{\isacharcolon}\ \ \ \ \ {\isachardoublequote}{\isasymturnstile}\ P\ {\isasymand}\isactrlsub D\ b\ {\isasymlongrightarrow}\isactrlsub D\ {\isacharbrackleft}{\isacharhash}\ p{\isacharbrackright}{\isacharparenleft}{\isasymlambda}u{\isachardot}\ P{\isacharparenright}\ {\isasymLongrightarrow}\ {\isasymturnstile}\ P\ {\isasymlongrightarrow}\isactrlsub D\ {\isacharbrackleft}{\isacharhash}\ WHILE\ b\ DO\ p\ END{\isacharbrackright}{\isacharparenleft}{\isasymlambda}x{\isachardot}\ P\ {\isasymand}\isactrlsub D\ {\isasymnot}\isactrlsub D\ b{\isacharparenright}{\isachardoublequote}\isanewline
read{\isacharunderscore}new{\isacharcolon}\ \ \ \ \ \ {\isachardoublequote}{\isasymturnstile}\ {\isacharbrackleft}{\isacharhash}\ r{\isasymleftarrow}newRef\ a{\isacharbrackright}{\isacharparenleft}\ Ret\ a\ {\isacharequal}\isactrlsub D\ {\isacharasterisk}r{\isacharparenright}{\isachardoublequote}\isanewline
read{\isacharunderscore}new{\isacharunderscore}other{\isacharcolon}\ {\isachardoublequote}{\isasymturnstile}\ {\isacharparenleft}Ret\ x\ {\isacharequal}\isactrlsub D\ {\isacharasterisk}r{\isacharparenright}\ {\isasymlongrightarrow}\isactrlsub D\ {\isacharbrackleft}{\isacharhash}\ s\ {\isasymleftarrow}\ newRef\ y{\isacharbrackright}{\isacharparenleft}{\isacharparenleft}Ret\ x\ {\isacharequal}\isactrlsub D\ {\isacharasterisk}r{\isacharparenright}\ {\isasymor}\isactrlsub D\ Ret\ {\isacharparenleft}r{\isacharequal}s{\isacharparenright}{\isacharparenright}{\isachardoublequote}\isanewline
\isanewline
\isanewline
\isanewline
\isamarkupfalse%
\isacommand{lemma}\ read{\isacharunderscore}write{\isacharunderscore}other{\isacharcolon}\ {\isachardoublequote}{\isasymturnstile}\ {\isacharparenleft}\ {\isacharasterisk}r\ {\isacharequal}\isactrlsub D\ Ret\ x{\isacharparenright}\ {\isasymlongrightarrow}\isactrlsub D\ {\isacharbrackleft}{\isacharhash}\ s\ {\isacharcolon}{\isacharequal}\ y{\isacharbrackright}{\isacharparenleft}{\isasymlambda}uu{\isachardot}\ Ret\ {\isacharparenleft}r{\isasymnoteq}s{\isacharparenright}\ {\isasymlongrightarrow}\isactrlsub D\ {\isacharparenleft}\ {\isacharasterisk}r\ {\isacharequal}\isactrlsub D\ Ret\ x{\isacharparenright}{\isacharparenright}{\isachardoublequote}\isanewline
\isamarkupfalse%
\isacommand{proof}\ {\isacharminus}\isanewline
\ \ \isamarkupfalse%
\isacommand{have}\ {\isachardoublequote}{\isasymturnstile}\ {\isasymUp}\ {\isacharparenleft}do\ {\isacharbraceleft}u{\isasymleftarrow}readRef\ r{\isacharsemicolon}\ ret\ {\isacharparenleft}u\ {\isacharequal}\ x{\isacharparenright}{\isacharbraceright}{\isacharparenright}\ {\isasymlongrightarrow}\isactrlsub D\isanewline
\ \ \ \ \ \ \ \ \ \ \ \ {\isacharbrackleft}{\isacharhash}\ s\ {\isacharcolon}{\isacharequal}\ y{\isacharbrackright}{\isacharparenleft}{\isasymlambda}uu{\isachardot}\ Ret\ {\isacharparenleft}r{\isasymnoteq}s{\isacharparenright}\ {\isasymlongrightarrow}\isactrlsub D\ {\isasymUp}\ {\isacharparenleft}do\ {\isacharbraceleft}u{\isasymleftarrow}readRef\ r{\isacharsemicolon}\ ret\ {\isacharparenleft}u\ {\isacharequal}\ x{\isacharparenright}{\isacharbraceright}{\isacharparenright}{\isacharparenright}{\isachardoublequote}\isanewline
\ \ \ \ \isamarkupfalse%
\isacommand{by}\ {\isacharparenleft}rule\ read{\isacharunderscore}write{\isacharunderscore}other{\isacharunderscore}gen{\isacharparenright}\isanewline
\ \ \isamarkupfalse%
\isacommand{thus}\ {\isacharquery}thesis\isanewline
\ \ \ \ \isamarkupfalse%
\isacommand{by}\ {\isacharparenleft}simp\ add{\isacharcolon}\ MonEq{\isacharunderscore}def\ liftM{\isadigit{2}}{\isacharunderscore}def\ Dsef{\isacharunderscore}def\ Ret{\isacharunderscore}def\ Abs{\isacharunderscore}Dsef{\isacharunderscore}inverse\ dsef{\isacharunderscore}read{\isacharparenright}\isanewline
\isamarkupfalse%
\isacommand{qed}\isamarkupfalse%
%
\begin{isamarkuptext}%
It is not really necessary to step back to the do-notation for 
  \isa{read{\isacharunderscore}write{\isacharunderscore}other{\isacharunderscore}gen}.%
\end{isamarkuptext}%
\isamarkuptrue%
\isacommand{lemma}\ {\isachardoublequote}{\isasymturnstile}\ {\isacharasterisk}r\ {\isacharequal}\isactrlsub D\ Ret\ b\ {\isasymand}\isactrlsub D\ Ret\ {\isacharparenleft}f\ b{\isacharparenright}\ {\isasymlongrightarrow}\isactrlsub D\ {\isasymUp}\ {\isacharparenleft}do\ {\isacharbraceleft}a{\isasymleftarrow}readRef\ r{\isacharsemicolon}\ ret\ {\isacharparenleft}f\ a\ {\isasymand}\ a\ {\isacharequal}\ b{\isacharparenright}{\isacharbraceright}{\isacharparenright}{\isachardoublequote}\isamarkupfalse%
\isamarkupfalse%
\isamarkupfalse%
\isamarkupfalse%
\isamarkupfalse%
%
\begin{isamarkuptext}%
Definitions of oddity and evenness of natural numbers, as well as an algorithm
  for computing Russian multiplication \isa{rumult}.
  \label{isa:rumult-spec}%
\end{isamarkuptext}%
\isamarkuptrue%
\isacommand{constdefs}\isanewline
\ \ nat{\isacharunderscore}even\ \ {\isacharcolon}{\isacharcolon}\ {\isachardoublequote}nat\ {\isasymRightarrow}\ bool{\isachardoublequote}\isanewline
\ \ {\isachardoublequote}nat{\isacharunderscore}even\ n\ {\isasymequiv}\ {\isadigit{2}}\ dvd\ n{\isachardoublequote}\isanewline
\ \ nat{\isacharunderscore}odd\ \ \ {\isacharcolon}{\isacharcolon}\ {\isachardoublequote}nat\ {\isasymRightarrow}\ bool{\isachardoublequote}\isanewline
\ \ {\isachardoublequote}nat{\isacharunderscore}odd\ n\ {\isasymequiv}\ {\isasymnot}\ nat{\isacharunderscore}even\ n{\isachardoublequote}\isanewline
\ \ rumult\ \ \ \ \ {\isacharcolon}{\isacharcolon}\ {\isachardoublequote}nat\ {\isasymRightarrow}\ nat\ {\isasymRightarrow}\ nat\ ref\ {\isasymRightarrow}\ nat\ ref\ {\isasymRightarrow}\ nat\ ref\ {\isasymRightarrow}\ nat\ T{\isachardoublequote}\isanewline
\ \ {\isachardoublequote}rumult\ a\ b\ x\ y\ r\ {\isasymequiv}\ do\ {\isacharbraceleft}x{\isacharcolon}{\isacharequal}a{\isacharsemicolon}\ y{\isacharcolon}{\isacharequal}b{\isacharsemicolon}\ r{\isacharcolon}{\isacharequal}{\isadigit{0}}{\isacharsemicolon}\isanewline
\ \ \ \ \ \ \ \ \ \ \ \ \ \ \ \ \ \ \ \ \ \ \ \ \ \ WHILE\ {\isacharparenleft}{\isasymUp}\ {\isacharparenleft}do\ {\isacharbraceleft}u{\isasymleftarrow}readRef\ x{\isacharsemicolon}\ ret\ {\isacharparenleft}{\isadigit{0}}\ {\isacharless}\ u{\isacharparenright}{\isacharbraceright}{\isacharparenright}{\isacharparenright}\isanewline
\ \ \ \ \ \ \ \ \ \ \ \ \ \ \ \ \ \ \ \ \ \ \ \ \ \ \ \ \ DO\ do\ {\isacharbraceleft}u{\isasymleftarrow}readRef\ x{\isacharsemicolon}\ v{\isasymleftarrow}readRef\ y{\isacharsemicolon}\ w{\isasymleftarrow}readRef\ r{\isacharsemicolon}\isanewline
\ \ \ \ \ \ \ \ \ \ \ \ \ \ \ \ \ \ \ \ \ \ \ \ \ \ \ \ \ \ \ \ \ \ \ \ if\ {\isacharparenleft}nat{\isacharunderscore}odd\ u{\isacharparenright}\ then\ {\isacharparenleft}r\ {\isacharcolon}{\isacharequal}\ w\ {\isacharplus}\ v{\isacharparenright}\ else\ ret\ {\isacharparenleft}{\isacharparenright}{\isacharsemicolon}\isanewline
\ \ \ \ \ \ \ \ \ \ \ \ \ \ \ \ \ \ \ \ \ \ \ \ \ \ \ \ \ \ \ \ \ \ \ \ x\ {\isacharcolon}{\isacharequal}\ u\ div\ {\isadigit{2}}{\isacharsemicolon}\ y\ {\isacharcolon}{\isacharequal}\ v\ {\isacharasterisk}\ {\isadigit{2}}{\isacharbraceright}\ END{\isacharsemicolon}\ readRef\ r{\isacharbraceright}{\isachardoublequote}\isamarkupfalse%
%
\isamarkupsubsection{General Auxiliary Lemmas%
}
\isamarkuptrue%
%
\begin{isamarkuptext}%
Following are several auxiliary lemmas which are not general enough to be placed
  inside the general theory files, but which are used more than once below -- and thus
  justify their mere existence.%
\end{isamarkuptext}%
\isamarkuptrue%
%
\begin{isamarkuptext}%
Some weakening rules.%
\end{isamarkuptext}%
\isamarkuptrue%
\isacommand{lemma}\ pdl{\isacharunderscore}conj{\isacharunderscore}imp{\isacharunderscore}wk{\isadigit{1}}{\isacharcolon}\ {\isachardoublequote}{\isasymturnstile}\ A\ {\isasymlongrightarrow}\isactrlsub D\ C\ {\isasymLongrightarrow}\ {\isasymturnstile}\ A\ {\isasymand}\isactrlsub D\ B\ {\isasymlongrightarrow}\isactrlsub D\ C{\isachardoublequote}\isanewline
\isamarkupfalse%
\isacommand{proof}\ {\isacharminus}\isanewline
\ \ \isamarkupfalse%
\isacommand{assume}\ {\isachardoublequote}{\isasymturnstile}\ A\ {\isasymlongrightarrow}\isactrlsub D\ C{\isachardoublequote}\isanewline
\ \ \isamarkupfalse%
\isacommand{have}\ {\isachardoublequote}{\isasymturnstile}\ {\isacharparenleft}A\ {\isasymlongrightarrow}\isactrlsub D\ C{\isacharparenright}\ {\isasymlongrightarrow}\isactrlsub D\ A\ {\isasymand}\isactrlsub D\ B\ {\isasymlongrightarrow}\isactrlsub D\ C{\isachardoublequote}\isanewline
\ \ \ \ \isamarkupfalse%
\isacommand{by}\ {\isacharparenleft}simp\ add{\isacharcolon}\ pdl{\isacharunderscore}taut{\isacharparenright}\isanewline
\ \ \isamarkupfalse%
\isacommand{thus}\ {\isacharquery}thesis\ \isamarkupfalse%
\isacommand{by}\ {\isacharparenleft}rule\ pdl{\isacharunderscore}mp{\isacharparenright}\isanewline
\isamarkupfalse%
\isacommand{qed}\isanewline
\isanewline
\isamarkupfalse%
\isacommand{lemma}\ pdl{\isacharunderscore}conj{\isacharunderscore}imp{\isacharunderscore}wk{\isadigit{2}}{\isacharcolon}\ {\isachardoublequote}{\isasymturnstile}\ B\ {\isasymlongrightarrow}\isactrlsub D\ C\ {\isasymLongrightarrow}\ {\isasymturnstile}\ A\ {\isasymand}\isactrlsub D\ B\ {\isasymlongrightarrow}\isactrlsub D\ C{\isachardoublequote}\isanewline
\isamarkupfalse%
\isacommand{proof}\ {\isacharminus}\isanewline
\ \ \isamarkupfalse%
\isacommand{assume}\ {\isachardoublequote}{\isasymturnstile}\ B\ {\isasymlongrightarrow}\isactrlsub D\ C{\isachardoublequote}\isanewline
\ \ \isamarkupfalse%
\isacommand{have}\ {\isachardoublequote}{\isasymturnstile}\ {\isacharparenleft}B\ {\isasymlongrightarrow}\isactrlsub D\ C{\isacharparenright}\ {\isasymlongrightarrow}\isactrlsub D\ A\ {\isasymand}\isactrlsub D\ B\ {\isasymlongrightarrow}\isactrlsub D\ C{\isachardoublequote}\isanewline
\ \ \ \ \isamarkupfalse%
\isacommand{by}\ {\isacharparenleft}simp\ add{\isacharcolon}\ pdl{\isacharunderscore}taut{\isacharparenright}\isanewline
\ \ \isamarkupfalse%
\isacommand{thus}\ {\isacharquery}thesis\ \isamarkupfalse%
\isacommand{by}\ {\isacharparenleft}rule\ pdl{\isacharunderscore}mp{\isacharparenright}\isanewline
\isamarkupfalse%
\isacommand{qed}\isamarkupfalse%
%
\begin{isamarkuptext}%
The following can be used to prove a specific goal by proving two parts separately. It is
  similar to \isa{pdl{\isacharunderscore}iffD{\isadigit{2}}\ {\isacharbrackleft}\ OF\ box{\isacharunderscore}conj{\isacharunderscore}distrib{\isacharunderscore}lifted{\isadigit{1}}\ {\isacharcomma}\ THEN\ pdl{\isacharunderscore}mp\ {\isacharbrackright}}, which is

  \begin{isabelle}%
{\isasymturnstile}\ {\isacharparenleft}A{\isacharunderscore}{\isadigit{2}}\ {\isasymlongrightarrow}\isactrlsub D\ {\isacharbrackleft}{\isacharhash}\ p{\isacharunderscore}{\isadigit{2}}{\isacharbrackright}P{\isacharunderscore}{\isadigit{2}}{\isacharparenright}\ {\isasymand}\isactrlsub D\ {\isacharparenleft}A{\isacharunderscore}{\isadigit{2}}\ {\isasymlongrightarrow}\isactrlsub D\ {\isacharbrackleft}{\isacharhash}\ p{\isacharunderscore}{\isadigit{2}}{\isacharbrackright}Q{\isacharunderscore}{\isadigit{2}}{\isacharparenright}\ {\isasymLongrightarrow}\isanewline
{\isasymturnstile}\ A{\isacharunderscore}{\isadigit{2}}\ {\isasymlongrightarrow}\isactrlsub D\ {\isacharbrackleft}{\isacharhash}\ p{\isacharunderscore}{\isadigit{2}}{\isacharbrackright}{\isacharparenleft}{\isasymlambda}x{\isachardot}\ P{\isacharunderscore}{\isadigit{2}}\ x\ {\isasymand}\isactrlsub D\ Q{\isacharunderscore}{\isadigit{2}}\ x{\isacharparenright}%
\end{isabelle}.%
\end{isamarkuptext}%
\isamarkuptrue%
\isacommand{lemma}\ pdl{\isacharunderscore}conj{\isacharunderscore}imp{\isacharunderscore}box{\isacharunderscore}split{\isacharcolon}\ {\isachardoublequote}{\isasymlbrakk}{\isasymturnstile}\ A\ {\isasymlongrightarrow}\isactrlsub D\ {\isacharbrackleft}{\isacharhash}\ p{\isacharbrackright}C{\isacharsemicolon}\ {\isasymturnstile}\ B\ {\isasymlongrightarrow}\isactrlsub D\ {\isacharbrackleft}{\isacharhash}\ p{\isacharbrackright}D{\isasymrbrakk}\ {\isasymLongrightarrow}\ {\isasymturnstile}\ A\ {\isasymand}\isactrlsub D\ B\ {\isasymlongrightarrow}\isactrlsub D\ {\isacharbrackleft}{\isacharhash}\ x{\isasymleftarrow}p{\isacharbrackright}{\isacharparenleft}C\ x\ {\isasymand}\isactrlsub D\ D\ x{\isacharparenright}{\isachardoublequote}\isanewline
\isamarkupfalse%
\isacommand{proof}\ {\isacharparenleft}rule\ pdl{\isacharunderscore}iffD{\isadigit{2}}{\isacharbrackleft}OF\ box{\isacharunderscore}conj{\isacharunderscore}distrib{\isacharunderscore}lifted{\isadigit{1}}{\isacharcomma}\ THEN\ pdl{\isacharunderscore}mp{\isacharbrackright}{\isacharparenright}\isanewline
\ \ \isamarkupfalse%
\isacommand{assume}\ a{\isadigit{1}}{\isacharcolon}\ {\isachardoublequote}{\isasymturnstile}\ A\ {\isasymlongrightarrow}\isactrlsub D\ {\isacharbrackleft}{\isacharhash}\ p{\isacharbrackright}C{\isachardoublequote}\ \isakeyword{and}\ a{\isadigit{2}}{\isacharcolon}\ {\isachardoublequote}{\isasymturnstile}\ B\ {\isasymlongrightarrow}\isactrlsub D\ {\isacharbrackleft}{\isacharhash}\ p{\isacharbrackright}D{\isachardoublequote}\isanewline
\ \ \isamarkupfalse%
\isacommand{show}\ {\isachardoublequote}{\isasymturnstile}\ {\isacharparenleft}A\ {\isasymand}\isactrlsub D\ B\ {\isasymlongrightarrow}\isactrlsub D\ {\isacharbrackleft}{\isacharhash}\ p{\isacharbrackright}C{\isacharparenright}\ {\isasymand}\isactrlsub D\ {\isacharparenleft}A\ {\isasymand}\isactrlsub D\ B\ {\isasymlongrightarrow}\isactrlsub D\ {\isacharbrackleft}{\isacharhash}\ p{\isacharbrackright}D{\isacharparenright}{\isachardoublequote}\isanewline
\ \ \isamarkupfalse%
\isacommand{proof}\ {\isacharparenleft}rule\ pdl{\isacharunderscore}conjI{\isacharparenright}\isanewline
\ \ \ \ \isamarkupfalse%
\isacommand{show}\ {\isachardoublequote}{\isasymturnstile}\ A\ {\isasymand}\isactrlsub D\ B\ {\isasymlongrightarrow}\isactrlsub D\ {\isacharbrackleft}{\isacharhash}\ p{\isacharbrackright}C{\isachardoublequote}\isanewline
\ \ \ \ \isamarkupfalse%
\isacommand{proof}\ {\isacharparenleft}rule\ pdl{\isacharunderscore}conj{\isacharunderscore}imp{\isacharunderscore}wk{\isadigit{1}}{\isacharparenright}\ \isanewline
\ \ \ \ \ \ \isamarkupfalse%
\isacommand{show}\ {\isachardoublequote}{\isasymturnstile}\ A\ {\isasymlongrightarrow}\isactrlsub D\ {\isacharbrackleft}{\isacharhash}\ p{\isacharbrackright}C{\isachardoublequote}\ \isamarkupfalse%
\isacommand{{\isachardot}}\isanewline
\ \ \ \ \isamarkupfalse%
\isacommand{qed}\isanewline
\ \ \isamarkupfalse%
\isacommand{next}\isanewline
\ \ \ \ \isamarkupfalse%
\isacommand{show}\ {\isachardoublequote}{\isasymturnstile}\ A\ {\isasymand}\isactrlsub D\ B\ {\isasymlongrightarrow}\isactrlsub D\ {\isacharbrackleft}{\isacharhash}\ p{\isacharbrackright}D{\isachardoublequote}\isanewline
\ \ \ \ \isamarkupfalse%
\isacommand{proof}\ {\isacharparenleft}rule\ pdl{\isacharunderscore}conj{\isacharunderscore}imp{\isacharunderscore}wk{\isadigit{2}}{\isacharparenright}\isanewline
\ \ \ \ \ \ \isamarkupfalse%
\isacommand{show}\ {\isachardoublequote}{\isasymturnstile}\ B\ {\isasymlongrightarrow}\isactrlsub D\ {\isacharbrackleft}{\isacharhash}\ p{\isacharbrackright}D{\isachardoublequote}\ \isamarkupfalse%
\isacommand{{\isachardot}}\isanewline
\ \ \ \ \isamarkupfalse%
\isacommand{qed}\isanewline
\ \ \isamarkupfalse%
\isacommand{qed}\isanewline
\isamarkupfalse%
\isacommand{qed}\isamarkupfalse%
%
\begin{isamarkuptext}%
Since dsef programs may be discarded, a formula is equal to itself prefixed
  by such a program.%
\end{isamarkuptext}%
\isamarkuptrue%
\isacommand{lemma}\ dsef{\isacharunderscore}form{\isacharunderscore}eq{\isacharcolon}\ {\isachardoublequote}dsef\ p\ {\isasymLongrightarrow}\ P\ {\isacharequal}\ {\isasymUp}\ {\isacharparenleft}do\ {\isacharbraceleft}a{\isasymleftarrow}p{\isacharsemicolon}\ {\isasymDown}\ P{\isacharbraceright}{\isacharparenright}{\isachardoublequote}\isanewline
\isamarkupfalse%
\isacommand{proof}\ {\isacharminus}\isanewline
\ \ \isamarkupfalse%
\isacommand{assume}\ a{\isadigit{1}}{\isacharcolon}\ {\isachardoublequote}dsef\ p{\isachardoublequote}\isanewline
\ \ \isamarkupfalse%
\isacommand{have}\ f{\isadigit{1}}{\isacharcolon}\ {\isachardoublequote}do\ {\isacharbraceleft}a{\isasymleftarrow}p{\isacharsemicolon}\ {\isasymDown}\ P{\isacharbraceright}\ {\isacharequal}\ {\isasymDown}\ P{\isachardoublequote}\isanewline
\ \ \isamarkupfalse%
\isacommand{proof}\ {\isacharparenleft}rule\ dis{\isacharunderscore}left{\isadigit{2}}{\isacharparenright}\isanewline
\ \ \ \ \isamarkupfalse%
\isacommand{show}\ {\isachardoublequote}dis\ p{\isachardoublequote}\isanewline
\ \ \ \ \ \ \isamarkupfalse%
\isacommand{by}\ {\isacharparenleft}rule\ dsef{\isacharunderscore}dis{\isacharbrackleft}OF\ a{\isadigit{1}}{\isacharbrackright}{\isacharparenright}\isanewline
\ \ \isamarkupfalse%
\isacommand{qed}\isanewline
\ \ \isamarkupfalse%
\isacommand{thus}\ {\isacharquery}thesis\ \isanewline
\ \ \isamarkupfalse%
\isacommand{proof}\ {\isacharminus}\isanewline
\ \ \ \ \isamarkupfalse%
\isacommand{have}\ {\isachardoublequote}P\ \ {\isacharequal}\ {\isasymUp}\ {\isacharparenleft}{\isasymDown}\ P{\isacharparenright}{\isachardoublequote}\isanewline
\ \ \ \ \ \ \isamarkupfalse%
\isacommand{by}\ {\isacharparenleft}rule\ Rep{\isacharunderscore}Dsef{\isacharunderscore}inverse{\isacharbrackleft}symmetric{\isacharbrackright}{\isacharparenright}\isanewline
\ \ \ \ \isamarkupfalse%
\isacommand{with}\ f{\isadigit{1}}\ \isamarkupfalse%
\isacommand{show}\ {\isacharquery}thesis\ \isamarkupfalse%
\isacommand{by}\ simp\isanewline
\ \ \isamarkupfalse%
\isacommand{qed}\isanewline
\isamarkupfalse%
\isacommand{qed}\isamarkupfalse%
%
\begin{isamarkuptext}%
A rendition of \isa{pdl{\isacharunderscore}dsefB}.%
\end{isamarkuptext}%
\isamarkuptrue%
\isacommand{lemma}\ dsefB{\isacharunderscore}D{\isacharcolon}\ {\isachardoublequote}dsef\ p\ {\isasymLongrightarrow}\ {\isasymturnstile}\ P\ {\isasymlongrightarrow}\isactrlsub D\ {\isacharbrackleft}{\isacharhash}\ x{\isasymleftarrow}p{\isacharbrackright}P{\isachardoublequote}\isanewline
\isamarkupfalse%
\isacommand{by}{\isacharparenleft}subst\ dsef{\isacharunderscore}form{\isacharunderscore}eq{\isacharbrackleft}of\ p\ P{\isacharbrackright}{\isacharcomma}\ assumption{\isacharcomma}\ rule\ pdl{\isacharunderscore}iffD{\isadigit{1}}{\isacharbrackleft}OF\ pdl{\isacharunderscore}dsefB{\isacharbrackright}{\isacharparenright}\isamarkupfalse%
%
\begin{isamarkuptext}%
An even number is equal to the sum of its div-halves.%
\end{isamarkuptext}%
\isamarkuptrue%
\isacommand{lemma}\ even{\isacharunderscore}div{\isacharunderscore}eq{\isacharcolon}\ {\isachardoublequote}nat{\isacharunderscore}even\ n\ {\isacharequal}\ {\isacharparenleft}n\ div\ {\isadigit{2}}\ {\isacharplus}\ n\ div\ {\isadigit{2}}\ {\isacharequal}\ n{\isacharparenright}{\isachardoublequote}\isanewline
\ \ \isamarkupfalse%
\isacommand{apply}{\isacharparenleft}unfold\ nat{\isacharunderscore}even{\isacharunderscore}def{\isacharparenright}\isanewline
\ \ \isamarkupfalse%
\isacommand{by}\ arith\isamarkupfalse%
%
\begin{isamarkuptext}%
Dividing $n$ by two and adding the result to itself yields a number one less
  than $n$.%
\end{isamarkuptext}%
\isamarkuptrue%
\isacommand{lemma}\ odd{\isacharunderscore}div{\isacharunderscore}eq{\isacharcolon}\ {\isachardoublequote}nat{\isacharunderscore}odd\ {\isacharparenleft}x{\isacharcolon}{\isacharcolon}nat{\isacharparenright}\ {\isacharequal}\ {\isacharparenleft}x\ div\ {\isadigit{2}}\ {\isacharplus}\ x\ div\ {\isadigit{2}}\ {\isacharplus}\ {\isadigit{1}}\ {\isacharequal}\ x{\isacharparenright}{\isachardoublequote}\isanewline
\ \ \isamarkupfalse%
\isacommand{apply}{\isacharparenleft}simp\ add{\isacharcolon}\ nat{\isacharunderscore}odd{\isacharunderscore}def\ nat{\isacharunderscore}even{\isacharunderscore}def{\isacharparenright}\isanewline
\ \ \isamarkupfalse%
\isacommand{by}\ {\isacharparenleft}arith{\isacharparenright}\isamarkupfalse%
%
\begin{isamarkuptext}%
A slight variant of \isa{pdl{\isacharunderscore}dsefB} for stateless formulas.%
\end{isamarkuptext}%
\isamarkuptrue%
\isacommand{lemma}\ pdl{\isacharunderscore}dsefB{\isacharunderscore}ret{\isacharcolon}\ {\isachardoublequote}dsef\ p\ {\isasymLongrightarrow}\ {\isasymturnstile}\ {\isasymUp}\ {\isacharparenleft}do\ {\isacharbraceleft}a{\isasymleftarrow}p{\isacharsemicolon}\ ret\ {\isacharparenleft}P\ a{\isacharparenright}{\isacharbraceright}{\isacharparenright}\ {\isasymlongleftrightarrow}\isactrlsub D\ {\isacharbrackleft}{\isacharhash}\ a{\isasymleftarrow}p{\isacharbrackright}{\isacharparenleft}Ret\ {\isacharparenleft}P\ a{\isacharparenright}{\isacharparenright}{\isachardoublequote}\isanewline
\ \ \isamarkupfalse%
\isacommand{apply}{\isacharparenleft}subgoal{\isacharunderscore}tac\ {\isachardoublequote}{\isasymforall}a{\isachardot}\ ret\ {\isacharparenleft}P\ a{\isacharparenright}\ {\isacharequal}\ {\isasymDown}\ Ret\ {\isacharparenleft}P\ a{\isacharparenright}{\isachardoublequote}{\isacharparenright}\isanewline
\ \ \isamarkupfalse%
\isacommand{apply}{\isacharparenleft}simp{\isacharparenright}\isanewline
\ \ \isamarkupfalse%
\isacommand{apply}{\isacharparenleft}rule\ pdl{\isacharunderscore}dsefB{\isacharparenright}\isanewline
\ \ \isamarkupfalse%
\isacommand{apply}{\isacharparenleft}assumption{\isacharparenright}\isanewline
\ \ \isamarkupfalse%
\isacommand{apply}{\isacharparenleft}simp\ add{\isacharcolon}\ Ret{\isacharunderscore}ret{\isacharparenright}\isanewline
\isamarkupfalse%
\isacommand{done}\isamarkupfalse%
%
\isamarkupsubsection{Problem-Specific Auxiliary Lemmas%
}
\isamarkuptrue%
%
\begin{isamarkuptext}%
The following lemmas are required for the final correctness proof to go through, but
  are of rather limited interest in general.%
\end{isamarkuptext}%
\isamarkuptrue%
\isacommand{lemma}\ var{\isacharunderscore}aux{\isadigit{1}}{\isacharcolon}\ {\isachardoublequote}{\isasymturnstile}\ {\isacharparenleft}\ {\isacharasterisk}y\ {\isacharequal}\isactrlsub D\ Ret\ b\ {\isasymand}\isactrlsub D\ Ret\ {\isacharparenleft}x\ {\isasymnoteq}\ y\ {\isasymand}\ y\ {\isasymnoteq}\ r\ {\isasymand}\ x\ {\isasymnoteq}\ r{\isacharparenright}\ {\isasymand}\isactrlsub D\ {\isacharparenleft}Ret\ {\isacharparenleft}x\ {\isasymnoteq}\ y{\isacharparenright}\ {\isasymlongrightarrow}\isactrlsub D\ {\isacharasterisk}x\ {\isacharequal}\isactrlsub D\ Ret\ a{\isacharparenright}\ {\isacharparenright}\ {\isasymlongrightarrow}\isactrlsub D\isanewline
\ \ \ \ \ \ \ \ \ \ \ \ \ \ \ {\isacharparenleft}\ {\isacharasterisk}x\ {\isacharequal}\isactrlsub D\ Ret\ a\ {\isasymand}\isactrlsub D\ {\isacharasterisk}y\ {\isacharequal}\isactrlsub D\ Ret\ b\ {\isasymand}\isactrlsub D\ Ret\ {\isacharparenleft}x\ {\isasymnoteq}\ y\ {\isasymand}\ y\ {\isasymnoteq}\ r\ {\isasymand}\ x\ {\isasymnoteq}\ r{\isacharparenright}\ {\isacharparenright}{\isachardoublequote}\isanewline
\ \ \isamarkupfalse%
\isacommand{by}\ {\isacharparenleft}simp\ add{\isacharcolon}\ conjD{\isacharunderscore}Ret{\isacharunderscore}hom\ pdl{\isacharunderscore}taut{\isacharparenright}\isanewline
\isanewline
\isanewline
\isamarkupfalse%
\isacommand{lemma}\ var{\isacharunderscore}aux{\isadigit{2}}{\isacharcolon}\ {\isachardoublequote}{\isasymturnstile}\ {\isacharparenleft}{\isacharparenleft}\ {\isacharasterisk}r\ {\isacharequal}\isactrlsub D\ Ret\ {\isadigit{0}}\ {\isasymand}\isactrlsub D\ Ret\ {\isacharparenleft}x\ {\isasymnoteq}\ y\ {\isasymand}\ y\ {\isasymnoteq}\ r\ {\isasymand}\ x\ {\isasymnoteq}\ r{\isacharparenright}{\isacharparenright}\ {\isasymand}\isactrlsub D\ {\isacharparenleft}Ret\ {\isacharparenleft}x\ {\isasymnoteq}\ r{\isacharparenright}\ {\isasymlongrightarrow}\isactrlsub D\ {\isacharasterisk}x\ {\isacharequal}\isactrlsub D\ Ret\ a{\isacharparenright}{\isacharparenright}\ {\isasymand}\isactrlsub D\isanewline
\ \ \ \ \ \ \ \ \ \ \ \ \ \ \ \ \ \ \ {\isacharparenleft}Ret\ {\isacharparenleft}y\ {\isasymnoteq}\ r{\isacharparenright}\ {\isasymlongrightarrow}\isactrlsub D\ {\isacharasterisk}y\ {\isacharequal}\isactrlsub D\ Ret\ b{\isacharparenright}\ {\isasymlongrightarrow}\isactrlsub D\isanewline
\ \ \ \ \ \ \ \ \ \ \ \ \ \ \ \ \ \ \ {\isacharparenleft}\ {\isacharasterisk}x\ {\isacharequal}\isactrlsub D\ Ret\ a\ {\isasymand}\isactrlsub D\ {\isacharasterisk}y\ {\isacharequal}\isactrlsub D\ Ret\ b\ {\isasymand}\isactrlsub D\ {\isacharasterisk}r\ {\isacharequal}\isactrlsub D\ Ret\ {\isacharparenleft}{\isadigit{0}}{\isacharcolon}{\isacharcolon}nat{\isacharparenright}\ {\isasymand}\isactrlsub D\ Ret\ {\isacharparenleft}x\ {\isasymnoteq}\ y\ {\isasymand}\ y\ {\isasymnoteq}\ r\ {\isasymand}\ x\ {\isasymnoteq}\ r{\isacharparenright}{\isacharparenright}{\isachardoublequote}\isanewline
\ \ \isamarkupfalse%
\isacommand{by}\ {\isacharparenleft}simp\ add{\isacharcolon}\ conjD{\isacharunderscore}Ret{\isacharunderscore}hom\ pdl{\isacharunderscore}taut{\isacharparenright}\isamarkupfalse%
%
\begin{isamarkuptext}%
The following proof it typical: since some formulas are built from do-terms and then lifted
  into \isa{bool\ D}, the usual proof rules will not get us far. The standard scheme in this 
  case is to proceed as documented in the following side remarks.%
\end{isamarkuptext}%
\isamarkuptrue%
\isacommand{lemma}\ derive{\isacharunderscore}inv{\isacharunderscore}aux{\isacharcolon}\ {\isachardoublequote}\ {\isasymturnstile}\ {\isacharasterisk}x\ {\isacharequal}\isactrlsub D\ Ret\ a\ {\isasymand}\isactrlsub D\ {\isacharasterisk}y\ {\isacharequal}\isactrlsub D\ Ret\ b\ {\isasymand}\isactrlsub D\ {\isacharasterisk}r\ {\isacharequal}\isactrlsub D\ Ret\ {\isacharparenleft}{\isadigit{0}}{\isacharcolon}{\isacharcolon}nat{\isacharparenright}\ {\isasymand}\isactrlsub D\ Ret\ {\isacharparenleft}x\ {\isasymnoteq}\ y\ {\isasymand}\ y\ {\isasymnoteq}\ r\ {\isasymand}\ x\ {\isasymnoteq}\ r{\isacharparenright}\ \isanewline
\ \ \ \ \ \ \ \ \ \ \ \ \ \ \ \ \ \ \ \ \ \ \ \ \ {\isasymlongrightarrow}\isactrlsub D\ Ret\ {\isacharparenleft}x\ {\isasymnoteq}\ y\ {\isasymand}\ y\ {\isasymnoteq}\ r\ {\isasymand}\ x\ {\isasymnoteq}\ r{\isacharparenright}\ {\isasymand}\isactrlsub D\ \isanewline
\ \ \ \ \ \ \ \ \ \ \ \ \ \ \ \ \ \ \ \ \ \ \ \ \ \ \ \ \ \ \ {\isasymUp}\ {\isacharparenleft}do\ {\isacharbraceleft}u{\isasymleftarrow}readRef\ x{\isacharsemicolon}\ v{\isasymleftarrow}readRef\ y{\isacharsemicolon}\ w{\isasymleftarrow}readRef\ r{\isacharsemicolon}\ ret\ {\isacharparenleft}u{\isacharasterisk}v{\isacharplus}w\ {\isacharequal}\ a{\isacharasterisk}b{\isacharparenright}{\isacharbraceright}{\isacharparenright}{\isachardoublequote}\isanewline
\ \ {\isacharparenleft}\isakeyword{is}\ {\isachardoublequote}{\isasymturnstile}\ {\isacharquery}x\ {\isasymand}\isactrlsub D\ {\isacharquery}y\ {\isasymand}\isactrlsub D\ {\isacharquery}r\ {\isasymand}\isactrlsub D\ {\isacharquery}diff\ {\isasymlongrightarrow}\isactrlsub D\ {\isacharquery}diff\ {\isasymand}\isactrlsub D\ {\isacharquery}seq{\isachardoublequote}{\isacharparenright}\isanewline
\isamarkupfalse%
\isacommand{proof}\ {\isacharminus}\isanewline
\ \ %
\isamarkupcmt{Simplify the goal by proving something tautologously equivalent.%
}
\isanewline
\ \ \isamarkupfalse%
\isacommand{have}\ {\isachardoublequote}{\isasymturnstile}\ {\isacharparenleft}{\isacharquery}x\ {\isasymand}\isactrlsub D\ {\isacharquery}y\ {\isasymand}\isactrlsub D\ {\isacharquery}r\ {\isasymlongrightarrow}\isactrlsub D\ {\isacharquery}seq{\isacharparenright}\ {\isasymlongrightarrow}\isactrlsub D\isanewline
\ \ \ \ \ \ \ \ \ \ {\isacharparenleft}{\isacharquery}x\ {\isasymand}\isactrlsub D\ {\isacharquery}y\ {\isasymand}\isactrlsub D\ {\isacharquery}r\ {\isasymand}\isactrlsub D\ {\isacharquery}diff\ {\isasymlongrightarrow}\isactrlsub D\ {\isacharquery}diff\ {\isasymand}\isactrlsub D\ {\isacharquery}seq{\isacharparenright}{\isachardoublequote}\ \isamarkupfalse%
\isacommand{by}\ {\isacharparenleft}simp\ add{\isacharcolon}\ pdl{\isacharunderscore}taut{\isacharparenright}\isanewline
\ \ \isamarkupfalse%
\isacommand{moreover}\isanewline
\ \ \isamarkupfalse%
\isacommand{have}\ {\isachardoublequote}{\isasymturnstile}\ {\isacharquery}x\ {\isasymand}\isactrlsub D\ {\isacharquery}y\ {\isasymand}\isactrlsub D\ {\isacharquery}r\ {\isasymlongrightarrow}\isactrlsub D\ {\isacharquery}seq{\isachardoublequote}\isanewline
\ \ \ \ %
\isamarkupcmt{Turn the formula into a straight program sequence%
}
\isanewline
\ \ \ \ \isamarkupfalse%
\isacommand{apply}{\isacharparenleft}simp\ add{\isacharcolon}\ liftM{\isadigit{2}}{\isacharunderscore}def\ impD{\isacharunderscore}def\ conjD{\isacharunderscore}def\ MonEq{\isacharunderscore}def\ dsef{\isacharunderscore}read\ Abs{\isacharunderscore}Dsef{\isacharunderscore}inverse\ Dsef{\isacharunderscore}def\ Ret{\isacharunderscore}ret{\isacharparenright}\isanewline
\ \ \ \ \isamarkupfalse%
\isacommand{apply}{\isacharparenleft}simp\ add{\isacharcolon}\ dsef{\isacharunderscore}read\ Abs{\isacharunderscore}Dsef{\isacharunderscore}inverse\ Dsef{\isacharunderscore}def\ dsef{\isacharunderscore}seq{\isacharparenright}\isanewline
\ \ \ \ \isamarkupfalse%
\isacommand{apply}{\isacharparenleft}simp\ add{\isacharcolon}\ mon{\isacharunderscore}ctr\ del{\isacharcolon}\ bind{\isacharunderscore}assoc{\isacharparenright}\isanewline
\ \ \ \ %
\isamarkupcmt{Sort programs so that equal ones are next to each other%
}
\isanewline
\ \ \ \ \isamarkupfalse%
\isacommand{apply}{\isacharparenleft}simp\ del{\isacharcolon}\ dsef{\isacharunderscore}ret\ add{\isacharcolon}\ commute{\isacharunderscore}dsef{\isacharbrackleft}of\ {\isachardoublequote}readRef\ r{\isachardoublequote}\ {\isachardoublequote}readRef\ x{\isachardoublequote}{\isacharbrackright}\ dsef{\isacharunderscore}read{\isacharparenright}\isanewline
\ \ \ \ \isamarkupfalse%
\isacommand{apply}{\isacharparenleft}simp\ del{\isacharcolon}\ dsef{\isacharunderscore}ret\ add{\isacharcolon}\ commute{\isacharunderscore}dsef{\isacharbrackleft}of\ {\isachardoublequote}readRef\ y{\isachardoublequote}\ {\isachardoublequote}readRef\ x{\isachardoublequote}{\isacharbrackright}\ dsef{\isacharunderscore}read{\isacharparenright}\isanewline
\ \ \ \ \isamarkupfalse%
\isacommand{apply}{\isacharparenleft}simp\ del{\isacharcolon}\ dsef{\isacharunderscore}ret\ add{\isacharcolon}\ commute{\isacharunderscore}dsef{\isacharbrackleft}of\ {\isachardoublequote}readRef\ r{\isachardoublequote}\ {\isachardoublequote}readRef\ y{\isachardoublequote}{\isacharbrackright}\ dsef{\isacharunderscore}read{\isacharparenright}\isanewline
\ \ \ \ %
\isamarkupcmt{Remove duplicate occurrences of all programs%
}
\isanewline
\ \ \ \ \isamarkupfalse%
\isacommand{apply}{\isacharparenleft}simp\ add{\isacharcolon}\ dsef{\isacharunderscore}cp{\isacharbrackleft}OF\ dsef{\isacharunderscore}read{\isacharbrackleft}of\ {\isachardoublequote}x{\isachardoublequote}{\isacharbrackright}{\isacharbrackright}\ cp{\isacharunderscore}arb{\isacharparenright}\isanewline
\ \ \ \ \isamarkupfalse%
\isacommand{apply}{\isacharparenleft}simp\ add{\isacharcolon}\ dsef{\isacharunderscore}cp{\isacharbrackleft}OF\ dsef{\isacharunderscore}read{\isacharbrackleft}of\ {\isachardoublequote}y{\isachardoublequote}{\isacharbrackright}{\isacharbrackright}\ cp{\isacharunderscore}arb{\isacharparenright}\isanewline
\ \ \ \ \isamarkupfalse%
\isacommand{apply}{\isacharparenleft}simp\ add{\isacharcolon}\ dsef{\isacharunderscore}cp{\isacharbrackleft}OF\ dsef{\isacharunderscore}read{\isacharbrackleft}of\ {\isachardoublequote}r{\isachardoublequote}{\isacharbrackright}{\isacharbrackright}\ cp{\isacharunderscore}arb{\isacharparenright}\isanewline
\ \ \ \ %
\isamarkupcmt{Finally prove the returned stateless formula and conclude by reducing 
          the program to \isa{ret\ True}%
}
\isanewline
\ \ \ \ \isamarkupfalse%
\isacommand{apply}{\isacharparenleft}simp\ add{\isacharcolon}\ dsef{\isacharunderscore}dis{\isacharbrackleft}OF\ dsef{\isacharunderscore}read{\isacharbrackright}\ dis{\isacharunderscore}left{\isadigit{2}}{\isacharparenright}\isanewline
\ \ \ \ \isamarkupfalse%
\isacommand{apply}{\isacharparenleft}simp\ add{\isacharcolon}\ Valid{\isacharunderscore}simp\ Abs{\isacharunderscore}Dsef{\isacharunderscore}inverse\ Dsef{\isacharunderscore}def{\isacharparenright}\isanewline
\ \ \ \ \isamarkupfalse%
\isacommand{done}\isanewline
\ \ \isamarkupfalse%
\isacommand{ultimately}\ \isamarkupfalse%
\isacommand{show}\ {\isacharquery}thesis\ \isamarkupfalse%
\isacommand{by}\ {\isacharparenleft}rule\ pdl{\isacharunderscore}mp{\isacharparenright}\isanewline
\isamarkupfalse%
\isacommand{qed}\isanewline
\isanewline
\isanewline
\isamarkupfalse%
\isacommand{lemma}\ doterm{\isacharunderscore}eq{\isadigit{1}}{\isacharunderscore}aux{\isacharcolon}\ {\isachardoublequote}do\ {\isacharbraceleft}u{\isasymleftarrow}readRef\ x{\isacharsemicolon}\ v{\isasymleftarrow}readRef\ y{\isacharsemicolon}\ w{\isasymleftarrow}readRef\ r{\isacharsemicolon}\ ret\ {\isacharparenleft}u\ {\isacharasterisk}\ v\ {\isacharplus}\ w\ {\isacharequal}\ a\ {\isacharasterisk}\ b{\isacharparenright}{\isacharbraceright}\ {\isacharequal}\isanewline
\ \ \ \ \ \ \ \ \ \ \ \ \ \ \ \ \ \ \ \ \ \ \ do\ {\isacharbraceleft}u{\isasymleftarrow}readRef\ x{\isacharsemicolon}\ {\isasymDown}\ {\isacharparenleft}{\isasymUp}\ {\isacharparenleft}do\ {\isacharbraceleft}v{\isasymleftarrow}readRef\ y{\isacharsemicolon}\ w{\isasymleftarrow}readRef\ r{\isacharsemicolon}\ ret\ {\isacharparenleft}u\ {\isacharasterisk}\ v\ {\isacharplus}\ w\ {\isacharequal}\ a\ {\isacharasterisk}\ b{\isacharparenright}{\isacharbraceright}{\isacharparenright}{\isacharparenright}{\isacharbraceright}{\isachardoublequote}\isamarkupfalse%
\isamarkupfalse%
\isamarkupfalse%
\isamarkupfalse%
\isamarkupfalse%
\isamarkupfalse%
\isamarkupfalse%
\isamarkupfalse%
\isamarkupfalse%
\isamarkupfalse%
\isamarkupfalse%
\isanewline
\isanewline
\isamarkupfalse%
\isacommand{lemma}\ doterm{\isacharunderscore}eq{\isadigit{2}}{\isacharunderscore}aux{\isacharcolon}\ {\isachardoublequote}do\ {\isacharbraceleft}v{\isasymleftarrow}readRef\ y{\isacharsemicolon}\ w{\isasymleftarrow}readRef\ r{\isacharsemicolon}\ ret\ {\isacharparenleft}u\ {\isacharasterisk}\ v\ {\isacharplus}\ w\ {\isacharequal}\ a\ {\isacharasterisk}\ b{\isacharparenright}{\isacharbraceright}\ {\isacharequal}\isanewline
\ \ \ \ \ \ \ \ \ \ \ \ \ \ \ \ \ \ \ \ \ \ \ do\ {\isacharbraceleft}v{\isasymleftarrow}readRef\ y{\isacharsemicolon}\ {\isasymDown}\ {\isacharparenleft}{\isasymUp}\ {\isacharparenleft}do\ {\isacharbraceleft}w{\isasymleftarrow}readRef\ r{\isacharsemicolon}\ ret\ {\isacharparenleft}u\ {\isacharasterisk}\ v\ {\isacharplus}\ w\ {\isacharequal}\ a\ {\isacharasterisk}\ b{\isacharparenright}{\isacharbraceright}{\isacharparenright}{\isacharparenright}{\isacharbraceright}{\isachardoublequote}\isamarkupfalse%
\isamarkupfalse%
\isamarkupfalse%
\isamarkupfalse%
\isamarkupfalse%
\isamarkupfalse%
\isamarkupfalse%
\isamarkupfalse%
\isamarkupfalse%
\isamarkupfalse%
\isamarkupfalse%
\isanewline
\isanewline
\isamarkupfalse%
\isacommand{lemma}\ arith{\isacharunderscore}aux{\isacharcolon}\ {\isachardoublequote}{\isasymlbrakk}nat{\isacharunderscore}odd\ u{\isacharsemicolon}\ u\ {\isacharasterisk}\ v\ {\isacharplus}\ w\ {\isacharequal}\ a\ {\isacharasterisk}\ b{\isasymrbrakk}\ {\isasymLongrightarrow}\ {\isacharparenleft}u\ div\ {\isadigit{2}}\ {\isacharplus}\ u\ div\ {\isadigit{2}}{\isacharparenright}\ {\isacharasterisk}\ v\ {\isacharplus}\ {\isacharparenleft}w\ {\isacharplus}\ v{\isacharparenright}\ {\isacharequal}\ a\ {\isacharasterisk}\ b{\isachardoublequote}\isamarkupfalse%
\isamarkupfalse%
\isamarkupfalse%
\isamarkupfalse%
\isamarkupfalse%
\isamarkupfalse%
\isamarkupfalse%
\isamarkupfalse%
\isamarkupfalse%
\isamarkupfalse%
\isamarkupfalse%
\isamarkupfalse%
\isamarkupfalse%
\isamarkupfalse%
\isamarkupfalse%
\isamarkupfalse%
\isamarkupfalse%
\isamarkupfalse%
\isamarkupfalse%
\isamarkupfalse%
\isamarkupfalse%
\isanewline
\isanewline
\isamarkupfalse%
\isacommand{lemma}\ rel{\isadigit{1}}{\isacharunderscore}aux{\isacharcolon}\ {\isachardoublequote}nat{\isacharunderscore}odd\ u\ {\isasymLongrightarrow}\ {\isasymturnstile}\ \ {\isacharparenleft}\ Ret\ {\isacharparenleft}x\ {\isasymnoteq}\ y\ {\isasymand}\ y\ {\isasymnoteq}\ r\ {\isasymand}\ x\ {\isasymnoteq}\ r{\isacharparenright}\ {\isasymand}\isactrlsub D\ {\isacharasterisk}r\ {\isacharequal}\isactrlsub D\ Ret\ {\isacharparenleft}w\ {\isacharplus}\ v{\isacharparenright}\ {\isasymand}\isactrlsub D\ Ret\ {\isacharparenleft}u\ {\isacharasterisk}\ v\ {\isacharplus}\ w\ {\isacharequal}\ a\ {\isacharasterisk}\ b{\isacharparenright}\ {\isacharparenright}\ {\isasymlongrightarrow}\isactrlsub D\isanewline
\ \ \ \ \ \ \ \ \ \ \ \ \ \ \ \ \ \ \ \ Ret\ {\isacharparenleft}x{\isasymnoteq}y\ {\isasymand}\ y{\isasymnoteq}r\ {\isasymand}\ x{\isasymnoteq}r{\isacharparenright}\ {\isasymand}\isactrlsub D\ {\isasymUp}\ {\isacharparenleft}do\ {\isacharbraceleft}w{\isasymleftarrow}readRef\ r{\isacharsemicolon}\ ret\ {\isacharparenleft}{\isacharparenleft}u\ div\ {\isadigit{2}}\ {\isacharplus}\ u\ div\ {\isadigit{2}}{\isacharparenright}\ {\isacharasterisk}\ v\ {\isacharplus}\ w\ {\isacharequal}\ a\ {\isacharasterisk}\ b{\isacharparenright}{\isacharbraceright}{\isacharparenright}{\isachardoublequote}\isanewline
\ \ {\isacharparenleft}\isakeyword{is}\ {\isachardoublequote}{\isacharquery}odd\ {\isasymLongrightarrow}\ {\isasymturnstile}\ {\isacharparenleft}{\isacharquery}diff\ {\isasymand}\isactrlsub D\ {\isacharquery}r\ {\isasymand}\isactrlsub D\ {\isacharquery}ar{\isacharparenright}\ {\isasymlongrightarrow}\isactrlsub D\ {\isacharquery}diff\ {\isasymand}\isactrlsub D\ {\isacharquery}seq{\isachardoublequote}{\isacharparenright}\isamarkupfalse%
\isamarkupfalse%
\isamarkupfalse%
\isamarkupfalse%
\isamarkupfalse%
\isamarkupfalse%
\isamarkupfalse%
\isamarkupfalse%
\isamarkupfalse%
\isamarkupfalse%
\isamarkupfalse%
\isamarkupfalse%
\isamarkupfalse%
\isamarkupfalse%
\isamarkupfalse%
\isamarkupfalse%
\isamarkupfalse%
\isamarkupfalse%
\isamarkupfalse%
\isamarkupfalse%
\isamarkupfalse%
\isamarkupfalse%
\isanewline
\isanewline
\isamarkupfalse%
\isacommand{lemma}\ wrt{\isacharunderscore}other{\isacharunderscore}aux{\isacharcolon}\ {\isachardoublequote}{\isasymturnstile}\ Ret\ {\isacharparenleft}\ x{\isasymnoteq}y\ {\isasymand}\ y{\isasymnoteq}r\ {\isasymand}\ x{\isasymnoteq}r\ {\isacharparenright}\ {\isasymand}\isactrlsub D\ {\isasymUp}\ {\isacharparenleft}do\ {\isacharbraceleft}w{\isasymleftarrow}readRef\ r{\isacharsemicolon}\ ret\ {\isacharparenleft}f\ w{\isacharparenright}{\isacharbraceright}{\isacharparenright}\ {\isasymlongrightarrow}\isactrlsub D\ \isanewline
\ \ \ \ \ \ \ \ \ \ \ \ \ \ \ \ \ \ \ \ \ \ \ \ {\isacharbrackleft}{\isacharhash}\ x\ {\isacharcolon}{\isacharequal}\ a{\isacharbrackright}{\isacharparenleft}{\isasymlambda}uu{\isachardot}\ Ret\ {\isacharparenleft}x{\isasymnoteq}y\ {\isasymand}\ y{\isasymnoteq}r\ {\isasymand}\ x{\isasymnoteq}r{\isacharparenright}\ {\isasymand}\isactrlsub D\ {\isasymUp}\ {\isacharparenleft}do\ {\isacharbraceleft}w{\isasymleftarrow}readRef\ r{\isacharsemicolon}\ ret\ {\isacharparenleft}f\ w{\isacharparenright}{\isacharbraceright}{\isacharparenright}{\isacharparenright}{\isachardoublequote}\isamarkupfalse%
\isamarkupfalse%
\isamarkupfalse%
\isamarkupfalse%
\isamarkupfalse%
\isamarkupfalse%
\isamarkupfalse%
\isanewline
\isanewline
\isamarkupfalse%
\isacommand{lemma}\ wrt{\isacharunderscore}other{\isadigit{2}}{\isacharunderscore}aux{\isacharcolon}\ \ {\isachardoublequote}{\isasymturnstile}\ Ret\ {\isacharparenleft}\ x{\isasymnoteq}y\ {\isasymand}\ y{\isasymnoteq}r\ {\isasymand}\ x{\isasymnoteq}r\ {\isacharparenright}\ {\isasymand}\isactrlsub D\ {\isasymUp}\ {\isacharparenleft}do\ {\isacharbraceleft}w{\isasymleftarrow}readRef\ r{\isacharsemicolon}\ ret\ {\isacharparenleft}f\ w{\isacharparenright}{\isacharbraceright}{\isacharparenright}\ {\isasymlongrightarrow}\isactrlsub D\ \isanewline
\ \ \ \ \ \ \ \ \ \ \ \ \ \ \ \ \ \ \ \ \ \ \ \ {\isacharbrackleft}{\isacharhash}\ y\ {\isacharcolon}{\isacharequal}\ b{\isacharbrackright}{\isacharparenleft}{\isasymlambda}uu{\isachardot}\ Ret\ {\isacharparenleft}x{\isasymnoteq}y\ {\isasymand}\ y{\isasymnoteq}r\ {\isasymand}\ x{\isasymnoteq}r{\isacharparenright}\ {\isasymand}\isactrlsub D\ {\isasymUp}\ {\isacharparenleft}do\ {\isacharbraceleft}w{\isasymleftarrow}readRef\ r{\isacharsemicolon}\ ret\ {\isacharparenleft}f\ w{\isacharparenright}{\isacharbraceright}{\isacharparenright}{\isacharparenright}{\isachardoublequote}\isamarkupfalse%
\isamarkupfalse%
\isamarkupfalse%
\isamarkupfalse%
\isamarkupfalse%
\isamarkupfalse%
\isamarkupfalse%
\isanewline
\isanewline
\isamarkupfalse%
\isacommand{lemma}\ rd{\isacharunderscore}seq{\isacharunderscore}aux{\isacharcolon}\ {\isachardoublequote}{\isasymturnstile}\ {\isasymUp}\ {\isacharparenleft}do\ {\isacharbraceleft}w{\isasymleftarrow}readRef\ r{\isacharsemicolon}\ ret\ {\isacharparenleft}f\ a\ w{\isacharparenright}{\isacharbraceright}{\isacharparenright}\ {\isasymand}\isactrlsub D\ {\isacharasterisk}x\ {\isacharequal}\isactrlsub D\ Ret\ a\ {\isasymlongrightarrow}\isactrlsub D\isanewline
\ \ \ \ \ \ \ \ \ \ \ \ \ \ \ \ \ \ \ \ \ {\isasymUp}\ {\isacharparenleft}do\ {\isacharbraceleft}u{\isasymleftarrow}readRef\ x{\isacharsemicolon}\ w{\isasymleftarrow}readRef\ r{\isacharsemicolon}\ ret\ {\isacharparenleft}f\ u\ w{\isacharparenright}{\isacharbraceright}{\isacharparenright}{\isachardoublequote}\isamarkupfalse%
\isamarkupfalse%
\isamarkupfalse%
\isamarkupfalse%
\isamarkupfalse%
\isamarkupfalse%
\isamarkupfalse%
\isamarkupfalse%
\isamarkupfalse%
\isanewline
\isanewline
\isamarkupfalse%
\isacommand{lemma}\ arith{\isadigit{2}}{\isacharunderscore}aux{\isacharcolon}\ {\isachardoublequote}{\isacharparenleft}u\ div\ {\isacharparenleft}{\isadigit{2}}{\isacharcolon}{\isacharcolon}nat{\isacharparenright}\ {\isacharplus}\ u\ div\ {\isadigit{2}}{\isacharparenright}\ {\isacharasterisk}\ v\ {\isacharplus}\ w\ {\isacharequal}\ a\ {\isacharasterisk}\ b\ {\isasymlongrightarrow}\ u\ div\ {\isadigit{2}}\ {\isacharasterisk}\ {\isacharparenleft}v\ {\isacharasterisk}\ {\isadigit{2}}{\isacharparenright}\ {\isacharplus}\ w\ {\isacharequal}\ a\ {\isacharasterisk}\ b{\isachardoublequote}\isamarkupfalse%
\isamarkupfalse%
\isamarkupfalse%
\isamarkupfalse%
\isamarkupfalse%
\isamarkupfalse%
\isamarkupfalse%
\isamarkupfalse%
\isamarkupfalse%
\isamarkupfalse%
\isamarkupfalse%
\isamarkupfalse%
\isamarkupfalse%
\isamarkupfalse%
\isamarkupfalse%
\isamarkupfalse%
\isamarkupfalse%
\isanewline
\isanewline
\isamarkupfalse%
\isacommand{lemma}\ asm{\isacharunderscore}results{\isacharunderscore}aux{\isacharcolon}\ {\isachardoublequote}\ {\isasymturnstile}\ {\isacharparenleft}Ret\ {\isacharparenleft}x\ {\isasymnoteq}\ y{\isacharparenright}\ {\isasymlongrightarrow}\isactrlsub D\ {\isacharasterisk}x\ {\isacharequal}\isactrlsub D\ Ret\ {\isacharparenleft}u\ div\ {\isacharparenleft}{\isadigit{2}}{\isacharcolon}{\isacharcolon}nat{\isacharparenright}{\isacharparenright}{\isacharparenright}\ {\isasymand}\isactrlsub D\isanewline
\ \ \ \ \ \ \ \ \ {\isacharasterisk}y\ {\isacharequal}\isactrlsub D\ Ret\ {\isacharparenleft}v\ {\isacharasterisk}\ {\isadigit{2}}{\isacharparenright}\ {\isasymand}\isactrlsub D\isanewline
\ \ \ \ \ \ \ \ \ Ret\ {\isacharparenleft}x\ {\isasymnoteq}\ y\ {\isasymand}\ y\ {\isasymnoteq}\ r\ {\isasymand}\ x\ {\isasymnoteq}\ r{\isacharparenright}\ {\isasymand}\isactrlsub D\ {\isasymUp}\ {\isacharparenleft}do\ {\isacharbraceleft}w{\isasymleftarrow}readRef\ r{\isacharsemicolon}\ ret\ {\isacharparenleft}{\isacharparenleft}u\ div\ {\isadigit{2}}\ {\isacharplus}\ u\ div\ {\isadigit{2}}{\isacharparenright}\ {\isacharasterisk}\ v\ {\isacharplus}\ w\ {\isacharequal}\ a\ {\isacharasterisk}\ b{\isacharparenright}{\isacharbraceright}{\isacharparenright}\ {\isasymlongrightarrow}\isactrlsub D\isanewline
\ \ \ \ \ \ \ \ \ Ret\ {\isacharparenleft}x\ {\isasymnoteq}\ y\ {\isasymand}\ y\ {\isasymnoteq}\ r\ {\isasymand}\ x\ {\isasymnoteq}\ r{\isacharparenright}\ {\isasymand}\isactrlsub D\ {\isasymUp}\ {\isacharparenleft}do\ {\isacharbraceleft}u{\isasymleftarrow}readRef\ x{\isacharsemicolon}\ v{\isasymleftarrow}readRef\ y{\isacharsemicolon}\ w{\isasymleftarrow}readRef\ r{\isacharsemicolon}\ ret\ {\isacharparenleft}u\ {\isacharasterisk}\ v\ {\isacharplus}\ w\ {\isacharequal}\ a\ {\isacharasterisk}\ b{\isacharparenright}{\isacharbraceright}{\isacharparenright}{\isachardoublequote}\isamarkupfalse%
\isamarkupfalse%
\isamarkupfalse%
\isamarkupfalse%
\isamarkupfalse%
\isamarkupfalse%
\isamarkupfalse%
\isamarkupfalse%
\isamarkupfalse%
\isamarkupfalse%
\isamarkupfalse%
\isamarkupfalse%
\isamarkupfalse%
\isamarkupfalse%
%
\begin{isamarkuptext}%
Yet another dsef formula extension.%
\end{isamarkuptext}%
\isamarkuptrue%
\isacommand{lemma}\ yadfe{\isacharcolon}\ {\isachardoublequote}\ {\isasymlbrakk}dsef\ p{\isacharsemicolon}\ dsef\ q{\isacharsemicolon}\ dsef\ r{\isacharsemicolon}\ {\isasymforall}x\ y\ z{\isachardot}\ f\ x\ y\ z{\isasymrbrakk}\ {\isasymLongrightarrow}\ {\isasymturnstile}\ {\isasymUp}\ {\isacharparenleft}do\ {\isacharbraceleft}x{\isasymleftarrow}p{\isacharsemicolon}\ y{\isasymleftarrow}q{\isacharsemicolon}\ z{\isasymleftarrow}r{\isacharsemicolon}\ ret\ {\isacharparenleft}f\ x\ y\ z{\isacharparenright}{\isacharbraceright}{\isacharparenright}{\isachardoublequote}\isanewline
\isamarkupfalse%
\isacommand{proof}\ {\isacharminus}\isanewline
\ \ \isamarkupfalse%
\isacommand{assume}\ ds{\isacharcolon}\ {\isachardoublequote}dsef\ p{\isachardoublequote}\ {\isachardoublequote}dsef\ q{\isachardoublequote}\ {\isachardoublequote}dsef\ r{\isachardoublequote}\isanewline
\ \ \isamarkupfalse%
\isacommand{assume}\ a{\isadigit{1}}{\isacharcolon}\ {\isachardoublequote}{\isasymforall}x\ y\ z{\isachardot}\ f\ x\ y\ z{\isachardoublequote}\isanewline
\ \ \isamarkupfalse%
\isacommand{hence}\ {\isachardoublequote}{\isasymDown}\ {\isacharparenleft}{\isasymUp}\ {\isacharparenleft}do\ {\isacharbraceleft}x{\isasymleftarrow}p{\isacharsemicolon}\ y{\isasymleftarrow}q{\isacharsemicolon}\ z{\isasymleftarrow}r{\isacharsemicolon}\ ret\ {\isacharparenleft}f\ x\ y\ z{\isacharparenright}{\isacharbraceright}{\isacharparenright}{\isacharparenright}\ {\isacharequal}\ \isanewline
\ \ \ \ \ \ \ \ \ {\isasymDown}\ {\isacharparenleft}{\isasymUp}\ {\isacharparenleft}do\ {\isacharbraceleft}x{\isasymleftarrow}p{\isacharsemicolon}\ y{\isasymleftarrow}q{\isacharsemicolon}\ z{\isasymleftarrow}r{\isacharsemicolon}\ ret\ True{\isacharbraceright}{\isacharparenright}{\isacharparenright}{\isachardoublequote}\isanewline
\ \ \ \ \isamarkupfalse%
\isacommand{by}\ {\isacharparenleft}simp{\isacharparenright}\isanewline
\ \ \isamarkupfalse%
\isacommand{also}\ \isamarkupfalse%
\isacommand{from}\ ds\ \isamarkupfalse%
\isacommand{have}\ {\isachardoublequote}{\isasymdots}\ {\isacharequal}\ ret\ True{\isachardoublequote}\ \isanewline
\ \ \ \ \isamarkupfalse%
\isacommand{by}\ {\isacharparenleft}simp\ add{\isacharcolon}\ Abs{\isacharunderscore}Dsef{\isacharunderscore}inverse\ Dsef{\isacharunderscore}def\ dsef{\isacharunderscore}seq\ dis{\isacharunderscore}left{\isadigit{2}}\ dsef{\isacharunderscore}dis{\isacharparenright}\isanewline
\ \ \isamarkupfalse%
\isacommand{finally}\ \isamarkupfalse%
\isacommand{show}\ {\isacharquery}thesis\ \isamarkupfalse%
\isacommand{by}\ {\isacharparenleft}simp\ add{\isacharcolon}\ Valid{\isacharunderscore}simp{\isacharparenright}\isanewline
\isamarkupfalse%
\isacommand{qed}\isanewline
\isanewline
\isanewline
\isamarkupfalse%
\isacommand{lemma}\ conclude{\isacharunderscore}aux{\isacharcolon}\ {\isachardoublequote}\ {\isasymturnstile}\ {\isacharparenleft}Ret\ {\isacharparenleft}x\ {\isasymnoteq}\ y\ {\isasymand}\ y\ {\isasymnoteq}\ r\ {\isasymand}\ x\ {\isasymnoteq}\ r{\isacharparenright}\ {\isasymand}\isactrlsub D\ \isanewline
\ \ \ \ \ \ \ \ \ {\isasymUp}\ {\isacharparenleft}do\ {\isacharbraceleft}u{\isasymleftarrow}readRef\ x{\isacharsemicolon}\ v{\isasymleftarrow}readRef\ y{\isacharsemicolon}\ w{\isasymleftarrow}readRef\ r{\isacharsemicolon}\ ret\ {\isacharparenleft}u\ {\isacharasterisk}\ v\ {\isacharplus}\ w\ {\isacharequal}\ {\isacharparenleft}a{\isacharcolon}{\isacharcolon}nat{\isacharparenright}\ {\isacharasterisk}\ b{\isacharparenright}{\isacharbraceright}{\isacharparenright}{\isacharparenright}\ {\isasymand}\isactrlsub D\isanewline
\ \ \ \ \ \ \ \ \ {\isasymnot}\isactrlsub D\ {\isasymUp}\ {\isacharparenleft}do\ {\isacharbraceleft}u{\isasymleftarrow}readRef\ x{\isacharsemicolon}\ ret\ {\isacharparenleft}{\isadigit{0}}\ {\isacharless}\ u{\isacharparenright}{\isacharbraceright}{\isacharparenright}\ {\isasymlongrightarrow}\isactrlsub D\isanewline
\ \ \ \ \ \ \ \ \ {\isacharbrackleft}{\isacharhash}\ readRef\ r{\isacharbrackright}{\isacharparenleft}{\isasymlambda}x{\isachardot}\ Ret\ {\isacharparenleft}x\ {\isacharequal}\ a\ {\isacharasterisk}\ b{\isacharparenright}{\isacharparenright}{\isachardoublequote}\isamarkupfalse%
\isamarkupfalse%
\isamarkupfalse%
\isamarkupfalse%
\isamarkupfalse%
\isamarkupfalse%
\isamarkupfalse%
\isamarkupfalse%
\isamarkupfalse%
\isamarkupfalse%
\isamarkupfalse%
\isamarkupfalse%
\isamarkupfalse%
\isamarkupfalse%
\isamarkupfalse%
\isamarkupfalse%
%
\isamarkupsubsection{Correctness of Russian Multiplication%
}
\isamarkuptrue%
%
\begin{isamarkuptext}%
Equipped with all these prerequisites, the correctness proof of Russian multiplication
  is `at your fingertips'\texttrademark. We will not display the actual rule applications but
  only the important proof goals arising in between.
  \label{isa:rumult-proof}%
\end{isamarkuptext}%
\isamarkuptrue%
\isacommand{theorem}\ russian{\isacharunderscore}mult{\isacharcolon}\ {\isachardoublequote}{\isasymturnstile}\ {\isacharparenleft}Ret\ {\isacharparenleft}\ x{\isasymnoteq}y\ {\isasymand}\ y{\isasymnoteq}r\ {\isasymand}\ x{\isasymnoteq}r{\isacharparenright}{\isacharparenright}\ {\isasymlongrightarrow}\isactrlsub D\ {\isacharbrackleft}{\isacharhash}\ rumult\ a\ b\ x\ y\ r{\isacharbrackright}{\isacharparenleft}{\isasymlambda}x{\isachardot}\ Ret\ {\isacharparenleft}x\ {\isacharequal}\ a\ {\isacharasterisk}\ b{\isacharparenright}{\isacharparenright}{\isachardoublequote}\isanewline
\ \ \isamarkupfalse%
\isacommand{apply}{\isacharparenleft}unfold\ rumult{\isacharunderscore}def{\isacharparenright}\ %
\isamarkupcmt{First, unfold the definition of \isa{rumult}%
}
\isanewline
\ \ \isamarkupfalse%
\isacommand{apply}{\isacharparenleft}simp\ only{\isacharcolon}\ seq{\isacharunderscore}def{\isacharparenright}\isanewline
\ \ \isamarkupfalse%
\isacommand{apply}{\isacharparenleft}rule\ pdl{\isacharunderscore}plugB{\isacharunderscore}lifted{\isadigit{1}}{\isacharparenright}\isamarkupfalse%
%
\begin{isamarkuptxt}%
Establish the `strongest postcondition' of the assignment to \isa{x}

    \begin{isabelle}%
{\isasymturnstile}\ Ret\ {\isacharparenleft}x\ {\isasymnoteq}\ y\ {\isasymand}\ y\ {\isasymnoteq}\ r\ {\isasymand}\ x\ {\isasymnoteq}\ r{\isacharparenright}\ {\isasymlongrightarrow}\isactrlsub D\ {\isacharbrackleft}{\isacharhash}\ rumult\ a\ b\ x\ y\ r{\isacharbrackright}{\isacharparenleft}{\isasymlambda}x{\isachardot}\ Ret\ {\isacharparenleft}x\ {\isacharequal}\ a\ {\isacharasterisk}\ b{\isacharparenright}{\isacharparenright}\isanewline
\ {\isadigit{1}}{\isachardot}\ {\isasymturnstile}\ Ret\ {\isacharparenleft}x\ {\isasymnoteq}\ y\ {\isasymand}\ y\ {\isasymnoteq}\ r\ {\isasymand}\ x\ {\isasymnoteq}\ r{\isacharparenright}\ {\isasymlongrightarrow}\isactrlsub D\ {\isacharbrackleft}{\isacharhash}\ x\ {\isacharcolon}{\isacharequal}\ a{\isacharbrackright}{\isacharquery}B%
\end{isabelle}%
\end{isamarkuptxt}%
\isamarkuptrue%
\isamarkupfalse%
\isamarkupfalse%
\isamarkupfalse%
\isamarkupfalse%
\isamarkupfalse%
\isamarkupfalse%
\isamarkupfalse%
%
\begin{isamarkuptxt}%
From this postcondition proceed with assignment to \isa{y}

      \begin{isabelle}%
{\isasymturnstile}\ Ret\ {\isacharparenleft}x\ {\isasymnoteq}\ y\ {\isasymand}\ y\ {\isasymnoteq}\ r\ {\isasymand}\ x\ {\isasymnoteq}\ r{\isacharparenright}\ {\isasymlongrightarrow}\isactrlsub D\ {\isacharbrackleft}{\isacharhash}\ rumult\ a\ b\ x\ y\ r{\isacharbrackright}{\isacharparenleft}{\isasymlambda}x{\isachardot}\ Ret\ {\isacharparenleft}x\ {\isacharequal}\ a\ {\isacharasterisk}\ b{\isacharparenright}{\isacharparenright}\isanewline
\ {\isadigit{1}}{\isachardot}\ {\isasymAnd}xa{\isachardot}\ {\isasymturnstile}\ Ret\ {\isacharparenleft}x\ {\isasymnoteq}\ y\ {\isasymand}\ y\ {\isasymnoteq}\ r\ {\isasymand}\ x\ {\isasymnoteq}\ r{\isacharparenright}\ {\isasymand}\isactrlsub D\ {\isacharasterisk}x\ {\isacharequal}\isactrlsub D\ Ret\ a\ {\isasymlongrightarrow}\isactrlsub D\ {\isacharbrackleft}{\isacharhash}\ y\ {\isacharcolon}{\isacharequal}\ b{\isacharbrackright}{\isacharquery}B{\isadigit{9}}\ xa%
\end{isabelle}%
\end{isamarkuptxt}%
\isamarkuptrue%
\isamarkupfalse%
\isamarkupfalse%
\isamarkupfalse%
\isamarkupfalse%
\isamarkupfalse%
\isamarkupfalse%
\isamarkupfalse%
\isamarkupfalse%
\isamarkupfalse%
\isamarkupfalse%
\isamarkupfalse%
\isamarkupfalse%
\isamarkupfalse%
\isamarkupfalse%
%
\begin{isamarkuptxt}%
After the final assignment to \isa{r} all variables will have their initial values

    \begin{isabelle}%
{\isasymturnstile}\ Ret\ {\isacharparenleft}x\ {\isasymnoteq}\ y\ {\isasymand}\ y\ {\isasymnoteq}\ r\ {\isasymand}\ x\ {\isasymnoteq}\ r{\isacharparenright}\ {\isasymlongrightarrow}\isactrlsub D\ {\isacharbrackleft}{\isacharhash}\ rumult\ a\ b\ x\ y\ r{\isacharbrackright}{\isacharparenleft}{\isasymlambda}x{\isachardot}\ Ret\ {\isacharparenleft}x\ {\isacharequal}\ a\ {\isacharasterisk}\ b{\isacharparenright}{\isacharparenright}\isanewline
\ {\isadigit{1}}{\isachardot}\ {\isasymAnd}xa\ xaa{\isachardot}\isanewline
\isaindent{\ {\isadigit{1}}{\isachardot}\ \ \ \ }{\isasymturnstile}\ {\isacharasterisk}x\ {\isacharequal}\isactrlsub D\ Ret\ a\ {\isasymand}\isactrlsub D\ {\isacharasterisk}y\ {\isacharequal}\isactrlsub D\ Ret\ b\ {\isasymand}\isactrlsub D\ Ret\ {\isacharparenleft}x\ {\isasymnoteq}\ y\ {\isasymand}\ y\ {\isasymnoteq}\ r\ {\isasymand}\ x\ {\isasymnoteq}\ r{\isacharparenright}\ {\isasymlongrightarrow}\isactrlsub D\isanewline
\isaindent{\ {\isadigit{1}}{\isachardot}\ \ \ \ {\isasymturnstile}\ }{\isacharbrackleft}{\isacharhash}\ r\ {\isacharcolon}{\isacharequal}\ {\isadigit{0}}{\isacharbrackright}{\isacharquery}B{\isadigit{2}}{\isadigit{7}}\ xa\ xaa%
\end{isabelle}%
\end{isamarkuptxt}%
\isamarkuptrue%
\isamarkupfalse%
\isamarkupfalse%
\isamarkupfalse%
\isamarkupfalse%
\isamarkupfalse%
\isamarkupfalse%
\isamarkupfalse%
\isamarkupfalse%
\isamarkupfalse%
\isamarkupfalse%
\isamarkupfalse%
\isamarkupfalse%
\isamarkupfalse%
\isamarkupfalse%
\isamarkupfalse%
\isamarkupfalse%
\isamarkupfalse%
\isamarkupfalse%
\isamarkupfalse%
%
\begin{isamarkuptxt}%
Now we have arrived at the while-loop, with the invariant readily established.

      \begin{isabelle}%
{\isasymturnstile}\ Ret\ {\isacharparenleft}x\ {\isasymnoteq}\ y\ {\isasymand}\ y\ {\isasymnoteq}\ r\ {\isasymand}\ x\ {\isasymnoteq}\ r{\isacharparenright}\ {\isasymlongrightarrow}\isactrlsub D\ {\isacharbrackleft}{\isacharhash}\ rumult\ a\ b\ x\ y\ r{\isacharbrackright}{\isacharparenleft}{\isasymlambda}x{\isachardot}\ Ret\ {\isacharparenleft}x\ {\isacharequal}\ a\ {\isacharasterisk}\ b{\isacharparenright}{\isacharparenright}\isanewline
\ {\isadigit{1}}{\isachardot}\ {\isasymAnd}xa\ xaa\ xb{\isachardot}\isanewline
\isaindent{\ {\isadigit{1}}{\isachardot}\ \ \ \ }{\isasymturnstile}\ Ret\ {\isacharparenleft}x\ {\isasymnoteq}\ y\ {\isasymand}\ y\ {\isasymnoteq}\ r\ {\isasymand}\ x\ {\isasymnoteq}\ r{\isacharparenright}\ {\isasymand}\isactrlsub D\isanewline
\isaindent{\ {\isadigit{1}}{\isachardot}\ \ \ \ {\isasymturnstile}\ }{\isasymUp}\ {\isacharparenleft}do\ {\isacharbraceleft}u{\isasymleftarrow}readRef\ x{\isacharsemicolon}\isanewline
\isaindent{\ {\isadigit{1}}{\isachardot}\ \ \ \ {\isasymturnstile}\ {\isasymUp}\ {\isacharparenleft}do\ {\isacharbraceleft}}v{\isasymleftarrow}readRef\ y{\isacharsemicolon}\ w{\isasymleftarrow}readRef\ r{\isacharsemicolon}\ ret\ {\isacharparenleft}u\ {\isacharasterisk}\ v\ {\isacharplus}\ w\ {\isacharequal}\ a\ {\isacharasterisk}\ b{\isacharparenright}{\isacharbraceright}{\isacharparenright}\ {\isasymlongrightarrow}\isactrlsub D\isanewline
\isaindent{\ {\isadigit{1}}{\isachardot}\ \ \ \ {\isasymturnstile}\ }{\isacharbrackleft}{\isacharhash}\ do\ {\isacharbraceleft}x{\isasymleftarrow}WHILE\ {\isasymUp}\ {\isacharparenleft}do\ {\isacharbraceleft}u{\isasymleftarrow}readRef\ x{\isacharsemicolon}\ ret\ {\isacharparenleft}{\isadigit{0}}\ {\isacharless}\ u{\isacharparenright}{\isacharbraceright}{\isacharparenright}\ \isanewline
\isaindent{\ {\isadigit{1}}{\isachardot}\ \ \ \ {\isasymturnstile}\ {\isacharbrackleft}{\isacharhash}\ do\ {\isacharbraceleft}x{\isasymleftarrow}}DO\ do\ {\isacharbraceleft}u{\isasymleftarrow}readRef\ x{\isacharsemicolon}\isanewline
\isaindent{\ {\isadigit{1}}{\isachardot}\ \ \ \ {\isasymturnstile}\ {\isacharbrackleft}{\isacharhash}\ do\ {\isacharbraceleft}x{\isasymleftarrow}DO\ do\ {\isacharbraceleft}}v{\isasymleftarrow}readRef\ y{\isacharsemicolon}\isanewline
\isaindent{\ {\isadigit{1}}{\isachardot}\ \ \ \ {\isasymturnstile}\ {\isacharbrackleft}{\isacharhash}\ do\ {\isacharbraceleft}x{\isasymleftarrow}DO\ do\ {\isacharbraceleft}}w{\isasymleftarrow}readRef\ r{\isacharsemicolon}\isanewline
\isaindent{\ {\isadigit{1}}{\isachardot}\ \ \ \ {\isasymturnstile}\ {\isacharbrackleft}{\isacharhash}\ do\ {\isacharbraceleft}x{\isasymleftarrow}DO\ do\ {\isacharbraceleft}}xa{\isasymleftarrow}if\ nat{\isacharunderscore}odd\ u\ then\ r\ {\isacharcolon}{\isacharequal}\ w\ {\isacharplus}\ v\ else\ ret\ {\isacharparenleft}{\isacharparenright}{\isacharsemicolon}\isanewline
\isaindent{\ {\isadigit{1}}{\isachardot}\ \ \ \ {\isasymturnstile}\ {\isacharbrackleft}{\isacharhash}\ do\ {\isacharbraceleft}x{\isasymleftarrow}DO\ do\ {\isacharbraceleft}}x{\isasymleftarrow}x\ {\isacharcolon}{\isacharequal}\ u\ div\ {\isadigit{2}}{\isacharsemicolon}\ y\ {\isacharcolon}{\isacharequal}\ v\ {\isacharasterisk}\ {\isadigit{2}}{\isacharbraceright}\ \isanewline
\isaindent{\ {\isadigit{1}}{\isachardot}\ \ \ \ {\isasymturnstile}\ {\isacharbrackleft}{\isacharhash}\ do\ {\isacharbraceleft}x{\isasymleftarrow}}END{\isacharsemicolon}\isanewline
\isaindent{\ {\isadigit{1}}{\isachardot}\ \ \ \ {\isasymturnstile}\ {\isacharbrackleft}{\isacharhash}\ do\ {\isacharbraceleft}}readRef\ r{\isacharbraceright}{\isacharbrackright}{\isacharparenleft}{\isasymlambda}x{\isachardot}\ Ret\ {\isacharparenleft}x\ {\isacharequal}\ a\ {\isacharasterisk}\ b{\isacharparenright}{\isacharparenright}%
\end{isabelle}%
\end{isamarkuptxt}%
\ \ \isamarkuptrue%
\isacommand{apply}{\isacharparenleft}rule\ pdl{\isacharunderscore}plugB{\isacharunderscore}lifted{\isadigit{1}}{\isacharparenright}\isanewline
\ \ \ \ \isamarkupfalse%
\isacommand{apply}{\isacharparenleft}rule\ while{\isacharunderscore}par{\isacharparenright}\ \ %
\isamarkupcmt{applied the while rule%
}
\isamarkupfalse%
%
\begin{isamarkuptxt}%
After splitting off the while-loop as a single box formula, we can apply the while
      rule, so that we obtain the following proof goal, telling us to establish the invariant after
      one run of the loop body:

      \begin{isabelle}%
{\isasymturnstile}\ Ret\ {\isacharparenleft}x\ {\isasymnoteq}\ y\ {\isasymand}\ y\ {\isasymnoteq}\ r\ {\isasymand}\ x\ {\isasymnoteq}\ r{\isacharparenright}\ {\isasymlongrightarrow}\isactrlsub D\ {\isacharbrackleft}{\isacharhash}\ rumult\ a\ b\ x\ y\ r{\isacharbrackright}{\isacharparenleft}{\isasymlambda}x{\isachardot}\ Ret\ {\isacharparenleft}x\ {\isacharequal}\ a\ {\isacharasterisk}\ b{\isacharparenright}{\isacharparenright}\isanewline
\ {\isadigit{1}}{\isachardot}\ {\isasymAnd}xa\ xaa\ xb{\isachardot}\isanewline
\isaindent{\ {\isadigit{1}}{\isachardot}\ \ \ \ }{\isasymturnstile}\ {\isacharparenleft}Ret\ {\isacharparenleft}x\ {\isasymnoteq}\ y\ {\isasymand}\ y\ {\isasymnoteq}\ r\ {\isasymand}\ x\ {\isasymnoteq}\ r{\isacharparenright}\ {\isasymand}\isactrlsub D\isanewline
\isaindent{\ {\isadigit{1}}{\isachardot}\ \ \ \ {\isasymturnstile}\ {\isacharparenleft}}{\isasymUp}\ {\isacharparenleft}do\ {\isacharbraceleft}u{\isasymleftarrow}readRef\ x{\isacharsemicolon}\isanewline
\isaindent{\ {\isadigit{1}}{\isachardot}\ \ \ \ {\isasymturnstile}\ {\isacharparenleft}{\isasymUp}\ {\isacharparenleft}do\ {\isacharbraceleft}}v{\isasymleftarrow}readRef\ y{\isacharsemicolon}\ w{\isasymleftarrow}readRef\ r{\isacharsemicolon}\ ret\ {\isacharparenleft}u\ {\isacharasterisk}\ v\ {\isacharplus}\ w\ {\isacharequal}\ a\ {\isacharasterisk}\ b{\isacharparenright}{\isacharbraceright}{\isacharparenright}{\isacharparenright}\ {\isasymand}\isactrlsub D\isanewline
\isaindent{\ {\isadigit{1}}{\isachardot}\ \ \ \ {\isasymturnstile}\ }{\isasymUp}\ {\isacharparenleft}do\ {\isacharbraceleft}u{\isasymleftarrow}readRef\ x{\isacharsemicolon}\ ret\ {\isacharparenleft}{\isadigit{0}}\ {\isacharless}\ u{\isacharparenright}{\isacharbraceright}{\isacharparenright}\ {\isasymlongrightarrow}\isactrlsub D\isanewline
\isaindent{\ {\isadigit{1}}{\isachardot}\ \ \ \ {\isasymturnstile}\ }{\isacharbrackleft}{\isacharhash}\ do\ {\isacharbraceleft}u{\isasymleftarrow}readRef\ x{\isacharsemicolon}\isanewline
\isaindent{\ {\isadigit{1}}{\isachardot}\ \ \ \ {\isasymturnstile}\ {\isacharbrackleft}{\isacharhash}\ do\ {\isacharbraceleft}}v{\isasymleftarrow}readRef\ y{\isacharsemicolon}\isanewline
\isaindent{\ {\isadigit{1}}{\isachardot}\ \ \ \ {\isasymturnstile}\ {\isacharbrackleft}{\isacharhash}\ do\ {\isacharbraceleft}}w{\isasymleftarrow}readRef\ r{\isacharsemicolon}\isanewline
\isaindent{\ {\isadigit{1}}{\isachardot}\ \ \ \ {\isasymturnstile}\ {\isacharbrackleft}{\isacharhash}\ do\ {\isacharbraceleft}}xa{\isasymleftarrow}if\ nat{\isacharunderscore}odd\ u\ then\ r\ {\isacharcolon}{\isacharequal}\ w\ {\isacharplus}\ v\ else\ ret\ {\isacharparenleft}{\isacharparenright}{\isacharsemicolon}\isanewline
\isaindent{\ {\isadigit{1}}{\isachardot}\ \ \ \ {\isasymturnstile}\ {\isacharbrackleft}{\isacharhash}\ do\ {\isacharbraceleft}}x{\isasymleftarrow}x\ {\isacharcolon}{\isacharequal}\ u\ div\ {\isadigit{2}}{\isacharsemicolon}\isanewline
\isaindent{\ {\isadigit{1}}{\isachardot}\ \ \ \ {\isasymturnstile}\ {\isacharbrackleft}{\isacharhash}\ do\ {\isacharbraceleft}}y\ {\isacharcolon}{\isacharequal}\ v\ {\isacharasterisk}\isanewline
\isaindent{\ {\isadigit{1}}{\isachardot}\ \ \ \ {\isasymturnstile}\ {\isacharbrackleft}{\isacharhash}\ do\ {\isacharbraceleft}y\ {\isacharcolon}{\isacharequal}\ }{\isadigit{2}}{\isacharbraceright}{\isacharbrackright}{\isacharparenleft}{\isasymlambda}u{\isachardot}\ Ret\ {\isacharparenleft}x\ {\isasymnoteq}\ y\ {\isasymand}\ y\ {\isasymnoteq}\ r\ {\isasymand}\ x\ {\isasymnoteq}\ r{\isacharparenright}\ {\isasymand}\isactrlsub D\isanewline
\isaindent{\ {\isadigit{1}}{\isachardot}\ \ \ \ {\isasymturnstile}\ {\isacharbrackleft}{\isacharhash}\ do\ {\isacharbraceleft}y\ {\isacharcolon}{\isacharequal}\ {\isadigit{2}}{\isacharbraceright}{\isacharbrackright}{\isacharparenleft}{\isasymlambda}u{\isachardot}\ }{\isasymUp}\ {\isacharparenleft}do\ {\isacharbraceleft}u{\isasymleftarrow}readRef\ x{\isacharsemicolon}\isanewline
\isaindent{\ {\isadigit{1}}{\isachardot}\ \ \ \ {\isasymturnstile}\ {\isacharbrackleft}{\isacharhash}\ do\ {\isacharbraceleft}y\ {\isacharcolon}{\isacharequal}\ {\isadigit{2}}{\isacharbraceright}{\isacharbrackright}{\isacharparenleft}{\isasymlambda}u{\isachardot}\ {\isasymUp}\ {\isacharparenleft}do\ {\isacharbraceleft}}v{\isasymleftarrow}readRef\ y{\isacharsemicolon}\isanewline
\isaindent{\ {\isadigit{1}}{\isachardot}\ \ \ \ {\isasymturnstile}\ {\isacharbrackleft}{\isacharhash}\ do\ {\isacharbraceleft}y\ {\isacharcolon}{\isacharequal}\ {\isadigit{2}}{\isacharbraceright}{\isacharbrackright}{\isacharparenleft}{\isasymlambda}u{\isachardot}\ {\isasymUp}\ {\isacharparenleft}do\ {\isacharbraceleft}}w{\isasymleftarrow}readRef\ r{\isacharsemicolon}\ ret\ {\isacharparenleft}u\ {\isacharasterisk}\ v\ {\isacharplus}\ w\ {\isacharequal}\ a\ {\isacharasterisk}\ b{\isacharparenright}{\isacharbraceright}{\isacharparenright}{\isacharparenright}%
\end{isabelle}%
\end{isamarkuptxt}%
\isamarkuptrue%
\isamarkupfalse%
\isamarkupfalse%
\isamarkupfalse%
\isamarkupfalse%
\isamarkupfalse%
\isamarkupfalse%
\isamarkupfalse%
\isamarkupfalse%
\isamarkupfalse%
\isamarkupfalse%
\isamarkupfalse%
\isamarkupfalse%
\isamarkupfalse%
\isamarkupfalse%
\isamarkupfalse%
\isamarkupfalse%
\isamarkupfalse%
\isamarkupfalse%
\isamarkupfalse%
\isamarkupfalse%
\isamarkupfalse%
\isamarkupfalse%
\isamarkupfalse%
\isamarkupfalse%
\isamarkupfalse%
\isamarkupfalse%
\isamarkupfalse%
\isamarkupfalse%
\isamarkupfalse%
\isamarkupfalse%
\isamarkupfalse%
\isamarkupfalse%
\isamarkupfalse%
\isamarkupfalse%
\isamarkupfalse%
\isamarkupfalse%
\isamarkupfalse%
\isamarkupfalse%
\isamarkupfalse%
\isamarkupfalse%
\isamarkupfalse%
%
\begin{isamarkuptxt}%
After having worked off all read operations, we again have to establish the strongest
      postcondition that is required after the if-statement.

      \begin{isabelle}%
{\isasymturnstile}\ Ret\ {\isacharparenleft}x\ {\isasymnoteq}\ y\ {\isasymand}\ y\ {\isasymnoteq}\ r\ {\isasymand}\ x\ {\isasymnoteq}\ r{\isacharparenright}\ {\isasymlongrightarrow}\isactrlsub D\ {\isacharbrackleft}{\isacharhash}\ rumult\ a\ b\ x\ y\ r{\isacharbrackright}{\isacharparenleft}{\isasymlambda}x{\isachardot}\ Ret\ {\isacharparenleft}x\ {\isacharequal}\ a\ {\isacharasterisk}\ b{\isacharparenright}{\isacharparenright}\isanewline
\ {\isadigit{1}}{\isachardot}\ {\isasymAnd}u\ v\ w{\isachardot}\isanewline
\isaindent{\ {\isadigit{1}}{\isachardot}\ \ \ \ }{\isasymturnstile}\ Ret\ {\isacharparenleft}{\isadigit{0}}\ {\isacharless}\ u{\isacharparenright}\ {\isasymand}\isactrlsub D\isanewline
\isaindent{\ {\isadigit{1}}{\isachardot}\ \ \ \ {\isasymturnstile}\ }Ret\ {\isacharparenleft}x\ {\isasymnoteq}\ y\ {\isasymand}\ y\ {\isasymnoteq}\ r\ {\isasymand}\ x\ {\isasymnoteq}\ r{\isacharparenright}\ {\isasymand}\isactrlsub D\isanewline
\isaindent{\ {\isadigit{1}}{\isachardot}\ \ \ \ {\isasymturnstile}\ }Ret\ {\isacharparenleft}u\ {\isacharasterisk}\ v\ {\isacharplus}\ w\ {\isacharequal}\ a\ {\isacharasterisk}\ b{\isacharparenright}\ {\isasymand}\isactrlsub D\isanewline
\isaindent{\ {\isadigit{1}}{\isachardot}\ \ \ \ {\isasymturnstile}\ }{\isasymUp}\ {\isacharparenleft}do\ {\isacharbraceleft}w{\isasymleftarrow}readRef\ r{\isacharsemicolon}\ ret\ {\isacharparenleft}u\ {\isacharasterisk}\ v\ {\isacharplus}\ w\ {\isacharequal}\ a\ {\isacharasterisk}\ b{\isacharparenright}{\isacharbraceright}{\isacharparenright}\ {\isasymlongrightarrow}\isactrlsub D\isanewline
\isaindent{\ {\isadigit{1}}{\isachardot}\ \ \ \ {\isasymturnstile}\ }{\isacharbrackleft}{\isacharhash}\ if\ nat{\isacharunderscore}odd\ u\ then\ r\ {\isacharcolon}{\isacharequal}\ w\ {\isacharplus}\ v\ else\ ret\ {\isacharparenleft}{\isacharparenright}{\isacharbrackright}{\isacharquery}B{\isadigit{1}}{\isadigit{1}}{\isadigit{1}}\ u\ v\ w%
\end{isabelle}%
\end{isamarkuptxt}%
\isamarkuptrue%
\isamarkupfalse%
\isamarkupfalse%
\isamarkupfalse%
\isamarkupfalse%
\isamarkupfalse%
\isamarkupfalse%
\isamarkupfalse%
\isamarkupfalse%
\isamarkupfalse%
\isamarkupfalse%
\isamarkupfalse%
\isamarkupfalse%
\isamarkupfalse%
\isamarkupfalse%
\isamarkupfalse%
\isamarkupfalse%
\isamarkupfalse%
\isamarkupfalse%
\isamarkupfalse%
\isamarkupfalse%
\isamarkupfalse%
\isamarkupfalse%
%
\begin{isamarkuptxt}%
Here we see what the just mentioned postcondition looks like: it says that the following
      relation (found in the premiss of the implication) holds:
      
      \begin{isabelle}%
{\isasymturnstile}\ Ret\ {\isacharparenleft}x\ {\isasymnoteq}\ y\ {\isasymand}\ y\ {\isasymnoteq}\ r\ {\isasymand}\ x\ {\isasymnoteq}\ r{\isacharparenright}\ {\isasymlongrightarrow}\isactrlsub D\ {\isacharbrackleft}{\isacharhash}\ rumult\ a\ b\ x\ y\ r{\isacharbrackright}{\isacharparenleft}{\isasymlambda}x{\isachardot}\ Ret\ {\isacharparenleft}x\ {\isacharequal}\ a\ {\isacharasterisk}\ b{\isacharparenright}{\isacharparenright}\isanewline
\ {\isadigit{1}}{\isachardot}\ {\isasymAnd}u\ v\ w\ xa{\isachardot}\isanewline
\isaindent{\ {\isadigit{1}}{\isachardot}\ \ \ \ }{\isasymturnstile}\ Ret\ {\isacharparenleft}x\ {\isasymnoteq}\ y\ {\isasymand}\ y\ {\isasymnoteq}\ r\ {\isasymand}\ x\ {\isasymnoteq}\ r{\isacharparenright}\ {\isasymand}\isactrlsub D\isanewline
\isaindent{\ {\isadigit{1}}{\isachardot}\ \ \ \ {\isasymturnstile}\ }{\isasymUp}\ {\isacharparenleft}do\ {\isacharbraceleft}w{\isasymleftarrow}readRef\ r{\isacharsemicolon}\ ret\ {\isacharparenleft}{\isacharparenleft}u\ div\ {\isadigit{2}}\ {\isacharplus}\ u\ div\ {\isadigit{2}}{\isacharparenright}\ {\isacharasterisk}\ v\ {\isacharplus}\ w\ {\isacharequal}\ a\ {\isacharasterisk}\ b{\isacharparenright}{\isacharbraceright}{\isacharparenright}\ {\isasymlongrightarrow}\isactrlsub D\isanewline
\isaindent{\ {\isadigit{1}}{\isachardot}\ \ \ \ {\isasymturnstile}\ }{\isacharbrackleft}{\isacharhash}\ x\ {\isacharcolon}{\isacharequal}\ u\ div\ {\isadigit{2}}{\isacharbrackright}{\isacharquery}B{\isadigit{1}}{\isadigit{4}}{\isadigit{2}}\ u\ v\ w\ xa%
\end{isabelle}%
\end{isamarkuptxt}%
\isamarkuptrue%
\isamarkupfalse%
\isamarkupfalse%
\isamarkupfalse%
\isamarkupfalse%
\isamarkupfalse%
\isamarkupfalse%
\isamarkupfalse%
%
\begin{isamarkuptxt}%
Now only the assignment to \isa{y} remains.

      \begin{isabelle}%
{\isasymturnstile}\ Ret\ {\isacharparenleft}x\ {\isasymnoteq}\ y\ {\isasymand}\ y\ {\isasymnoteq}\ r\ {\isasymand}\ x\ {\isasymnoteq}\ r{\isacharparenright}\ {\isasymlongrightarrow}\isactrlsub D\ {\isacharbrackleft}{\isacharhash}\ rumult\ a\ b\ x\ y\ r{\isacharbrackright}{\isacharparenleft}{\isasymlambda}x{\isachardot}\ Ret\ {\isacharparenleft}x\ {\isacharequal}\ a\ {\isacharasterisk}\ b{\isacharparenright}{\isacharparenright}\isanewline
\ {\isadigit{1}}{\isachardot}\ {\isasymAnd}u\ v\ w\ xa\ xaa{\isachardot}\isanewline
\isaindent{\ {\isadigit{1}}{\isachardot}\ \ \ \ }{\isasymturnstile}\ {\isacharasterisk}x\ {\isacharequal}\isactrlsub D\ Ret\ {\isacharparenleft}u\ div\ {\isadigit{2}}{\isacharparenright}\ {\isasymand}\isactrlsub D\isanewline
\isaindent{\ {\isadigit{1}}{\isachardot}\ \ \ \ {\isasymturnstile}\ }Ret\ {\isacharparenleft}x\ {\isasymnoteq}\ y\ {\isasymand}\ y\ {\isasymnoteq}\ r\ {\isasymand}\ x\ {\isasymnoteq}\ r{\isacharparenright}\ {\isasymand}\isactrlsub D\isanewline
\isaindent{\ {\isadigit{1}}{\isachardot}\ \ \ \ {\isasymturnstile}\ }{\isasymUp}\ {\isacharparenleft}do\ {\isacharbraceleft}w{\isasymleftarrow}readRef\ r{\isacharsemicolon}\ ret\ {\isacharparenleft}{\isacharparenleft}u\ div\ {\isadigit{2}}\ {\isacharplus}\ u\ div\ {\isadigit{2}}{\isacharparenright}\ {\isacharasterisk}\ v\ {\isacharplus}\ w\ {\isacharequal}\ a\ {\isacharasterisk}\ b{\isacharparenright}{\isacharbraceright}{\isacharparenright}\ {\isasymlongrightarrow}\isactrlsub D\isanewline
\isaindent{\ {\isadigit{1}}{\isachardot}\ \ \ \ {\isasymturnstile}\ }{\isacharbrackleft}{\isacharhash}\ y\ {\isacharcolon}{\isacharequal}\ v\ {\isacharasterisk}\ {\isadigit{2}}{\isacharbrackright}{\isacharquery}B{\isadigit{1}}{\isadigit{5}}{\isadigit{1}}\ u\ v\ w\ xa\ xaa%
\end{isabelle}%
\end{isamarkuptxt}%
\isamarkuptrue%
\isamarkupfalse%
\isamarkupfalse%
\isamarkupfalse%
\isamarkupfalse%
\isamarkupfalse%
\isamarkupfalse%
\isamarkupfalse%
\isamarkupfalse%
\isamarkupfalse%
%
\begin{isamarkuptxt}%
We finally succeeded in re-establishing the loop invariant after one
      execution of the loop
      body. The final part is just to read reference \isa{r}, which is easily done.
      
       \begin{isabelle}%
{\isasymturnstile}\ Ret\ {\isacharparenleft}x\ {\isasymnoteq}\ y\ {\isasymand}\ y\ {\isasymnoteq}\ r\ {\isasymand}\ x\ {\isasymnoteq}\ r{\isacharparenright}\ {\isasymlongrightarrow}\isactrlsub D\ {\isacharbrackleft}{\isacharhash}\ rumult\ a\ b\ x\ y\ r{\isacharbrackright}{\isacharparenleft}{\isasymlambda}x{\isachardot}\ Ret\ {\isacharparenleft}x\ {\isacharequal}\ a\ {\isacharasterisk}\ b{\isacharparenright}{\isacharparenright}\isanewline
\ {\isadigit{1}}{\isachardot}\ {\isasymAnd}xa\ xaa\ xb\ xc{\isachardot}\isanewline
\isaindent{\ {\isadigit{1}}{\isachardot}\ \ \ \ }{\isasymturnstile}\ {\isacharparenleft}Ret\ {\isacharparenleft}x\ {\isasymnoteq}\ y\ {\isasymand}\ y\ {\isasymnoteq}\ r\ {\isasymand}\ x\ {\isasymnoteq}\ r{\isacharparenright}\ {\isasymand}\isactrlsub D\isanewline
\isaindent{\ {\isadigit{1}}{\isachardot}\ \ \ \ {\isasymturnstile}\ {\isacharparenleft}}{\isasymUp}\ {\isacharparenleft}do\ {\isacharbraceleft}u{\isasymleftarrow}readRef\ x{\isacharsemicolon}\isanewline
\isaindent{\ {\isadigit{1}}{\isachardot}\ \ \ \ {\isasymturnstile}\ {\isacharparenleft}{\isasymUp}\ {\isacharparenleft}do\ {\isacharbraceleft}}v{\isasymleftarrow}readRef\ y{\isacharsemicolon}\ w{\isasymleftarrow}readRef\ r{\isacharsemicolon}\ ret\ {\isacharparenleft}u\ {\isacharasterisk}\ v\ {\isacharplus}\ w\ {\isacharequal}\ a\ {\isacharasterisk}\ b{\isacharparenright}{\isacharbraceright}{\isacharparenright}{\isacharparenright}\ {\isasymand}\isactrlsub D\isanewline
\isaindent{\ {\isadigit{1}}{\isachardot}\ \ \ \ {\isasymturnstile}\ }{\isasymnot}\isactrlsub D\ {\isasymUp}\ {\isacharparenleft}do\ {\isacharbraceleft}u{\isasymleftarrow}readRef\ x{\isacharsemicolon}\ ret\ {\isacharparenleft}{\isadigit{0}}\ {\isacharless}\ u{\isacharparenright}{\isacharbraceright}{\isacharparenright}\ {\isasymlongrightarrow}\isactrlsub D\isanewline
\isaindent{\ {\isadigit{1}}{\isachardot}\ \ \ \ {\isasymturnstile}\ }{\isacharbrackleft}{\isacharhash}\ readRef\ r{\isacharbrackright}{\isacharparenleft}{\isasymlambda}x{\isachardot}\ Ret\ {\isacharparenleft}x\ {\isacharequal}\ a\ {\isacharasterisk}\ b{\isacharparenright}{\isacharparenright}%
\end{isabelle}%
\end{isamarkuptxt}%
\ \ \isamarkuptrue%
\isacommand{apply}{\isacharparenleft}rule\ conclude{\isacharunderscore}aux{\isacharparenright}\ \ %
\isamarkupcmt{\dots Just 124 straightforward proof steps later%
}
\isanewline
\isamarkupfalse%
\isacommand{done}\isanewline
\isanewline
\isanewline
\isamarkupfalse%
\isacommand{end}\isanewline
\isanewline
\isamarkupfalse%
\end{isabellebody}%
%%% Local Variables:
%%% mode: latex
%%% TeX-master: "root"
%%% End:

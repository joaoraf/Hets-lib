

\chapter{Haskell Implementation of mbody }
\label{cha:hask-impl-mbody}

We present here a complete Haskell implementation of the $\op{mbody}$ construct
described in Section \ref{sec:param-except}. As an example application the
pattern match algorithm that has been verified in Section
\ref{sec:corr-patt-match} and in \cite{WalterEA05} is used.
\begin{alltt}
module MBodyTrans
where

import Control.Monad.Error
import Control.Monad.State

data Exception a = Excpt String
                 | Ret a
                 | DropOff
                   deriving (Show)


-- Needed for class dependencies; actually only for fail 
-- which is not used in our calculus.
instance Error (Exception a) where
    noMsg = Excpt ""
    strMsg = Excpt
\end{alltt}
Rather than defining the binding in an exception monad by ourselves, we make use
of the \emph{exception monad transformer} \code{ErrorT} from the Haskell
libraries. The type of exceptions consists of three alternatives; exceptions
either signal failure through \code{Excpt} with a message attached, or they
carry a return value of some method, or they indicate that a method has
illegally terminated normally (\code{DropOff}).  For simplicity, \code{continue}
and \code{break} have been left out, but could easily be added.

For every monad $m$ we can construct a monad $(\op{Ex}\ m\ e)$ that behaves just
like $m$ in the absence of an exception, but also allows exceptions to be thrown
and caught. 
\begin{alltt}
type Ex m e a = ErrorT (Exception e) m a
\end{alltt}
Recall the instance definition of \code{ErrorT} from the Haskell libraries:
\begin{alltt}
instance (Monad m, Error e) => Monad (ErrorT e m) where
        return a = ErrorT $ return (Right a)
        m >>= k  = ErrorT $ do
                a <- runErrorT m
                case a of
                        Left  l -> return (Left l)
                        Right r -> runErrorT (k r)
        fail msg = ErrorT $ return (Left (strMsg msg))
\end{alltt}  % $ ugly highlighter!
which precisely captures the intended behaviour of the binding in the presence
of an exception, namely that the right-hand argument is only evaluated if the
left one terminated normally. The function \code{runErrorT} simply unpacks the
inner monad, \IE drops the constructor \code{ErrorT}.

The concrete state monad that will be used below needs a single reference of
type \code{Int}, but the general variable mapping can be defined as follows. A
variable map consists of an ID for the next reference and a function mapping
references to their values:
\begin{alltt}
type Ref = Int

type VMap a = (Int, Ref -> a)
\end{alltt}

Next comes the \code{mbody} construction that catches \code{Ret} exceptions and
converts them into normal return values. All other exceptions are propagated
unchanged.  This implementation is polymorphic in the exception type of the
result and thus allows for switching between monads.  Whether the input
computation should be polymorphic in its return type or whether $\unit$ should be
enforced is a matter of taste. \code{mret} emulates the actual Java
\code{return} statements -- whereas \code{return} is the usual monadic $\ret$
function. 
\label{page:mbody-def}
\begin{alltt}
mret :: Monad m=> e -> Ex m e a
mret x = throwError (Ret x)

mbody :: Monad m=> Ex m e () -> Ex m e1 e
mbody p = ErrorT $ do
          a <- runErrorT p      -- binding in the "inner" monad
          case a of
            Right () -> return (Left DropOff)
            Left e   -> case e of
              Ret x    -> return (Right x)
              Excpt s  -> return (Left (Excpt s))
              DropOff  -> return (Left DropOff)
\end{alltt}
% $ stupid highlighter!

There are three state-related operations on exception state monads: reading,
writing and creation of variables. A generic while loop for the exception
state monad is also easily defined.
\begin{alltt}
readVar :: Ref -> Ex (State (VMap a)) e a
readVar r = do (_, f) <- get
               return (f r)

wrtVar :: Ref -> a -> Ex (State (VMap a)) e ()
wrtVar r x = do (n, f) <- get
                put (n, \textbackslash{} k-> if k == r then x else f k)

newVar :: a -> Ex (State (VMap a)) e Ref
newVar v = do (n, f) <- get
              put (n+1, \textbackslash{} k-> if k == n then v else f k)
              return n

while :: Monad m=> Ex m e Bool -> Ex m e () -> Ex m e ()
while b p = do v <- b
               if v then do p; while b p
                    else return ()
\end{alltt}

The pattern match algorithm, as described in \cite{WalterEA05,HuismanJacobs00}.
For testing purposes, here's how to evaluate pmatch with an initial state with
all references defaulting to 0:\\[1.5ex]
\centerline{\texttt{evalState (runErrorT (pmatch base1 pat1)) (0, const 0)}}\\[1.5ex]
which will
evaluate (correctly) to \code{Right 9}
\begin{alltt}
pmatch :: String -> String -> Ex (State (VMap Int)) e Int
pmatch base pat = mbody $ do
    r <- newVar 0
    s <- newVar 0
    while (return True)
          (do u <- readVar r
              v <- readVar s
              if u == length pat
                then mret v
                else if v + u == length base 
                  then throwError (Excpt "Pattern not found")
                  else if base!!(v+u) == pat!!u
                    then wrtVar r (u+1)
                    else do wrtVar s (v+1); wrtVar r 0)

-- Some sample patterns
base1 :: String
base1 = "puff the magic dragon"

pat1 :: String
pat1 = "magic"

pat2 :: String
pat2 = "mary"
\end{alltt}
% $


\chapter{Table of  Rules of Isabelle/HOL}
\label{cha:freq-used-rules}

Since the main purpose of the implementation in Isabelle was to set up a new
logic, only few deep theorems of Isabelle/HOL itself, on which the logic is
based, have been made use of.  Further, many rules are applied implicitly
when employing the simplifier or the classical reasoner. The following is a list
of the rules that appear verbatim in the implementation.

\begin{table}[h]
  \centering \renewcommand{\arraystretch}{1.3}
  \begin{tabular}{|l@{$\quad$}l|}\hline
    \irule{allI}        & $(\bigwedge x\bdot P\, x) \Longrightarrow \forall x\bdot P\, x$\\\hline
    \irule{arg\_cong}    & $x = y \Longrightarrow f\ x = f\ y$\\\hline
    \irule{cong}        & $\llbracket f = g;\; x = y\rrbracket \Longrightarrow f\ x = g\ y$\\\hline
    \irule{conjE}       & $\llbracket P\land Q;\; \llbracket P;\; Q\rrbracket \Longrightarrow R\rrbracket \Longrightarrow R$\\\hline
    \irule{conjI}       & $\llbracket P;\; Q\rrbracket \Longrightarrow P \land Q$ \\\hline
    \irule{conjunct1}   & $\llbracket P \land Q\rrbracket \Longrightarrow P$ \\\hline
    \irule{exE}         & $\llbracket\exists x\bdot P\ x;\; \bigwedge x\bdot P\ x \Longrightarrow Q\rrbracket \Longrightarrow Q$\\\hline
    \irule{FalseE}      & $\mathit{False} \Longrightarrow P$\\\hline
    \irule{iffD1}       & $\llbracket Q = P;\; Q\rrbracket \Longrightarrow P$\\\hline
    \irule{iffD2}       & $\llbracket P = Q;\; Q\rrbracket \Longrightarrow P$\\\hline
    \irule{iffI}        & $\llbracket P \Longrightarrow Q;\; Q \Longrightarrow P\rrbracket \Longrightarrow P = Q$\\\hline
    \irule{impI}        & $(P \Longrightarrow Q) \Longrightarrow P \longrightarrow Q$\\\hline
    \irule{mp}          & $\llbracket P \longrightarrow Q;\; P\rrbracket \Longrightarrow Q$\\\hline
    \irule{notE}        & $\llbracket\lnot P;\; P\rrbracket \Longrightarrow R$\\\hline
    \irule{notI}        & $(P \Longrightarrow \mathit{False}) \Longrightarrow \lnot P$\\\hline
    \irule{refl}        & $t = t$\\\hline
    \irule{spec}        & $\forall x.\ P\ x \Longrightarrow P\ y$ \\\hline
    \irule{subst}       & $\llbracket s = t;\; P\ s\rrbracket \Longrightarrow P\ t$\\\hline
  \end{tabular}
  \caption{Derived rules of inference for HOL}
  \label{tab:derived-rules}
\end{table}


\chapter{Isabelle Theories}
\label{cha:isabelle-theories}

The following sections present the concrete implementation of the calculus of
dynamic logic in Isabelle. The typesetting has been automatically taken care of
by the \verb|isatool| mechanism of the Isabelle distribution, which directly
extracts this information from the given theory files. This chapter is intended
for reference usage and not so much for being perused sequentially. Refer to
Chapter \ref{cha:implementation} for a conceptual description of the
implementation.

%
\begin{isabellebody}%
\def\isabellecontext{Monads}%
%
\isamarkupheader{Basic Monad Definitions and Laws.%
}
\isamarkuptrue%
\isacommand{theory}\ Monads\ {\isacharequal}\ Main{\isacharcolon}\isamarkupfalse%
%
\label{sec:monads-thy}
%
\begin{isamarkuptext}%
For the lack of constructor classes in Isabelle, we initially
 use functor \isa{T} as a parameter standing for the monad in question.
 \label{isa:monads-def}%
\end{isamarkuptext}%
\isamarkuptrue%
\isacommand{typedecl}\ {\isacharprime}a\ T\isanewline
\isanewline
\isamarkupfalse%
\isacommand{arities}\ T\ {\isacharcolon}{\isacharcolon}\ {\isacharparenleft}type{\isacharparenright}type\isamarkupfalse%
%
\begin{isamarkuptext}%
Monadic operations, decorated with Haskell-style syntax.%
\end{isamarkuptext}%
\isamarkuptrue%
\isacommand{consts}\isanewline
\ bind\ {\isacharcolon}{\isacharcolon}\ {\isachardoublequote}{\isacharprime}a\ T\ {\isasymRightarrow}\ {\isacharparenleft}{\isacharprime}a\ {\isasymRightarrow}\ {\isacharprime}b\ T{\isacharparenright}\ {\isasymRightarrow}\ {\isacharprime}b\ T{\isachardoublequote}\ \ \ \ \ {\isacharparenleft}\isakeyword{infixl}\ {\isachardoublequote}{\isasymggreater}{\isacharequal}{\isachardoublequote}\ {\isadigit{2}}{\isadigit{0}}{\isacharparenright}\isanewline
\ ret\ \ {\isacharcolon}{\isacharcolon}\ {\isachardoublequote}{\isacharprime}a\ {\isasymRightarrow}\ {\isacharprime}a\ T{\isachardoublequote}\isanewline
\isanewline
\isamarkupfalse%
\isacommand{constdefs}\isanewline
\ seq\ {\isacharcolon}{\isacharcolon}\ {\isachardoublequote}{\isacharprime}a\ T\ {\isasymRightarrow}\ {\isacharprime}b\ T\ {\isasymRightarrow}\ {\isacharprime}b\ T{\isachardoublequote}\ \ \ \ \ \ \ \ \ \ \ \ {\isacharparenleft}\isakeyword{infixl}\ {\isachardoublequote}{\isasymggreater}{\isachardoublequote}\ {\isadigit{2}}{\isadigit{0}}{\isacharparenright}\isanewline
\ {\isachardoublequote}p\ {\isasymggreater}\ q\ {\isasymequiv}\ {\isacharparenleft}p\ {\isasymggreater}{\isacharequal}\ {\isacharparenleft}{\isasymlambda}x{\isachardot}\ q{\isacharparenright}{\isacharparenright}{\isachardoublequote}\isamarkupfalse%
%
\begin{isamarkuptext}%
The usual monad laws for bind and ret (not the Kleisli triple ones)
  including injectivity of \isa{ret} for convenience.
  \label{isa:monads-laws}%
\end{isamarkuptext}%
\isamarkuptrue%
\isacommand{axioms}\isanewline
\ bind{\isacharunderscore}assoc\ {\isacharbrackleft}simp{\isacharbrackright}{\isacharcolon}\ {\isachardoublequote}{\isacharparenleft}p\ {\isasymggreater}{\isacharequal}\ {\isacharparenleft}{\isasymlambda}x{\isachardot}\ f\ x\ {\isasymggreater}{\isacharequal}\ g{\isacharparenright}{\isacharparenright}\ {\isacharequal}\ {\isacharparenleft}p\ {\isasymggreater}{\isacharequal}\ f\ {\isasymggreater}{\isacharequal}\ g{\isacharparenright}{\isachardoublequote}\isanewline
\ ret{\isacharunderscore}lunit\ {\isacharbrackleft}simp{\isacharbrackright}{\isacharcolon}\ {\isachardoublequote}{\isacharparenleft}ret\ x\ {\isasymggreater}{\isacharequal}\ f{\isacharparenright}\ {\isacharequal}\ f\ x{\isachardoublequote}\isanewline
\ ret{\isacharunderscore}runit\ {\isacharbrackleft}simp{\isacharbrackright}{\isacharcolon}\ {\isachardoublequote}{\isacharparenleft}p\ {\isasymggreater}{\isacharequal}\ ret{\isacharparenright}\ {\isacharequal}\ p{\isachardoublequote}\isanewline
\ ret{\isacharunderscore}inject{\isacharcolon}\ {\isachardoublequote}ret\ x\ {\isacharequal}\ ret\ z\ {\isasymLongrightarrow}\ x\ {\isacharequal}\ z{\isachardoublequote}\isanewline
\isanewline
\isamarkupfalse%
\isacommand{lemma}\ seq{\isacharunderscore}assoc\ {\isacharbrackleft}simp{\isacharbrackright}{\isacharcolon}\ {\isachardoublequote}{\isacharparenleft}p\ {\isasymggreater}\ {\isacharparenleft}q\ {\isasymggreater}\ r{\isacharparenright}{\isacharparenright}\ {\isacharequal}\ {\isacharparenleft}p\ {\isasymggreater}\ q\ {\isasymggreater}\ r{\isacharparenright}{\isachardoublequote}\isanewline
\ \isamarkupfalse%
\isacommand{by}\ {\isacharparenleft}simp\ add{\isacharcolon}\ seq{\isacharunderscore}def{\isacharparenright}\isamarkupfalse%
%
\begin{isamarkuptext}%
This sets up a Haskell-style `\isa{do\ {\isacharbraceleft}x{\isasymleftarrow}p{\isacharsemicolon}\ q{\isacharbraceright}}' syntax 
  with multiple bindings inside one \isa{do} term.
  \label{isa:do-notation}%
\end{isamarkuptext}%
\isamarkuptrue%
\isacommand{nonterminals}\isanewline
\ monseq\isanewline
\isanewline
\isamarkupfalse%
\isacommand{syntax}\ {\isacharparenleft}xsymbols{\isacharparenright}\isanewline
\ {\isachardoublequote}{\isacharunderscore}monseq{\isachardoublequote}\ \ {\isacharcolon}{\isacharcolon}\ {\isachardoublequote}monseq\ {\isasymRightarrow}\ {\isacharprime}a\ T{\isachardoublequote}\ \ \ \ \ \ \ \ \ \ \ \ \ \ \ \ \ \ {\isacharparenleft}{\isachardoublequote}{\isacharparenleft}do\ {\isacharbraceleft}{\isacharparenleft}{\isacharunderscore}{\isacharparenright}{\isacharbraceright}{\isacharparenright}{\isachardoublequote}\ \ \ \ {\isacharbrackleft}{\isadigit{5}}{\isacharbrackright}\ {\isadigit{1}}{\isadigit{0}}{\isadigit{0}}{\isacharparenright}\isanewline
\ {\isachardoublequote}{\isacharunderscore}mongen{\isachardoublequote}\ \ {\isacharcolon}{\isacharcolon}\ {\isachardoublequote}{\isacharbrackleft}pttrn{\isacharcomma}\ {\isacharprime}a\ T{\isacharcomma}\ monseq{\isacharbrackright}{\isasymRightarrow}\ monseq{\isachardoublequote}\ \ \ {\isacharparenleft}{\isachardoublequote}{\isacharparenleft}{\isacharunderscore}{\isasymleftarrow}{\isacharparenleft}{\isacharunderscore}{\isacharparenright}{\isacharsemicolon}{\isacharslash}\ {\isacharunderscore}{\isacharparenright}{\isachardoublequote}\ \ {\isacharbrackleft}{\isadigit{1}}{\isadigit{0}}{\isacharcomma}\ {\isadigit{6}}{\isacharcomma}\ {\isadigit{5}}{\isacharbrackright}\ {\isadigit{5}}{\isacharparenright}\isanewline
\ {\isachardoublequote}{\isacharunderscore}monexp{\isachardoublequote}\ \ {\isacharcolon}{\isacharcolon}\ {\isachardoublequote}{\isacharbrackleft}{\isacharprime}a\ T{\isacharcomma}\ monseq{\isacharbrackright}{\isasymRightarrow}\ monseq{\isachardoublequote}\ \ \ \ \ \ \ \ \ {\isacharparenleft}{\isachardoublequote}{\isacharparenleft}{\isacharunderscore}{\isacharsemicolon}{\isacharslash}\ {\isacharunderscore}{\isacharparenright}{\isachardoublequote}\ \ \ \ \ \ \ {\isacharbrackleft}{\isadigit{6}}{\isacharcomma}\ {\isadigit{5}}{\isacharbrackright}\ {\isadigit{5}}{\isacharparenright}\isanewline
\ {\isachardoublequote}{\isacharunderscore}monexp{\isadigit{0}}{\isachardoublequote}\ {\isacharcolon}{\isacharcolon}\ {\isachardoublequote}{\isacharbrackleft}{\isacharprime}a\ T{\isacharbrackright}\ {\isasymRightarrow}\ monseq{\isachardoublequote}\ \ \ \ \ \ \ \ \ \ \ \ \ \ \ \ {\isacharparenleft}{\isachardoublequote}{\isacharparenleft}{\isacharunderscore}{\isacharparenright}{\isachardoublequote}\ \ \ \ \ \ \ \ \ {\isadigit{5}}{\isacharparenright}\isanewline
\isanewline
\isamarkupfalse%
\isacommand{translations}\isanewline
\ %
\isamarkupcmt{input macros; replace do-notation by \isa{op\ {\isasymggreater}{\isacharequal}}/\isa{op\ {\isasymggreater}}%
}
\isanewline
\ {\isachardoublequote}{\isacharunderscore}monseq{\isacharparenleft}{\isacharunderscore}mongen\ x\ p\ q{\isacharparenright}{\isachardoublequote}\ \ \ \ {\isasymrightharpoonup}\ {\isachardoublequote}p\ {\isasymggreater}{\isacharequal}\ {\isacharparenleft}{\isacharpercent}x{\isachardot}\ {\isacharparenleft}{\isacharunderscore}monseq\ q{\isacharparenright}{\isacharparenright}{\isachardoublequote}\isanewline
\ {\isachardoublequote}{\isacharunderscore}monseq{\isacharparenleft}{\isacharunderscore}monexp\ p\ q{\isacharparenright}{\isachardoublequote}\ \ \ \ \ \ {\isasymrightharpoonup}\ {\isachardoublequote}p\ {\isasymggreater}\ {\isacharparenleft}{\isacharunderscore}monseq\ q{\isacharparenright}{\isachardoublequote}\isanewline
\ {\isachardoublequote}{\isacharunderscore}monseq{\isacharparenleft}{\isacharunderscore}monexp{\isadigit{0}}\ q{\isacharparenright}{\isachardoublequote}\ \ \ \ \ \ \ {\isasymrightharpoonup}\ {\isachardoublequote}q{\isachardoublequote}\isanewline
\ %
\isamarkupcmt{Retranslation of into the do-notation%
}
\isanewline
\ {\isachardoublequote}{\isacharunderscore}monseq{\isacharparenleft}{\isacharunderscore}mongen\ x\ p\ q{\isacharparenright}{\isachardoublequote}\ \ \ \ {\isasymleftharpoondown}\ {\isachardoublequote}p\ {\isasymggreater}{\isacharequal}\ {\isacharparenleft}{\isacharpercent}x{\isachardot}\ q{\isacharparenright}{\isachardoublequote}\isanewline
\ {\isachardoublequote}{\isacharunderscore}monseq{\isacharparenleft}{\isacharunderscore}monexp\ p\ q{\isacharparenright}{\isachardoublequote}\ \ \ \ \ \ {\isasymleftharpoondown}\ {\isachardoublequote}p\ {\isasymggreater}\ q{\isachardoublequote}\isanewline
\ %
\isamarkupcmt{Normalization macros `flattening' do-terms%
}
\isanewline
\ {\isachardoublequote}{\isacharunderscore}monseq{\isacharparenleft}{\isacharunderscore}mongen\ x\ p\ q{\isacharparenright}{\isachardoublequote}\ \ \ \ {\isasymleftharpoondown}\ {\isachardoublequote}{\isacharunderscore}monseq\ {\isacharparenleft}{\isacharunderscore}mongen\ x\ p\ {\isacharparenleft}{\isacharunderscore}monseq\ q{\isacharparenright}{\isacharparenright}{\isachardoublequote}\isanewline
\ {\isachardoublequote}{\isacharunderscore}monseq{\isacharparenleft}{\isacharunderscore}monexp\ p\ q{\isacharparenright}{\isachardoublequote}\ \ \ \ \ \ {\isasymleftharpoondown}\ {\isachardoublequote}{\isacharunderscore}monseq\ {\isacharparenleft}{\isacharunderscore}monexp\ p\ {\isacharparenleft}{\isacharunderscore}monseq\ q{\isacharparenright}{\isacharparenright}{\isachardoublequote}\isamarkupfalse%
%
\begin{isamarkuptext}%
Actually, this rule does not contract, but rather expand monadic 
  sequences, but for historical reasons\dots%
\end{isamarkuptext}%
\isamarkuptrue%
\isacommand{lemma}\ mon{\isacharunderscore}ctr{\isacharcolon}\ {\isachardoublequote}{\isacharparenleft}do\ {\isacharbraceleft}x\ {\isasymleftarrow}\ {\isacharparenleft}do\ {\isacharbraceleft}y\ {\isasymleftarrow}\ p{\isacharsemicolon}\ q\ y{\isacharbraceright}{\isacharparenright}{\isacharsemicolon}\ f\ x{\isacharbraceright}{\isacharparenright}\ {\isacharequal}\ {\isacharparenleft}do\ {\isacharbraceleft}y\ {\isasymleftarrow}\ p{\isacharsemicolon}\ x\ {\isasymleftarrow}\ q\ y{\isacharsemicolon}\ f\ x{\isacharbraceright}{\isacharparenright}{\isachardoublequote}\isanewline
\ \ \isamarkupfalse%
\isacommand{by}{\isacharparenleft}rule\ bind{\isacharunderscore}assoc{\isacharbrackleft}symmetric{\isacharbrackright}{\isacharparenright}\isanewline
\isanewline
\ \ \isanewline
\isanewline
\isamarkupfalse%
\isacommand{end}\isanewline
\isamarkupfalse%
\end{isabellebody}%
%%% Local Variables:
%%% mode: latex
%%% TeX-master: "root"
%%% End:


%
\begin{isabellebody}%
\def\isabellecontext{MonProp}%
%
\isamarkupheader{Basic Notions of Monadic Programs%
}
\isamarkuptrue%
\isacommand{theory}\ MonProp\ {\isacharequal}\ Monads{\isacharcolon}\isamarkupfalse%
%
\label{sec:monprop-thy}
%
\isamarkupsubsection{Discardability and Copyability%
}
\isamarkuptrue%
%
\begin{isamarkuptext}%
Properties of monadic programs which are needed for the further development,
  e.g. for the definition of a subtype \isa{{\isacharprime}a\ D} of deterministically
  side-effect free (\isa{dsef}) programs.
  \label{isa:mon-properties}%
\end{isamarkuptext}%
\isamarkuptrue%
\isacommand{constdefs}\isanewline
\ \ %
\isamarkupcmt{Discardable programs%
}
\isanewline
\ \ dis\ {\isacharcolon}{\isacharcolon}\ {\isachardoublequote}{\isacharprime}a\ T\ {\isasymRightarrow}\ bool{\isachardoublequote}\isanewline
\ \ {\isachardoublequote}dis{\isacharparenleft}p{\isacharparenright}\ {\isasymequiv}\ {\isacharparenleft}do\ {\isacharbraceleft}x{\isasymleftarrow}p{\isacharsemicolon}\ ret{\isacharparenleft}{\isacharparenright}{\isacharbraceright}{\isacharparenright}\ {\isacharequal}\ ret\ {\isacharparenleft}{\isacharparenright}{\isachardoublequote}\isanewline
\ \ %
\isamarkupcmt{Copyable programs%
}
\isanewline
\ \ cp\ \ {\isacharcolon}{\isacharcolon}\ {\isachardoublequote}{\isacharprime}a\ T\ {\isasymRightarrow}\ bool{\isachardoublequote}\isanewline
\ \ {\isachardoublequote}cp{\isacharparenleft}p{\isacharparenright}\ {\isasymequiv}\ {\isacharparenleft}do\ {\isacharbraceleft}x{\isasymleftarrow}p{\isacharsemicolon}\ y{\isasymleftarrow}p{\isacharsemicolon}\ ret{\isacharparenleft}x{\isacharcomma}y{\isacharparenright}{\isacharbraceright}{\isacharparenright}\ {\isacharequal}\ {\isacharparenleft}do\ {\isacharbraceleft}x{\isasymleftarrow}p{\isacharsemicolon}\ ret{\isacharparenleft}x{\isacharcomma}x{\isacharparenright}{\isacharbraceright}{\isacharparenright}{\isachardoublequote}\isanewline
\ \ %
\isamarkupcmt{\isa{dsef} programs are \isa{cp} and \isa{dis} and
        commute with all such programs%
}
\isanewline
\ \ dsef\ {\isacharcolon}{\isacharcolon}\ {\isachardoublequote}{\isacharprime}a\ T\ {\isasymRightarrow}\ bool{\isachardoublequote}\isanewline
\ \ {\isachardoublequote}dsef{\isacharparenleft}p{\isacharparenright}\ {\isasymequiv}\ cp{\isacharparenleft}p{\isacharparenright}\ {\isasymand}\ dis{\isacharparenleft}p{\isacharparenright}\ {\isasymand}\ {\isacharparenleft}{\isasymforall}q{\isacharcolon}{\isacharcolon}bool\ T{\isachardot}\ cp{\isacharparenleft}q{\isacharparenright}\ {\isasymand}\ dis{\isacharparenleft}q{\isacharparenright}\ {\isasymlongrightarrow}\ \isanewline
\ \ \ \ \ \ \ \ \ \ \ \ \ \ \ \ \ \ \ \ \ \ \ \ \ \ \ \ \ \ \ \ \ \ \ \ cp{\isacharparenleft}do\ {\isacharbraceleft}x{\isasymleftarrow}p{\isacharsemicolon}\ y{\isasymleftarrow}q{\isacharsemicolon}\ ret{\isacharparenleft}x{\isacharcomma}y{\isacharparenright}{\isacharbraceright}{\isacharparenright}{\isacharparenright}{\isachardoublequote}\isanewline
\isanewline
\isanewline
\isamarkupfalse%
\isacommand{lemma}\ dsef{\isacharunderscore}cp{\isacharcolon}\ {\isachardoublequote}dsef\ p\ {\isasymLongrightarrow}\ cp\ p{\isachardoublequote}\isanewline
\ \ \isamarkupfalse%
\isacommand{apply}{\isacharparenleft}unfold\ dsef{\isacharunderscore}def{\isacharparenright}\isanewline
\isamarkupfalse%
\isacommand{by}\ blast\isanewline
\isanewline
\isamarkupfalse%
\isacommand{lemma}\ dsef{\isacharunderscore}dis{\isacharcolon}\ {\isachardoublequote}dsef\ p\ {\isasymLongrightarrow}\ dis\ p{\isachardoublequote}\isanewline
\ \ \isamarkupfalse%
\isacommand{apply}{\isacharparenleft}unfold\ dsef{\isacharunderscore}def{\isacharparenright}\isanewline
\isamarkupfalse%
\isacommand{by}\ blast\isamarkupfalse%
%
\begin{isamarkuptext}%
This is Lemma 4.5 of \cite{SchroederMossakowski:PDL} that allows us to actually discard
  discardable programs in front of arbitrary programs.%
\end{isamarkuptext}%
\isamarkuptrue%
\isacommand{lemma}\ dis{\isacharunderscore}left{\isacharcolon}\ {\isachardoublequote}dis{\isacharparenleft}p{\isacharparenright}\ {\isasymLongrightarrow}\ do\ {\isacharbraceleft}p{\isacharsemicolon}\ q{\isacharbraceright}\ {\isacharequal}\ q{\isachardoublequote}\isanewline
\isamarkupfalse%
\isacommand{proof}\ {\isacharminus}\isanewline
\ \ \isamarkupfalse%
\isacommand{assume}\ d{\isacharcolon}\ {\isachardoublequote}dis{\isacharparenleft}p{\isacharparenright}{\isachardoublequote}\isanewline
\ \ \isamarkupfalse%
\isacommand{have}\ {\isachardoublequote}do\ {\isacharbraceleft}p{\isacharsemicolon}\ q{\isacharbraceright}\ {\isacharequal}\ do\ {\isacharbraceleft}p{\isacharsemicolon}\ ret\ {\isacharparenleft}{\isacharparenright}{\isacharsemicolon}\ q{\isacharbraceright}{\isachardoublequote}\isanewline
\ \ \ \ \isamarkupfalse%
\isacommand{by}\ {\isacharparenleft}simp\ add{\isacharcolon}\ seq{\isacharunderscore}def{\isacharparenright}\isanewline
\ \ \isamarkupfalse%
\isacommand{also}\ \isamarkupfalse%
\isacommand{from}\ d\ \isamarkupfalse%
\isacommand{have}\ {\isachardoublequote}{\isasymdots}\ {\isacharequal}\ do\ {\isacharbraceleft}ret\ {\isacharparenleft}{\isacharparenright}{\isacharsemicolon}\ q{\isacharbraceright}{\isachardoublequote}\isanewline
\ \ \ \ \isamarkupfalse%
\isacommand{by}\ {\isacharparenleft}simp\ add{\isacharcolon}\ dis{\isacharunderscore}def\ seq{\isacharunderscore}def\ del{\isacharcolon}\ ret{\isacharunderscore}lunit{\isacharparenright}\isanewline
\ \ \isamarkupfalse%
\isacommand{also}\ \isamarkupfalse%
\isacommand{have}\ {\isachardoublequote}{\isasymdots}\ {\isacharequal}\ q{\isachardoublequote}\ \isamarkupfalse%
\isacommand{by}\ {\isacharparenleft}simp\ add{\isacharcolon}\ seq{\isacharunderscore}def{\isacharparenright}\isanewline
\ \ \isamarkupfalse%
\isacommand{finally}\ \isamarkupfalse%
\isacommand{show}\ {\isacharquery}thesis\ \isamarkupfalse%
\isacommand{{\isachardot}}\isanewline
\isamarkupfalse%
\isacommand{qed}\isamarkupfalse%
%
\begin{isamarkuptext}%
Essentially the same as \isa{dis{\isacharunderscore}left}, but expressed
  with binding.%
\end{isamarkuptext}%
\isamarkuptrue%
\isacommand{lemma}\ dis{\isacharunderscore}left{\isadigit{2}}{\isacharcolon}\ {\isachardoublequote}dis\ p\ {\isasymLongrightarrow}\ do\ {\isacharbraceleft}x{\isasymleftarrow}p{\isacharsemicolon}\ q{\isacharbraceright}\ {\isacharequal}\ q{\isachardoublequote}\isanewline
\isamarkupfalse%
\isacommand{proof}\ {\isacharminus}\isanewline
\ \ \isamarkupfalse%
\isacommand{assume}\ a{\isacharcolon}\ {\isachardoublequote}dis\ p{\isachardoublequote}\isanewline
\ \ \isamarkupfalse%
\isacommand{have}\ {\isachardoublequote}do\ {\isacharbraceleft}x{\isasymleftarrow}p{\isacharsemicolon}\ q{\isacharbraceright}\ {\isacharequal}\ do\ {\isacharbraceleft}p{\isacharsemicolon}\ q{\isacharbraceright}{\isachardoublequote}\ \isamarkupfalse%
\isacommand{by}\ {\isacharparenleft}simp\ only{\isacharcolon}\ seq{\isacharunderscore}def{\isacharparenright}\isanewline
\ \ \isamarkupfalse%
\isacommand{also}\ \isamarkupfalse%
\isacommand{from}\ a\ \isamarkupfalse%
\isacommand{have}\ {\isachardoublequote}{\isasymdots}\ {\isacharequal}\ q{\isachardoublequote}\ \isamarkupfalse%
\isacommand{by}\ {\isacharparenleft}rule\ dis{\isacharunderscore}left{\isacharparenright}\isanewline
\ \ \isamarkupfalse%
\isacommand{finally}\ \isamarkupfalse%
\isacommand{show}\ {\isacharquery}thesis\ \isamarkupfalse%
\isacommand{{\isachardot}}\isanewline
\isamarkupfalse%
\isacommand{qed}\isamarkupfalse%
%
\begin{isamarkuptext}%
This is Lemma 4.22 of \cite{SchroederMossakowski:PDL} which allows us to insert or remove
  copies of \isa{cp} programs whose result values may be substituted for each other
  in the following program sequence \isa{r}.%
\end{isamarkuptext}%
\isamarkuptrue%
\isacommand{lemma}\ cp{\isacharunderscore}arb{\isacharcolon}\ {\isachardoublequote}cp\ p\ {\isasymLongrightarrow}\ do\ {\isacharbraceleft}x{\isasymleftarrow}p{\isacharsemicolon}\ y{\isasymleftarrow}p{\isacharsemicolon}\ r\ x\ y{\isacharbraceright}\ {\isacharequal}\ do\ {\isacharbraceleft}x{\isasymleftarrow}p{\isacharsemicolon}\ r\ x\ x{\isacharbraceright}{\isachardoublequote}\isanewline
\isamarkupfalse%
\isacommand{proof}\ {\isacharparenleft}unfold\ cp{\isacharunderscore}def{\isacharparenright}\isanewline
\ \ \isamarkupfalse%
\isacommand{assume}\ c{\isacharcolon}\ {\isachardoublequote}\ do\ {\isacharbraceleft}x{\isasymleftarrow}p{\isacharsemicolon}\ y{\isasymleftarrow}p{\isacharsemicolon}\ ret\ {\isacharparenleft}x{\isacharcomma}\ y{\isacharparenright}{\isacharbraceright}\ {\isacharequal}\ do\ {\isacharbraceleft}x{\isasymleftarrow}p{\isacharsemicolon}\ ret\ {\isacharparenleft}x{\isacharcomma}\ x{\isacharparenright}{\isacharbraceright}{\isachardoublequote}\isanewline
\ \ \isamarkupfalse%
\isacommand{have}\ {\isachardoublequote}do\ {\isacharbraceleft}x{\isasymleftarrow}p{\isacharsemicolon}\ y{\isasymleftarrow}p{\isacharsemicolon}\ r\ x\ y{\isacharbraceright}\ {\isacharequal}\ do\ {\isacharbraceleft}x{\isasymleftarrow}p{\isacharsemicolon}\ y{\isasymleftarrow}p{\isacharsemicolon}\ z{\isasymleftarrow}ret{\isacharparenleft}x{\isacharcomma}y{\isacharparenright}{\isacharsemicolon}\ r\ {\isacharparenleft}fst\ z{\isacharparenright}\ {\isacharparenleft}snd\ z{\isacharparenright}{\isacharbraceright}{\isachardoublequote}\isanewline
\ \ \ \ \isamarkupfalse%
\isacommand{by}\ {\isacharparenleft}simp{\isacharparenright}\isanewline
\ \ \isamarkupfalse%
\isacommand{also}\ \isamarkupfalse%
\isacommand{have}\ {\isachardoublequote}{\isasymdots}\ {\isacharequal}\ do\ {\isacharbraceleft}z{\isasymleftarrow}do\ {\isacharbraceleft}x{\isasymleftarrow}p{\isacharsemicolon}\ y{\isasymleftarrow}p{\isacharsemicolon}\ ret{\isacharparenleft}x{\isacharcomma}y{\isacharparenright}{\isacharbraceright}{\isacharsemicolon}\ r\ {\isacharparenleft}fst\ z{\isacharparenright}\ {\isacharparenleft}snd\ z{\isacharparenright}{\isacharbraceright}{\isachardoublequote}\isanewline
\ \ \ \ \isamarkupfalse%
\isacommand{by}\ {\isacharparenleft}simp\ add{\isacharcolon}\ mon{\isacharunderscore}ctr{\isacharparenright}\isanewline
\ \ \isamarkupfalse%
\isacommand{also}\ \isamarkupfalse%
\isacommand{from}\ c\ \isamarkupfalse%
\isacommand{have}\ {\isachardoublequote}{\isasymdots}\ {\isacharequal}\ do\ {\isacharbraceleft}z{\isasymleftarrow}do\ {\isacharbraceleft}x{\isasymleftarrow}p{\isacharsemicolon}\ ret{\isacharparenleft}x{\isacharcomma}x{\isacharparenright}{\isacharbraceright}{\isacharsemicolon}\ r\ {\isacharparenleft}fst\ z{\isacharparenright}\ {\isacharparenleft}snd\ z{\isacharparenright}{\isacharbraceright}{\isachardoublequote}\isanewline
\ \ \ \ \isamarkupfalse%
\isacommand{by}\ simp\isanewline
\ \ \isamarkupfalse%
\isacommand{also}\ \isamarkupfalse%
\isacommand{have}\ {\isachardoublequote}{\isasymdots}\ {\isacharequal}\ do\ {\isacharbraceleft}x{\isasymleftarrow}p{\isacharsemicolon}\ z{\isasymleftarrow}ret\ {\isacharparenleft}x{\isacharcomma}x{\isacharparenright}{\isacharsemicolon}\ r\ {\isacharparenleft}fst\ z{\isacharparenright}\ {\isacharparenleft}snd\ z{\isacharparenright}{\isacharbraceright}{\isachardoublequote}\isanewline
\ \ \ \ \isamarkupfalse%
\isacommand{by}\ {\isacharparenleft}simp\ add{\isacharcolon}\ mon{\isacharunderscore}ctr{\isacharparenright}\isanewline
\ \ \isamarkupfalse%
\isacommand{also}\ \isamarkupfalse%
\isacommand{have}\ {\isachardoublequote}{\isasymdots}\ {\isacharequal}\ do\ {\isacharbraceleft}x{\isasymleftarrow}p{\isacharsemicolon}\ r\ x\ x{\isacharbraceright}{\isachardoublequote}\isanewline
\ \ \ \ \isamarkupfalse%
\isacommand{by}\ simp\isanewline
\ \ \isamarkupfalse%
\isacommand{finally}\ \isamarkupfalse%
\isacommand{show}\ {\isacharquery}thesis\ \isamarkupfalse%
\isacommand{{\isachardot}}\isanewline
\isamarkupfalse%
\isacommand{qed}\isamarkupfalse%
%
\begin{isamarkuptext}%
This is Lemma 4.23 of \cite{SchroederMossakowski:PDL}, asserting a weak composability of copyable programs.
  It is generally not the case that sequences of copyable programs constitute
  a copyable program.%
\end{isamarkuptext}%
\isamarkuptrue%
\isacommand{lemma}\ weak{\isacharunderscore}cp{\isacharunderscore}seq{\isacharcolon}\ {\isachardoublequote}cp\ p\ {\isasymLongrightarrow}\ cp\ {\isacharparenleft}do\ {\isacharbraceleft}x{\isasymleftarrow}p{\isacharsemicolon}\ ret\ {\isacharparenleft}f\ x{\isacharparenright}{\isacharbraceright}{\isacharparenright}{\isachardoublequote}\isanewline
\isamarkupfalse%
\isacommand{proof}\ {\isacharminus}\isanewline
\ \ \isamarkupfalse%
\isacommand{assume}\ c{\isacharcolon}\ {\isachardoublequote}cp\ p{\isachardoublequote}\isanewline
\ \ \isamarkupfalse%
\isacommand{let}\ {\isacharquery}q\ {\isacharequal}\ {\isachardoublequote}do\ {\isacharbraceleft}x{\isasymleftarrow}p{\isacharsemicolon}\ ret\ {\isacharparenleft}f\ x{\isacharparenright}{\isacharbraceright}{\isachardoublequote}\isanewline
\ \ \isamarkupfalse%
\isacommand{have}\ {\isachardoublequote}do\ {\isacharbraceleft}u{\isasymleftarrow}{\isacharquery}q{\isacharsemicolon}\ v{\isasymleftarrow}{\isacharquery}q{\isacharsemicolon}\ ret{\isacharparenleft}u{\isacharcomma}v{\isacharparenright}{\isacharbraceright}\ {\isacharequal}\ do\ {\isacharbraceleft}x{\isasymleftarrow}p{\isacharsemicolon}\ u{\isasymleftarrow}ret\ {\isacharparenleft}f\ x{\isacharparenright}{\isacharsemicolon}\ y{\isasymleftarrow}p{\isacharsemicolon}\ v{\isasymleftarrow}ret\ {\isacharparenleft}f\ y{\isacharparenright}{\isacharsemicolon}\ ret{\isacharparenleft}u{\isacharcomma}v{\isacharparenright}{\isacharbraceright}{\isachardoublequote}\isanewline
\ \ \ \ \isamarkupfalse%
\isacommand{by}\ {\isacharparenleft}simp\ add{\isacharcolon}\ mon{\isacharunderscore}ctr{\isacharparenright}\isanewline
\ \ \isamarkupfalse%
\isacommand{also}\ \isamarkupfalse%
\isacommand{have}\ {\isachardoublequote}{\isasymdots}\ {\isacharequal}\ do\ {\isacharbraceleft}x{\isasymleftarrow}p{\isacharsemicolon}\ y{\isasymleftarrow}p{\isacharsemicolon}\ ret\ {\isacharparenleft}f\ x{\isacharcomma}\ f\ y{\isacharparenright}{\isacharbraceright}{\isachardoublequote}\isanewline
\ \ \ \ \isamarkupfalse%
\isacommand{by}\ simp\isanewline
\ \ \isamarkupfalse%
\isacommand{also}\ \isamarkupfalse%
\isacommand{from}\ c\ \ \isamarkupfalse%
\isacommand{have}\ {\isachardoublequote}{\isasymdots}\ {\isacharequal}\ do\ {\isacharbraceleft}x{\isasymleftarrow}p{\isacharsemicolon}\ ret\ {\isacharparenleft}f\ x{\isacharcomma}\ f\ x{\isacharparenright}{\isacharbraceright}{\isachardoublequote}\isanewline
\ \ \ \ \isamarkupfalse%
\isacommand{by}\ {\isacharparenleft}simp\ add{\isacharcolon}\ cp{\isacharunderscore}arb{\isacharparenright}\isanewline
\ \ \isamarkupfalse%
\isacommand{also}\ \isamarkupfalse%
\isacommand{have}\ {\isachardoublequote}{\isasymdots}\ {\isacharequal}\ do\ {\isacharbraceleft}x{\isasymleftarrow}p{\isacharsemicolon}\ u{\isasymleftarrow}ret\ {\isacharparenleft}f\ x{\isacharparenright}{\isacharsemicolon}\ ret{\isacharparenleft}u{\isacharcomma}u{\isacharparenright}{\isacharbraceright}{\isachardoublequote}\isanewline
\ \ \ \ \isamarkupfalse%
\isacommand{by}\ simp\isanewline
\ \ \isamarkupfalse%
\isacommand{also}\ \isamarkupfalse%
\isacommand{have}\ {\isachardoublequote}{\isasymdots}\ {\isacharequal}\ do\ {\isacharbraceleft}u{\isasymleftarrow}{\isacharquery}q{\isacharsemicolon}\ ret{\isacharparenleft}u{\isacharcomma}u{\isacharparenright}{\isacharbraceright}{\isachardoublequote}\isanewline
\ \ \ \ \isamarkupfalse%
\isacommand{by}\ {\isacharparenleft}simp\ add{\isacharcolon}\ mon{\isacharunderscore}ctr{\isacharparenright}\isanewline
\ \ \isamarkupfalse%
\isacommand{finally}\ \isamarkupfalse%
\isacommand{show}\ {\isacharquery}thesis\ \isamarkupfalse%
\isacommand{by}\ {\isacharparenleft}simp\ add{\isacharcolon}\ cp{\isacharunderscore}def{\isacharparenright}\isanewline
\isamarkupfalse%
\isacommand{qed}\isamarkupfalse%
%
\begin{isamarkuptext}%
One can reduce the copyability of a program of a certain form to a simpler
  form.%
\end{isamarkuptext}%
\isamarkuptrue%
\isacommand{lemma}\ cp{\isacharunderscore}seq{\isacharunderscore}ret{\isacharcolon}\ {\isachardoublequote}cp\ {\isacharparenleft}do\ {\isacharbraceleft}x{\isasymleftarrow}p{\isacharsemicolon}\ y{\isasymleftarrow}q{\isacharsemicolon}\ ret{\isacharparenleft}x{\isacharcomma}y{\isacharparenright}{\isacharbraceright}{\isacharparenright}\ {\isasymLongrightarrow}\ cp\ {\isacharparenleft}do\ {\isacharbraceleft}x{\isasymleftarrow}p{\isacharsemicolon}\ y{\isasymleftarrow}q{\isacharsemicolon}\ ret\ {\isacharparenleft}f\ x\ y{\isacharparenright}{\isacharbraceright}{\isacharparenright}{\isachardoublequote}\isanewline
\isamarkupfalse%
\isacommand{proof}\ {\isacharminus}\isanewline
\ \ \isamarkupfalse%
\isacommand{assume}\ {\isachardoublequote}cp\ {\isacharparenleft}do\ {\isacharbraceleft}x{\isasymleftarrow}p{\isacharsemicolon}\ y{\isasymleftarrow}q{\isacharsemicolon}\ ret{\isacharparenleft}x{\isacharcomma}y{\isacharparenright}{\isacharbraceright}{\isacharparenright}{\isachardoublequote}\isanewline
\ \ \isamarkupfalse%
\isacommand{hence}\ c{\isacharcolon}\ {\isachardoublequote}cp\ {\isacharparenleft}do\ {\isacharbraceleft}u{\isasymleftarrow}do\ {\isacharbraceleft}x{\isasymleftarrow}p{\isacharsemicolon}\ y{\isasymleftarrow}q{\isacharsemicolon}\ ret{\isacharparenleft}x{\isacharcomma}y{\isacharparenright}{\isacharbraceright}{\isacharsemicolon}\ ret\ {\isacharparenleft}f\ {\isacharparenleft}fst\ u{\isacharparenright}\ {\isacharparenleft}snd\ u{\isacharparenright}{\isacharparenright}{\isacharbraceright}{\isacharparenright}{\isachardoublequote}\ \isanewline
\ \ \ \ \isamarkupfalse%
\isacommand{by}\ {\isacharparenleft}simp\ add{\isacharcolon}\ weak{\isacharunderscore}cp{\isacharunderscore}seq{\isacharparenright}\isanewline
\ \ \isamarkupfalse%
\isacommand{have}\ {\isachardoublequote}do\ {\isacharbraceleft}u{\isasymleftarrow}do\ {\isacharbraceleft}x{\isasymleftarrow}p{\isacharsemicolon}\ y{\isasymleftarrow}q{\isacharsemicolon}\ ret{\isacharparenleft}x{\isacharcomma}y{\isacharparenright}{\isacharbraceright}{\isacharsemicolon}\ ret\ {\isacharparenleft}f\ {\isacharparenleft}fst\ u{\isacharparenright}\ {\isacharparenleft}snd\ u{\isacharparenright}{\isacharparenright}{\isacharbraceright}\isanewline
\ \ \ \ \ \ \ \ {\isacharequal}\ do\ {\isacharbraceleft}x{\isasymleftarrow}p{\isacharsemicolon}\ y{\isasymleftarrow}q{\isacharsemicolon}\ ret\ {\isacharparenleft}f\ x\ y{\isacharparenright}{\isacharbraceright}{\isachardoublequote}\isanewline
\ \ \ \ \isamarkupfalse%
\isacommand{by}\ {\isacharparenleft}simp\ add{\isacharcolon}\ mon{\isacharunderscore}ctr{\isacharparenright}\isanewline
\ \ \isamarkupfalse%
\isacommand{with}\ c\ \isamarkupfalse%
\isacommand{show}\ {\isacharquery}thesis\ \isamarkupfalse%
\isacommand{by}\ simp\isanewline
\isamarkupfalse%
\isacommand{qed}\isamarkupfalse%
%
\begin{isamarkuptext}%
We also have a weak notion of stability under sequencing for \isa{dsef}
  programs.%
\end{isamarkuptext}%
\isamarkuptrue%
\isacommand{lemma}\ weak{\isacharunderscore}dis{\isacharunderscore}seq{\isacharcolon}\ {\isachardoublequote}dis\ p\ {\isasymLongrightarrow}\ dis\ {\isacharparenleft}do\ {\isacharbraceleft}x{\isasymleftarrow}p{\isacharsemicolon}\ ret\ {\isacharparenleft}f\ x{\isacharparenright}{\isacharbraceright}{\isacharparenright}{\isachardoublequote}\isanewline
\isamarkupfalse%
\isacommand{proof}\ {\isacharminus}\isanewline
\ \ \isamarkupfalse%
\isacommand{assume}\ d{\isacharcolon}\ {\isachardoublequote}dis\ p{\isachardoublequote}\isanewline
\ \ \isamarkupfalse%
\isacommand{have}\ {\isachardoublequote}do\ {\isacharbraceleft}z{\isasymleftarrow}do\ {\isacharbraceleft}x{\isasymleftarrow}p{\isacharsemicolon}\ ret\ {\isacharparenleft}f\ x{\isacharparenright}{\isacharbraceright}{\isacharsemicolon}\ ret\ {\isacharparenleft}{\isacharparenright}{\isacharbraceright}\ {\isacharequal}\ do\ {\isacharbraceleft}x{\isasymleftarrow}p{\isacharsemicolon}\ z{\isasymleftarrow}ret\ {\isacharparenleft}f\ x{\isacharparenright}{\isacharsemicolon}\ ret\ {\isacharparenleft}{\isacharparenright}{\isacharbraceright}{\isachardoublequote}\isanewline
\ \ \ \ \isamarkupfalse%
\isacommand{by}\ {\isacharparenleft}simp\ only{\isacharcolon}\ mon{\isacharunderscore}ctr{\isacharparenright}\isanewline
\ \ \isamarkupfalse%
\isacommand{also}\ \isamarkupfalse%
\isacommand{have}\ {\isachardoublequote}{\isasymdots}\ {\isacharequal}\ do\ {\isacharbraceleft}x{\isasymleftarrow}p{\isacharsemicolon}\ ret{\isacharparenleft}{\isacharparenright}{\isacharbraceright}{\isachardoublequote}\isanewline
\ \ \ \ \isamarkupfalse%
\isacommand{by}\ simp\isanewline
\ \ \isamarkupfalse%
\isacommand{also}\ \isamarkupfalse%
\isacommand{from}\ d\ \isamarkupfalse%
\isacommand{have}\ {\isachardoublequote}{\isasymdots}\ {\isacharequal}\ ret\ {\isacharparenleft}{\isacharparenright}{\isachardoublequote}\ \isamarkupfalse%
\isacommand{by}\ {\isacharparenleft}simp\ add{\isacharcolon}\ dis{\isacharunderscore}def{\isacharparenright}\isanewline
\ \ \isamarkupfalse%
\isacommand{finally}\ \isamarkupfalse%
\isacommand{show}\ {\isacharquery}thesis\ \isamarkupfalse%
\isacommand{by}\ {\isacharparenleft}simp\ add{\isacharcolon}\ dis{\isacharunderscore}def{\isacharparenright}\isanewline
\isamarkupfalse%
\isacommand{qed}\isamarkupfalse%
%
\begin{isamarkuptext}%
The following lemmas \isa{commute{\isacharunderscore}X{\isacharunderscore}Y} are proofs of the Propositions 
  4.24 of \cite{SchroederMossakowski:PDL} where \isa{X} is the respective premiss
  and \isa{Y} is the conclusion.
  \label{isa:commute-1-2}%
\end{isamarkuptext}%
\isamarkuptrue%
\isacommand{lemma}\ commute{\isacharunderscore}{\isadigit{1}}{\isacharunderscore}{\isadigit{2}}{\isacharcolon}\ {\isachardoublequote}{\isasymlbrakk}cp\ q{\isacharsemicolon}\ cp\ p{\isacharsemicolon}\ dis\ q{\isacharsemicolon}\ dis\ p{\isasymrbrakk}\ {\isasymLongrightarrow}\ cp\ {\isacharparenleft}do\ {\isacharbraceleft}x{\isasymleftarrow}p{\isacharsemicolon}\ y{\isasymleftarrow}q{\isacharsemicolon}\ ret{\isacharparenleft}x{\isacharcomma}y{\isacharparenright}{\isacharbraceright}{\isacharparenright}\isanewline
\ \ \ \ \ \ \ \ \ \ \ \ \ \ \ \ \ \ \ \ {\isasymLongrightarrow}\ do\ {\isacharbraceleft}x{\isasymleftarrow}p{\isacharsemicolon}\ y{\isasymleftarrow}q{\isacharsemicolon}\ ret{\isacharparenleft}x{\isacharcomma}y{\isacharparenright}{\isacharbraceright}\ {\isacharequal}\ do\ {\isacharbraceleft}y{\isasymleftarrow}q{\isacharsemicolon}\ x{\isasymleftarrow}p{\isacharsemicolon}\ ret{\isacharparenleft}x{\isacharcomma}y{\isacharparenright}{\isacharbraceright}{\isachardoublequote}\isanewline
\isamarkupfalse%
\isacommand{proof}\ {\isacharminus}\isanewline
\ \ \isamarkupfalse%
\isacommand{assume}\ a{\isacharcolon}\ {\isachardoublequote}cp\ q{\isachardoublequote}\ {\isachardoublequote}cp\ p{\isachardoublequote}\ {\isachardoublequote}dis\ q{\isachardoublequote}\ {\isachardoublequote}dis\ p{\isachardoublequote}\isanewline
\ \ \isamarkupfalse%
\isacommand{assume}\ c{\isacharcolon}\ {\isachardoublequote}cp\ {\isacharparenleft}do\ {\isacharbraceleft}x{\isasymleftarrow}p{\isacharsemicolon}\ y{\isasymleftarrow}q{\isacharsemicolon}\ ret{\isacharparenleft}x{\isacharcomma}y{\isacharparenright}{\isacharbraceright}{\isacharparenright}{\isachardoublequote}\isanewline
\ \ \isamarkupfalse%
\isacommand{let}\ {\isacharquery}s\ {\isacharequal}\ {\isachardoublequote}do\ {\isacharbraceleft}x{\isasymleftarrow}p{\isacharsemicolon}\ y{\isasymleftarrow}q{\isacharsemicolon}\ ret{\isacharparenleft}x{\isacharcomma}y{\isacharparenright}{\isacharbraceright}{\isachardoublequote}\isanewline
\ \ \isamarkupfalse%
\isacommand{have}\ {\isachardoublequote}{\isacharquery}s\ {\isacharequal}\ do\ {\isacharbraceleft}z{\isasymleftarrow}{\isacharquery}s{\isacharsemicolon}\ ret\ {\isacharparenleft}fst\ z{\isacharcomma}\ snd\ z{\isacharparenright}{\isacharbraceright}{\isachardoublequote}\ \isamarkupfalse%
\isacommand{by}\ simp\isanewline
\ \ \isamarkupfalse%
\isacommand{also}\ \isamarkupfalse%
\isacommand{from}\ c\ \isamarkupfalse%
\isacommand{have}\ {\isachardoublequote}{\isasymdots}\ {\isacharequal}\ do\ {\isacharbraceleft}w{\isasymleftarrow}{\isacharquery}s{\isacharsemicolon}\ z{\isasymleftarrow}{\isacharquery}s{\isacharsemicolon}\ ret\ {\isacharparenleft}fst\ z{\isacharcomma}\ snd\ w{\isacharparenright}{\isacharbraceright}{\isachardoublequote}\ \isamarkupfalse%
\isacommand{by}\ {\isacharparenleft}simp\ add{\isacharcolon}\ cp{\isacharunderscore}arb{\isacharparenright}\isanewline
\ \ \isamarkupfalse%
\isacommand{also}\ \isamarkupfalse%
\isacommand{from}\ a\ \isamarkupfalse%
\isacommand{have}\ {\isachardoublequote}{\isasymdots}\ {\isacharequal}\ do\ {\isacharbraceleft}v{\isasymleftarrow}q{\isacharsemicolon}\ x{\isasymleftarrow}p{\isacharsemicolon}\ ret{\isacharparenleft}x{\isacharcomma}v{\isacharparenright}{\isacharbraceright}{\isachardoublequote}\ \isamarkupfalse%
\isacommand{by}\ {\isacharparenleft}simp\ add{\isacharcolon}\ mon{\isacharunderscore}ctr\ dis{\isacharunderscore}left{\isadigit{2}}{\isacharparenright}\isanewline
\ \ \isamarkupfalse%
\isacommand{finally}\ \isamarkupfalse%
\isacommand{show}\ {\isacharquery}thesis\ \isamarkupfalse%
\isacommand{{\isachardot}}\isanewline
\isamarkupfalse%
\isacommand{qed}\isanewline
\isanewline
\isanewline
\isamarkupfalse%
\isacommand{lemma}\ commute{\isacharunderscore}{\isadigit{2}}{\isacharunderscore}{\isadigit{3}}{\isacharcolon}\ {\isachardoublequote}{\isasymlbrakk}cp\ q{\isacharsemicolon}\ cp\ p{\isacharsemicolon}\ dis\ q{\isacharsemicolon}\ dis\ p{\isasymrbrakk}\ {\isasymLongrightarrow}\ \isanewline
\ \ \ \ \ \ \ \ \ \ \ \ \ \ \ \ \ \ \ \ do\ {\isacharbraceleft}x{\isasymleftarrow}p{\isacharsemicolon}\ y{\isasymleftarrow}q{\isacharsemicolon}\ ret{\isacharparenleft}x{\isacharcomma}y{\isacharparenright}{\isacharbraceright}\ {\isacharequal}\ do\ {\isacharbraceleft}y{\isasymleftarrow}q{\isacharsemicolon}\ x{\isasymleftarrow}p{\isacharsemicolon}\ ret{\isacharparenleft}x{\isacharcomma}y{\isacharparenright}{\isacharbraceright}\ {\isasymLongrightarrow}\isanewline
\ \ \ \ \ \ \ \ \ \ \ \ \ \ \ \ \ \ \ \ {\isasymforall}r{\isachardot}\ do\ {\isacharbraceleft}x{\isasymleftarrow}p{\isacharsemicolon}\ y{\isasymleftarrow}q{\isacharsemicolon}\ r\ x\ y{\isacharbraceright}\ {\isacharequal}\ do\ {\isacharbraceleft}y{\isasymleftarrow}q{\isacharsemicolon}\ x{\isasymleftarrow}p{\isacharsemicolon}\ r\ x\ y{\isacharbraceright}{\isachardoublequote}\isanewline
\isamarkupfalse%
\isacommand{proof}\isanewline
\ \ \isamarkupfalse%
\isacommand{fix}\ r\isanewline
\ \ \isamarkupfalse%
\isacommand{assume}\ a{\isacharcolon}\ {\isachardoublequote}cp\ q{\isachardoublequote}\ {\isachardoublequote}cp\ p{\isachardoublequote}\ {\isachardoublequote}dis\ q{\isachardoublequote}\ {\isachardoublequote}dis\ p{\isachardoublequote}\isanewline
\ \ \isamarkupfalse%
\isacommand{assume}\ b{\isacharcolon}\ {\isachardoublequote}do\ {\isacharbraceleft}x{\isasymleftarrow}p{\isacharsemicolon}\ y{\isasymleftarrow}q{\isacharsemicolon}\ ret{\isacharparenleft}x{\isacharcomma}y{\isacharparenright}{\isacharbraceright}\ {\isacharequal}\ do\ {\isacharbraceleft}y{\isasymleftarrow}q{\isacharsemicolon}\ x{\isasymleftarrow}p{\isacharsemicolon}\ ret{\isacharparenleft}x{\isacharcomma}y{\isacharparenright}{\isacharbraceright}{\isachardoublequote}\isanewline
\ \ \isamarkupfalse%
\isacommand{have}\ {\isachardoublequote}do\ {\isacharbraceleft}x{\isasymleftarrow}p{\isacharsemicolon}\ y{\isasymleftarrow}q{\isacharsemicolon}\ r\ x\ y{\isacharbraceright}\ {\isacharequal}\ do\ {\isacharbraceleft}x{\isasymleftarrow}p{\isacharsemicolon}\ y{\isasymleftarrow}q{\isacharsemicolon}\ z{\isasymleftarrow}ret{\isacharparenleft}x{\isacharcomma}y{\isacharparenright}{\isacharsemicolon}\ r\ {\isacharparenleft}fst\ z{\isacharparenright}\ {\isacharparenleft}snd\ z{\isacharparenright}{\isacharbraceright}{\isachardoublequote}\isanewline
\ \ \ \ \isamarkupfalse%
\isacommand{by}\ simp\isanewline
\ \ \isamarkupfalse%
\isacommand{also}\ \isamarkupfalse%
\isacommand{have}\ {\isachardoublequote}{\isasymdots}\ {\isacharequal}\ do\ {\isacharbraceleft}z{\isasymleftarrow}do\ {\isacharbraceleft}x{\isasymleftarrow}p{\isacharsemicolon}\ y{\isasymleftarrow}q{\isacharsemicolon}\ ret{\isacharparenleft}x{\isacharcomma}y{\isacharparenright}{\isacharbraceright}{\isacharsemicolon}\ r\ {\isacharparenleft}fst\ z{\isacharparenright}\ {\isacharparenleft}snd\ z{\isacharparenright}{\isacharbraceright}{\isachardoublequote}\isanewline
\ \ \ \ \isamarkupfalse%
\isacommand{by}\ {\isacharparenleft}simp\ only{\isacharcolon}\ mon{\isacharunderscore}ctr{\isacharparenright}\isanewline
\ \ \isamarkupfalse%
\isacommand{also}\ \isamarkupfalse%
\isacommand{from}\ b\ \isamarkupfalse%
\isacommand{have}\ {\isachardoublequote}{\isasymdots}\ {\isacharequal}\ do\ {\isacharbraceleft}z{\isasymleftarrow}do\ {\isacharbraceleft}y{\isasymleftarrow}q{\isacharsemicolon}\ x{\isasymleftarrow}p{\isacharsemicolon}\ ret{\isacharparenleft}x{\isacharcomma}y{\isacharparenright}{\isacharbraceright}{\isacharsemicolon}\ r\ {\isacharparenleft}fst\ z{\isacharparenright}\ {\isacharparenleft}snd\ z{\isacharparenright}{\isacharbraceright}{\isachardoublequote}\isanewline
\ \ \ \ \isamarkupfalse%
\isacommand{by}\ simp\isanewline
\ \ \isamarkupfalse%
\isacommand{also}\ \isamarkupfalse%
\isacommand{have}\ {\isachardoublequote}{\isasymdots}\ {\isacharequal}\ do\ {\isacharbraceleft}y{\isasymleftarrow}q{\isacharsemicolon}\ x{\isasymleftarrow}p{\isacharsemicolon}\ r\ x\ y{\isacharbraceright}{\isachardoublequote}\ \isamarkupfalse%
\isacommand{by}\ {\isacharparenleft}simp\ add{\isacharcolon}\ mon{\isacharunderscore}ctr{\isacharparenright}\isanewline
\ \ \isamarkupfalse%
\isacommand{finally}\ \isamarkupfalse%
\isacommand{show}\ {\isachardoublequote}do\ {\isacharbraceleft}x{\isasymleftarrow}p{\isacharsemicolon}\ y{\isasymleftarrow}q{\isacharsemicolon}\ r\ x\ y{\isacharbraceright}\ {\isacharequal}\ do\ {\isacharbraceleft}y{\isasymleftarrow}q{\isacharsemicolon}\ x{\isasymleftarrow}p{\isacharsemicolon}\ r\ x\ y{\isacharbraceright}{\isachardoublequote}\ \isamarkupfalse%
\isacommand{{\isachardot}}\isanewline
\isamarkupfalse%
\isacommand{qed}\isanewline
\isamarkupfalse%
\isamarkupfalse%
\isamarkupfalse%
%
\begin{isamarkuptext}%
In this case, type annotations are necessary, since we cannot
  quantify over types of programs. The type for \isa{r} given here
  is precisely what is needed for the proof to go through.
  \label{isa:commute-3-1}%
\end{isamarkuptext}%
\isamarkuptrue%
\isacommand{lemma}\ commute{\isacharunderscore}{\isadigit{3}}{\isacharunderscore}{\isadigit{1}}{\isacharcolon}\ {\isachardoublequote}{\isasymlbrakk}cp\ q{\isacharsemicolon}\ cp\ p{\isacharsemicolon}\ dis\ q{\isacharsemicolon}\ dis\ p{\isasymrbrakk}\ {\isasymLongrightarrow}\isanewline
\ \ \ \ \ \ \ \ \ \ \ \ \ \ \ \ \ \ \ \ {\isasymforall}r{\isacharcolon}{\isacharcolon}{\isacharprime}a\ {\isasymRightarrow}\ {\isacharprime}b\ {\isasymRightarrow}\ {\isacharparenleft}{\isacharparenleft}{\isacharprime}a{\isacharasterisk}{\isacharprime}b{\isacharparenright}{\isacharasterisk}{\isacharparenleft}{\isacharprime}a{\isacharasterisk}{\isacharprime}b{\isacharparenright}{\isacharparenright}\ T{\isachardot}\isanewline
\ \ \ \ \ \ \ \ \ \ \ \ \ \ \ \ \ \ \ \ \ \ do\ {\isacharbraceleft}x{\isasymleftarrow}p{\isacharsemicolon}\ y{\isasymleftarrow}q{\isacharsemicolon}\ r\ x\ y{\isacharbraceright}\ {\isacharequal}\ do\ {\isacharbraceleft}y{\isasymleftarrow}q{\isacharsemicolon}\ x{\isasymleftarrow}p{\isacharsemicolon}\ r\ x\ y{\isacharbraceright}\ {\isasymLongrightarrow}\isanewline
\ \ \ \ \ \ \ \ \ \ \ \ \ \ \ \ \ \ \ \ cp\ {\isacharparenleft}do\ {\isacharbraceleft}x{\isasymleftarrow}p{\isacharsemicolon}\ y{\isasymleftarrow}q{\isacharsemicolon}\ ret{\isacharparenleft}x{\isacharcomma}y{\isacharparenright}{\isacharcolon}{\isacharcolon}{\isacharparenleft}{\isacharprime}a\ {\isacharasterisk}\ {\isacharprime}b{\isacharparenright}\ T{\isacharbraceright}{\isacharparenright}{\isachardoublequote}\isanewline
\isamarkupfalse%
\isacommand{proof}\ {\isacharminus}\isanewline
\ \ \isamarkupfalse%
\isacommand{let}\ {\isacharquery}s\ {\isacharequal}\ {\isachardoublequote}do\ {\isacharbraceleft}x{\isasymleftarrow}p{\isacharsemicolon}\ y{\isasymleftarrow}q{\isacharsemicolon}\ ret{\isacharparenleft}x{\isacharcomma}y{\isacharparenright}{\isacharbraceright}{\isachardoublequote}\isanewline
\ \ \isamarkupfalse%
\isacommand{assume}\ a{\isacharcolon}\ {\isachardoublequote}cp\ q{\isachardoublequote}\ {\isachardoublequote}cp\ p{\isachardoublequote}\ {\isachardoublequote}dis\ q{\isachardoublequote}\ {\isachardoublequote}dis\ p{\isachardoublequote}\isanewline
\ \ \isamarkupfalse%
\isacommand{assume}\ c{\isacharcolon}\ {\isachardoublequote}{\isasymforall}r{\isacharcolon}{\isacharcolon}{\isacharprime}a\ {\isasymRightarrow}\ {\isacharprime}b\ {\isasymRightarrow}\ {\isacharparenleft}{\isacharparenleft}{\isacharprime}a{\isacharasterisk}{\isacharprime}b{\isacharparenright}{\isacharasterisk}{\isacharparenleft}{\isacharprime}a{\isacharasterisk}{\isacharprime}b{\isacharparenright}{\isacharparenright}\ T{\isachardot}\ \isanewline
\ \ \ \ \ \ \ \ \ \ \ \ \ \ \ \ do\ {\isacharbraceleft}x{\isasymleftarrow}p{\isacharsemicolon}\ y{\isasymleftarrow}q{\isacharsemicolon}\ r\ x\ y{\isacharbraceright}\ {\isacharequal}\ do\ {\isacharbraceleft}y{\isasymleftarrow}q{\isacharsemicolon}\ x{\isasymleftarrow}p{\isacharsemicolon}\ r\ x\ y{\isacharbraceright}{\isachardoublequote}\isanewline
\ \ \isamarkupfalse%
\isacommand{have}\ {\isachardoublequote}do\ {\isacharbraceleft}w{\isasymleftarrow}{\isacharquery}s{\isacharsemicolon}\ z{\isasymleftarrow}{\isacharquery}s{\isacharsemicolon}\ ret\ {\isacharparenleft}w{\isacharcomma}z{\isacharparenright}{\isacharbraceright}\ {\isacharequal}\ do\ {\isacharbraceleft}u{\isasymleftarrow}p{\isacharsemicolon}\ v{\isasymleftarrow}q{\isacharsemicolon}\ x{\isasymleftarrow}p{\isacharsemicolon}\ y{\isasymleftarrow}q{\isacharsemicolon}\ ret{\isacharparenleft}{\isacharparenleft}u{\isacharcomma}v{\isacharparenright}{\isacharcomma}{\isacharparenleft}x{\isacharcomma}y{\isacharparenright}{\isacharparenright}{\isacharbraceright}{\isachardoublequote}\isanewline
\ \ \ \ \isamarkupfalse%
\isacommand{by}\ {\isacharparenleft}simp\ add{\isacharcolon}\ mon{\isacharunderscore}ctr{\isacharparenright}\isanewline
\ \ \isamarkupfalse%
\isacommand{also}\ \isamarkupfalse%
\isacommand{from}\ c\ \isamarkupfalse%
\isacommand{have}\ {\isachardoublequote}{\isasymdots}\ {\isacharequal}\ do\ {\isacharbraceleft}u{\isasymleftarrow}p{\isacharsemicolon}\ x{\isasymleftarrow}p{\isacharsemicolon}\ v{\isasymleftarrow}q{\isacharsemicolon}\ y{\isasymleftarrow}q{\isacharsemicolon}\ ret{\isacharparenleft}{\isacharparenleft}u{\isacharcomma}v{\isacharparenright}{\isacharcomma}{\isacharparenleft}x{\isacharcomma}y{\isacharparenright}{\isacharparenright}{\isacharbraceright}{\isachardoublequote}\ \isamarkupfalse%
\isacommand{by}\ simp\isanewline
\ \ \isamarkupfalse%
\isacommand{also}\ \isamarkupfalse%
\isacommand{from}\ a\ \isamarkupfalse%
\isacommand{have}\ {\isachardoublequote}{\isasymdots}\ {\isacharequal}\ do\ {\isacharbraceleft}u{\isasymleftarrow}p{\isacharsemicolon}\ v{\isasymleftarrow}q{\isacharsemicolon}\ ret{\isacharparenleft}{\isacharparenleft}u{\isacharcomma}v{\isacharparenright}{\isacharcomma}{\isacharparenleft}u{\isacharcomma}v{\isacharparenright}{\isacharparenright}{\isacharbraceright}{\isachardoublequote}\ \isamarkupfalse%
\isacommand{by}\ {\isacharparenleft}simp\ only{\isacharcolon}\ cp{\isacharunderscore}arb{\isacharparenright}\isanewline
\ \ \isamarkupfalse%
\isacommand{also}\ \isamarkupfalse%
\isacommand{have}\ {\isachardoublequote}{\isasymdots}\ {\isacharequal}\ do\ {\isacharbraceleft}w{\isasymleftarrow}{\isacharquery}s{\isacharsemicolon}\ ret{\isacharparenleft}w{\isacharcomma}w{\isacharparenright}{\isacharbraceright}{\isachardoublequote}\ \isamarkupfalse%
\isacommand{by}\ {\isacharparenleft}simp\ add{\isacharcolon}mon{\isacharunderscore}ctr{\isacharparenright}\isanewline
\ \ \isamarkupfalse%
\isacommand{finally}\ \isamarkupfalse%
\isacommand{show}\ {\isacharquery}thesis\ \isamarkupfalse%
\isacommand{by}\ {\isacharparenleft}simp\ add{\isacharcolon}\ cp{\isacharunderscore}def{\isacharparenright}\isanewline
\isamarkupfalse%
\isacommand{qed}\isanewline
\isanewline
\isanewline
\isamarkupfalse%
\isacommand{lemma}\ commute{\isacharunderscore}{\isadigit{1}}{\isacharunderscore}{\isadigit{3}}{\isacharcolon}\ {\isachardoublequote}{\isasymlbrakk}cp\ q{\isacharsemicolon}\ cp\ p{\isacharsemicolon}\ dis\ q{\isacharsemicolon}\ dis\ p{\isasymrbrakk}\ {\isasymLongrightarrow}\isanewline
\ \ \ \ \ \ \ \ \ \ \ \ \ \ \ \ \ \ \ \ cp\ {\isacharparenleft}do\ {\isacharbraceleft}x{\isasymleftarrow}p{\isacharsemicolon}\ y{\isasymleftarrow}q{\isacharsemicolon}\ ret{\isacharparenleft}x{\isacharcomma}y{\isacharparenright}{\isacharbraceright}{\isacharparenright}\ {\isasymLongrightarrow}\isanewline
\ \ \ \ \ \ \ \ \ \ \ \ \ \ \ \ \ \ \ \ {\isasymforall}r{\isachardot}\ do\ {\isacharbraceleft}x{\isasymleftarrow}p{\isacharsemicolon}\ y{\isasymleftarrow}q{\isacharsemicolon}\ r\ x\ y{\isacharbraceright}\ {\isacharequal}\ do\ {\isacharbraceleft}y{\isasymleftarrow}q{\isacharsemicolon}\ x{\isasymleftarrow}p{\isacharsemicolon}\ r\ x\ y{\isacharbraceright}{\isachardoublequote}\isanewline
\ \ %
\isamarkupcmt{More or less just transitivity of implication%
}
\isanewline
\ \ \isamarkupfalse%
\isacommand{apply}{\isacharparenleft}rule\ commute{\isacharunderscore}{\isadigit{2}}{\isacharunderscore}{\isadigit{3}}{\isacharparenright}\isanewline
\ \ \isamarkupfalse%
\isacommand{apply}{\isacharparenleft}simp{\isacharunderscore}all{\isacharparenright}\isanewline
\ \ \isamarkupfalse%
\isacommand{apply}{\isacharparenleft}rule\ commute{\isacharunderscore}{\isadigit{1}}{\isacharunderscore}{\isadigit{2}}{\isacharparenright}\isanewline
\ \ \isamarkupfalse%
\isacommand{apply}{\isacharparenleft}simp{\isacharunderscore}all{\isacharparenright}\isanewline
\isamarkupfalse%
\isacommand{done}\isamarkupfalse%
%
\begin{isamarkuptext}%
This weird axiom is needed to obtain the general commutativity of 
  a discardable and copyable program from its commuting  with all 
  \isa{bool}-valued programs.
  \label{isa:commute-bool-arb}%
\end{isamarkuptext}%
\isamarkuptrue%
\isacommand{axioms}\isanewline
\ \ commute{\isacharunderscore}bool{\isacharunderscore}arb{\isacharcolon}\ {\isachardoublequote}{\isasymlbrakk}dis\ p{\isacharsemicolon}\ cp\ p{\isacharsemicolon}\ {\isasymforall}q{\isadigit{1}}{\isacharcolon}{\isacharcolon}bool\ T{\isachardot}\ cp{\isacharparenleft}q{\isadigit{1}}{\isacharparenright}\ {\isasymand}\ dis{\isacharparenleft}q{\isadigit{1}}{\isacharparenright}\ {\isasymlongrightarrow}\ \isanewline
\ \ \ \ \ \ \ \ \ \ \ \ \ \ \ \ \ \ \ \ \ \ \ \ \ \ \ \ \ \ \ \ \ \ \ \ \ \ \ \ cp{\isacharparenleft}do\ {\isacharbraceleft}x{\isasymleftarrow}p{\isacharsemicolon}\ y{\isasymleftarrow}q{\isadigit{1}}{\isacharsemicolon}\ ret{\isacharparenleft}x{\isacharcomma}y{\isacharparenright}{\isacharbraceright}{\isacharparenright}{\isasymrbrakk}\ {\isasymLongrightarrow}\isanewline
\ \ \ \ \ \ \ \ \ \ \ \ \ \ \ \ \ \ \ {\isacharparenleft}{\isasymforall}q{\isachardot}\ cp{\isacharparenleft}q{\isacharparenright}\ {\isasymand}\ dis{\isacharparenleft}q{\isacharparenright}\ {\isasymlongrightarrow}\ cp{\isacharparenleft}do\ {\isacharbraceleft}x{\isasymleftarrow}p{\isacharsemicolon}\ y{\isasymleftarrow}q{\isacharsemicolon}\ ret{\isacharparenleft}x{\isacharcomma}y{\isacharparenright}{\isacharbraceright}{\isacharparenright}{\isacharparenright}{\isachardoublequote}\isamarkupfalse%
%
\begin{isamarkuptext}%
In order to introduce the subtype of \isa{dsef} programs, we
  must prove it is not empty.%
\end{isamarkuptext}%
\isamarkuptrue%
\isacommand{theorem}\ dsef{\isacharunderscore}ret\ {\isacharbrackleft}simp{\isacharbrackright}{\isacharcolon}\ {\isachardoublequote}dsef\ {\isacharparenleft}ret\ x{\isacharparenright}{\isachardoublequote}\isanewline
\isamarkupfalse%
\isacommand{proof}\ {\isacharparenleft}unfold\ dsef{\isacharunderscore}def{\isacharparenright}\isanewline
\ \ \isamarkupfalse%
\isacommand{have}\ {\isachardoublequote}cp\ {\isacharparenleft}ret\ x{\isacharparenright}{\isachardoublequote}\ \isamarkupfalse%
\isacommand{by}\ {\isacharparenleft}simp\ add{\isacharcolon}\ cp{\isacharunderscore}def{\isacharparenright}\isanewline
\ \ \isamarkupfalse%
\isacommand{moreover}\ \isamarkupfalse%
\isacommand{have}\ {\isachardoublequote}dis\ {\isacharparenleft}ret\ x{\isacharparenright}{\isachardoublequote}\ \isamarkupfalse%
\isacommand{by}\ {\isacharparenleft}simp\ add{\isacharcolon}\ dis{\isacharunderscore}def{\isacharparenright}\isanewline
\ \ \isamarkupfalse%
\isacommand{moreover}\ \isamarkupfalse%
\isacommand{have}\ {\isachardoublequote}{\isacharparenleft}{\isasymforall}q{\isachardot}\ cp\ q\ {\isasymand}\ dis\ q\ {\isasymlongrightarrow}\ cp\ {\isacharparenleft}do\ {\isacharbraceleft}x{\isasymleftarrow}ret\ x{\isacharsemicolon}\ y{\isasymleftarrow}q{\isacharsemicolon}\ ret\ {\isacharparenleft}x{\isacharcomma}\ y{\isacharparenright}{\isacharbraceright}{\isacharparenright}{\isacharparenright}{\isachardoublequote}\isanewline
\ \ \ \ \isamarkupfalse%
\isacommand{by}\ {\isacharparenleft}simp\ add{\isacharcolon}\ weak{\isacharunderscore}cp{\isacharunderscore}seq{\isacharparenright}\isanewline
\ \ \isamarkupfalse%
\isacommand{ultimately}\ \isamarkupfalse%
\isacommand{show}\ {\isachardoublequote}cp\ {\isacharparenleft}ret\ x{\isacharparenright}\ {\isasymand}\ dis\ {\isacharparenleft}ret\ x{\isacharparenright}\ {\isasymand}\ \isanewline
\ \ \ \ \ \ \ \ \ \ \ \ \ \ \ \ \ \ \ {\isacharparenleft}{\isasymforall}q{\isachardot}\ cp\ q\ {\isasymand}\ dis\ q\ {\isasymlongrightarrow}\ cp\ {\isacharparenleft}do\ {\isacharbraceleft}x{\isasymleftarrow}ret\ x{\isacharsemicolon}\ y{\isasymleftarrow}q{\isacharsemicolon}\ ret\ {\isacharparenleft}x{\isacharcomma}\ y{\isacharparenright}{\isacharbraceright}{\isacharparenright}{\isacharparenright}{\isachardoublequote}\isanewline
\ \ \ \ \isamarkupfalse%
\isacommand{by}\ blast\isanewline
\isamarkupfalse%
\isacommand{qed}\isamarkupfalse%
%
\isamarkupsubsection{Introducing the Subtype of \emph{dsef} Programs%
}
\isamarkuptrue%
%
\begin{isamarkuptext}%
Introducing the subtype \isa{{\isacharprime}a\ D} of \isa{{\isacharprime}a\ T} comprising the 
\isa{dsef} programs;
  since Isabelle lacks true subtyping, it is simply declared as a new type
  with coercion functions 
    \isa{Rep{\isacharunderscore}Dsef\ {\isacharcolon}{\isacharcolon}\ {\isacharprime}a\ D\ {\isasymRightarrow}\ {\isacharprime}a\ T}
  and 
    \isa{Abs{\isacharunderscore}Dsef\ {\isacharcolon}{\isacharcolon}\ {\isacharprime}a\ T\ {\isasymRightarrow}\ {\isacharprime}a\ D}
  where \isa{Abs{\isacharunderscore}Dsef\ p} is of course only sensibly defined if \isa{dsef\ p}
   holds.
  \label{isa:intro-dsef}%
\end{isamarkuptext}%
\isamarkuptrue%
\isacommand{typedef}\ {\isacharparenleft}Dsef{\isacharparenright}\ {\isacharparenleft}{\isacharprime}a{\isacharparenright}\ D\ {\isacharequal}\ {\isachardoublequote}{\isacharbraceleft}p{\isacharcolon}{\isacharcolon}{\isacharprime}a\ T{\isachardot}\ dsef\ p{\isacharbraceright}{\isachardoublequote}\isanewline
\ \ \isamarkupfalse%
\isacommand{apply}{\isacharparenleft}rule\ exI{\isacharbrackleft}of\ {\isacharunderscore}\ {\isachardoublequote}ret\ x{\isachardoublequote}{\isacharbrackright}{\isacharparenright}\isanewline
\ \ \isamarkupfalse%
\isacommand{apply}{\isacharparenleft}blast\ intro{\isacharcolon}\ dsef{\isacharunderscore}ret{\isacharparenright}\isanewline
\isamarkupfalse%
\isacommand{done}\isamarkupfalse%
%
\begin{isamarkuptext}%
Minimizing the clutter caused by \isa{Abs{\isacharunderscore}Dsef} and \isa{Rep{\isacharunderscore}Dsef}.%
\end{isamarkuptext}%
\isamarkuptrue%
\isacommand{syntax}\isanewline
\ \ {\isachardoublequote}{\isacharunderscore}absdsef{\isachardoublequote}\ \ \ \ \ \ \ \ \ \ \ \ {\isacharcolon}{\isacharcolon}\ {\isachardoublequote}{\isacharprime}a\ T\ {\isasymRightarrow}\ {\isacharprime}a\ D{\isachardoublequote}\ \ \ \ \ \ {\isacharparenleft}{\isachardoublequote}{\isasymUp}\ {\isacharunderscore}{\isachardoublequote}\ {\isacharbrackleft}{\isadigit{2}}{\isadigit{0}}{\isadigit{0}}{\isacharbrackright}\ {\isadigit{1}}{\isadigit{9}}{\isadigit{9}}{\isacharparenright}\isanewline
\ \ {\isachardoublequote}{\isacharunderscore}repdsef{\isachardoublequote}\ \ \ \ \ \ \ \ \ \ \ \ {\isacharcolon}{\isacharcolon}\ {\isachardoublequote}{\isacharprime}a\ D\ {\isasymRightarrow}\ {\isacharprime}a\ T{\isachardoublequote}\ \ \ \ \ \ {\isacharparenleft}{\isachardoublequote}{\isasymDown}\ {\isacharunderscore}{\isachardoublequote}\ {\isacharbrackleft}{\isadigit{2}}{\isadigit{0}}{\isadigit{0}}{\isacharbrackright}\ {\isadigit{1}}{\isadigit{9}}{\isadigit{9}}{\isacharparenright}\isanewline
\isamarkupfalse%
\isacommand{translations}\isanewline
\ \ {\isachardoublequote}{\isasymUp}\ p{\isachardoublequote}\ \ \ \ \ \ {\isasymrightleftharpoons}\ \ \ \ \ {\isachardoublequote}Abs{\isacharunderscore}Dsef\ p{\isachardoublequote}\isanewline
\ \ {\isachardoublequote}{\isasymDown}\ p{\isachardoublequote}\ \ \ \ \ \ {\isasymrightleftharpoons}\ \ \ \ \ {\isachardoublequote}Rep{\isacharunderscore}Dsef\ p{\isachardoublequote}\isamarkupfalse%
%
\begin{isamarkuptext}%
All representatives of terms of type \isa{{\isacharprime}a\ D} are dsef and thus in 
        particular discardable and copyable.%
\end{isamarkuptext}%
\isamarkuptrue%
\isacommand{lemma}\ dsef{\isacharunderscore}Rep{\isacharunderscore}Dsef\ {\isacharbrackleft}simp{\isacharbrackright}{\isacharcolon}\ {\isachardoublequote}dsef\ {\isacharparenleft}{\isasymDown}\ a{\isacharparenright}{\isachardoublequote}\isanewline
\isamarkupfalse%
\isacommand{proof}\ {\isacharparenleft}induct\ a\ rule{\isacharcolon}\ Abs{\isacharunderscore}Dsef{\isacharunderscore}induct{\isacharparenright}\isanewline
\ \ \isamarkupfalse%
\isacommand{fix}\ a\isanewline
\ \ \isamarkupfalse%
\isacommand{assume}\ {\isachardoublequote}a\ {\isacharcolon}\ Dsef{\isachardoublequote}\isanewline
\ \ \isamarkupfalse%
\isacommand{thus}\ {\isachardoublequote}dsef\ {\isacharparenleft}{\isasymDown}\ {\isacharparenleft}{\isasymUp}\ a{\isacharparenright}{\isacharparenright}{\isachardoublequote}\isanewline
\ \ \ \ \isamarkupfalse%
\isacommand{by}\ {\isacharparenleft}simp\ add{\isacharcolon}\ Abs{\isacharunderscore}Dsef{\isacharunderscore}inverse\ Dsef{\isacharunderscore}def{\isacharparenright}\isanewline
\isamarkupfalse%
\isacommand{qed}\isanewline
\isanewline
\isamarkupfalse%
\isacommand{lemma}\ dis{\isacharunderscore}Rep{\isacharunderscore}Dsef{\isacharcolon}\ {\isachardoublequote}dis\ {\isacharparenleft}{\isasymDown}\ a{\isacharparenright}{\isachardoublequote}\isanewline
\ \ \isamarkupfalse%
\isacommand{apply}{\isacharparenleft}insert\ dsef{\isacharunderscore}Rep{\isacharunderscore}Dsef{\isacharbrackleft}of\ a{\isacharbrackright}{\isacharparenright}\isanewline
\ \ \isamarkupfalse%
\isacommand{apply}{\isacharparenleft}unfold\ dsef{\isacharunderscore}def{\isacharparenright}\isanewline
\ \ \isamarkupfalse%
\isacommand{apply}{\isacharparenleft}blast{\isacharparenright}\isanewline
\isamarkupfalse%
\isacommand{done}\isanewline
\isanewline
\isamarkupfalse%
\isacommand{lemma}\ cp{\isacharunderscore}Rep{\isacharunderscore}Dsef{\isacharcolon}\ {\isachardoublequote}cp\ {\isacharparenleft}{\isasymDown}\ a{\isacharparenright}{\isachardoublequote}\isanewline
\ \ \isamarkupfalse%
\isacommand{apply}{\isacharparenleft}insert\ dsef{\isacharunderscore}Rep{\isacharunderscore}Dsef{\isacharbrackleft}of\ a{\isacharbrackright}{\isacharparenright}\isanewline
\ \ \isamarkupfalse%
\isacommand{apply}{\isacharparenleft}unfold\ dsef{\isacharunderscore}def{\isacharparenright}\isanewline
\ \ \isamarkupfalse%
\isacommand{apply}{\isacharparenleft}blast{\isacharparenright}\isanewline
\isamarkupfalse%
\isacommand{done}\isamarkupfalse%
%
\begin{isamarkuptext}%
\textbf{Convention:} We will denote functions in \isa{D} that are simply
abstracted versions
  of appropriate functions in \isa{T} by the same name with the first
  letter capitalised.%
\end{isamarkuptext}%
\isamarkuptrue%
\isacommand{constdefs}\ \isanewline
\ \ Ret\ {\isacharcolon}{\isacharcolon}\ {\isachardoublequote}{\isacharprime}a\ {\isasymRightarrow}\ {\isacharprime}a\ D{\isachardoublequote}\isanewline
\ \ {\isachardoublequote}Ret\ x\ {\isasymequiv}\ {\isasymUp}\ {\isacharparenleft}ret\ x{\isacharparenright}{\isachardoublequote}\isanewline
\isanewline
\isanewline
\isamarkupfalse%
\isacommand{lemma}\ Ret{\isacharunderscore}ret{\isacharcolon}\ {\isachardoublequote}{\isasymDown}\ {\isacharparenleft}Ret\ x{\isacharparenright}\ {\isacharequal}\ ret\ x{\isachardoublequote}\isanewline
\isamarkupfalse%
\isacommand{proof}\ {\isacharminus}\isanewline
\ \ \isamarkupfalse%
\isacommand{have}\ {\isachardoublequote}{\isasymDown}\ {\isacharparenleft}Ret\ x{\isacharparenright}\ {\isacharequal}\ {\isasymDown}\ {\isacharparenleft}{\isasymUp}\ {\isacharparenleft}ret\ x{\isacharparenright}{\isacharparenright}{\isachardoublequote}\ \isamarkupfalse%
\isacommand{by}\ {\isacharparenleft}simp\ only{\isacharcolon}\ Ret{\isacharunderscore}def{\isacharparenright}\isanewline
\ \ \isamarkupfalse%
\isacommand{also}\ \isamarkupfalse%
\isacommand{have}\ {\isachardoublequote}{\isasymdots}\ {\isacharequal}\ ret\ x{\isachardoublequote}\ \isamarkupfalse%
\isacommand{by}\ {\isacharparenleft}simp\ add{\isacharcolon}\ Dsef{\isacharunderscore}def\ Abs{\isacharunderscore}Dsef{\isacharunderscore}inverse{\isacharparenright}\isanewline
\ \ \isamarkupfalse%
\isacommand{finally}\ \isamarkupfalse%
\isacommand{show}\ {\isacharquery}thesis\ \isamarkupfalse%
\isacommand{{\isachardot}}\isanewline
\isamarkupfalse%
\isacommand{qed}\isamarkupfalse%
%
\begin{isamarkuptext}%
Lifting operations will allow us to introduce monadic connectives \isa{{\isasymand}{\isacharcomma}\ {\isasymor}}, etc.
  by simply lifting the HOL ones. Theorem \isa{{\isachardoublequote}dsef{\isacharunderscore}ret{\isachardoublequote}} will
  assert these to be \isa{dsef} (see below).  
  \label{isa:lift-op}%
\end{isamarkuptext}%
\isamarkuptrue%
\isacommand{constdefs}\isanewline
\ liftM\ {\isacharcolon}{\isacharcolon}\ {\isachardoublequote}{\isacharbrackleft}{\isacharprime}a\ {\isasymRightarrow}\ {\isacharprime}b{\isacharcomma}\ {\isacharprime}a\ T{\isacharbrackright}\ {\isasymRightarrow}\ {\isacharprime}b\ T{\isachardoublequote}\isanewline
\ {\isachardoublequote}liftM\ f\ p\ {\isasymequiv}\ do\ {\isacharbraceleft}x\ {\isasymleftarrow}\ p{\isacharsemicolon}\ ret\ {\isacharparenleft}f\ x{\isacharparenright}{\isacharbraceright}{\isachardoublequote}\isanewline
\ liftM{\isadigit{2}}\ {\isacharcolon}{\isacharcolon}\ {\isachardoublequote}{\isacharbrackleft}{\isacharprime}a\ {\isasymRightarrow}\ {\isacharprime}b\ {\isasymRightarrow}\ {\isacharprime}c{\isacharcomma}\ {\isacharprime}a\ T{\isacharcomma}\ {\isacharprime}b\ T{\isacharbrackright}\ {\isasymRightarrow}\ {\isacharprime}c\ T{\isachardoublequote}\isanewline
\ {\isachardoublequote}liftM{\isadigit{2}}\ f\ p\ q\ {\isasymequiv}\ do\ {\isacharbraceleft}x\ {\isasymleftarrow}\ p{\isacharsemicolon}\ y\ {\isasymleftarrow}\ q{\isacharsemicolon}\ ret\ {\isacharparenleft}f\ x\ y{\isacharparenright}{\isacharbraceright}{\isachardoublequote}\isanewline
\ liftM{\isadigit{3}}\ {\isacharcolon}{\isacharcolon}\ {\isachardoublequote}{\isacharbrackleft}{\isacharprime}a\ {\isasymRightarrow}\ {\isacharprime}b\ {\isasymRightarrow}\ {\isacharprime}c\ {\isasymRightarrow}\ {\isacharprime}d{\isacharcomma}\ {\isacharprime}a\ T{\isacharcomma}\ {\isacharprime}b\ T{\isacharcomma}\ {\isacharprime}c\ T{\isacharbrackright}\ {\isasymRightarrow}\ {\isacharprime}d\ T{\isachardoublequote}\isanewline
\ {\isachardoublequote}liftM{\isadigit{3}}\ f\ p\ q\ r\ {\isasymequiv}\ do\ {\isacharbraceleft}x\ {\isasymleftarrow}\ p{\isacharsemicolon}\ y\ {\isasymleftarrow}\ q{\isacharsemicolon}\ z\ {\isasymleftarrow}\ r{\isacharsemicolon}\ ret\ {\isacharparenleft}f\ x\ y\ z{\isacharparenright}{\isacharbraceright}{\isachardoublequote}\isanewline
\ %
\isamarkupcmt{The most general form of lifting; the above may be expressed by it%
}
\isanewline
\ ap\ {\isacharcolon}{\isacharcolon}\ {\isachardoublequote}{\isacharbrackleft}{\isacharparenleft}{\isacharprime}a\ {\isasymRightarrow}\ {\isacharprime}b{\isacharparenright}\ T{\isacharcomma}\ {\isacharprime}a\ T{\isacharbrackright}\ {\isasymRightarrow}\ {\isacharprime}b\ T{\isachardoublequote}\ \ \ \ \ \ \ \ \ {\isacharparenleft}\isakeyword{infixl}\ {\isachardoublequote}{\isachardollar}{\isachardollar}{\isachardoublequote}\ {\isadigit{1}}{\isadigit{0}}{\isadigit{0}}{\isacharparenright}\isanewline
\ {\isachardoublequote}ap\ m\ p\ {\isasymequiv}\ do\ {\isacharbraceleft}f\ {\isasymleftarrow}\ m{\isacharsemicolon}\ x\ {\isasymleftarrow}\ p{\isacharsemicolon}\ ret\ {\isacharparenleft}f\ x{\isacharparenright}{\isacharbraceright}{\isachardoublequote}\isanewline
\isanewline
\isamarkupfalse%
\isacommand{lemma}\ liftM{\isacharunderscore}ap{\isacharcolon}\ {\isachardoublequote}liftM\ f\ x\ {\isacharequal}\ {\isacharparenleft}ret\ f\ {\isachardollar}{\isachardollar}\ x{\isacharparenright}{\isachardoublequote}\isanewline
\isamarkupfalse%
\isacommand{by}\ {\isacharparenleft}simp\ add{\isacharcolon}\ ap{\isacharunderscore}def\ liftM{\isacharunderscore}def{\isacharparenright}\isanewline
\isanewline
\isamarkupfalse%
\isacommand{lemma}\ liftM{\isadigit{2}}{\isacharunderscore}ap{\isacharcolon}\ {\isachardoublequote}liftM{\isadigit{2}}\ f\ x\ y\ {\isacharequal}\ {\isacharparenleft}ret\ f\ {\isachardollar}{\isachardollar}\ x\ {\isachardollar}{\isachardollar}\ y{\isacharparenright}{\isachardoublequote}\isanewline
\isamarkupfalse%
\isacommand{by}\ {\isacharparenleft}simp\ add{\isacharcolon}\ mon{\isacharunderscore}ctr\ ap{\isacharunderscore}def\ liftM{\isadigit{2}}{\isacharunderscore}def{\isacharparenright}\isanewline
\isanewline
\isamarkupfalse%
\isacommand{lemma}\ liftM{\isadigit{3}}{\isacharunderscore}ap{\isacharcolon}\ {\isachardoublequote}liftM{\isadigit{3}}\ f\ x\ y\ z\ {\isacharequal}\ ret\ f\ {\isachardollar}{\isachardollar}\ x\ {\isachardollar}{\isachardollar}\ y\ {\isachardollar}{\isachardollar}\ z{\isachardoublequote}\isanewline
\ \isamarkupfalse%
\isacommand{by}{\isacharparenleft}simp\ add{\isacharcolon}\ mon{\isacharunderscore}ctr\ ap{\isacharunderscore}def\ liftM{\isadigit{3}}{\isacharunderscore}def{\isacharparenright}\isanewline
\isanewline
\isanewline
\isamarkupfalse%
\isacommand{theorem}\ dsef{\isacharunderscore}ret{\isacharunderscore}ap{\isacharcolon}\ {\isachardoublequote}dsef\ p\ {\isasymLongrightarrow}\ dsef\ {\isacharparenleft}ret\ f\ {\isachardollar}{\isachardollar}\ p{\isacharparenright}{\isachardoublequote}\isanewline
\ \ \isamarkupfalse%
\isacommand{apply}{\isacharparenleft}simp\ add{\isacharcolon}\ ap{\isacharunderscore}def\ dsef{\isacharunderscore}def{\isacharparenright}\isanewline
\ \ \isamarkupfalse%
\isacommand{apply}{\isacharparenleft}clarify{\isacharparenright}\isanewline
\ \ \isamarkupfalse%
\isacommand{apply}{\isacharparenleft}rule\ conjI{\isacharparenright}\isanewline
\ \ \isamarkupfalse%
\isacommand{apply}{\isacharparenleft}erule\ weak{\isacharunderscore}cp{\isacharunderscore}seq{\isacharparenright}\isanewline
\ \ \isamarkupfalse%
\isacommand{apply}{\isacharparenleft}rule\ conjI{\isacharparenright}\isanewline
\ \ \isamarkupfalse%
\isacommand{apply}{\isacharparenleft}erule\ weak{\isacharunderscore}dis{\isacharunderscore}seq{\isacharparenright}\isanewline
\ \ \isamarkupfalse%
\isacommand{apply}{\isacharparenleft}clarify{\isacharparenright}\isanewline
\ \ \isamarkupfalse%
\isacommand{apply}{\isacharparenleft}drule{\isacharunderscore}tac\ x\ {\isacharequal}\ q\ \isakeyword{in}\ spec{\isacharparenright}\isanewline
\ \ \isamarkupfalse%
\isacommand{apply}{\isacharparenleft}simp\ add{\isacharcolon}\ mon{\isacharunderscore}ctr\ weak{\isacharunderscore}cp{\isacharunderscore}seq{\isacharparenright}\isanewline
\ \ \isamarkupfalse%
\isacommand{apply}{\isacharparenleft}simp\ {\isacharparenleft}no{\isacharunderscore}asm{\isacharunderscore}simp{\isacharparenright}\ only{\isacharcolon}\ cp{\isacharunderscore}seq{\isacharunderscore}ret{\isacharparenright}\isanewline
\isamarkupfalse%
\isacommand{done}\isamarkupfalse%
%
\begin{isamarkuptext}%
\isa{dsef} programs may be swapped. \label{isa:commute-dsef}%
\end{isamarkuptext}%
\isamarkuptrue%
\isacommand{lemma}\ commute{\isacharunderscore}dsef{\isacharcolon}\ {\isachardoublequote}{\isasymlbrakk}dsef\ p{\isacharsemicolon}\ dsef\ q{\isasymrbrakk}\ {\isasymLongrightarrow}\ \isanewline
\ \ \ \ \ \ \ \ \ \ \ \ \ \ \ \ \ \ \ \ \ \ {\isasymforall}r{\isachardot}\ do\ {\isacharbraceleft}x{\isasymleftarrow}p{\isacharsemicolon}\ y{\isasymleftarrow}q{\isacharsemicolon}\ r\ x\ y{\isacharbraceright}\ {\isacharequal}\ do\ {\isacharbraceleft}y{\isasymleftarrow}q{\isacharsemicolon}\ x{\isasymleftarrow}p{\isacharsemicolon}\ r\ x\ y{\isacharbraceright}{\isachardoublequote}\isanewline
\ \ \isamarkupfalse%
\isacommand{apply}{\isacharparenleft}rule\ commute{\isacharunderscore}{\isadigit{1}}{\isacharunderscore}{\isadigit{3}}{\isacharparenright}\isanewline
\ \ \isamarkupfalse%
\isacommand{apply}{\isacharparenleft}simp{\isacharunderscore}all\ add{\isacharcolon}\ dsef{\isacharunderscore}def{\isacharparenright}\isanewline
\ \ \isamarkupfalse%
\isacommand{apply}{\isacharparenleft}clarify{\isacharparenright}\isanewline
\ \ \isamarkupfalse%
\isacommand{apply}{\isacharparenleft}drule\ commute{\isacharunderscore}bool{\isacharunderscore}arb{\isacharparenright}\isanewline
\ \ \isamarkupfalse%
\isacommand{apply}{\isacharparenleft}assumption{\isacharparenright}{\isacharplus}\isanewline
\ \ \isamarkupfalse%
\isacommand{apply}{\isacharparenleft}drule{\isacharunderscore}tac\ x\ {\isacharequal}\ q\ \isakeyword{in}\ spec{\isacharparenright}\isanewline
\isamarkupfalse%
\isacommand{by}{\isacharparenleft}blast{\isacharparenright}\isanewline
\ \ \isanewline
\isamarkupfalse%
\isacommand{lemma}\ commute{\isacharunderscore}bool{\isacharcolon}\ {\isachardoublequote}{\isasymlbrakk}dsef\ p{\isacharsemicolon}\ cp\ {\isacharparenleft}q{\isacharcolon}{\isacharcolon}bool\ T{\isacharparenright}{\isacharsemicolon}\ dis\ q{\isasymrbrakk}\ {\isasymLongrightarrow}\ \isanewline
\ \ \ \ \ \ \ \ \ \ \ \ \ \ \ \ \ \ \ \ \ {\isasymforall}r{\isachardot}\ do\ {\isacharbraceleft}x{\isasymleftarrow}p{\isacharsemicolon}\ y{\isasymleftarrow}q{\isacharsemicolon}\ r\ x\ y{\isacharbraceright}\ {\isacharequal}\ do\ {\isacharbraceleft}y{\isasymleftarrow}q{\isacharsemicolon}\ x{\isasymleftarrow}p{\isacharsemicolon}\ r\ x\ y{\isacharbraceright}{\isachardoublequote}\isanewline
\isamarkupfalse%
\isacommand{by}\ {\isacharparenleft}rule\ commute{\isacharunderscore}{\isadigit{1}}{\isacharunderscore}{\isadigit{3}}{\isacharcomma}\ simp{\isacharunderscore}all\ add{\isacharcolon}\ dsef{\isacharunderscore}def{\isacharparenright}\isamarkupfalse%
%
\begin{isamarkuptext}%
A formalisation of the essential fact that \isa{dsef} programs
  are actually stable under sequencing; this has only been proposed
  in \cite{SchroederMossakowski:PDL}, but has not been shown.
  \label{isa:dsef-seq}%
\end{isamarkuptext}%
\isamarkuptrue%
\isacommand{theorem}\ dsef{\isacharunderscore}seq{\isacharcolon}\ {\isachardoublequote}{\isasymlbrakk}dsef\ p{\isacharsemicolon}\ {\isasymforall}x{\isachardot}\ dsef\ {\isacharparenleft}q\ x{\isacharparenright}{\isasymrbrakk}\ {\isasymLongrightarrow}\ dsef\ {\isacharparenleft}do\ {\isacharbraceleft}x{\isasymleftarrow}p{\isacharsemicolon}\ q\ x{\isacharbraceright}{\isacharparenright}{\isachardoublequote}\isanewline
\isamarkupfalse%
\isacommand{proof}\ {\isacharminus}\isanewline
\ \ \isamarkupfalse%
\isacommand{assume}\ a{\isadigit{1}}{\isacharcolon}\ {\isachardoublequote}dsef\ p{\isachardoublequote}\ \isanewline
\ \ \isamarkupfalse%
\isacommand{assume}\ a{\isadigit{2}}{\isacharcolon}\ {\isachardoublequote}{\isasymforall}x{\isachardot}\ dsef\ {\isacharparenleft}q\ x{\isacharparenright}{\isachardoublequote}\isanewline
\ \ \isamarkupfalse%
\isacommand{from}\ a{\isadigit{1}}\ \isamarkupfalse%
\isacommand{have}\ disp{\isacharcolon}\ {\isachardoublequote}dis\ p{\isachardoublequote}\ \isamarkupfalse%
\isacommand{by}\ {\isacharparenleft}rule\ dsef{\isacharunderscore}dis{\isacharparenright}\isanewline
\ \ \isamarkupfalse%
\isacommand{from}\ a{\isadigit{1}}\ \isamarkupfalse%
\isacommand{have}\ cpp{\isacharcolon}\ {\isachardoublequote}cp\ p{\isachardoublequote}\ \isamarkupfalse%
\isacommand{by}\ {\isacharparenleft}rule\ dsef{\isacharunderscore}cp{\isacharparenright}\isanewline
\ \ \isamarkupfalse%
\isacommand{from}\ a{\isadigit{2}}\ \isamarkupfalse%
\isacommand{have}\ disq{\isacharcolon}\ {\isachardoublequote}{\isasymforall}x{\isachardot}\ dis\ {\isacharparenleft}q\ x{\isacharparenright}{\isachardoublequote}\ \isamarkupfalse%
\isacommand{by}\ {\isacharparenleft}unfold\ dsef{\isacharunderscore}def{\isacharcomma}\ blast{\isacharparenright}\isanewline
\ \ \isamarkupfalse%
\isacommand{from}\ a{\isadigit{2}}\ \isamarkupfalse%
\isacommand{have}\ cpq{\isacharcolon}\ {\isachardoublequote}{\isasymforall}x{\isachardot}\ cp\ {\isacharparenleft}q\ x{\isacharparenright}{\isachardoublequote}\ \isamarkupfalse%
\isacommand{by}\ {\isacharparenleft}unfold\ dsef{\isacharunderscore}def{\isacharcomma}\ blast{\isacharparenright}\isanewline
\ \ \isamarkupfalse%
\isacommand{let}\ {\isacharquery}s\ {\isacharequal}\ {\isachardoublequote}do\ {\isacharbraceleft}x{\isasymleftarrow}p{\isacharsemicolon}\ q\ x{\isacharbraceright}{\isachardoublequote}\isanewline
\ \ %
\isamarkupcmt{The proof proceeds in three parts, each one asserting some property stated
        in the definition of \isa{dsef} terms. Firstly, \isa{dsef} terms 
        are discardable.%
}
\isanewline
\ \ \isamarkupfalse%
\isacommand{have}\ {\isachardoublequote}dis\ {\isacharquery}s{\isachardoublequote}\isanewline
\ \ \isamarkupfalse%
\isacommand{proof}\ {\isacharminus}\isanewline
\ \ \ \ \isamarkupfalse%
\isacommand{have}\ {\isachardoublequote}do\ {\isacharbraceleft}x{\isasymleftarrow}{\isacharquery}s{\isacharsemicolon}\ ret\ {\isacharparenleft}{\isacharparenright}{\isacharbraceright}\ {\isacharequal}\ do\ {\isacharbraceleft}x{\isasymleftarrow}p{\isacharsemicolon}\ q\ x{\isacharsemicolon}\ ret\ {\isacharparenleft}{\isacharparenright}{\isacharbraceright}{\isachardoublequote}\ \isamarkupfalse%
\isacommand{by}\ {\isacharparenleft}simp\ add{\isacharcolon}\ seq{\isacharunderscore}def{\isacharparenright}\isanewline
\ \ \ \ \isamarkupfalse%
\isacommand{also}\ \isamarkupfalse%
\isacommand{from}\ disp\ disq\ \isanewline
\ \ \ \ \isamarkupfalse%
\isacommand{have}\ {\isachardoublequote}{\isasymdots}\ {\isacharequal}\ ret\ {\isacharparenleft}{\isacharparenright}{\isachardoublequote}\ \isamarkupfalse%
\isacommand{by}\ {\isacharparenleft}simp\ add{\isacharcolon}\ dis{\isacharunderscore}left\ dis{\isacharunderscore}left{\isadigit{2}}{\isacharparenright}\isanewline
\ \ \ \ \isamarkupfalse%
\isacommand{finally}\ \isamarkupfalse%
\isacommand{show}\ {\isacharquery}thesis\ \isamarkupfalse%
\isacommand{by}\ {\isacharparenleft}simp\ add{\isacharcolon}\ dis{\isacharunderscore}def{\isacharparenright}\isanewline
\ \ \isamarkupfalse%
\isacommand{qed}\isanewline
\ \ %
\isamarkupcmt{\isa{dsef} terms are also copyable. We unfold the definition and prove
        the required equation directly.%
}
\isanewline
\ \ \isamarkupfalse%
\isacommand{moreover}\ \isamarkupfalse%
\isacommand{have}\ {\isachardoublequote}cp\ {\isacharquery}s{\isachardoublequote}\isanewline
\ \ \isamarkupfalse%
\isacommand{proof}\ {\isacharminus}\isanewline
\ \ \ \ \isamarkupfalse%
\isacommand{have}\ {\isachardoublequote}do\ {\isacharbraceleft}x{\isasymleftarrow}{\isacharquery}s{\isacharsemicolon}\ y{\isasymleftarrow}{\isacharquery}s{\isacharsemicolon}\ ret\ {\isacharparenleft}x{\isacharcomma}y{\isacharparenright}{\isacharbraceright}\ {\isacharequal}\ \isanewline
\ \ \ \ \ \ do\ {\isacharbraceleft}u{\isasymleftarrow}p{\isacharsemicolon}\ x{\isasymleftarrow}q\ u{\isacharsemicolon}\ v{\isasymleftarrow}p{\isacharsemicolon}\ y{\isasymleftarrow}q\ v{\isacharsemicolon}\ ret\ {\isacharparenleft}x{\isacharcomma}y{\isacharparenright}{\isacharbraceright}{\isachardoublequote}\isanewline
\ \ \ \ \ \ \isamarkupfalse%
\isacommand{by}\ simp\isanewline
\ \ \ \ \isamarkupfalse%
\isacommand{also}\ \isamarkupfalse%
\isacommand{have}\ {\isachardoublequote}{\isasymdots}\ {\isacharequal}\ do\ {\isacharbraceleft}u{\isasymleftarrow}p{\isacharsemicolon}\ v{\isasymleftarrow}p{\isacharsemicolon}\ x{\isasymleftarrow}q\ u{\isacharsemicolon}\ y{\isasymleftarrow}q\ v{\isacharsemicolon}\ ret\ {\isacharparenleft}x{\isacharcomma}y{\isacharparenright}{\isacharbraceright}{\isachardoublequote}\isanewline
\ \ \ \ \isamarkupfalse%
\isacommand{proof}\ {\isacharminus}\isanewline
\ \ \ \ \ \ %
\isamarkupcmt{This swapping step is a bit more difficult; we have to assist 
            the simplifier by the following general statement:%
}
\isanewline
\ \ \ \ \ \ \isamarkupfalse%
\isacommand{have}\ {\isachardoublequote}{\isasymforall}u{\isachardot}\ do\ {\isacharbraceleft}x{\isasymleftarrow}q\ u{\isacharsemicolon}\ v{\isasymleftarrow}p{\isacharsemicolon}\ y{\isasymleftarrow}q\ v{\isacharsemicolon}\ ret\ {\isacharparenleft}x{\isacharcomma}y{\isacharparenright}{\isacharbraceright}\ {\isacharequal}\ do\ {\isacharbraceleft}v{\isasymleftarrow}p{\isacharsemicolon}\ x{\isasymleftarrow}q\ u{\isacharsemicolon}\ y{\isasymleftarrow}q\ v{\isacharsemicolon}\ ret\ {\isacharparenleft}x{\isacharcomma}y{\isacharparenright}{\isacharbraceright}{\isachardoublequote}\isanewline
\ \ \ \ \ \ \ \ {\isacharparenleft}\isakeyword{is}\ {\isachardoublequote}{\isasymforall}u{\isachardot}\ {\isacharquery}A\ u\ {\isacharequal}\ {\isacharquery}B\ u{\isachardoublequote}{\isacharparenright}\isanewline
\ \ \ \ \ \ \isamarkupfalse%
\isacommand{proof}\ \isanewline
\ \ \ \ \ \ \ \ \isamarkupfalse%
\isacommand{fix}\ u\isanewline
\ \ \ \ \ \ \ \ \isamarkupfalse%
\isacommand{from}\ a{\isadigit{2}}\ \isamarkupfalse%
\isacommand{have}\ {\isachardoublequote}dsef\ {\isacharparenleft}q\ u{\isacharparenright}{\isachardoublequote}\ \isamarkupfalse%
\isacommand{by}\ {\isacharparenleft}rule\ spec{\isacharparenright}\isanewline
\ \ \ \ \ \ \ \ \isamarkupfalse%
\isacommand{from}\ this\ a{\isadigit{1}}\ \isanewline
\ \ \ \ \ \ \ \ \isamarkupfalse%
\isacommand{have}\ {\isachardoublequote}{\isasymforall}r{\isacharcolon}{\isacharcolon}{\isacharprime}b{\isasymRightarrow}{\isacharprime}a{\isasymRightarrow}{\isacharparenleft}{\isacharprime}b{\isacharasterisk}{\isacharprime}b{\isacharparenright}\ T{\isachardot}\ do\ {\isacharbraceleft}x{\isasymleftarrow}q\ u{\isacharsemicolon}\ v{\isasymleftarrow}p{\isacharsemicolon}\ r\ x\ v{\isacharbraceright}\ {\isacharequal}\ do\ {\isacharbraceleft}v{\isasymleftarrow}p{\isacharsemicolon}\ x{\isasymleftarrow}q\ u{\isacharsemicolon}\ r\ x\ v{\isacharbraceright}{\isachardoublequote}\isanewline
\ \ \ \ \ \ \ \ \ \ \isamarkupfalse%
\isacommand{by}\ {\isacharparenleft}rule\ commute{\isacharunderscore}dsef{\isacharparenright}\isanewline
\ \ \ \ \ \ \ \ \isamarkupfalse%
\isacommand{thus}\ {\isachardoublequote}{\isacharquery}A\ u\ {\isacharequal}\ {\isacharquery}B\ u{\isachardoublequote}\ \isamarkupfalse%
\isacommand{by}\ {\isacharparenleft}rule\ spec{\isacharparenright}\isanewline
\ \ \ \ \ \ \isamarkupfalse%
\isacommand{qed}\isanewline
\ \ \ \ \ \ \isamarkupfalse%
\isacommand{thus}\ {\isacharquery}thesis\ \isamarkupfalse%
\isacommand{by}\ simp\isanewline
\ \ \ \ \isamarkupfalse%
\isacommand{qed}\isanewline
\ \ \ \ \isamarkupfalse%
\isacommand{also}\ \isamarkupfalse%
\isacommand{from}\ cpp\ cpq\ \isamarkupfalse%
\isacommand{have}\ {\isachardoublequote}{\isasymdots}\ {\isacharequal}\ do\ {\isacharbraceleft}u{\isasymleftarrow}p{\isacharsemicolon}\ x{\isasymleftarrow}q\ u{\isacharsemicolon}\ ret\ {\isacharparenleft}x{\isacharcomma}x{\isacharparenright}{\isacharbraceright}{\isachardoublequote}\isanewline
\ \ \ \ \ \ \isamarkupfalse%
\isacommand{by}\ {\isacharparenleft}simp\ add{\isacharcolon}\ cp{\isacharunderscore}arb{\isacharparenright}\isanewline
\ \ \ \ \isamarkupfalse%
\isacommand{finally}\ \isamarkupfalse%
\isacommand{show}\ {\isacharquery}thesis\ \isamarkupfalse%
\isacommand{by}\ {\isacharparenleft}simp\ add{\isacharcolon}\ cp{\isacharunderscore}def{\isacharparenright}\isanewline
\ \ \isamarkupfalse%
\isacommand{qed}\isanewline
\ \ %
\isamarkupcmt{The final step is that \isa{p\ {\isasymggreater}{\isacharequal}\ q} commutes with bool-valued programs:%
}
\isanewline
\ \ \isamarkupfalse%
\isacommand{moreover}\ \isamarkupfalse%
\isacommand{have}\ {\isachardoublequote}{\isasymforall}q{\isacharcolon}{\isacharcolon}bool\ T{\isachardot}\ cp\ q\ {\isasymand}\ dis\ q\ {\isasymlongrightarrow}\ cp\ {\isacharparenleft}do\ {\isacharbraceleft}x{\isasymleftarrow}{\isacharquery}s{\isacharsemicolon}\ y{\isasymleftarrow}q{\isacharsemicolon}\ ret{\isacharparenleft}x{\isacharcomma}y{\isacharparenright}{\isacharbraceright}{\isacharparenright}{\isachardoublequote}\isanewline
\ \ \isamarkupfalse%
\isacommand{proof}\isanewline
\ \ \ \ %
\isamarkupcmt{The proof is carried out by a so called raw proof block, where the succeeding
          application of blast spares us having to do the trivial proof steps.%
}
\isanewline
\ \ \ \ \isamarkupfalse%
\isacommand{fix}\ qa\isanewline
\ \ \ \ \isamarkupfalse%
\isacommand{{\isacharbraceleft}}\ \isamarkupfalse%
\isacommand{assume}\ cpqa{\isacharcolon}\ {\isachardoublequote}cp\ {\isacharparenleft}qa{\isacharcolon}{\isacharcolon}bool\ T{\isacharparenright}{\isachardoublequote}\isanewline
\ \ \ \ \ \ \isamarkupfalse%
\isacommand{assume}\ disqa{\isacharcolon}\ {\isachardoublequote}dis\ qa{\isachardoublequote}\isanewline
\ \ \ \ \ \ \isamarkupfalse%
\isacommand{have}\ {\isachardoublequote}cp\ {\isacharparenleft}do\ {\isacharbraceleft}x{\isasymleftarrow}do{\isacharbraceleft}u{\isasymleftarrow}p{\isacharsemicolon}\ q\ u{\isacharbraceright}{\isacharsemicolon}\ y{\isasymleftarrow}qa{\isacharsemicolon}\ ret\ {\isacharparenleft}x{\isacharcomma}\ y{\isacharparenright}{\isacharbraceright}{\isacharparenright}{\isachardoublequote}\isanewline
\ \ \ \ \ \ \isamarkupfalse%
\isacommand{proof}\ {\isacharminus}\isanewline
\ \ \ \ \ \ \ \ \isamarkupfalse%
\isacommand{let}\ {\isacharquery}w\ {\isacharequal}\ {\isachardoublequote}do\ {\isacharbraceleft}x{\isasymleftarrow}do{\isacharbraceleft}u{\isasymleftarrow}p{\isacharsemicolon}\ q\ u{\isacharbraceright}{\isacharsemicolon}\ y{\isasymleftarrow}qa{\isacharsemicolon}\ ret\ {\isacharparenleft}x{\isacharcomma}\ y{\isacharparenright}{\isacharbraceright}{\isachardoublequote}\isanewline
\ \ \ \ \ \ \ \ \isamarkupfalse%
\isacommand{have}\ {\isachardoublequote}do\ {\isacharbraceleft}x{\isasymleftarrow}{\isacharquery}w{\isacharsemicolon}\ y{\isasymleftarrow}{\isacharquery}w{\isacharsemicolon}\ ret\ {\isacharparenleft}x{\isacharcomma}y{\isacharparenright}{\isacharbraceright}\ {\isacharequal}\ \isanewline
\ \ \ \ \ \ \ \ \ \ \ \ \ \ do\ {\isacharbraceleft}u{\isasymleftarrow}p{\isacharsemicolon}\ x{\isasymleftarrow}q\ u{\isacharsemicolon}\ y{\isasymleftarrow}qa{\isacharsemicolon}\ u{\isacharprime}{\isasymleftarrow}p{\isacharsemicolon}\ x{\isacharprime}{\isasymleftarrow}q\ u{\isacharprime}{\isacharsemicolon}\ y{\isacharprime}{\isasymleftarrow}qa{\isacharsemicolon}\ ret{\isacharparenleft}{\isacharparenleft}x{\isacharcomma}y{\isacharparenright}{\isacharcomma}{\isacharparenleft}x{\isacharprime}{\isacharcomma}y{\isacharprime}{\isacharparenright}{\isacharparenright}{\isacharbraceright}{\isachardoublequote}\isanewline
\ \ \ \ \ \ \ \ \ \ \isamarkupfalse%
\isacommand{by}\ {\isacharparenleft}simp\ del{\isacharcolon}\ bind{\isacharunderscore}assoc\ add{\isacharcolon}\ mon{\isacharunderscore}ctr{\isacharparenright}\isanewline
\ \ \ \ \ \ \ \ \isamarkupfalse%
\isacommand{also}\ \isamarkupfalse%
\isacommand{from}\ a{\isadigit{1}}\ cpqa\ disqa\ \isanewline
\ \ \ \ \ \ \ \ \isamarkupfalse%
\isacommand{have}\ {\isachardoublequote}{\isasymdots}\ {\isacharequal}\ do\ {\isacharbraceleft}u{\isasymleftarrow}p{\isacharsemicolon}\ x{\isasymleftarrow}q\ u{\isacharsemicolon}\ u{\isacharprime}{\isasymleftarrow}p{\isacharsemicolon}\ y{\isasymleftarrow}qa{\isacharsemicolon}\ x{\isacharprime}{\isasymleftarrow}q\ u{\isacharprime}{\isacharsemicolon}\ y{\isacharprime}{\isasymleftarrow}qa{\isacharsemicolon}\ ret{\isacharparenleft}{\isacharparenleft}x{\isacharcomma}y{\isacharparenright}{\isacharcomma}{\isacharparenleft}x{\isacharprime}{\isacharcomma}y{\isacharprime}{\isacharparenright}{\isacharparenright}{\isacharbraceright}{\isachardoublequote}\isanewline
\ \ \ \ \ \ \ \ \ \ \isamarkupfalse%
\isacommand{by}\ {\isacharparenleft}simp\ add{\isacharcolon}\ commute{\isacharunderscore}bool{\isacharparenright}\isanewline
\ \ \ \ \ \ \ \ \isamarkupfalse%
\isacommand{also}\ \isamarkupfalse%
\isacommand{from}\ a{\isadigit{1}}\ a{\isadigit{2}}\isanewline
\ \ \ \ \ \ \ \ \isamarkupfalse%
\isacommand{have}\ {\isachardoublequote}{\isasymdots}\ {\isacharequal}\ do\ {\isacharbraceleft}u{\isasymleftarrow}p{\isacharsemicolon}\ u{\isacharprime}{\isasymleftarrow}p{\isacharsemicolon}\ x{\isasymleftarrow}q\ u{\isacharsemicolon}\ y{\isasymleftarrow}qa{\isacharsemicolon}\ x{\isacharprime}{\isasymleftarrow}q\ u{\isacharprime}{\isacharsemicolon}\ y{\isacharprime}{\isasymleftarrow}qa{\isacharsemicolon}\ ret{\isacharparenleft}{\isacharparenleft}x{\isacharcomma}y{\isacharparenright}{\isacharcomma}{\isacharparenleft}x{\isacharprime}{\isacharcomma}y{\isacharprime}{\isacharparenright}{\isacharparenright}{\isacharbraceright}{\isachardoublequote}\isanewline
\ \ \ \ \ \ \ \ \isamarkupfalse%
\isacommand{proof}\ {\isacharminus}\ \isanewline
\ \ \ \ \ \ \ \ \ \ %
\isamarkupcmt{This fact is needed to help the simplifier solve the goal%
}
\isanewline
\ \ \ \ \ \ \ \ \ \ \isamarkupfalse%
\isacommand{have}\ {\isachardoublequote}{\isasymforall}u{\isachardot}\ do\ {\isacharbraceleft}x{\isasymleftarrow}q\ u{\isacharsemicolon}\ u{\isacharprime}{\isasymleftarrow}p{\isacharsemicolon}\ y{\isasymleftarrow}qa{\isacharsemicolon}\ x{\isacharprime}{\isasymleftarrow}q\ u{\isacharprime}{\isacharsemicolon}\ y{\isacharprime}{\isasymleftarrow}qa{\isacharsemicolon}\ ret{\isacharparenleft}{\isacharparenleft}x{\isacharcomma}y{\isacharparenright}{\isacharcomma}{\isacharparenleft}x{\isacharprime}{\isacharcomma}y{\isacharprime}{\isacharparenright}{\isacharparenright}{\isacharbraceright}\ {\isacharequal}\isanewline
\ \ \ \ \ \ \ \ \ \ \ \ \ \ \ \ \ \ \ \ do\ {\isacharbraceleft}u{\isacharprime}{\isasymleftarrow}p{\isacharsemicolon}\ x{\isasymleftarrow}q\ u{\isacharsemicolon}\ y{\isasymleftarrow}qa{\isacharsemicolon}\ x{\isacharprime}{\isasymleftarrow}q\ u{\isacharprime}{\isacharsemicolon}\ y{\isacharprime}{\isasymleftarrow}qa{\isacharsemicolon}\ ret{\isacharparenleft}{\isacharparenleft}x{\isacharcomma}y{\isacharparenright}{\isacharcomma}{\isacharparenleft}x{\isacharprime}{\isacharcomma}y{\isacharprime}{\isacharparenright}{\isacharparenright}{\isacharbraceright}{\isachardoublequote}\isanewline
\ \ \ \ \ \ \ \ \ \ \ \ {\isacharparenleft}\isakeyword{is}\ {\isachardoublequote}{\isasymforall}u{\isachardot}\ {\isacharquery}A\ u\ {\isacharequal}\ {\isacharquery}B\ u{\isachardoublequote}{\isacharparenright}\isanewline
\ \ \ \ \ \ \ \ \ \ \isamarkupfalse%
\isacommand{proof}\isanewline
\ \ \ \ \ \ \ \ \ \ \ \ \isamarkupfalse%
\isacommand{fix}\ u\isanewline
\ \ \ \ \ \ \ \ \ \ \ \ \isamarkupfalse%
\isacommand{from}\ a{\isadigit{2}}\ \isamarkupfalse%
\isacommand{have}\ {\isachardoublequote}dsef\ {\isacharparenleft}q\ u{\isacharparenright}{\isachardoublequote}\ \isamarkupfalse%
\isacommand{by}\ {\isacharparenleft}rule\ spec{\isacharparenright}\isanewline
\ \ \ \ \ \ \ \ \ \ \ \ \isamarkupfalse%
\isacommand{from}\ this\ a{\isadigit{1}}\ \isamarkupfalse%
\isacommand{have}\ {\isachardoublequote}{\isasymforall}r{\isachardot}\ do\ {\isacharbraceleft}x{\isasymleftarrow}q\ u{\isacharsemicolon}\ u{\isacharprime}{\isasymleftarrow}p{\isacharsemicolon}\ r\ x\ u{\isacharprime}{\isacharbraceright}\ {\isacharequal}\ do\ {\isacharbraceleft}u{\isacharprime}{\isasymleftarrow}p{\isacharsemicolon}\ x{\isasymleftarrow}q\ u{\isacharsemicolon}\ r\ x\ u{\isacharprime}{\isacharbraceright}{\isachardoublequote}\isanewline
\ \ \ \ \ \ \ \ \ \ \ \ \ \ \isamarkupfalse%
\isacommand{by}\ {\isacharparenleft}rule\ commute{\isacharunderscore}dsef{\isacharparenright}\isanewline
\ \ \ \ \ \ \ \ \ \ \ \ \isamarkupfalse%
\isacommand{thus}\ {\isachardoublequote}{\isacharquery}A\ u\ {\isacharequal}\ {\isacharquery}B\ u{\isachardoublequote}\ \isamarkupfalse%
\isacommand{by}\ {\isacharparenleft}rule\ spec{\isacharparenright}\isanewline
\ \ \ \ \ \ \ \ \ \ \isamarkupfalse%
\isacommand{qed}\isanewline
\ \ \ \ \ \ \ \ \ \ \isamarkupfalse%
\isacommand{thus}\ {\isacharquery}thesis\ \isamarkupfalse%
\isacommand{by}\ simp\isanewline
\ \ \ \ \ \ \ \ \isamarkupfalse%
\isacommand{qed}\isanewline
\ \ \ \ \ \ \ \ \isamarkupfalse%
\isacommand{also}\ \isamarkupfalse%
\isacommand{from}\ a{\isadigit{2}}\ cpqa\ disqa\ \isanewline
\ \ \ \ \ \ \ \ \isamarkupfalse%
\isacommand{have}\ {\isachardoublequote}{\isasymdots}\ {\isacharequal}\ do\ {\isacharbraceleft}u{\isasymleftarrow}p{\isacharsemicolon}\ u{\isacharprime}{\isasymleftarrow}p{\isacharsemicolon}\ x{\isasymleftarrow}q\ u{\isacharsemicolon}\ x{\isacharprime}{\isasymleftarrow}q\ u{\isacharprime}{\isacharsemicolon}\ y{\isasymleftarrow}qa{\isacharsemicolon}\ y{\isacharprime}{\isasymleftarrow}qa{\isacharsemicolon}\ ret{\isacharparenleft}{\isacharparenleft}x{\isacharcomma}y{\isacharparenright}{\isacharcomma}{\isacharparenleft}x{\isacharprime}{\isacharcomma}y{\isacharprime}{\isacharparenright}{\isacharparenright}{\isacharbraceright}{\isachardoublequote}\isanewline
\ \ \ \ \ \ \ \ \ \ \isamarkupfalse%
\isacommand{by}\ {\isacharparenleft}simp\ add{\isacharcolon}\ commute{\isacharunderscore}bool{\isacharparenright}\isanewline
\ \ \ \ \ \ \ \ \isamarkupfalse%
\isacommand{also}\ \isamarkupfalse%
\isacommand{from}\ cpp\ cpq\ cpqa\ \isamarkupfalse%
\isacommand{have}\ {\isachardoublequote}{\isasymdots}\ {\isacharequal}\ do\ {\isacharbraceleft}u{\isasymleftarrow}p{\isacharsemicolon}\ x{\isasymleftarrow}q\ u{\isacharsemicolon}\ y{\isasymleftarrow}qa{\isacharsemicolon}\ ret{\isacharparenleft}{\isacharparenleft}x{\isacharcomma}y{\isacharparenright}{\isacharcomma}{\isacharparenleft}x{\isacharcomma}y{\isacharparenright}{\isacharparenright}{\isacharbraceright}{\isachardoublequote}\isanewline
\ \ \ \ \ \ \ \ \ \ \isamarkupfalse%
\isacommand{by}\ {\isacharparenleft}simp\ add{\isacharcolon}\ cp{\isacharunderscore}arb{\isacharparenright}\isanewline
\ \ \ \ \ \ \ \ \isamarkupfalse%
\isacommand{finally}\ \isamarkupfalse%
\isacommand{show}\ {\isacharquery}thesis\ \isamarkupfalse%
\isacommand{by}\ {\isacharparenleft}simp\ del{\isacharcolon}\ bind{\isacharunderscore}assoc\ add{\isacharcolon}\ mon{\isacharunderscore}ctr\ cp{\isacharunderscore}def{\isacharparenright}\isanewline
\ \ \ \ \ \ \isamarkupfalse%
\isacommand{qed}\isanewline
\ \ \ \ \isamarkupfalse%
\isacommand{{\isacharbraceright}}\isanewline
\ \ \ \ \isamarkupfalse%
\isacommand{thus}\ {\isachardoublequote}cp\ {\isacharparenleft}qa{\isacharcolon}{\isacharcolon}bool\ T{\isacharparenright}\ {\isasymand}\ dis\ qa\ {\isasymlongrightarrow}\ cp\ {\isacharparenleft}do\ {\isacharbraceleft}x{\isasymleftarrow}do\ {\isacharbraceleft}u{\isasymleftarrow}p{\isacharsemicolon}\ q\ u{\isacharbraceright}{\isacharsemicolon}\ y{\isasymleftarrow}qa{\isacharsemicolon}\ ret\ {\isacharparenleft}x{\isacharcomma}\ y{\isacharparenright}{\isacharbraceright}{\isacharparenright}{\isachardoublequote}\isanewline
\ \ \ \ \ \ \isamarkupfalse%
\isacommand{by}\ blast\isanewline
\ \ \isamarkupfalse%
\isacommand{qed}\isanewline
\ \ \isamarkupfalse%
\isacommand{ultimately}\ \isamarkupfalse%
\isacommand{show}\ {\isachardoublequote}dsef\ {\isacharquery}s{\isachardoublequote}\ \isamarkupfalse%
\isacommand{by}\ {\isacharparenleft}simp\ add{\isacharcolon}dsef{\isacharunderscore}def{\isacharparenright}\isanewline
\isamarkupfalse%
\isacommand{qed}\isamarkupfalse%
%
\begin{isamarkuptext}%
Given that \isa{dsef} programs are stable under sequencing, this
        weak form, which comes in handy sometimes, can easily be proved.%
\end{isamarkuptext}%
\isamarkuptrue%
\isacommand{lemma}\ weak{\isacharunderscore}dsef{\isacharunderscore}seq{\isacharcolon}\ {\isachardoublequote}dsef\ p\ {\isasymLongrightarrow}\ dsef\ {\isacharparenleft}do\ {\isacharbraceleft}x{\isasymleftarrow}p{\isacharsemicolon}\ ret\ {\isacharparenleft}f\ x{\isacharparenright}{\isacharbraceright}{\isacharparenright}{\isachardoublequote}\isanewline
\ \ \isamarkupfalse%
\isacommand{by}{\isacharparenleft}simp\ add{\isacharcolon}\ dsef{\isacharunderscore}seq{\isacharparenright}\isamarkupfalse%
%
\begin{isamarkuptext}%
With the help of theorem \isa{dsef{\isacharunderscore}seq} the following proof is
  immediate.%
\end{isamarkuptext}%
\isamarkuptrue%
\isacommand{lemma}\ dsef{\isacharunderscore}liftM{\isadigit{2}}{\isacharcolon}\ {\isachardoublequote}{\isasymlbrakk}dsef\ p{\isacharsemicolon}\ dsef\ q{\isasymrbrakk}\ {\isasymLongrightarrow}\ dsef\ {\isacharparenleft}liftM{\isadigit{2}}\ f\ p\ q{\isacharparenright}{\isachardoublequote}\isanewline
\isamarkupfalse%
\isacommand{proof}\ {\isacharminus}\isanewline
\ \ \isamarkupfalse%
\isacommand{assume}\ a{\isadigit{1}}{\isacharcolon}\ {\isachardoublequote}dsef\ p{\isachardoublequote}\ \isakeyword{and}\ a{\isadigit{2}}{\isacharcolon}\ {\isachardoublequote}dsef\ q{\isachardoublequote}\isanewline
\ \ \isamarkupfalse%
\isacommand{from}\ a{\isadigit{1}}\ \isamarkupfalse%
\isacommand{have}\ {\isachardoublequote}dsef\ {\isacharparenleft}do\ {\isacharbraceleft}x{\isasymleftarrow}p{\isacharsemicolon}\ y{\isasymleftarrow}q{\isacharsemicolon}\ ret\ {\isacharparenleft}f\ x\ y{\isacharparenright}{\isacharbraceright}{\isacharparenright}{\isachardoublequote}\isanewline
\ \ \isamarkupfalse%
\isacommand{proof}\ {\isacharparenleft}rule\ dsef{\isacharunderscore}seq{\isacharparenright}\isanewline
\ \ \ \ \isamarkupfalse%
\isacommand{show}\ {\isachardoublequote}\ {\isasymforall}x{\isachardot}\ dsef\ {\isacharparenleft}do\ {\isacharbraceleft}y{\isasymleftarrow}q{\isacharsemicolon}\ ret\ {\isacharparenleft}f\ x\ y{\isacharparenright}{\isacharbraceright}{\isacharparenright}{\isachardoublequote}\isanewline
\ \ \ \ \isamarkupfalse%
\isacommand{proof}\ \isanewline
\ \ \ \ \ \ \isamarkupfalse%
\isacommand{fix}\ x\ \isamarkupfalse%
\isacommand{from}\ a{\isadigit{2}}\ \isamarkupfalse%
\isacommand{show}\ {\isachardoublequote}dsef\ {\isacharparenleft}do\ {\isacharbraceleft}y{\isasymleftarrow}q{\isacharsemicolon}\ ret\ {\isacharparenleft}f\ x\ y{\isacharparenright}{\isacharbraceright}{\isacharparenright}{\isachardoublequote}\isanewline
\ \ \ \ \ \ \isamarkupfalse%
\isacommand{proof}\ {\isacharparenleft}rule\ dsef{\isacharunderscore}seq{\isacharparenright}\isanewline
\ \ \ \ \ \ \ \ \isamarkupfalse%
\isacommand{show}\ {\isachardoublequote}\ {\isasymforall}y{\isachardot}\ dsef\ {\isacharparenleft}ret\ {\isacharparenleft}f\ x\ y{\isacharparenright}{\isacharparenright}{\isachardoublequote}\isanewline
\ \ \ \ \ \ \ \ \isamarkupfalse%
\isacommand{proof}\ \isanewline
\ \ \ \ \ \ \ \ \ \ \isamarkupfalse%
\isacommand{fix}\ y\ \isamarkupfalse%
\isacommand{show}\ {\isachardoublequote}dsef\ {\isacharparenleft}ret\ {\isacharparenleft}f\ x\ y{\isacharparenright}{\isacharparenright}{\isachardoublequote}\ \isamarkupfalse%
\isacommand{by}\ {\isacharparenleft}rule\ dsef{\isacharunderscore}ret{\isacharparenright}\isanewline
\ \ \ \ \ \ \ \ \isamarkupfalse%
\isacommand{qed}\isanewline
\ \ \ \ \ \ \isamarkupfalse%
\isacommand{qed}\isanewline
\ \ \ \ \isamarkupfalse%
\isacommand{qed}\isanewline
\ \ \isamarkupfalse%
\isacommand{qed}\isanewline
\ \ \isamarkupfalse%
\isacommand{thus}\ {\isachardoublequote}dsef\ {\isacharparenleft}liftM{\isadigit{2}}\ f\ p\ q{\isacharparenright}{\isachardoublequote}\ \isamarkupfalse%
\isacommand{by}\ {\isacharparenleft}simp\ only{\isacharcolon}\ liftM{\isadigit{2}}{\isacharunderscore}def{\isacharparenright}\isanewline
\isamarkupfalse%
\isacommand{qed}\isanewline
\isanewline
\isanewline
\isamarkupfalse%
\isacommand{lemma}\ Abs{\isacharunderscore}Dsef{\isacharunderscore}inverse{\isacharunderscore}liftM{\isadigit{2}}\ {\isacharbrackleft}simp{\isacharbrackright}{\isacharcolon}\ {\isachardoublequote}{\isasymlbrakk}dsef\ p{\isacharsemicolon}\ dsef\ q{\isasymrbrakk}\ {\isasymLongrightarrow}\ \isanewline
\ \ {\isasymDown}\ {\isacharparenleft}{\isasymUp}\ {\isacharparenleft}liftM{\isadigit{2}}\ f\ p\ q{\isacharparenright}{\isacharparenright}\ {\isacharequal}\ liftM{\isadigit{2}}\ f\ p\ q{\isachardoublequote}\isanewline
\isamarkupfalse%
\isacommand{by}\ {\isacharparenleft}simp\ add{\isacharcolon}\ Abs{\isacharunderscore}Dsef{\isacharunderscore}inverse\ Dsef{\isacharunderscore}def\ dsef{\isacharunderscore}liftM{\isadigit{2}}{\isacharparenright}\isanewline
\isanewline
\isanewline
\isamarkupfalse%
\isacommand{end}\isamarkupfalse%
\end{isabellebody}%
%%% Local Variables:
%%% mode: latex
%%% TeX-master: "root"
%%% End:


%
\begin{isabellebody}%
\def\isabellecontext{MonLogic}%
%
\isamarkupheader{Introducing Propositional Connectives%
}
\isamarkuptrue%
\isacommand{theory}\ MonLogic\ {\isacharequal}\ MonProp{\isacharcolon}\isamarkupfalse%
%
\label{sec:monlogic-thy}
%
\isamarkupsubsection{Propositional Connectives%
}
\isamarkuptrue%
%
\begin{isamarkuptext}%
As usual in intuitionistic logics, we introduce conjunction,
  disjunction and implication independently of each other.
  \label{isa:logical-const}%
\end{isamarkuptext}%
\isamarkuptrue%
\isacommand{consts}\isanewline
\ {\isachardoublequote}Valid{\isachardoublequote}\ \ \ \ \ \ \ {\isacharcolon}{\isacharcolon}\ {\isachardoublequote}bool\ D\ {\isasymRightarrow}\ bool{\isachardoublequote}\ \ \ \ \ \ \ \ \ \ \ \ \ \ \ {\isacharparenleft}{\isachardoublequote}{\isacharparenleft}{\isasymturnstile}\ {\isacharunderscore}{\isacharparenright}{\isachardoublequote}\ {\isadigit{1}}{\isadigit{5}}{\isacharparenright}\isanewline
\ {\isachardoublequote}{\isasymand}\isactrlsub D{\isachardoublequote}\ \ \ \ \ \ \ \ \ {\isacharcolon}{\isacharcolon}\ {\isachardoublequote}{\isacharbrackleft}bool\ D{\isacharcomma}\ bool\ D{\isacharbrackright}\ {\isasymRightarrow}\ bool\ D{\isachardoublequote}\ \ \ \ \ {\isacharparenleft}\isakeyword{infixr}\ {\isadigit{3}}{\isadigit{5}}{\isacharparenright}\isanewline
\ {\isachardoublequote}{\isasymor}\isactrlsub D{\isachardoublequote}\ \ \ \ \ \ \ \ \ {\isacharcolon}{\isacharcolon}\ {\isachardoublequote}{\isacharbrackleft}bool\ D{\isacharcomma}\ bool\ D{\isacharbrackright}\ {\isasymRightarrow}\ bool\ D{\isachardoublequote}\ \ \ \ \ {\isacharparenleft}\isakeyword{infixr}\ {\isadigit{3}}{\isadigit{0}}{\isacharparenright}\isanewline
\ {\isachardoublequote}{\isasymlongrightarrow}\isactrlsub D{\isachardoublequote}\ \ \ \ \ \ \ \ {\isacharcolon}{\isacharcolon}\ {\isachardoublequote}{\isacharbrackleft}bool\ D{\isacharcomma}\ bool\ D{\isacharbrackright}\ {\isasymRightarrow}\ bool\ D{\isachardoublequote}\ \ \ \ {\isacharparenleft}\isakeyword{infixr}\ {\isadigit{2}}{\isadigit{5}}{\isacharparenright}\isamarkupfalse%
%
\begin{isamarkuptext}%
According with the definition in \cite{SchroederMossakowski:PDL}, the connectives
  are simply lifted from HOL, and validity amounts to being
  equal to a program always returning \isa{True}.%
\end{isamarkuptext}%
\isamarkuptrue%
\isacommand{defs}\isanewline
\ \ Valid{\isacharunderscore}def{\isacharcolon}\ {\isachardoublequote}{\isasymturnstile}\ P\ {\isasymequiv}\ {\isasymDown}\ P\ {\isacharequal}\ do\ {\isacharbraceleft}x{\isasymleftarrow}{\isacharparenleft}{\isasymDown}\ P{\isacharparenright}{\isacharsemicolon}\ ret\ True{\isacharbraceright}{\isachardoublequote}\isanewline
\ \ conjD{\isacharunderscore}def{\isacharcolon}\ {\isachardoublequote}P\ {\isasymand}\isactrlsub D\ Q\ {\isasymequiv}\ {\isasymUp}\ {\isacharparenleft}liftM{\isadigit{2}}\ {\isacharparenleft}op\ {\isasymand}{\isacharparenright}\ {\isacharparenleft}{\isasymDown}\ P{\isacharparenright}\ {\isacharparenleft}{\isasymDown}\ Q{\isacharparenright}{\isacharparenright}{\isachardoublequote}\isanewline
\ \ disjD{\isacharunderscore}def{\isacharcolon}\ {\isachardoublequote}P\ {\isasymor}\isactrlsub D\ Q\ {\isasymequiv}\ {\isasymUp}\ {\isacharparenleft}liftM{\isadigit{2}}\ {\isacharparenleft}op\ {\isasymor}{\isacharparenright}\ {\isacharparenleft}{\isasymDown}\ P{\isacharparenright}\ {\isacharparenleft}{\isasymDown}\ Q{\isacharparenright}{\isacharparenright}{\isachardoublequote}\isanewline
\ \ impD{\isacharunderscore}def{\isacharcolon}\ \ {\isachardoublequote}P\ {\isasymlongrightarrow}\isactrlsub D\ Q\ {\isasymequiv}\ {\isasymUp}\ {\isacharparenleft}liftM{\isadigit{2}}\ {\isacharparenleft}op\ {\isasymlongrightarrow}{\isacharparenright}\ {\isacharparenleft}{\isasymDown}\ P{\isacharparenright}\ {\isacharparenleft}{\isasymDown}\ Q{\isacharparenright}{\isacharparenright}{\isachardoublequote}\isanewline
\isanewline
\isanewline
\isamarkupfalse%
\isacommand{constdefs}\isanewline
\ iffD\ \ \ \ \ \ \ \ {\isacharcolon}{\isacharcolon}\ {\isachardoublequote}{\isacharbrackleft}bool\ D{\isacharcomma}\ bool\ D{\isacharbrackright}\ {\isasymRightarrow}\ bool\ D{\isachardoublequote}\ \ \ {\isacharparenleft}\isakeyword{infixr}\ {\isachardoublequote}{\isasymlongleftrightarrow}\isactrlsub D{\isachardoublequote}\ {\isadigit{2}}{\isadigit{0}}{\isacharparenright}\isanewline
\ {\isachardoublequote}P\ {\isasymlongleftrightarrow}\isactrlsub D\ Q\ {\isasymequiv}\ {\isacharparenleft}P\ {\isasymlongrightarrow}\isactrlsub D\ Q{\isacharparenright}\ {\isasymand}\isactrlsub D\ {\isacharparenleft}Q\ {\isasymlongrightarrow}\isactrlsub D\ P{\isacharparenright}{\isachardoublequote}\isanewline
\ NotD\ \ \ \ \ \ \ \ {\isacharcolon}{\isacharcolon}\ {\isachardoublequote}bool\ D\ {\isasymRightarrow}\ bool\ D{\isachardoublequote}\ \ \ \ \ \ \ \ \ \ \ \ \ \ {\isacharparenleft}{\isachardoublequote}{\isasymnot}\isactrlsub D\ {\isacharunderscore}{\isachardoublequote}\ {\isacharbrackleft}{\isadigit{4}}{\isadigit{0}}{\isacharbrackright}\ {\isadigit{4}}{\isadigit{0}}{\isacharparenright}\isanewline
\ \ {\isachardoublequote}{\isasymnot}\isactrlsub D\ P\ {\isasymequiv}\ P\ {\isasymlongrightarrow}\isactrlsub D\ Ret\ False{\isachardoublequote}\isamarkupfalse%
%
\begin{isamarkuptext}%
Because of discardability, the definition of \isa{Valid}, which 
  was simply taken over from the definition of global validity of terms of type
  \isa{bool\ T}, can be simplified.
  \label{isa:valid-simp}%
\end{isamarkuptext}%
\isamarkuptrue%
\isacommand{lemma}\ Valid{\isacharunderscore}simp{\isacharcolon}\ {\isachardoublequote}{\isacharparenleft}{\isasymturnstile}\ p{\isacharparenright}\ {\isacharequal}\ {\isacharparenleft}{\isasymDown}\ p\ {\isacharequal}\ ret\ True{\isacharparenright}{\isachardoublequote}\isanewline
\isamarkupfalse%
\isacommand{proof}\isanewline
\ \ \isamarkupfalse%
\isacommand{assume}\ vp{\isacharcolon}\ {\isachardoublequote}{\isasymturnstile}\ p{\isachardoublequote}\isanewline
\ \ \isamarkupfalse%
\isacommand{show}\ {\isachardoublequote}{\isasymDown}\ p\ {\isacharequal}\ ret\ True{\isachardoublequote}\isanewline
\ \ \isamarkupfalse%
\isacommand{proof}\ {\isacharminus}\isanewline
\ \ \ \ \isamarkupfalse%
\isacommand{from}\ vp\ \isamarkupfalse%
\isacommand{have}\ {\isachardoublequote}{\isasymDown}\ p\ {\isacharequal}\ do\ {\isacharbraceleft}{\isasymDown}\ p{\isacharsemicolon}\ ret\ True{\isacharbraceright}{\isachardoublequote}\isanewline
\ \ \ \ \ \ \isamarkupfalse%
\isacommand{by}\ {\isacharparenleft}simp\ only{\isacharcolon}\ Valid{\isacharunderscore}def\ seq{\isacharunderscore}def{\isacharparenright}\isanewline
\ \ \ \ \isamarkupfalse%
\isacommand{also}\ \isamarkupfalse%
\isacommand{have}\ {\isachardoublequote}{\isasymdots}\ {\isacharequal}\ ret\ True{\isachardoublequote}\ \isamarkupfalse%
\isacommand{by}\ {\isacharparenleft}rule\ dis{\isacharunderscore}left{\isacharcomma}\ rule\ dis{\isacharunderscore}Rep{\isacharunderscore}Dsef{\isacharparenright}\ \ \isanewline
\ \ \ \ \isamarkupfalse%
\isacommand{finally}\ \isamarkupfalse%
\isacommand{show}\ {\isacharquery}thesis\ \isamarkupfalse%
\isacommand{{\isachardot}}\isanewline
\ \ \isamarkupfalse%
\isacommand{qed}\isanewline
\isamarkupfalse%
\isacommand{next}\isanewline
\ \ \isamarkupfalse%
\isacommand{assume}\ {\isachardoublequote}{\isasymDown}\ p\ {\isacharequal}\ ret\ True{\isachardoublequote}\isanewline
\ \ \isamarkupfalse%
\isacommand{hence}\ {\isachardoublequote}{\isasymDown}\ p\ {\isacharequal}\ do\ {\isacharbraceleft}x{\isasymleftarrow}{\isasymDown}\ p{\isacharsemicolon}\ ret\ True{\isacharbraceright}{\isachardoublequote}\ \isamarkupfalse%
\isacommand{by}\ simp\isanewline
\ \ \isamarkupfalse%
\isacommand{thus}\ {\isachardoublequote}{\isasymturnstile}\ p{\isachardoublequote}\ \isamarkupfalse%
\isacommand{by}\ {\isacharparenleft}simp\ only{\isacharcolon}\ Valid{\isacharunderscore}def{\isacharparenright}\isanewline
\isamarkupfalse%
\isacommand{qed}\isanewline
\isanewline
\isamarkupfalse%
\isacommand{lemma}\ Valid{\isacharunderscore}simpD{\isacharcolon}\ {\isachardoublequote}{\isacharparenleft}{\isasymturnstile}\ P{\isacharparenright}\ {\isacharequal}\ {\isacharparenleft}P\ {\isacharequal}\ Ret\ True{\isacharparenright}{\isachardoublequote}\isanewline
\ \ \isamarkupfalse%
\isacommand{apply}{\isacharparenleft}simp\ add{\isacharcolon}\ Valid{\isacharunderscore}simp\ Ret{\isacharunderscore}ret\ Ret{\isacharunderscore}def{\isacharparenright}\isanewline
\ \ \isamarkupfalse%
\isacommand{apply}{\isacharparenleft}induct{\isacharunderscore}tac\ P\ rule{\isacharcolon}\ \ Abs{\isacharunderscore}Dsef{\isacharunderscore}induct{\isacharparenright}\isanewline
\ \ \isamarkupfalse%
\isacommand{apply}{\isacharparenleft}simp\ add{\isacharcolon}\ Dsef{\isacharunderscore}def\ Abs{\isacharunderscore}Dsef{\isacharunderscore}inverse{\isacharparenright}\isanewline
\ \ \isamarkupfalse%
\isacommand{apply}{\isacharparenleft}rule\ Abs{\isacharunderscore}Dsef{\isacharunderscore}inject{\isacharbrackleft}symmetric{\isacharbrackright}{\isacharparenright}\isanewline
\ \ \isamarkupfalse%
\isacommand{by}\ {\isacharparenleft}simp{\isacharunderscore}all\ add{\isacharcolon}\ Dsef{\isacharunderscore}def{\isacharparenright}\isamarkupfalse%
%
\begin{isamarkuptext}%
There is a notion of homomorphism associated with lifted operations. The formulation
  does not really make clear what is intended, but the subsequent lemmas should
  illuminate the idea.
  \label{isa:lift-ret-hom}%
\end{isamarkuptext}%
\isamarkuptrue%
\isacommand{theorem}\ lift{\isacharunderscore}Ret{\isacharunderscore}hom{\isacharcolon}\ \ {\isachardoublequote}{\isacharparenleft}{\isasymUp}\ {\isacharparenleft}liftM{\isadigit{2}}\ f\ {\isacharparenleft}{\isasymDown}\ {\isacharparenleft}Ret\ a{\isacharparenright}{\isacharparenright}\ {\isacharparenleft}{\isasymDown}\ {\isacharparenleft}Ret\ b{\isacharparenright}{\isacharparenright}{\isacharparenright}{\isacharparenright}\ \isanewline
\ \ \ \ \ \ \ \ \ \ \ \ \ \ \ \ \ \ \ \ \ \ \ \ \ \ \ \ {\isacharequal}\ Ret\ {\isacharparenleft}f\ a\ b{\isacharparenright}{\isachardoublequote}\isanewline
\isamarkupfalse%
\isacommand{proof}\ {\isacharminus}\isanewline
\ \ \isamarkupfalse%
\isacommand{have}\ {\isachardoublequote}{\isasymUp}\ {\isacharparenleft}liftM{\isadigit{2}}\ f\ {\isacharparenleft}{\isasymDown}\ {\isacharparenleft}Ret\ a{\isacharparenright}{\isacharparenright}\ {\isacharparenleft}{\isasymDown}\ {\isacharparenleft}Ret\ b{\isacharparenright}{\isacharparenright}{\isacharparenright}\isanewline
\ \ \ \ \ \ \ \ {\isacharequal}\ {\isasymUp}\ {\isacharparenleft}do\ {\isacharbraceleft}x{\isasymleftarrow}{\isacharparenleft}{\isasymDown}\ {\isacharparenleft}Ret\ a{\isacharparenright}{\isacharparenright}{\isacharsemicolon}\ y{\isasymleftarrow}{\isacharparenleft}{\isasymDown}\ {\isacharparenleft}Ret\ b{\isacharparenright}{\isacharparenright}{\isacharsemicolon}\ ret\ {\isacharparenleft}f\ x\ y{\isacharparenright}{\isacharbraceright}{\isacharparenright}{\isachardoublequote}\isanewline
\ \ \ \ \isamarkupfalse%
\isacommand{by}\ {\isacharparenleft}simp\ only{\isacharcolon}\ liftM{\isadigit{2}}{\isacharunderscore}def{\isacharparenright}\isanewline
\ \ \isamarkupfalse%
\isacommand{also}\ \isamarkupfalse%
\isacommand{have}\ {\isachardoublequote}{\isasymdots}\ {\isacharequal}\ {\isasymUp}\ {\isacharparenleft}do\ {\isacharbraceleft}x{\isasymleftarrow}{\isacharparenleft}{\isasymDown}\ {\isacharparenleft}{\isasymUp}\ {\isacharparenleft}ret\ a{\isacharparenright}{\isacharparenright}{\isacharparenright}{\isacharsemicolon}\ \isanewline
\ \ \ \ \ \ \ \ \ \ \ \ \ \ \ \ \ \ \ \ \ \ \ \ \ \ \ \ \ \ \ \ y{\isasymleftarrow}{\isacharparenleft}{\isasymDown}\ {\isacharparenleft}{\isasymUp}\ {\isacharparenleft}ret\ b{\isacharparenright}{\isacharparenright}{\isacharparenright}{\isacharsemicolon}\ ret\ {\isacharparenleft}f\ x\ y{\isacharparenright}{\isacharbraceright}{\isacharparenright}{\isachardoublequote}\isanewline
\ \ \ \ \isamarkupfalse%
\isacommand{by}\ {\isacharparenleft}simp\ add{\isacharcolon}\ Ret{\isacharunderscore}def{\isacharparenright}\isanewline
\ \ \isamarkupfalse%
\isacommand{also}\ \isamarkupfalse%
\isacommand{have}\ {\isachardoublequote}{\isasymdots}\ {\isacharequal}\ {\isasymUp}\ {\isacharparenleft}do\ {\isacharbraceleft}x{\isasymleftarrow}ret\ a{\isacharsemicolon}\ y{\isasymleftarrow}ret\ b{\isacharsemicolon}\ ret{\isacharparenleft}f\ x\ y{\isacharparenright}{\isacharbraceright}{\isacharparenright}{\isachardoublequote}\isanewline
\ \ \ \ \isamarkupfalse%
\isacommand{by}\ {\isacharparenleft}simp\ add{\isacharcolon}\ Dsef{\isacharunderscore}def\ Abs{\isacharunderscore}Dsef{\isacharunderscore}inverse{\isacharparenright}\isanewline
\ \ \isamarkupfalse%
\isacommand{also}\ \isamarkupfalse%
\isacommand{have}\ {\isachardoublequote}{\isasymdots}\ {\isacharequal}\ {\isasymUp}\ {\isacharparenleft}ret\ {\isacharparenleft}f\ a\ b{\isacharparenright}{\isacharparenright}{\isachardoublequote}\ \isamarkupfalse%
\isacommand{by}\ simp\isanewline
\ \ \isamarkupfalse%
\isacommand{also}\ \isamarkupfalse%
\isacommand{have}\ {\isachardoublequote}{\isasymdots}\ {\isacharequal}\ Ret\ {\isacharparenleft}f\ a\ b{\isacharparenright}{\isachardoublequote}\ \isamarkupfalse%
\isacommand{by}\ {\isacharparenleft}simp\ only{\isacharcolon}\ Ret{\isacharunderscore}def{\isacharparenright}\isanewline
\ \ \isamarkupfalse%
\isacommand{finally}\ \isamarkupfalse%
\isacommand{show}\ {\isacharquery}thesis\ \isamarkupfalse%
\isacommand{{\isachardot}}\isanewline
\isamarkupfalse%
\isacommand{qed}\isanewline
\isanewline
\isanewline
\isamarkupfalse%
\isacommand{lemma}\ conjD{\isacharunderscore}Ret{\isacharunderscore}hom{\isacharcolon}\ {\isachardoublequote}Ret\ {\isacharparenleft}a{\isasymand}b{\isacharparenright}\ {\isacharequal}\ {\isacharparenleft}{\isacharparenleft}Ret\ a{\isacharparenright}\ {\isasymand}\isactrlsub D\ {\isacharparenleft}Ret\ b{\isacharparenright}{\isacharparenright}{\isachardoublequote}\isanewline
\ \ \isamarkupfalse%
\isacommand{by}\ {\isacharparenleft}simp\ add{\isacharcolon}\ lift{\isacharunderscore}Ret{\isacharunderscore}hom\ conjD{\isacharunderscore}def{\isacharparenright}\isanewline
\isamarkupfalse%
\isacommand{lemma}\ disjD{\isacharunderscore}Ret{\isacharunderscore}hom{\isacharcolon}\ {\isachardoublequote}Ret\ {\isacharparenleft}a{\isasymor}b{\isacharparenright}\ {\isacharequal}\ {\isacharparenleft}{\isacharparenleft}Ret\ a{\isacharparenright}\ {\isasymor}\isactrlsub D\ {\isacharparenleft}Ret\ b{\isacharparenright}{\isacharparenright}{\isachardoublequote}\isanewline
\ \ \isamarkupfalse%
\isacommand{by}\ {\isacharparenleft}simp\ add{\isacharcolon}\ lift{\isacharunderscore}Ret{\isacharunderscore}hom\ disjD{\isacharunderscore}def{\isacharparenright}\isanewline
\isamarkupfalse%
\isacommand{lemma}\ impD{\isacharunderscore}Ret{\isacharunderscore}hom{\isacharcolon}\ {\isachardoublequote}Ret\ {\isacharparenleft}a{\isasymlongrightarrow}b{\isacharparenright}\ {\isacharequal}\ {\isacharparenleft}{\isacharparenleft}Ret\ a{\isacharparenright}\ {\isasymlongrightarrow}\isactrlsub D\ {\isacharparenleft}Ret\ b{\isacharparenright}{\isacharparenright}{\isachardoublequote}\isanewline
\ \ \isamarkupfalse%
\isacommand{by}\ {\isacharparenleft}simp\ add{\isacharcolon}\ lift{\isacharunderscore}Ret{\isacharunderscore}hom\ impD{\isacharunderscore}def{\isacharparenright}\isanewline
\isanewline
\isamarkupfalse%
\isacommand{lemma}\ NotD{\isacharunderscore}Ret{\isacharunderscore}hom{\isacharcolon}\ {\isachardoublequote}Ret\ {\isacharparenleft}{\isasymnot}\ P{\isacharparenright}\ {\isacharequal}\ {\isacharparenleft}{\isasymnot}\isactrlsub D\ {\isacharparenleft}Ret\ P{\isacharparenright}{\isacharparenright}{\isachardoublequote}\isanewline
\ \ \isamarkupfalse%
\isacommand{by}{\isacharparenleft}simp\ add{\isacharcolon}\ NotD{\isacharunderscore}def\ impD{\isacharunderscore}Ret{\isacharunderscore}hom{\isacharbrackleft}symmetric{\isacharbrackright}{\isacharparenright}\isamarkupfalse%
%
\begin{isamarkuptext}%
If a formula depending on variable \isa{x} is valid for all \isa{x}, then we
  may also `substitute' it by a \isa{dsef} term.%
\end{isamarkuptext}%
\isamarkuptrue%
\isacommand{lemma}\ dsef{\isacharunderscore}form{\isacharcolon}\ {\isachardoublequote}\ {\isasymforall}x{\isachardot}\ {\isasymturnstile}\ P\ x\ {\isasymLongrightarrow}\ {\isasymforall}b{\isachardot}\ {\isasymturnstile}\ {\isasymUp}\ {\isacharparenleft}do\ {\isacharbraceleft}a{\isasymleftarrow}{\isasymDown}\ b{\isacharsemicolon}\ {\isasymDown}\ {\isacharparenleft}P\ a{\isacharparenright}{\isacharbraceright}{\isacharparenright}{\isachardoublequote}\isanewline
\isamarkupfalse%
\isacommand{proof}\isanewline
\ \ \isamarkupfalse%
\isacommand{fix}\ b\isanewline
\ \ \isamarkupfalse%
\isacommand{assume}\ a{\isadigit{1}}{\isacharcolon}\ {\isachardoublequote}\ {\isasymforall}x{\isachardot}\ {\isasymturnstile}\ P\ x{\isachardoublequote}\isanewline
\ \ \isamarkupfalse%
\isacommand{hence}\ {\isachardoublequote}{\isasymDown}\ {\isacharparenleft}{\isasymUp}\ {\isacharparenleft}do\ {\isacharbraceleft}a{\isasymleftarrow}{\isasymDown}\ {\isacharparenleft}b{\isacharcolon}{\isacharcolon}{\isacharprime}a\ D{\isacharparenright}{\isacharsemicolon}\ {\isasymDown}\ {\isacharparenleft}P\ a{\isacharparenright}{\isacharbraceright}{\isacharparenright}{\isacharparenright}\ {\isacharequal}\ \isanewline
\ \ \ \ \ \ \ \ {\isasymDown}\ {\isacharparenleft}{\isasymUp}\ {\isacharparenleft}do\ {\isacharbraceleft}a{\isasymleftarrow}{\isasymDown}\ {\isacharparenleft}b{\isacharcolon}{\isacharcolon}{\isacharprime}a\ D{\isacharparenright}{\isacharsemicolon}\ ret\ True{\isacharbraceright}{\isacharparenright}{\isacharparenright}{\isachardoublequote}\isanewline
\ \ \ \ \isamarkupfalse%
\isacommand{by}\ {\isacharparenleft}simp\ add{\isacharcolon}\ Valid{\isacharunderscore}simp{\isacharparenright}\ \isanewline
\ \ \isamarkupfalse%
\isacommand{also}\ \isamarkupfalse%
\isacommand{have}\ {\isachardoublequote}{\isasymdots}\ {\isacharequal}\ do\ {\isacharbraceleft}a{\isasymleftarrow}{\isasymDown}\ b{\isacharsemicolon}\ ret\ True{\isacharbraceright}{\isachardoublequote}\isanewline
\ \ \isamarkupfalse%
\isacommand{proof}\ {\isacharparenleft}rule\ Abs{\isacharunderscore}Dsef{\isacharunderscore}inverse{\isacharparenright}\isanewline
\ \ \ \ \isamarkupfalse%
\isacommand{have}\ {\isachardoublequote}dsef\ {\isacharparenleft}do\ {\isacharbraceleft}a{\isasymleftarrow}{\isasymDown}\ b{\isacharsemicolon}\ ret\ True{\isacharbraceright}{\isacharparenright}{\isachardoublequote}\isanewline
\ \ \ \ \ \ \isamarkupfalse%
\isacommand{by}\ {\isacharparenleft}simp\ add{\isacharcolon}\ dsef{\isacharunderscore}ret\ dsef{\isacharunderscore}Rep{\isacharunderscore}Dsef\ dsef{\isacharunderscore}seq{\isacharparenright}\isanewline
\ \ \ \ \isamarkupfalse%
\isacommand{thus}\ {\isachardoublequote}do\ {\isacharbraceleft}a{\isasymleftarrow}{\isasymDown}\ b{\isacharsemicolon}\ ret\ True{\isacharbraceright}\ {\isasymin}\ Dsef{\isachardoublequote}\ \isamarkupfalse%
\isacommand{by}\ {\isacharparenleft}simp\ add{\isacharcolon}\ Dsef{\isacharunderscore}def{\isacharparenright}\isanewline
\ \ \isamarkupfalse%
\isacommand{qed}\isanewline
\ \ \isamarkupfalse%
\isacommand{also}\ \isamarkupfalse%
\isacommand{have}\ {\isachardoublequote}{\isasymdots}\ {\isacharequal}\ ret\ True{\isachardoublequote}\ \isamarkupfalse%
\isacommand{by}\ {\isacharparenleft}simp\ add{\isacharcolon}\ dis{\isacharunderscore}left{\isadigit{2}}\ dsef{\isacharunderscore}dis{\isacharbrackleft}OF\ dsef{\isacharunderscore}Rep{\isacharunderscore}Dsef{\isacharbrackright}{\isacharparenright}\isanewline
\ \ \isamarkupfalse%
\isacommand{finally}\ \isamarkupfalse%
\isacommand{show}\ {\isachardoublequote}{\isasymturnstile}\ {\isasymUp}\ {\isacharparenleft}do\ {\isacharbraceleft}a{\isasymleftarrow}{\isasymDown}\ {\isacharparenleft}b{\isacharcolon}{\isacharcolon}{\isacharprime}a\ D{\isacharparenright}{\isacharsemicolon}\ {\isasymDown}\ {\isacharparenleft}P\ a{\isacharparenright}{\isacharbraceright}{\isacharparenright}{\isachardoublequote}\isanewline
\ \ \ \ \isamarkupfalse%
\isacommand{by}\ {\isacharparenleft}simp\ add{\isacharcolon}\ Valid{\isacharunderscore}simp{\isacharparenright}\isanewline
\isamarkupfalse%
\isacommand{qed}\isamarkupfalse%
%
\begin{isamarkuptext}%
Every true formula may be injected into \isa{bool\ D} by \isa{Ret} to 
  yield a valid formula of dynamic logic. And the converse also holds!
  \label{isa:valid-ret}%
\end{isamarkuptext}%
\isamarkuptrue%
\isacommand{theorem}\ Valid{\isacharunderscore}Ret\ {\isacharbrackleft}simp{\isacharbrackright}{\isacharcolon}\ {\isachardoublequote}{\isacharparenleft}{\isasymturnstile}\ Ret\ P{\isacharparenright}\ {\isacharequal}\ P{\isachardoublequote}\isanewline
\isamarkupfalse%
\isacommand{proof}\ \isanewline
\ \ \isamarkupfalse%
\isacommand{assume}\ p{\isacharcolon}\ {\isachardoublequote}P{\isachardoublequote}\isanewline
\ \ \isamarkupfalse%
\isacommand{have}\ {\isachardoublequote}{\isasymDown}\ {\isacharparenleft}Ret\ P{\isacharparenright}\ {\isacharequal}\ do\ {\isacharbraceleft}x{\isasymleftarrow}{\isasymDown}\ {\isacharparenleft}Ret\ P{\isacharparenright}{\isacharsemicolon}\ ret\ True{\isacharbraceright}{\isachardoublequote}\isanewline
\ \ \isamarkupfalse%
\isacommand{proof}\ {\isacharminus}\isanewline
\ \ \ \ \isamarkupfalse%
\isacommand{have}\ {\isachardoublequote}dsef\ {\isacharparenleft}{\isasymDown}\ {\isacharparenleft}Ret\ P{\isacharparenright}{\isacharparenright}{\isachardoublequote}\ \isamarkupfalse%
\isacommand{by}\ {\isacharparenleft}rule\ dsef{\isacharunderscore}Rep{\isacharunderscore}Dsef{\isacharparenright}\isanewline
\ \ \ \ \isamarkupfalse%
\isacommand{hence}\ ds{\isacharcolon}\ {\isachardoublequote}dis\ {\isacharparenleft}{\isasymDown}\ {\isacharparenleft}Ret\ P{\isacharparenright}{\isacharparenright}{\isachardoublequote}\ \isamarkupfalse%
\isacommand{by}\ {\isacharparenleft}simp\ only{\isacharcolon}\ dsef{\isacharunderscore}def{\isacharparenright}\isanewline
\ \ \ \ \isamarkupfalse%
\isacommand{have}\ {\isachardoublequote}{\isasymDown}\ {\isacharparenleft}Ret\ P{\isacharparenright}\ {\isacharequal}\ ret\ P{\isachardoublequote}\ \isamarkupfalse%
\isacommand{by}\ {\isacharparenleft}rule\ Ret{\isacharunderscore}ret{\isacharparenright}\isanewline
\ \ \ \ \isamarkupfalse%
\isacommand{also}\ \isamarkupfalse%
\isacommand{from}\ p\ \isamarkupfalse%
\isacommand{have}\ {\isachardoublequote}{\isasymdots}\ {\isacharequal}\ ret\ True{\isachardoublequote}\ \isamarkupfalse%
\isacommand{by}\ simp\isanewline
\ \ \ \ \isamarkupfalse%
\isacommand{also}\ \isamarkupfalse%
\isacommand{from}\ ds\ \isamarkupfalse%
\isacommand{have}\ {\isachardoublequote}{\isasymdots}\ {\isacharequal}\ do\ {\isacharbraceleft}{\isasymDown}\ {\isacharparenleft}Ret\ P{\isacharparenright}{\isacharsemicolon}\ ret\ True{\isacharbraceright}{\isachardoublequote}\ \isamarkupfalse%
\isacommand{by}\ {\isacharparenleft}rule\ dis{\isacharunderscore}left{\isacharbrackleft}symmetric{\isacharbrackright}{\isacharparenright}\isanewline
\ \ \ \ \isamarkupfalse%
\isacommand{finally}\ \isamarkupfalse%
\isacommand{show}\ {\isacharquery}thesis\ \isamarkupfalse%
\isacommand{by}\ {\isacharparenleft}simp\ only{\isacharcolon}\ seq{\isacharunderscore}def{\isacharparenright}\isanewline
\ \ \isamarkupfalse%
\isacommand{qed}\isanewline
\ \ \isamarkupfalse%
\isacommand{thus}\ {\isachardoublequote}{\isasymturnstile}\ Ret\ P{\isachardoublequote}\ \isamarkupfalse%
\isacommand{by}\ {\isacharparenleft}simp\ only{\isacharcolon}\ Valid{\isacharunderscore}def{\isacharparenright}\isanewline
\isamarkupfalse%
\isacommand{next}\isanewline
\ \ \isamarkupfalse%
\isacommand{assume}\ rp{\isacharcolon}\ {\isachardoublequote}{\isasymturnstile}\ Ret\ P{\isachardoublequote}\isanewline
\ \ \isamarkupfalse%
\isacommand{hence}\ {\isachardoublequote}{\isasymDown}\ {\isacharparenleft}Ret\ P{\isacharparenright}\ {\isacharequal}\ ret\ True{\isachardoublequote}\ \isamarkupfalse%
\isacommand{by}\ {\isacharparenleft}rule\ iffD{\isadigit{1}}{\isacharbrackleft}OF\ Valid{\isacharunderscore}simp{\isacharbrackright}{\isacharparenright}\isanewline
\ \ \isamarkupfalse%
\isacommand{hence}\ {\isachardoublequote}ret\ P\ {\isacharequal}\ ret\ True{\isachardoublequote}\isanewline
\ \ \ \ \isamarkupfalse%
\isacommand{by}\ {\isacharparenleft}simp\ add{\isacharcolon}\ Ret{\isacharunderscore}def\ Dsef{\isacharunderscore}def\ Abs{\isacharunderscore}Dsef{\isacharunderscore}inverse{\isacharparenright}\isanewline
\ \ \isamarkupfalse%
\isacommand{hence}\ {\isachardoublequote}P\ {\isacharequal}\ True{\isachardoublequote}\ \isamarkupfalse%
\isacommand{by}\ {\isacharparenleft}rule\ ret{\isacharunderscore}inject{\isacharparenright}\isanewline
\ \ \isamarkupfalse%
\isacommand{thus}\ {\isachardoublequote}P{\isachardoublequote}\ \isamarkupfalse%
\isacommand{by}\ rules\isanewline
\isamarkupfalse%
\isacommand{qed}\isamarkupfalse%
%
\begin{isamarkuptext}%
A bit more tedious, but conversely to \isa{Valid{\isacharunderscore}simp} it is also true
  that every valid formula that is a negation equals \isa{ret\ False}.%
\end{isamarkuptext}%
\isamarkuptrue%
\isacommand{lemma}\ Valid{\isacharunderscore}not{\isacharunderscore}eq{\isacharunderscore}ret{\isacharunderscore}False{\isacharcolon}\ {\isachardoublequote}{\isacharparenleft}{\isasymturnstile}\ {\isasymnot}\isactrlsub D\ b{\isacharparenright}\ {\isacharequal}\ {\isacharparenleft}{\isasymDown}\ b\ {\isacharequal}\ ret\ False{\isacharparenright}{\isachardoublequote}\isanewline
\isamarkupfalse%
\isacommand{proof}\isanewline
\ \ \isamarkupfalse%
\isacommand{assume}\ {\isachardoublequote}{\isasymturnstile}\ {\isasymnot}\isactrlsub D\ b{\isachardoublequote}\isanewline
\ \ \isamarkupfalse%
\isacommand{hence}\ nt{\isacharcolon}\ {\isachardoublequote}{\isasymDown}\ {\isacharparenleft}{\isasymnot}\isactrlsub D\ b{\isacharparenright}\ {\isacharequal}\ ret\ True{\isachardoublequote}\ \isamarkupfalse%
\isacommand{by}\ {\isacharparenleft}simp\ add{\isacharcolon}\ Valid{\isacharunderscore}simp{\isacharparenright}\isanewline
\ \ \isamarkupfalse%
\isacommand{show}\ {\isachardoublequote}{\isasymDown}\ b\ {\isacharequal}\ ret\ False{\isachardoublequote}\isanewline
\ \ \isamarkupfalse%
\isacommand{proof}\ {\isacharminus}\isanewline
\ \ \ \ \isamarkupfalse%
\isacommand{have}\ {\isachardoublequote}dsef\ {\isacharparenleft}do\ {\isacharbraceleft}x{\isasymleftarrow}{\isasymDown}\ b{\isacharsemicolon}\ ret\ {\isacharparenleft}{\isasymnot}\ x{\isacharparenright}{\isacharbraceright}{\isacharparenright}{\isachardoublequote}\isanewline
\ \ \ \ \ \ \isamarkupfalse%
\isacommand{by}\ {\isacharparenleft}rule\ weak{\isacharunderscore}dsef{\isacharunderscore}seq{\isacharcomma}\ rule\ dsef{\isacharunderscore}Rep{\isacharunderscore}Dsef{\isacharparenright}\ \isanewline
\ \ \ \ \isamarkupfalse%
\isacommand{hence}\ bnnb{\isacharcolon}\ {\isachardoublequote}b\ {\isacharequal}\ {\isacharparenleft}{\isasymnot}\isactrlsub D\ {\isacharparenleft}{\isasymnot}\isactrlsub D\ b{\isacharparenright}{\isacharparenright}{\isachardoublequote}\isanewline
\ \ \ \ \ \ \isamarkupfalse%
\isacommand{by}\ {\isacharparenleft}simp\ add{\isacharcolon}\ NotD{\isacharunderscore}def\ impD{\isacharunderscore}def\ liftM{\isadigit{2}}{\isacharunderscore}def\ \isanewline
\ \ \ \ \ \ \ \ \ \ \ \ \ \ \ \ \ \ \ \ Ret{\isacharunderscore}ret\ Abs{\isacharunderscore}Dsef{\isacharunderscore}inverse\ Dsef{\isacharunderscore}def\ mon{\isacharunderscore}ctr\ Rep{\isacharunderscore}Dsef{\isacharunderscore}inverse{\isacharparenright}\ \isanewline
\ \ \ \ \isamarkupfalse%
\isacommand{from}\ nt\ \isamarkupfalse%
\isacommand{have}\ {\isachardoublequote}{\isasymUp}\ {\isacharparenleft}{\isasymDown}\ {\isacharparenleft}{\isasymnot}\isactrlsub D\ b{\isacharparenright}{\isacharparenright}\ {\isacharequal}\ Ret\ True{\isachardoublequote}\ \isamarkupfalse%
\isacommand{by}\ {\isacharparenleft}simp\ add{\isacharcolon}\ Ret{\isacharunderscore}def{\isacharparenright}\isanewline
\ \ \ \ \isamarkupfalse%
\isacommand{hence}\ {\isachardoublequote}{\isacharparenleft}{\isasymnot}\isactrlsub D\ b{\isacharparenright}\ {\isacharequal}\ Ret\ True{\isachardoublequote}\ \isamarkupfalse%
\isacommand{by}\ {\isacharparenleft}simp\ only{\isacharcolon}\ Rep{\isacharunderscore}Dsef{\isacharunderscore}inverse{\isacharparenright}\isanewline
\ \ \ \ \isamarkupfalse%
\isacommand{hence}\ {\isachardoublequote}{\isacharparenleft}{\isasymnot}\isactrlsub D\ {\isacharparenleft}{\isasymnot}\isactrlsub D\ b{\isacharparenright}{\isacharparenright}\ {\isacharequal}\ {\isacharparenleft}{\isasymnot}\isactrlsub D\ {\isacharparenleft}Ret\ True{\isacharparenright}{\isacharparenright}{\isachardoublequote}\ \isamarkupfalse%
\isacommand{by}\ simp\isanewline
\ \ \ \ \isamarkupfalse%
\isacommand{with}\ bnnb\ \isamarkupfalse%
\isacommand{have}\ {\isachardoublequote}b\ {\isacharequal}\ Ret\ {\isacharparenleft}{\isasymnot}\ True{\isacharparenright}{\isachardoublequote}\ \isamarkupfalse%
\isacommand{by}\ {\isacharparenleft}simp\ add{\isacharcolon}\ NotD{\isacharunderscore}Ret{\isacharunderscore}hom{\isacharbrackleft}symmetric{\isacharbrackright}{\isacharparenright}\isanewline
\ \ \ \ \isamarkupfalse%
\isacommand{thus}\ {\isacharquery}thesis\ \isamarkupfalse%
\isacommand{by}\ {\isacharparenleft}simp\ add{\isacharcolon}\ Ret{\isacharunderscore}ret{\isacharparenright}\isanewline
\ \ \isamarkupfalse%
\isacommand{qed}\isanewline
\isamarkupfalse%
\isacommand{next}\isanewline
\ \ \isamarkupfalse%
\isacommand{assume}\ {\isachardoublequote}{\isasymDown}\ b\ {\isacharequal}\ ret\ False{\isachardoublequote}\isanewline
\ \ \isamarkupfalse%
\isacommand{hence}\ {\isachardoublequote}{\isasymUp}\ {\isacharparenleft}{\isasymDown}\ b{\isacharparenright}\ {\isacharequal}\ {\isasymUp}\ {\isacharparenleft}ret\ False{\isacharparenright}{\isachardoublequote}\ \isamarkupfalse%
\isacommand{by}\ simp\isanewline
\ \ \isamarkupfalse%
\isacommand{hence}\ bf{\isacharcolon}\ {\isachardoublequote}b\ {\isacharequal}\ Ret\ False{\isachardoublequote}\ \isamarkupfalse%
\isacommand{by}\ {\isacharparenleft}simp\ add{\isacharcolon}\ Rep{\isacharunderscore}Dsef{\isacharunderscore}inverse\ Ret{\isacharunderscore}def{\isacharparenright}\isanewline
\ \ \isamarkupfalse%
\isacommand{have}\ {\isachardoublequote}{\isasymDown}\ {\isacharparenleft}{\isasymnot}\isactrlsub D\ b{\isacharparenright}\ {\isacharequal}\ ret\ True{\isachardoublequote}\isanewline
\ \ \isamarkupfalse%
\isacommand{proof}\ {\isacharminus}\isanewline
\ \ \ \ \isamarkupfalse%
\isacommand{from}\ bf\ \isamarkupfalse%
\isacommand{have}\ {\isachardoublequote}{\isasymDown}\ {\isacharparenleft}{\isasymnot}\isactrlsub D\ b{\isacharparenright}\ {\isacharequal}\ {\isasymDown}\ {\isacharparenleft}Ret\ False\ {\isasymlongrightarrow}\isactrlsub D\ Ret\ False{\isacharparenright}{\isachardoublequote}\isanewline
\ \ \ \ \ \ \isamarkupfalse%
\isacommand{by}\ {\isacharparenleft}simp\ add{\isacharcolon}\ NotD{\isacharunderscore}def{\isacharparenright}\isanewline
\ \ \ \ \isamarkupfalse%
\isacommand{also}\ \isamarkupfalse%
\isacommand{have}\ {\isachardoublequote}{\isasymdots}\ {\isacharequal}\ {\isasymDown}\ {\isacharparenleft}Ret\ True{\isacharparenright}{\isachardoublequote}\ \isanewline
\ \ \ \ \isamarkupfalse%
\isacommand{proof}\ {\isacharminus}\isanewline
\ \ \ \ \ \ \isamarkupfalse%
\isacommand{have}\ {\isachardoublequote}{\isacharparenleft}Ret\ False\ {\isasymlongrightarrow}\isactrlsub D\ Ret\ False{\isacharparenright}\ {\isacharequal}\ Ret\ {\isacharparenleft}False\ {\isasymlongrightarrow}\ False{\isacharparenright}{\isachardoublequote}\isanewline
\ \ \ \ \ \ \ \ \isamarkupfalse%
\isacommand{by}\ {\isacharparenleft}rule\ impD{\isacharunderscore}Ret{\isacharunderscore}hom{\isacharbrackleft}symmetric{\isacharbrackright}{\isacharparenright}\isanewline
\ \ \ \ \ \ \isamarkupfalse%
\isacommand{thus}\ {\isacharquery}thesis\ \isamarkupfalse%
\isacommand{by}\ simp\isanewline
\ \ \ \ \isamarkupfalse%
\isacommand{qed}\isanewline
\ \ \ \ \isamarkupfalse%
\isacommand{also}\ \isamarkupfalse%
\isacommand{have}\ {\isachardoublequote}{\isasymdots}\ {\isacharequal}\ ret\ True{\isachardoublequote}\ \isamarkupfalse%
\isacommand{by}\ {\isacharparenleft}rule\ Ret{\isacharunderscore}ret{\isacharparenright}\isanewline
\ \ \ \ \isamarkupfalse%
\isacommand{finally}\ \isamarkupfalse%
\isacommand{show}\ {\isacharquery}thesis\ \isamarkupfalse%
\isacommand{{\isachardot}}\isanewline
\ \ \isamarkupfalse%
\isacommand{qed}\isanewline
\ \ \isamarkupfalse%
\isacommand{thus}\ {\isachardoublequote}{\isasymturnstile}\ {\isasymnot}\isactrlsub D\ b{\isachardoublequote}\ \isamarkupfalse%
\isacommand{by}\ {\isacharparenleft}simp\ only{\isacharcolon}\ Valid{\isacharunderscore}simp{\isacharparenright}\isanewline
\isamarkupfalse%
\isacommand{qed}\isamarkupfalse%
%
\begin{isamarkuptext}%
Lemmas \isa{Valid{\isacharunderscore}simp}, \isa{Valid{\isacharunderscore}not{\isacharunderscore}eq{\isacharunderscore}ret{\isacharunderscore}False} 
  and \isa{Valid{\isacharunderscore}Ret} show that, since the classical type \isa{bool} 
  is taken as the carrier of truth values, the whole calculus is classical.%
\end{isamarkuptext}%
\isamarkuptrue%
%
\isamarkupsubsection{Setting up the Simplifier for Propositional Reasoning%
}
\isamarkuptrue%
%
\begin{isamarkuptext}%
Since natural deduction rules don't get us far in the calculus of global
  validity judgments (in particular, we do not have an analogon for
  the implication introduction rule), we algebraize it and perform
  proofs by term manipulation.
 
  All these axioms are in fact provable; it is just the shortage of time that
  forces us to impose them directly.  
  \label{isa:simp-pdl-taut}%
\end{isamarkuptext}%
\isamarkuptrue%
\isacommand{constdefs}\isanewline
\ \ xorD\ {\isacharcolon}{\isacharcolon}\ {\isachardoublequote}{\isacharbrackleft}bool\ D{\isacharcomma}\ bool\ D{\isacharbrackright}\ {\isasymRightarrow}\ bool\ D{\isachardoublequote}\ \ \ \ \ \ {\isacharparenleft}\isakeyword{infixr}\ {\isachardoublequote}{\isasymoplus}\isactrlsub D{\isachardoublequote}\ {\isadigit{2}}{\isadigit{0}}{\isacharparenright}\isanewline
\ \ {\isachardoublequote}xorD\ P\ Q\ \ {\isasymequiv}\ \ {\isacharparenleft}P\ {\isasymand}\isactrlsub D\ {\isasymnot}\isactrlsub D\ Q{\isacharparenright}\ {\isasymor}\isactrlsub D\ {\isacharparenleft}{\isasymnot}\isactrlsub D\ P\ {\isasymand}\isactrlsub D\ Q{\isacharparenright}{\isachardoublequote}\isanewline
\isanewline
\ \ \isanewline
\isamarkupfalse%
\isacommand{axioms}\isanewline
\ \ apl{\isacharunderscore}and{\isacharunderscore}assoc{\isacharcolon}\ \ \ {\isachardoublequote}{\isacharparenleft}{\isacharparenleft}P\ {\isasymand}\isactrlsub D\ Q{\isacharparenright}\ {\isasymand}\isactrlsub D\ R{\isacharparenright}\ {\isacharequal}\ {\isacharparenleft}P\ {\isasymand}\isactrlsub D\ {\isacharparenleft}Q\ {\isasymand}\isactrlsub D\ R{\isacharparenright}{\isacharparenright}{\isachardoublequote}\isanewline
\ \ apl{\isacharunderscore}xor{\isacharunderscore}assoc{\isacharcolon}\ \ \ \ {\isachardoublequote}{\isacharparenleft}{\isacharparenleft}P\ {\isasymoplus}\isactrlsub D\ Q{\isacharparenright}\ {\isasymoplus}\isactrlsub D\ R{\isacharparenright}\ {\isacharequal}\ {\isacharparenleft}P\ {\isasymoplus}\isactrlsub D\ {\isacharparenleft}Q\ {\isasymoplus}\isactrlsub D\ R{\isacharparenright}{\isacharparenright}{\isachardoublequote}\isanewline
\ \ apl{\isacharunderscore}and{\isacharunderscore}comm{\isacharcolon}\ \ \ \ \ {\isachardoublequote}{\isacharparenleft}P\ {\isasymand}\isactrlsub D\ Q{\isacharparenright}\ {\isacharequal}\ {\isacharparenleft}Q\ {\isasymand}\isactrlsub D\ P{\isacharparenright}{\isachardoublequote}\isanewline
\ \ apl{\isacharunderscore}xor{\isacharunderscore}comm{\isacharcolon}\ \ \ \ \ {\isachardoublequote}{\isacharparenleft}P\ {\isasymoplus}\isactrlsub D\ Q{\isacharparenright}\ {\isacharequal}\ {\isacharparenleft}Q\ {\isasymoplus}\isactrlsub D\ P{\isacharparenright}{\isachardoublequote}\isanewline
\ \ apl{\isacharunderscore}and{\isacharunderscore}LC{\isacharcolon}\ \ \ \ \ \ \ {\isachardoublequote}{\isacharparenleft}P\ {\isasymand}\isactrlsub D\ {\isacharparenleft}Q\ {\isasymand}\isactrlsub D\ R{\isacharparenright}{\isacharparenright}\ {\isacharequal}\ {\isacharparenleft}Q\ {\isasymand}\isactrlsub D\ {\isacharparenleft}P\ {\isasymand}\isactrlsub D\ R{\isacharparenright}{\isacharparenright}{\isachardoublequote}\isanewline
\ \ apl{\isacharunderscore}xor{\isacharunderscore}LC{\isacharcolon}\ \ \ \ \ \ \ {\isachardoublequote}{\isacharparenleft}P\ {\isasymoplus}\isactrlsub D\ {\isacharparenleft}Q\ {\isasymoplus}\isactrlsub D\ R{\isacharparenright}{\isacharparenright}\ {\isacharequal}\ {\isacharparenleft}Q\ {\isasymoplus}\isactrlsub D\ {\isacharparenleft}P\ {\isasymoplus}\isactrlsub D\ R{\isacharparenright}{\isacharparenright}{\isachardoublequote}\isanewline
\ \ apl{\isacharunderscore}and{\isacharunderscore}True{\isacharunderscore}r{\isacharcolon}\ \ \ {\isachardoublequote}{\isacharparenleft}P\ {\isasymand}\isactrlsub D\ Ret\ True{\isacharparenright}\ {\isacharequal}\ P{\isachardoublequote}\isanewline
\ \ apl{\isacharunderscore}and{\isacharunderscore}True{\isacharunderscore}l{\isacharcolon}\ \ \ {\isachardoublequote}{\isacharparenleft}Ret\ True\ {\isasymand}\isactrlsub D\ P{\isacharparenright}\ {\isacharequal}\ P{\isachardoublequote}\isanewline
\ \ apl{\isacharunderscore}and{\isacharunderscore}absorb{\isacharcolon}\ \ \ {\isachardoublequote}{\isacharparenleft}P\ {\isasymand}\isactrlsub D\ P{\isacharparenright}\ {\isacharequal}\ P{\isachardoublequote}\isanewline
\ \ apl{\isacharunderscore}and{\isacharunderscore}absorb{\isadigit{2}}{\isacharcolon}\ \ {\isachardoublequote}{\isacharparenleft}P\ {\isasymand}\isactrlsub D\ {\isacharparenleft}P\ {\isasymand}\isactrlsub D\ Q{\isacharparenright}{\isacharparenright}\ {\isacharequal}\ {\isacharparenleft}P\ {\isasymand}\isactrlsub D\ Q{\isacharparenright}{\isachardoublequote}\isanewline
\ \ apl{\isacharunderscore}and{\isacharunderscore}False{\isacharunderscore}l{\isacharcolon}\ \ {\isachardoublequote}{\isacharparenleft}Ret\ False\ {\isasymand}\isactrlsub D\ P{\isacharparenright}\ {\isacharequal}\ Ret\ False{\isachardoublequote}\isanewline
\ \ apl{\isacharunderscore}and{\isacharunderscore}False{\isacharunderscore}r{\isacharcolon}\ \ {\isachardoublequote}{\isacharparenleft}P\ {\isasymand}\isactrlsub D\ Ret\ False{\isacharparenright}\ {\isacharequal}\ Ret\ False{\isachardoublequote}\isanewline
\ \ apl{\isacharunderscore}xor{\isacharunderscore}False{\isacharunderscore}r{\isacharcolon}\ \ {\isachardoublequote}{\isacharparenleft}P\ {\isasymoplus}\isactrlsub D\ Ret\ False{\isacharparenright}\ {\isacharequal}\ P{\isachardoublequote}\isanewline
\ \ apl{\isacharunderscore}xor{\isacharunderscore}False{\isacharunderscore}l{\isacharcolon}\ \ {\isachardoublequote}{\isacharparenleft}Ret\ False\ {\isasymoplus}\isactrlsub D\ P{\isacharparenright}\ {\isacharequal}\ P{\isachardoublequote}\isanewline
\ \ apl{\isacharunderscore}xor{\isacharunderscore}contr{\isacharcolon}\ \ \ \ {\isachardoublequote}{\isacharparenleft}P\ {\isasymoplus}\isactrlsub D\ P{\isacharparenright}\ {\isacharequal}\ Ret\ False{\isachardoublequote}\isanewline
\ \ apl{\isacharunderscore}xor{\isacharunderscore}contr{\isadigit{2}}{\isacharcolon}\ \ \ {\isachardoublequote}{\isacharparenleft}P\ {\isasymoplus}\isactrlsub D\ {\isacharparenleft}P\ {\isasymoplus}\isactrlsub D\ Q{\isacharparenright}{\isacharparenright}\ {\isacharequal}\ Q{\isachardoublequote}\isanewline
\ \ apl{\isacharunderscore}and{\isacharunderscore}ldist{\isacharcolon}\ \ \ \ {\isachardoublequote}{\isacharparenleft}P\ {\isasymand}\isactrlsub D\ {\isacharparenleft}Q\ {\isasymoplus}\isactrlsub D\ R{\isacharparenright}{\isacharparenright}\ {\isacharequal}\ {\isacharparenleft}{\isacharparenleft}P\ {\isasymand}\isactrlsub D\ Q{\isacharparenright}\ {\isasymoplus}\isactrlsub D\ {\isacharparenleft}P\ {\isasymand}\isactrlsub D\ R{\isacharparenright}{\isacharparenright}{\isachardoublequote}\isanewline
\ \ apl{\isacharunderscore}and{\isacharunderscore}rdist{\isacharcolon}\ \ \ \ {\isachardoublequote}{\isacharparenleft}{\isacharparenleft}P\ {\isasymoplus}\isactrlsub D\ Q{\isacharparenright}\ {\isasymand}\isactrlsub D\ R{\isacharparenright}\ {\isacharequal}\ {\isacharparenleft}{\isacharparenleft}P\ {\isasymand}\isactrlsub D\ R{\isacharparenright}\ {\isasymoplus}\isactrlsub D\ {\isacharparenleft}Q\ {\isasymand}\isactrlsub D\ R{\isacharparenright}{\isacharparenright}{\isachardoublequote}\isanewline
\ \ %
\isamarkupcmt{Expressing the connectives by conjunction and exclusive or%
}
\isanewline
\ \ apl{\isacharunderscore}imp{\isacharunderscore}xor{\isacharcolon}\ \ \ \ \ \ {\isachardoublequote}{\isacharparenleft}P\ {\isasymlongrightarrow}\isactrlsub D\ Q{\isacharparenright}\ {\isacharequal}\ {\isacharparenleft}{\isacharparenleft}P\ {\isasymand}\isactrlsub D\ Q{\isacharparenright}\ {\isasymoplus}\isactrlsub D\ P\ {\isasymoplus}\isactrlsub D\ Ret\ True{\isacharparenright}{\isachardoublequote}\isanewline
\ \ apl{\isacharunderscore}or{\isacharunderscore}xor{\isacharcolon}\ \ \ \ \ \ \ {\isachardoublequote}{\isacharparenleft}P\ {\isasymor}\isactrlsub D\ Q{\isacharparenright}\ {\isacharequal}\ {\isacharparenleft}P\ {\isasymoplus}\isactrlsub D\ Q\ {\isasymoplus}\isactrlsub D\ {\isacharparenleft}P\ {\isasymand}\isactrlsub D\ Q{\isacharparenright}{\isacharparenright}{\isachardoublequote}\isanewline
\ \ apl{\isacharunderscore}not{\isacharunderscore}xor{\isacharcolon}\ \ \ \ \ \ {\isachardoublequote}{\isacharparenleft}{\isasymnot}\isactrlsub D\ P{\isacharparenright}\ {\isacharequal}\ {\isacharparenleft}P\ {\isasymoplus}\isactrlsub D\ Ret\ True{\isacharparenright}{\isachardoublequote}\isanewline
\ \ apl{\isacharunderscore}iff{\isacharunderscore}xor{\isacharcolon}\ \ \ \ \ \ {\isachardoublequote}{\isacharparenleft}P\ {\isasymlongleftrightarrow}\isactrlsub D\ Q{\isacharparenright}\ {\isacharequal}\ {\isacharparenleft}P\ {\isasymoplus}\isactrlsub D\ Q\ {\isasymoplus}\isactrlsub D\ Ret\ True{\isacharparenright}{\isachardoublequote}\isamarkupfalse%
%
\begin{isamarkuptext}%
\isa{pdl{\isacharunderscore}taut} is the collection of all these rules, so that
  they can be handed over to the simplifier conveniently.
  
  This set of rewrite rules is complete with respect to normalisation
  of propositional tautologies to their normal form 
  \isa{Ret\ True}. Hence, we can prove monadic tautologies 
  in one fell swoop by applying 
  the tactic \isa{{\isacharparenleft}simp\ only{\isacharcolon}\ pdl{\isacharunderscore}taut\ Valid{\isacharunderscore}Ret{\isacharparenright}}.%
\end{isamarkuptext}%
\isamarkuptrue%
\isacommand{lemmas}\ pdl{\isacharunderscore}taut\ {\isacharequal}\ \ %
\isamarkupcmt{\dots all axioms above%
}
\isanewline
\isanewline
\isanewline
\isamarkupfalse%
\isacommand{lemmas}\ mon{\isacharunderscore}prop{\isacharunderscore}reason\ {\isacharequal}\ Abs{\isacharunderscore}Dsef{\isacharunderscore}inverse\ dsef{\isacharunderscore}liftM{\isadigit{2}}\ \isanewline
\ \ Dsef{\isacharunderscore}def\ conjD{\isacharunderscore}def\ disjD{\isacharunderscore}def\ impD{\isacharunderscore}def\ NotD{\isacharunderscore}def\isamarkupfalse%
%
\begin{isamarkuptext}%
A proof showing in what manner the above axioms may be proved.%
\end{isamarkuptext}%
\isamarkuptrue%
\isacommand{lemma}\ {\isachardoublequote}{\isacharparenleft}P\ {\isasymand}\isactrlsub D\ {\isacharparenleft}{\isasymnot}\isactrlsub D\ P{\isacharparenright}{\isacharparenright}\ {\isacharequal}\ Ret\ False{\isachardoublequote}\isanewline
\ \ \isamarkupfalse%
\isacommand{apply}{\isacharparenleft}simp\ add{\isacharcolon}\ mon{\isacharunderscore}prop{\isacharunderscore}reason{\isacharcomma}\ simp\ only{\isacharcolon}\ liftM{\isadigit{2}}{\isacharunderscore}def{\isacharparenright}\isanewline
\ \ \isamarkupfalse%
\isacommand{apply}{\isacharparenleft}unfold\ Ret{\isacharunderscore}def{\isacharparenright}\isanewline
\ \ \isamarkupfalse%
\isacommand{apply}{\isacharparenleft}rule\ cong{\isacharbrackleft}of\ Abs{\isacharunderscore}Dsef\ Abs{\isacharunderscore}Dsef{\isacharbrackright}{\isacharcomma}\ rule\ refl{\isacharparenright}\isanewline
\ \ \isamarkupfalse%
\isacommand{apply}{\isacharparenleft}simp\ add{\isacharcolon}\ Abs{\isacharunderscore}Dsef{\isacharunderscore}inverse\ Dsef{\isacharunderscore}def{\isacharparenright}\isanewline
\ \ \isamarkupfalse%
\isacommand{apply}{\isacharparenleft}simp\ add{\isacharcolon}\ mon{\isacharunderscore}ctr\ del{\isacharcolon}\ bind{\isacharunderscore}assoc{\isacharparenright}\isanewline
\ \ \isamarkupfalse%
\isacommand{apply}{\isacharparenleft}simp\ add{\isacharcolon}\ cp{\isacharunderscore}arb\ dsef{\isacharunderscore}cp{\isacharbrackleft}OF\ dsef{\isacharunderscore}Rep{\isacharunderscore}Dsef{\isacharbrackright}{\isacharparenright}\isanewline
\ \ \isamarkupfalse%
\isacommand{apply}{\isacharparenleft}rule\ dis{\isacharunderscore}left{\isadigit{2}}{\isacharparenright}\isanewline
\ \ \isamarkupfalse%
\isacommand{apply}{\isacharparenleft}rule\ dsef{\isacharunderscore}dis{\isacharbrackleft}OF\ dsef{\isacharunderscore}Rep{\isacharunderscore}Dsef{\isacharbrackright}{\isacharparenright}\isanewline
\isamarkupfalse%
\isacommand{done}\isamarkupfalse%
%
\begin{isamarkuptext}%
And another one, following the same scheme, only that the simplifier now
  needs help from the classical reasoner to finish.%
\end{isamarkuptext}%
\isamarkuptrue%
\isacommand{lemma}\ {\isachardoublequote}{\isacharparenleft}P\ {\isasymoplus}\isactrlsub D\ Q{\isacharparenright}\ {\isacharequal}\ {\isacharparenleft}Q\ {\isasymoplus}\isactrlsub D\ P{\isacharparenright}{\isachardoublequote}\isanewline
\ \ \isamarkupfalse%
\isacommand{apply}{\isacharparenleft}simp\ add{\isacharcolon}\ disjD{\isacharunderscore}def\ conjD{\isacharunderscore}def\ NotD{\isacharunderscore}def\ impD{\isacharunderscore}def\ liftM{\isadigit{2}}{\isacharunderscore}def\ xorD{\isacharunderscore}def\ Ret{\isacharunderscore}def{\isacharparenright}\isanewline
\ \ \isamarkupfalse%
\isacommand{apply}{\isacharparenleft}simp\ add{\isacharcolon}\ Abs{\isacharunderscore}Dsef{\isacharunderscore}inverse\ Dsef{\isacharunderscore}def\ dsef{\isacharunderscore}seq\ dsef{\isacharunderscore}Rep{\isacharunderscore}Dsef\ mon{\isacharunderscore}ctr\ del{\isacharcolon}\ bind{\isacharunderscore}assoc{\isacharparenright}\isanewline
\ \ \isamarkupfalse%
\isacommand{apply}{\isacharparenleft}simp\ add{\isacharcolon}\ commute{\isacharunderscore}dsef{\isacharbrackleft}of\ {\isachardoublequote}{\isasymDown}\ Q{\isachardoublequote}\ {\isachardoublequote}{\isasymDown}\ P{\isachardoublequote}{\isacharbrackright}{\isacharparenright}\isanewline
\ \ \isamarkupfalse%
\isacommand{apply}{\isacharparenleft}simp\ add{\isacharcolon}\ dsef{\isacharunderscore}cp\ cp{\isacharunderscore}arb{\isacharparenright}\isanewline
\ \ \isamarkupfalse%
\isacommand{apply}{\isacharparenleft}subgoal{\isacharunderscore}tac\ {\isachardoublequote}{\isasymforall}x\ y{\isachardot}\ {\isacharparenleft}x\ {\isasymand}\ {\isasymnot}\ y\ {\isasymor}\ {\isasymnot}\ x\ {\isasymand}\ y{\isacharparenright}\ {\isacharequal}\ {\isacharparenleft}y\ {\isasymand}\ {\isasymnot}\ x\ {\isasymor}\ {\isasymnot}\ y\ {\isasymand}\ x{\isacharparenright}{\isachardoublequote}{\isacharcomma}\ simp{\isacharparenright}\isanewline
\ \ \isamarkupfalse%
\isacommand{by}\ blast\isamarkupfalse%
%
\isamarkupsubsection{Proof Rules%
}
\isamarkuptrue%
%
\begin{isamarkuptext}%
Proof rules, which can all be proved to be correct, since we have
  the semantics built into the logic (i.e. we can access it within HOL).
  Some proofs however simply employ the above tautology reasoner.
  \label{isa:proof-rules}%
\end{isamarkuptext}%
\isamarkuptrue%
\isacommand{theorem}\ pdl{\isacharunderscore}excluded{\isacharunderscore}middle{\isacharcolon}\ {\isachardoublequote}{\isasymturnstile}\ P\ {\isasymor}\isactrlsub D\ {\isacharparenleft}{\isasymnot}\isactrlsub D\ P{\isacharparenright}{\isachardoublequote}\isanewline
\ \ \isamarkupfalse%
\isacommand{by}\ {\isacharparenleft}simp\ add{\isacharcolon}\ pdl{\isacharunderscore}taut{\isacharparenright}\isanewline
\isanewline
\isanewline
\isamarkupfalse%
\isacommand{theorem}\ pdl{\isacharunderscore}mp{\isacharcolon}\ {\isachardoublequote}{\isasymlbrakk}{\isasymturnstile}\ P\ {\isasymlongrightarrow}\isactrlsub D\ Q{\isacharsemicolon}\ {\isasymturnstile}\ P{\isasymrbrakk}\ {\isasymLongrightarrow}\ {\isasymturnstile}\ Q{\isachardoublequote}\isanewline
\ \ \isamarkupfalse%
\isacommand{by}{\isacharparenleft}simp\ add{\isacharcolon}\ Valid{\isacharunderscore}simp\ impD{\isacharunderscore}def\ liftM{\isadigit{2}}{\isacharunderscore}def\ Rep{\isacharunderscore}Dsef{\isacharunderscore}inverse{\isacharparenright}\isamarkupfalse%
%
\begin{isamarkuptext}%
Disjunction introduction%
\end{isamarkuptext}%
\isamarkuptrue%
\isacommand{theorem}\ pdl{\isacharunderscore}disjI{\isadigit{1}}{\isacharcolon}\ {\isachardoublequote}{\isasymturnstile}\ P\ {\isasymLongrightarrow}\ {\isasymturnstile}\ {\isacharparenleft}P\ {\isasymor}\isactrlsub D\ Q{\isacharparenright}{\isachardoublequote}\isanewline
\isamarkupfalse%
\isacommand{proof}\ {\isacharminus}\isanewline
\ \ \isamarkupfalse%
\isacommand{assume}\ {\isachardoublequote}{\isasymturnstile}\ P{\isachardoublequote}\isanewline
\ \ \isamarkupfalse%
\isacommand{hence}\ pt{\isacharcolon}\ {\isachardoublequote}{\isasymDown}\ P\ {\isacharequal}\ ret\ True{\isachardoublequote}\ \isamarkupfalse%
\isacommand{by}\ {\isacharparenleft}simp\ only{\isacharcolon}\ Valid{\isacharunderscore}simp{\isacharparenright}\isanewline
\ \ \isamarkupfalse%
\isacommand{have}\ {\isachardoublequote}{\isasymDown}\ {\isacharparenleft}P\ {\isasymor}\isactrlsub D\ Q{\isacharparenright}\ {\isacharequal}\ ret\ True{\isachardoublequote}\ \isanewline
\ \ \isamarkupfalse%
\isacommand{proof}\ {\isacharminus}\isanewline
\ \ \ \ \isamarkupfalse%
\isacommand{have}\ {\isachardoublequote}{\isasymDown}\ {\isacharparenleft}{\isasymUp}\ {\isacharparenleft}liftM{\isadigit{2}}\ op\ {\isasymor}\ {\isacharparenleft}{\isasymDown}\ P{\isacharparenright}\ {\isacharparenleft}{\isasymDown}\ Q{\isacharparenright}{\isacharparenright}{\isacharparenright}\ {\isacharequal}\ ret\ True{\isachardoublequote}\isanewline
\ \ \ \ \isamarkupfalse%
\isacommand{proof}\ {\isacharminus}\isanewline
\ \ \ \ \ \ \isamarkupfalse%
\isacommand{have}\ {\isachardoublequote}{\isasymDown}\ {\isacharparenleft}{\isasymUp}\ {\isacharparenleft}do\ {\isacharbraceleft}x{\isasymleftarrow}{\isasymDown}\ Q{\isacharsemicolon}\ ret\ True{\isacharbraceright}{\isacharparenright}{\isacharparenright}\ {\isacharequal}\ ret\ True{\isachardoublequote}\isanewline
\ \ \ \ \ \ \isamarkupfalse%
\isacommand{proof}\ {\isacharminus}\isanewline
\ \ \ \ \ \ \ \ \isamarkupfalse%
\isacommand{have}\ {\isachardoublequote}{\isasymDown}\ {\isacharparenleft}{\isasymUp}\ {\isacharparenleft}do\ {\isacharbraceleft}x{\isasymleftarrow}{\isasymDown}\ Q{\isacharsemicolon}\ ret\ True{\isacharbraceright}{\isacharparenright}{\isacharparenright}\ {\isacharequal}\ \isanewline
\ \ \ \ \ \ \ \ \ \ do\ {\isacharbraceleft}x{\isasymleftarrow}{\isasymDown}\ Q{\isacharsemicolon}\ ret\ True{\isacharbraceright}{\isachardoublequote}\isanewline
\ \ \ \ \ \ \ \ \ \ \isamarkupfalse%
\isacommand{by}\ {\isacharparenleft}simp\ add{\isacharcolon}\ Abs{\isacharunderscore}Dsef{\isacharunderscore}inverse\ Dsef{\isacharunderscore}def\ weak{\isacharunderscore}dsef{\isacharunderscore}seq{\isacharparenright}\isanewline
\ \ \ \ \ \ \ \ \isamarkupfalse%
\isacommand{also}\ \isamarkupfalse%
\isacommand{have}\ {\isachardoublequote}{\isasymdots}\ {\isacharequal}\ do\ {\isacharbraceleft}{\isasymDown}\ Q{\isacharsemicolon}\ ret\ True{\isacharbraceright}{\isachardoublequote}\ \isamarkupfalse%
\isacommand{by}\ {\isacharparenleft}simp\ only{\isacharcolon}seq{\isacharunderscore}def{\isacharparenright}\isanewline
\ \ \ \ \ \ \ \ \isamarkupfalse%
\isacommand{also}\ \isamarkupfalse%
\isacommand{have}\ {\isachardoublequote}{\isasymdots}\ {\isacharequal}\ ret\ True{\isachardoublequote}\ \isamarkupfalse%
\isacommand{by}\ {\isacharparenleft}simp\ add{\isacharcolon}\ dis{\isacharunderscore}Rep{\isacharunderscore}Dsef\ dis{\isacharunderscore}left{\isacharparenright}\isanewline
\ \ \ \ \ \ \ \ \isamarkupfalse%
\isacommand{finally}\ \isamarkupfalse%
\isacommand{show}\ {\isacharquery}thesis\ \isamarkupfalse%
\isacommand{{\isachardot}}\isanewline
\ \ \ \ \ \ \isamarkupfalse%
\isacommand{qed}\isanewline
\ \ \ \ \ \ \isamarkupfalse%
\isacommand{with}\ pt\ \isamarkupfalse%
\isacommand{show}\ {\isacharquery}thesis\ \isamarkupfalse%
\isacommand{by}\ {\isacharparenleft}simp\ add{\isacharcolon}\ liftM{\isadigit{2}}{\isacharunderscore}def{\isacharparenright}\isanewline
\ \ \ \ \isamarkupfalse%
\isacommand{qed}\isanewline
\ \ \ \ \isamarkupfalse%
\isacommand{thus}\ {\isacharquery}thesis\ \isamarkupfalse%
\isacommand{by}\ {\isacharparenleft}simp\ only{\isacharcolon}\ disjD{\isacharunderscore}def{\isacharparenright}\isanewline
\ \ \isamarkupfalse%
\isacommand{qed}\isanewline
\ \ \isamarkupfalse%
\isacommand{thus}\ {\isachardoublequote}{\isasymturnstile}\ {\isacharparenleft}P\ {\isasymor}\isactrlsub D\ Q{\isacharparenright}{\isachardoublequote}\ \isamarkupfalse%
\isacommand{by}\ {\isacharparenleft}simp\ only{\isacharcolon}\ Valid{\isacharunderscore}simp{\isacharparenright}\isanewline
\isamarkupfalse%
\isacommand{qed}\isamarkupfalse%
%
\begin{isamarkuptext}%
Entirely analogous for this dual rule.%
\end{isamarkuptext}%
\isamarkuptrue%
\isacommand{theorem}\ pdl{\isacharunderscore}disjI{\isadigit{2}}{\isacharcolon}\ {\isachardoublequote}{\isasymturnstile}\ Q\ {\isasymLongrightarrow}\ {\isasymturnstile}\ {\isacharparenleft}P\ {\isasymor}\isactrlsub D\ Q{\isacharparenright}{\isachardoublequote}\isamarkupfalse%
\isamarkupfalse%
\isamarkupfalse%
\isamarkupfalse%
\isamarkupfalse%
\isamarkupfalse%
\isamarkupfalse%
\isamarkupfalse%
\isamarkupfalse%
\isamarkupfalse%
\isamarkupfalse%
\isamarkupfalse%
\isamarkupfalse%
\isamarkupfalse%
\isamarkupfalse%
\isamarkupfalse%
\isamarkupfalse%
\isamarkupfalse%
\isamarkupfalse%
\isamarkupfalse%
\isamarkupfalse%
\isamarkupfalse%
\isamarkupfalse%
\isamarkupfalse%
\isamarkupfalse%
\isamarkupfalse%
\isamarkupfalse%
\isamarkupfalse%
\isamarkupfalse%
\isamarkupfalse%
\isamarkupfalse%
\isamarkupfalse%
\isamarkupfalse%
%
\begin{isamarkuptext}%
The following proof proceeds by a standard pattern: First insert the 
  assumptions into some specifically tailored do-term and then 
  reduce this do-term to \isa{ret\ True} with the simplifier.%
\end{isamarkuptext}%
\isamarkuptrue%
\isacommand{theorem}\ pdl{\isacharunderscore}disjE{\isacharcolon}\ {\isachardoublequote}{\isasymlbrakk}\ {\isasymturnstile}\ P\ {\isasymor}\isactrlsub D\ Q{\isacharsemicolon}\ {\isasymturnstile}\ P\ {\isasymlongrightarrow}\isactrlsub D\ R{\isacharsemicolon}\ {\isasymturnstile}\ Q\ {\isasymlongrightarrow}\isactrlsub D\ R{\isasymrbrakk}\ {\isasymLongrightarrow}\ {\isasymturnstile}\ R{\isachardoublequote}\isanewline
\isamarkupfalse%
\isacommand{proof}\ {\isacharminus}\isanewline
\ \ \isamarkupfalse%
\isacommand{assume}\ a{\isadigit{1}}{\isacharcolon}\ {\isachardoublequote}{\isasymturnstile}\ P\ {\isasymor}\isactrlsub D\ Q{\isachardoublequote}\ {\isachardoublequote}{\isasymturnstile}\ P\ {\isasymlongrightarrow}\isactrlsub D\ R{\isachardoublequote}\ {\isachardoublequote}{\isasymturnstile}\ Q\ {\isasymlongrightarrow}\isactrlsub D\ R{\isachardoublequote}\isanewline
\ \ \isamarkupfalse%
\isacommand{note}\ copy\ {\isacharequal}\ dsef{\isacharunderscore}cp{\isacharbrackleft}OF\ dsef{\isacharunderscore}Rep{\isacharunderscore}Dsef{\isacharbrackright}\isanewline
\ \ \isamarkupfalse%
\isacommand{note}\ dsc\ \ {\isacharequal}\ dsef{\isacharunderscore}dis{\isacharbrackleft}OF\ dsef{\isacharunderscore}Rep{\isacharunderscore}Dsef{\isacharbrackright}\isanewline
\ \ %
\isamarkupcmt{1st part: blow up program \isa{{\isasymDown}\ R} to some giant term:%
}
\isanewline
\ \ \isamarkupfalse%
\isacommand{have}\ {\isachardoublequote}{\isasymDown}\ R\ {\isacharequal}\ do\ {\isacharbraceleft}u{\isasymleftarrow}ret\ True{\isacharsemicolon}\ v{\isasymleftarrow}ret\ True{\isacharsemicolon}\ w{\isasymleftarrow}ret\ True{\isacharsemicolon}\ r{\isasymleftarrow}{\isasymDown}\ R{\isacharsemicolon}\ ret{\isacharparenleft}u{\isasymlongrightarrow}v{\isasymlongrightarrow}w{\isasymlongrightarrow}r{\isacharparenright}{\isacharbraceright}{\isachardoublequote}\isanewline
\ \ \ \ \isamarkupfalse%
\isacommand{by}\ simp\isanewline
\ \ \isamarkupfalse%
\isacommand{also}\ \isamarkupfalse%
\isacommand{from}\ a{\isadigit{1}}\ \isamarkupfalse%
\isacommand{have}\ {\isachardoublequote}{\isasymdots}\ {\isacharequal}\ do\ {\isacharbraceleft}u{\isasymleftarrow}{\isacharparenleft}{\isasymDown}\ {\isacharparenleft}P\ {\isasymor}\isactrlsub D\ Q{\isacharparenright}{\isacharparenright}{\isacharsemicolon}\isanewline
\ \ \ \ \ \ \ \ \ \ \ \ \ \ \ \ \ \ \ \ \ \ \ \ \ \ \ \ \ v{\isasymleftarrow}{\isacharparenleft}{\isasymDown}\ {\isacharparenleft}P\ {\isasymlongrightarrow}\isactrlsub D\ R{\isacharparenright}{\isacharparenright}{\isacharsemicolon}\isanewline
\ \ \ \ \ \ \ \ \ \ \ \ \ \ \ \ \ \ \ \ \ \ \ \ \ \ \ \ \ w{\isasymleftarrow}{\isacharparenleft}{\isasymDown}\ {\isacharparenleft}Q\ {\isasymlongrightarrow}\isactrlsub D\ R{\isacharparenright}{\isacharparenright}{\isacharsemicolon}\isanewline
\ \ \ \ \ \ \ \ \ \ \ \ \ \ \ \ \ \ \ \ \ \ \ \ \ \ \ \ \ r{\isasymleftarrow}{\isasymDown}\ R{\isacharsemicolon}\ ret\ {\isacharparenleft}u{\isasymlongrightarrow}v{\isasymlongrightarrow}w{\isasymlongrightarrow}r{\isacharparenright}{\isacharbraceright}{\isachardoublequote}\isanewline
\ \ \ \ \isamarkupfalse%
\isacommand{by}\ {\isacharparenleft}simp\ add{\isacharcolon}\ Valid{\isacharunderscore}simp{\isacharparenright}\isanewline
\ \ %
\isamarkupcmt{2nd part: reduce this giant program to \isa{ret\ True} exploiting
        properties of dsef programs%
}
\isanewline
\ \ \isamarkupfalse%
\isacommand{also}\ \isamarkupfalse%
\isacommand{have}\ {\isachardoublequote}{\isasymdots}\ {\isacharequal}\ ret\ True{\isachardoublequote}\isanewline
\ \ \ \ \isamarkupfalse%
\isacommand{apply}{\isacharparenleft}simp\ add{\isacharcolon}\ mon{\isacharunderscore}prop{\isacharunderscore}reason\ liftM{\isadigit{2}}{\isacharunderscore}def\ dsef{\isacharunderscore}Rep{\isacharunderscore}Dsef\ dsef{\isacharunderscore}seq\ mon{\isacharunderscore}ctr\ del{\isacharcolon}\ bind{\isacharunderscore}assoc{\isacharparenright}\isanewline
\ \ \ \ \isamarkupfalse%
\isacommand{apply}{\isacharparenleft}simp\ add{\isacharcolon}\ commute{\isacharunderscore}dsef{\isacharbrackleft}of\ {\isachardoublequote}{\isasymDown}\ Q{\isachardoublequote}\ {\isachardoublequote}{\isasymDown}\ P{\isachardoublequote}{\isacharbrackright}{\isacharparenright}\isanewline
\ \ \ \ \isamarkupfalse%
\isacommand{apply}{\isacharparenleft}simp\ add{\isacharcolon}\ commute{\isacharunderscore}dsef{\isacharbrackleft}of\ {\isachardoublequote}{\isasymDown}\ R{\isachardoublequote}\ {\isachardoublequote}{\isasymDown}\ Q{\isachardoublequote}{\isacharbrackright}{\isacharparenright}\isanewline
\ \ \ \ \isamarkupfalse%
\isacommand{apply}{\isacharparenleft}simp\ add{\isacharcolon}\ dsef{\isacharunderscore}cp{\isacharbrackleft}OF\ dsef{\isacharunderscore}Rep{\isacharunderscore}Dsef{\isacharbrackright}\ cp{\isacharunderscore}arb\ del{\isacharcolon}\ bind{\isacharunderscore}assoc{\isacharparenright}\isanewline
\ \ \ \ \isamarkupfalse%
\isacommand{apply}{\isacharparenleft}simp\ add{\isacharcolon}\ dsef{\isacharunderscore}dis{\isacharbrackleft}OF\ dsef{\isacharunderscore}Rep{\isacharunderscore}Dsef{\isacharbrackright}\ dis{\isacharunderscore}left{\isadigit{2}}{\isacharparenright}\isanewline
\ \ \ \ \isamarkupfalse%
\isacommand{done}\isanewline
\ \ \isamarkupfalse%
\isacommand{finally}\ \isamarkupfalse%
\isacommand{show}\ {\isacharquery}thesis\ \isamarkupfalse%
\isacommand{by}\ {\isacharparenleft}simp\ only{\isacharcolon}\ Valid{\isacharunderscore}simp{\isacharparenright}\isanewline
\isamarkupfalse%
\isacommand{qed}\isanewline
\ \ \ \ \isanewline
\isanewline
\isamarkupfalse%
\isacommand{theorem}\ pdl{\isacharunderscore}conjI{\isacharcolon}\ {\isachardoublequote}{\isasymlbrakk}\ {\isasymturnstile}\ P{\isacharsemicolon}\ {\isasymturnstile}\ Q\ {\isasymrbrakk}\ {\isasymLongrightarrow}\ {\isasymturnstile}\ P\ {\isasymand}\isactrlsub D\ Q{\isachardoublequote}\isanewline
\isamarkupfalse%
\isacommand{proof}\ {\isacharminus}\isanewline
\ \ \isamarkupfalse%
\isacommand{assume}\ a{\isacharcolon}\ {\isachardoublequote}{\isasymturnstile}\ P{\isachardoublequote}\ {\isachardoublequote}{\isasymturnstile}\ Q{\isachardoublequote}\isanewline
\ \ \isamarkupfalse%
\isacommand{from}\ a\ \isamarkupfalse%
\isacommand{have}\ {\isachardoublequote}{\isasymDown}\ P\ {\isacharequal}\ ret\ True{\isachardoublequote}\ \isamarkupfalse%
\isacommand{by}\ {\isacharparenleft}simp\ add{\isacharcolon}\ Valid{\isacharunderscore}simp{\isacharparenright}\isanewline
\ \ \isamarkupfalse%
\isacommand{moreover}\isanewline
\ \ \isamarkupfalse%
\isacommand{from}\ a\ \isamarkupfalse%
\isacommand{have}\ {\isachardoublequote}{\isasymDown}\ Q\ {\isacharequal}\ ret\ True{\isachardoublequote}\ \isamarkupfalse%
\isacommand{by}\ {\isacharparenleft}simp\ add{\isacharcolon}\ Valid{\isacharunderscore}simp{\isacharparenright}\isanewline
\ \ \isamarkupfalse%
\isacommand{ultimately}\ \isanewline
\ \ \isamarkupfalse%
\isacommand{have}\ {\isachardoublequote}{\isasymDown}\ {\isacharparenleft}P\ {\isasymand}\isactrlsub D\ Q{\isacharparenright}\ {\isacharequal}\ ret\ True{\isachardoublequote}\isanewline
\ \ \ \ \isamarkupfalse%
\isacommand{by}\ {\isacharparenleft}simp\ add{\isacharcolon}\ mon{\isacharunderscore}prop{\isacharunderscore}reason\ liftM{\isadigit{2}}{\isacharunderscore}def{\isacharparenright}\ \isanewline
\ \ \isamarkupfalse%
\isacommand{thus}\ {\isacharquery}thesis\ \isamarkupfalse%
\isacommand{by}\ {\isacharparenleft}simp\ add{\isacharcolon}\ Valid{\isacharunderscore}simp{\isacharparenright}\isanewline
\isamarkupfalse%
\isacommand{qed}\isamarkupfalse%
%
\isamarkupsubsubsection{Derived rules of inference%
}
\isamarkuptrue%
\isacommand{theorem}\ pdl{\isacharunderscore}FalseE{\isacharcolon}\ {\isachardoublequote}{\isasymturnstile}\ Ret\ False\ {\isasymLongrightarrow}\ {\isasymturnstile}\ R{\isachardoublequote}\isanewline
\isamarkupfalse%
\isacommand{proof}\ {\isacharminus}\isanewline
\ \ \isamarkupfalse%
\isacommand{assume}\ {\isachardoublequote}{\isasymturnstile}\ Ret\ False{\isachardoublequote}\isanewline
\ \ \isamarkupfalse%
\isacommand{hence}\ {\isachardoublequote}False{\isachardoublequote}\ \isamarkupfalse%
\isacommand{by}\ {\isacharparenleft}rule\ iffD{\isadigit{1}}{\isacharbrackleft}OF\ Valid{\isacharunderscore}Ret{\isacharbrackright}{\isacharparenright}\isanewline
\ \ \isamarkupfalse%
\isacommand{thus}\ {\isachardoublequote}{\isasymturnstile}\ R{\isachardoublequote}\ \isamarkupfalse%
\isacommand{by}\ {\isacharparenleft}rule\ FalseE{\isacharparenright}\isanewline
\isamarkupfalse%
\isacommand{qed}\isanewline
\isanewline
\isanewline
\isamarkupfalse%
\isacommand{lemma}\ pdl{\isacharunderscore}notE{\isacharcolon}\ {\isachardoublequote}{\isasymlbrakk}\ {\isasymturnstile}\ P{\isacharsemicolon}\ {\isasymturnstile}\ {\isasymnot}\isactrlsub D\ P\ {\isasymrbrakk}\ {\isasymLongrightarrow}\ {\isasymturnstile}\ R{\isachardoublequote}\isanewline
\isamarkupfalse%
\isacommand{proof}\ {\isacharparenleft}unfold\ NotD{\isacharunderscore}def{\isacharparenright}\isanewline
\ \ \isamarkupfalse%
\isacommand{assume}\ p{\isacharcolon}\ {\isachardoublequote}{\isasymturnstile}\ P{\isachardoublequote}\ \isakeyword{and}\ np{\isacharcolon}\ {\isachardoublequote}{\isasymturnstile}\ P\ {\isasymlongrightarrow}\isactrlsub D\ Ret\ False{\isachardoublequote}\isanewline
\ \ \isamarkupfalse%
\isacommand{from}\ np\ p\ \isamarkupfalse%
\isacommand{have}\ {\isachardoublequote}{\isasymturnstile}\ Ret\ False{\isachardoublequote}\ \isamarkupfalse%
\isacommand{by}\ {\isacharparenleft}rule\ pdl{\isacharunderscore}mp{\isacharparenright}\isanewline
\ \ \isamarkupfalse%
\isacommand{thus}\ {\isachardoublequote}{\isasymturnstile}\ R{\isachardoublequote}\ \isamarkupfalse%
\isacommand{by}\ {\isacharparenleft}rule\ pdl{\isacharunderscore}FalseE{\isacharparenright}\isanewline
\isamarkupfalse%
\isacommand{qed}\isanewline
\isanewline
\isanewline
\isamarkupfalse%
\isacommand{lemma}\ pdl{\isacharunderscore}conjE{\isacharcolon}\ {\isachardoublequote}{\isasymlbrakk}\ {\isasymturnstile}\ P\ {\isasymand}\isactrlsub D\ Q{\isacharsemicolon}\ {\isasymlbrakk}{\isasymturnstile}\ P{\isacharsemicolon}\ {\isasymturnstile}\ Q{\isasymrbrakk}\ {\isasymLongrightarrow}\ {\isasymturnstile}\ R{\isasymrbrakk}\ {\isasymLongrightarrow}\ {\isasymturnstile}\ R{\isachardoublequote}\isanewline
\isamarkupfalse%
\isacommand{proof}\ {\isacharminus}\isanewline
\ \ \isamarkupfalse%
\isacommand{assume}\ a{\isadigit{1}}{\isacharcolon}\ {\isachardoublequote}{\isasymturnstile}\ P\ {\isasymand}\isactrlsub D\ Q{\isachardoublequote}\isanewline
\ \ \isamarkupfalse%
\isacommand{assume}\ a{\isadigit{2}}{\isacharcolon}\ {\isachardoublequote}{\isasymlbrakk}{\isasymturnstile}\ P{\isacharsemicolon}\ {\isasymturnstile}\ Q{\isasymrbrakk}\ {\isasymLongrightarrow}\ {\isasymturnstile}\ R{\isachardoublequote}\isanewline
\ \ \isamarkupfalse%
\isacommand{have}\ {\isachardoublequote}{\isasymturnstile}\ P{\isachardoublequote}\ \isanewline
\ \ \isamarkupfalse%
\isacommand{proof}\ {\isacharparenleft}rule\ pdl{\isacharunderscore}mp{\isacharparenright}\isanewline
\ \ \ \ \isamarkupfalse%
\isacommand{show}\ {\isachardoublequote}{\isasymturnstile}\ P\ {\isasymand}\isactrlsub D\ Q\ {\isasymlongrightarrow}\isactrlsub D\ P{\isachardoublequote}\ \isamarkupfalse%
\isacommand{by}\ {\isacharparenleft}simp\ add{\isacharcolon}\ pdl{\isacharunderscore}taut{\isacharparenright}\isanewline
\ \ \isamarkupfalse%
\isacommand{qed}\isanewline
\ \ \isamarkupfalse%
\isacommand{moreover}\isanewline
\ \ \isamarkupfalse%
\isacommand{have}\ {\isachardoublequote}{\isasymturnstile}\ Q{\isachardoublequote}\ \isanewline
\ \ \isamarkupfalse%
\isacommand{proof}\ {\isacharparenleft}rule\ pdl{\isacharunderscore}mp{\isacharparenright}\isanewline
\ \ \ \ \isamarkupfalse%
\isacommand{show}\ {\isachardoublequote}{\isasymturnstile}\ P\ {\isasymand}\isactrlsub D\ Q\ {\isasymlongrightarrow}\isactrlsub D\ Q{\isachardoublequote}\ \isamarkupfalse%
\isacommand{by}\ {\isacharparenleft}simp\ add{\isacharcolon}\ pdl{\isacharunderscore}taut{\isacharparenright}\isanewline
\ \ \isamarkupfalse%
\isacommand{qed}\isanewline
\ \ \isamarkupfalse%
\isacommand{moreover}\ \isamarkupfalse%
\isacommand{note}\ a{\isadigit{1}}\ a{\isadigit{2}}\isanewline
\ \ \isamarkupfalse%
\isacommand{ultimately}\ \isanewline
\ \ \isamarkupfalse%
\isacommand{show}\ {\isachardoublequote}{\isasymturnstile}\ R{\isachardoublequote}\ \isamarkupfalse%
\isacommand{by}\ {\isacharparenleft}rules{\isacharparenright}\isanewline
\isamarkupfalse%
\isacommand{qed}\isamarkupfalse%
%
\begin{isamarkuptext}%
Some further typical rules.%
\end{isamarkuptext}%
\isamarkuptrue%
\isacommand{lemma}\ pdl{\isacharunderscore}notI{\isacharcolon}\ {\isachardoublequote}{\isasymlbrakk}\ {\isasymturnstile}\ P{\isacharsemicolon}\ {\isasymturnstile}\ Ret\ False{\isasymrbrakk}\ {\isasymLongrightarrow}\ {\isasymturnstile}\ {\isasymnot}\isactrlsub D\ P{\isachardoublequote}\isanewline
\isamarkupfalse%
\isacommand{by}{\isacharparenleft}rule\ pdl{\isacharunderscore}FalseE{\isacharparenright}\isanewline
\isanewline
\isamarkupfalse%
\isacommand{lemma}\ pdl{\isacharunderscore}conjunct{\isadigit{1}}{\isacharcolon}\ {\isachardoublequote}{\isasymturnstile}\ P\ {\isasymand}\isactrlsub D\ Q\ {\isasymLongrightarrow}\ {\isasymturnstile}\ P{\isachardoublequote}\isanewline
\isamarkupfalse%
\isacommand{proof}\ {\isacharminus}\isanewline
\ \ \isamarkupfalse%
\isacommand{assume}\ {\isachardoublequote}{\isasymturnstile}\ P\ {\isasymand}\isactrlsub D\ Q{\isachardoublequote}\isanewline
\ \ \isamarkupfalse%
\isacommand{thus}\ {\isachardoublequote}{\isasymturnstile}\ P{\isachardoublequote}\isanewline
\ \ \isamarkupfalse%
\isacommand{proof}\ {\isacharparenleft}rule\ pdl{\isacharunderscore}conjE{\isacharparenright}\isanewline
\ \ \ \ \isamarkupfalse%
\isacommand{assume}\ {\isachardoublequote}{\isasymturnstile}\ P{\isachardoublequote}\isanewline
\ \ \ \ \isamarkupfalse%
\isacommand{thus}\ {\isacharquery}thesis\ \isamarkupfalse%
\isacommand{{\isachardot}}\isanewline
\ \ \isamarkupfalse%
\isacommand{qed}\isanewline
\isamarkupfalse%
\isacommand{qed}\isanewline
\isanewline
\isamarkupfalse%
\isacommand{lemma}\ pdl{\isacharunderscore}conjunct{\isadigit{2}}{\isacharcolon}\ \isakeyword{assumes}\ pq{\isacharcolon}\ {\isachardoublequote}{\isasymturnstile}\ P\ {\isasymand}\isactrlsub D\ Q{\isachardoublequote}\ \isakeyword{shows}\ {\isachardoublequote}{\isasymturnstile}\ Q{\isachardoublequote}\isanewline
\isamarkupfalse%
\isacommand{proof}\ {\isacharminus}\isanewline
\ \ \isamarkupfalse%
\isacommand{from}\ pq\ \isamarkupfalse%
\isacommand{show}\ {\isachardoublequote}{\isasymturnstile}\ Q{\isachardoublequote}\isanewline
\ \ \isamarkupfalse%
\isacommand{proof}\ {\isacharparenleft}rule\ pdl{\isacharunderscore}conjE{\isacharparenright}\isanewline
\ \ \ \ \isamarkupfalse%
\isacommand{assume}\ {\isachardoublequote}{\isasymturnstile}\ Q{\isachardoublequote}\isanewline
\ \ \ \ \isamarkupfalse%
\isacommand{thus}\ {\isacharquery}thesis\ \isamarkupfalse%
\isacommand{{\isachardot}}\isanewline
\ \ \isamarkupfalse%
\isacommand{qed}\isanewline
\isamarkupfalse%
\isacommand{qed}\isanewline
\ \ \ \ \isanewline
\isamarkupfalse%
\isacommand{lemma}\ pdl{\isacharunderscore}iffI{\isacharcolon}\ {\isachardoublequote}{\isasymlbrakk}{\isasymturnstile}\ P\ {\isasymlongrightarrow}\isactrlsub D\ Q{\isacharsemicolon}\ {\isasymturnstile}\ Q\ {\isasymlongrightarrow}\isactrlsub D\ P{\isasymrbrakk}\ {\isasymLongrightarrow}\ {\isasymturnstile}\ P\ {\isasymlongleftrightarrow}\isactrlsub D\ Q{\isachardoublequote}\isanewline
\isamarkupfalse%
\isacommand{proof}\ {\isacharparenleft}unfold\ iffD{\isacharunderscore}def{\isacharparenright}\isanewline
\ \ \isamarkupfalse%
\isacommand{assume}\ a{\isacharcolon}\ {\isachardoublequote}{\isasymturnstile}\ P\ {\isasymlongrightarrow}\isactrlsub D\ Q{\isachardoublequote}\ \isakeyword{and}\ b{\isacharcolon}\ {\isachardoublequote}{\isasymturnstile}\ Q\ {\isasymlongrightarrow}\isactrlsub D\ P{\isachardoublequote}\isanewline
\ \ \isamarkupfalse%
\isacommand{show}\ {\isachardoublequote}{\isasymturnstile}\ {\isacharparenleft}P\ {\isasymlongrightarrow}\isactrlsub D\ Q{\isacharparenright}\ {\isasymand}\isactrlsub D\ {\isacharparenleft}Q\ {\isasymlongrightarrow}\isactrlsub D\ P{\isacharparenright}{\isachardoublequote}\isanewline
\ \ \ \ \isamarkupfalse%
\isacommand{by}\ {\isacharparenleft}rule\ pdl{\isacharunderscore}conjI{\isacharparenright}\isanewline
\isamarkupfalse%
\isacommand{qed}\isanewline
\isanewline
\isamarkupfalse%
\isacommand{lemma}\ pdl{\isacharunderscore}iffE{\isacharcolon}\ {\isachardoublequote}{\isasymlbrakk}{\isasymturnstile}\ P\ {\isasymlongleftrightarrow}\isactrlsub D\ Q{\isacharsemicolon}\ {\isasymlbrakk}\ {\isasymturnstile}\ P\ {\isasymlongrightarrow}\isactrlsub D\ Q{\isacharsemicolon}\ {\isasymturnstile}\ Q\ {\isasymlongrightarrow}\isactrlsub D\ P\ {\isasymrbrakk}\ {\isasymLongrightarrow}\ {\isasymturnstile}\ R{\isasymrbrakk}\ {\isasymLongrightarrow}\ {\isasymturnstile}\ R{\isachardoublequote}\isanewline
\ \isamarkupfalse%
\isacommand{apply}{\isacharparenleft}unfold\ iffD{\isacharunderscore}def{\isacharparenright}\ \isanewline
\ \isamarkupfalse%
\isacommand{apply}{\isacharparenleft}erule\ pdl{\isacharunderscore}conjE{\isacharparenright}\isanewline
\isamarkupfalse%
\isacommand{by}\ blast\isanewline
\isanewline
\isamarkupfalse%
\isacommand{lemma}\ pdl{\isacharunderscore}sym{\isacharcolon}\ {\isachardoublequote}{\isacharparenleft}{\isasymturnstile}\ P\ {\isasymlongleftrightarrow}\isactrlsub D\ Q{\isacharparenright}\ {\isasymLongrightarrow}\ {\isacharparenleft}{\isasymturnstile}\ Q\ {\isasymlongleftrightarrow}\isactrlsub D\ P{\isacharparenright}{\isachardoublequote}\isanewline
\ \ \isamarkupfalse%
\isacommand{apply}{\isacharparenleft}erule\ pdl{\isacharunderscore}iffE{\isacharparenright}\isanewline
\isamarkupfalse%
\isacommand{by}{\isacharparenleft}rule\ pdl{\isacharunderscore}iffI{\isacharparenright}\isanewline
\isanewline
\isamarkupfalse%
\isacommand{lemma}\ pdl{\isacharunderscore}iffD{\isadigit{1}}{\isacharcolon}\ {\isachardoublequote}{\isasymturnstile}\ P\ {\isasymlongleftrightarrow}\isactrlsub D\ Q\ {\isasymLongrightarrow}\ {\isasymturnstile}\ P\ {\isasymlongrightarrow}\isactrlsub D\ Q{\isachardoublequote}\isanewline
\isamarkupfalse%
\isacommand{by}{\isacharparenleft}erule\ pdl{\isacharunderscore}iffE{\isacharparenright}\isanewline
\isanewline
\isamarkupfalse%
\isacommand{lemma}\ pdl{\isacharunderscore}iffD{\isadigit{2}}{\isacharcolon}\ {\isachardoublequote}{\isasymturnstile}\ P\ {\isasymlongleftrightarrow}\isactrlsub D\ Q\ {\isasymLongrightarrow}\ {\isasymturnstile}\ Q\ {\isasymlongrightarrow}\isactrlsub D\ P{\isachardoublequote}\isanewline
\isamarkupfalse%
\isacommand{by}\ {\isacharparenleft}erule\ pdl{\isacharunderscore}iffE{\isacharparenright}\isanewline
\isanewline
\isamarkupfalse%
\isacommand{lemma}\ pdl{\isacharunderscore}conjI{\isacharunderscore}lifted{\isacharcolon}\ \isanewline
\isakeyword{assumes}\ {\isachardoublequote}{\isasymturnstile}\ P\ {\isasymlongrightarrow}\isactrlsub D\ Q{\isachardoublequote}\ \isakeyword{and}\ {\isachardoublequote}{\isasymturnstile}\ P\ {\isasymlongrightarrow}\isactrlsub D\ R{\isachardoublequote}\ \isakeyword{shows}\ {\isachardoublequote}{\isasymturnstile}\ P\ {\isasymlongrightarrow}\isactrlsub D\ Q\ {\isasymand}\isactrlsub D\ R{\isachardoublequote}\isanewline
\isamarkupfalse%
\isacommand{proof}\ {\isacharminus}\isanewline
\ \ \isamarkupfalse%
\isacommand{have}\ {\isachardoublequote}{\isasymturnstile}\ {\isacharparenleft}P\ {\isasymlongrightarrow}\isactrlsub D\ Q{\isacharparenright}\ {\isasymlongrightarrow}\isactrlsub D\ {\isacharparenleft}P\ {\isasymlongrightarrow}\isactrlsub D\ R{\isacharparenright}\ {\isasymlongrightarrow}\isactrlsub D\ {\isacharparenleft}P\ {\isasymlongrightarrow}\isactrlsub D\ Q\ {\isasymand}\isactrlsub D\ R{\isacharparenright}{\isachardoublequote}\ \isanewline
\ \ \ \ \isamarkupfalse%
\isacommand{by}\ {\isacharparenleft}simp\ add{\isacharcolon}\ \ pdl{\isacharunderscore}taut{\isacharparenright}\isanewline
\ \ \isamarkupfalse%
\isacommand{thus}\ {\isacharquery}thesis\ \isamarkupfalse%
\isacommand{by}\ {\isacharparenleft}rule\ pdl{\isacharunderscore}mp{\isacharbrackleft}THEN\ pdl{\isacharunderscore}mp{\isacharbrackright}{\isacharparenright}\isanewline
\isamarkupfalse%
\isacommand{qed}\isanewline
\isanewline
\isamarkupfalse%
\isacommand{lemma}\ pdl{\isacharunderscore}eq{\isacharunderscore}iff{\isacharcolon}\ {\isachardoublequote}{\isasymlbrakk}\ P\ {\isacharequal}\ Q\ {\isasymrbrakk}\ {\isasymLongrightarrow}\ {\isasymturnstile}\ P\ {\isasymlongleftrightarrow}\isactrlsub D\ Q{\isachardoublequote}\isanewline
\isamarkupfalse%
\isacommand{by}\ {\isacharparenleft}simp\ only{\isacharcolon}\ pdl{\isacharunderscore}taut\ Valid{\isacharunderscore}Ret{\isacharparenright}\isanewline
\isanewline
\isanewline
\isamarkupfalse%
\isacommand{lemma}\ pdl{\isacharunderscore}iff{\isacharunderscore}sym{\isacharcolon}\ {\isachardoublequote}{\isasymturnstile}\ P\ {\isasymlongleftrightarrow}\isactrlsub D\ Q\ {\isasymLongrightarrow}\ {\isasymturnstile}\ Q\ {\isasymlongleftrightarrow}\isactrlsub D\ P{\isachardoublequote}\isanewline
\isamarkupfalse%
\isacommand{by}\ {\isacharparenleft}simp\ only{\isacharcolon}\ pdl{\isacharunderscore}taut\ Valid{\isacharunderscore}Ret{\isacharparenright}\isanewline
\isanewline
\isamarkupfalse%
\isacommand{lemma}\ pdl{\isacharunderscore}imp{\isacharunderscore}wk{\isacharcolon}\ {\isachardoublequote}{\isasymturnstile}\ P\ {\isasymLongrightarrow}\ {\isasymturnstile}\ Q\ {\isasymlongrightarrow}\isactrlsub D\ P{\isachardoublequote}\isanewline
\isamarkupfalse%
\isacommand{proof}\ {\isacharminus}\isanewline
\ \ \isamarkupfalse%
\isacommand{assume}\ {\isachardoublequote}{\isasymturnstile}\ P{\isachardoublequote}\isanewline
\ \ \isamarkupfalse%
\isacommand{have}\ {\isachardoublequote}{\isasymturnstile}\ P\ {\isasymlongrightarrow}\isactrlsub D\ Q\ {\isasymlongrightarrow}\isactrlsub D\ P{\isachardoublequote}\ \isamarkupfalse%
\isacommand{by}\ {\isacharparenleft}simp\ add{\isacharcolon}\ pdl{\isacharunderscore}taut{\isacharparenright}\isanewline
\ \ \isamarkupfalse%
\isacommand{thus}\ {\isacharquery}thesis\ \isamarkupfalse%
\isacommand{by}\ {\isacharparenleft}rule\ pdl{\isacharunderscore}mp{\isacharparenright}\isanewline
\isamarkupfalse%
\isacommand{qed}\isanewline
\isanewline
\isanewline
\isamarkupfalse%
\isacommand{lemma}\ pdl{\isacharunderscore}False{\isacharunderscore}imp{\isacharcolon}\ {\isachardoublequote}{\isasymturnstile}\ Ret\ False\ {\isasymlongrightarrow}\isactrlsub D\ P{\isachardoublequote}\isanewline
\ \ \isamarkupfalse%
\isacommand{by}\ {\isacharparenleft}simp\ add{\isacharcolon}\ pdl{\isacharunderscore}taut{\isacharparenright}\isanewline
\isanewline
\isanewline
\isamarkupfalse%
\isacommand{lemma}\ pdl{\isacharunderscore}imp{\isacharunderscore}trans{\isacharcolon}\ {\isachardoublequote}{\isasymlbrakk}{\isasymturnstile}\ A\ {\isasymlongrightarrow}\isactrlsub D\ B{\isacharsemicolon}\ {\isasymturnstile}\ B\ {\isasymlongrightarrow}\isactrlsub D\ C{\isasymrbrakk}\ {\isasymLongrightarrow}\ {\isasymturnstile}\ A\ {\isasymlongrightarrow}\isactrlsub D\ C{\isachardoublequote}\isanewline
\isamarkupfalse%
\isacommand{proof}\ {\isacharminus}\isanewline
\ \ \isamarkupfalse%
\isacommand{assume}\ a{\isadigit{1}}{\isacharcolon}\ {\isachardoublequote}{\isasymturnstile}\ A\ {\isasymlongrightarrow}\isactrlsub D\ B{\isachardoublequote}\ \isakeyword{and}\ a{\isadigit{2}}{\isacharcolon}\ {\isachardoublequote}{\isasymturnstile}\ B\ {\isasymlongrightarrow}\isactrlsub D\ C{\isachardoublequote}\isanewline
\ \ \isamarkupfalse%
\isacommand{have}\ {\isachardoublequote}{\isasymturnstile}\ {\isacharparenleft}A\ {\isasymlongrightarrow}\isactrlsub D\ B{\isacharparenright}\ {\isasymlongrightarrow}\isactrlsub D\ {\isacharparenleft}B\ {\isasymlongrightarrow}\isactrlsub D\ C{\isacharparenright}\ {\isasymlongrightarrow}\isactrlsub D\ A\ {\isasymlongrightarrow}\isactrlsub D\ C{\isachardoublequote}\ \isamarkupfalse%
\isacommand{by}\ {\isacharparenleft}simp\ only{\isacharcolon}\ pdl{\isacharunderscore}taut\ Valid{\isacharunderscore}Ret{\isacharparenright}\isanewline
\ \ \isamarkupfalse%
\isacommand{from}\ this\ a{\isadigit{1}}\ a{\isadigit{2}}\ \isamarkupfalse%
\isacommand{show}\ {\isacharquery}thesis\ \isamarkupfalse%
\isacommand{by}\ {\isacharparenleft}rule\ pdl{\isacharunderscore}mp{\isacharbrackleft}THEN\ pdl{\isacharunderscore}mp{\isacharbrackright}{\isacharparenright}\isanewline
\isamarkupfalse%
\isacommand{qed}\isamarkupfalse%
%
\begin{isamarkuptext}%
Some applications of the enhanced simplifier, which is now
         capable of proving prop. tautologies immediately.%
\end{isamarkuptext}%
\isamarkuptrue%
\isacommand{lemma}\ {\isachardoublequote}{\isasymturnstile}\ A\ {\isasymlongrightarrow}\isactrlsub D\ B\ {\isasymlongrightarrow}\isactrlsub D\ A{\isachardoublequote}\isanewline
\isamarkupfalse%
\isacommand{by}\ {\isacharparenleft}simp\ only{\isacharcolon}\ pdl{\isacharunderscore}taut\ Valid{\isacharunderscore}Ret{\isacharparenright}\isanewline
\isanewline
\isanewline
\isamarkupfalse%
\isacommand{lemma}\ {\isachardoublequote}{\isasymturnstile}\ {\isacharparenleft}P\ {\isasymand}\isactrlsub D\ Q\ {\isasymlongrightarrow}\isactrlsub D\ R{\isacharparenright}\ {\isasymlongleftrightarrow}\isactrlsub D\ {\isacharparenleft}P\ {\isasymlongrightarrow}\isactrlsub D\ Q\ {\isasymlongrightarrow}\isactrlsub D\ R{\isacharparenright}{\isachardoublequote}\isanewline
\isamarkupfalse%
\isacommand{by}\ {\isacharparenleft}simp\ only{\isacharcolon}\ pdl{\isacharunderscore}taut\ Valid{\isacharunderscore}Ret{\isacharparenright}\isanewline
\isanewline
\isanewline
\isamarkupfalse%
\isacommand{lemma}\ {\isachardoublequote}{\isasymturnstile}\ {\isacharparenleft}P\ {\isasymlongrightarrow}\isactrlsub D\ Q{\isacharparenright}\ {\isasymor}\isactrlsub D\ {\isacharparenleft}Q\ {\isasymlongrightarrow}\isactrlsub D\ P{\isacharparenright}{\isachardoublequote}\isanewline
\ \ \isamarkupfalse%
\isacommand{by}\ {\isacharparenleft}simp\ only{\isacharcolon}\ pdl{\isacharunderscore}taut\ Valid{\isacharunderscore}Ret{\isacharparenright}\isanewline
\isanewline
\isanewline
\isamarkupfalse%
\isacommand{lemma}\ {\isachardoublequote}{\isasymturnstile}\ {\isacharparenleft}P\ {\isasymlongrightarrow}\isactrlsub D\ Q{\isacharparenright}\ {\isasymand}\isactrlsub D\ {\isacharparenleft}{\isasymnot}\isactrlsub D\ P\ {\isasymlongrightarrow}\isactrlsub D\ R{\isacharparenright}\ {\isasymlongleftrightarrow}\isactrlsub D\ {\isacharparenleft}P\ {\isasymand}\isactrlsub D\ Q\ {\isasymor}\isactrlsub D\ {\isasymnot}\isactrlsub D\ P\ {\isasymand}\isactrlsub D\ R{\isacharparenright}{\isachardoublequote}\isanewline
\ \ \isamarkupfalse%
\isacommand{by}\ {\isacharparenleft}simp\ only{\isacharcolon}\ pdl{\isacharunderscore}taut\ Valid{\isacharunderscore}Ret{\isacharparenright}\isanewline
\isanewline
\isamarkupfalse%
\isacommand{end}\isanewline
\isamarkupfalse%
\end{isabellebody}%
%%% Local Variables:
%%% mode: latex
%%% TeX-master: "root"
%%% End:


%
\begin{isabellebody}%
\def\isabellecontext{MonEq}%
%
\isamarkupheader{Monadic Equality%
}
\isamarkuptrue%
\isacommand{theory}\ MonEq\ {\isacharequal}\ MonLogic{\isacharcolon}\isamarkupfalse%
%
\begin{isamarkuptext}%
\label{sec:moneq-thy}%
\end{isamarkuptext}%
\isamarkuptrue%
\isacommand{constdefs}\isanewline
\ \ {\isachardoublequote}MonEq{\isachardoublequote}\ \ {\isacharcolon}{\isacharcolon}\ {\isachardoublequote}{\isacharbrackleft}{\isacharprime}a\ D{\isacharcomma}\ {\isacharprime}a\ D{\isacharbrackright}\ {\isasymRightarrow}\ bool\ D{\isachardoublequote}\ \ \ \ {\isacharparenleft}\isakeyword{infixl}\ {\isachardoublequote}{\isacharequal}\isactrlsub D{\isachardoublequote}\ {\isadigit{6}}{\isadigit{0}}{\isacharparenright}\isanewline
\ \ {\isachardoublequote}MonEq\ a\ b\ {\isasymequiv}\ {\isasymUp}\ {\isacharparenleft}liftM{\isadigit{2}}\ {\isacharparenleft}op\ {\isacharequal}{\isacharparenright}\ {\isacharparenleft}{\isasymDown}\ a{\isacharparenright}\ {\isacharparenleft}{\isasymDown}\ b{\isacharparenright}{\isacharparenright}{\isachardoublequote}\isanewline
\isanewline
\isanewline
\isamarkupfalse%
\isacommand{lemma}\ MonEq{\isacharunderscore}Ret{\isacharunderscore}hom{\isacharcolon}\ {\isachardoublequote}{\isacharparenleft}{\isacharparenleft}Ret\ a{\isacharparenright}\ {\isacharequal}\isactrlsub D\ {\isacharparenleft}Ret\ b{\isacharparenright}{\isacharparenright}\ {\isacharequal}\ {\isacharparenleft}Ret\ {\isacharparenleft}a{\isacharequal}b{\isacharparenright}{\isacharparenright}{\isachardoublequote}\isanewline
\isamarkupfalse%
\isacommand{by}\ {\isacharparenleft}simp\ add{\isacharcolon}\ lift{\isacharunderscore}Ret{\isacharunderscore}hom\ MonEq{\isacharunderscore}def{\isacharparenright}\isamarkupfalse%
%
\begin{isamarkuptext}%
Transitivity of monadic equality.%
\end{isamarkuptext}%
\isamarkuptrue%
\isacommand{lemma}\ mon{\isacharunderscore}eq{\isacharunderscore}trans{\isacharcolon}\ {\isachardoublequote}{\isasymlbrakk}{\isasymturnstile}\ a\ {\isacharequal}\isactrlsub D\ b{\isacharsemicolon}\ {\isasymturnstile}\ b\ {\isacharequal}\isactrlsub D\ c{\isasymrbrakk}\ {\isasymLongrightarrow}\ {\isasymturnstile}\ a\ {\isacharequal}\isactrlsub D\ c{\isachardoublequote}\isanewline
\isamarkupfalse%
\isacommand{proof}\ {\isacharminus}\isanewline
\ \ \isamarkupfalse%
\isacommand{assume}\ ab{\isacharcolon}\ {\isachardoublequote}{\isasymturnstile}\ a\ {\isacharequal}\isactrlsub D\ b{\isachardoublequote}\ \isakeyword{and}\ bc{\isacharcolon}\ {\isachardoublequote}{\isasymturnstile}\ b\ {\isacharequal}\isactrlsub D\ c{\isachardoublequote}\isanewline
\ \ \isamarkupfalse%
\isacommand{have}\ {\isachardoublequote}{\isasymturnstile}\ {\isacharparenleft}a\ {\isacharequal}\isactrlsub D\ b{\isacharparenright}\ {\isasymlongrightarrow}\isactrlsub D\ {\isacharparenleft}b\ {\isacharequal}\isactrlsub D\ c{\isacharparenright}\ {\isasymlongrightarrow}\isactrlsub D\ {\isacharparenleft}a\ {\isacharequal}\isactrlsub D\ c{\isacharparenright}{\isachardoublequote}\isanewline
\ \ \ \ \isamarkupfalse%
\isacommand{apply}{\isacharparenleft}simp\ add{\isacharcolon}\ MonEq{\isacharunderscore}def\ impD{\isacharunderscore}def\ liftM{\isadigit{2}}{\isacharunderscore}def{\isacharparenright}\isanewline
\ \ \ \ \isamarkupfalse%
\isacommand{apply}{\isacharparenleft}simp\ add{\isacharcolon}\ Abs{\isacharunderscore}Dsef{\isacharunderscore}inverse\ dsef{\isacharunderscore}Rep{\isacharunderscore}Dsef\ Dsef{\isacharunderscore}def\ dsef{\isacharunderscore}seq\ mon{\isacharunderscore}ctr\ del{\isacharcolon}\ bind{\isacharunderscore}assoc{\isacharparenright}\isanewline
\ \ \ \ \isamarkupfalse%
\isacommand{apply}{\isacharparenleft}simp\ add{\isacharcolon}\ cp{\isacharunderscore}arb\ dsef{\isacharunderscore}cp{\isacharbrackleft}OF\ dsef{\isacharunderscore}Rep{\isacharunderscore}Dsef{\isacharbrackright}{\isacharparenright}\ \isanewline
\ \ \ \ \isamarkupfalse%
\isacommand{apply}{\isacharparenleft}simp\ add{\isacharcolon}\ commute{\isacharunderscore}dsef{\isacharbrackleft}of\ {\isachardoublequote}{\isasymDown}\ c{\isachardoublequote}\ {\isachardoublequote}{\isasymDown}\ a{\isachardoublequote}{\isacharbrackright}{\isacharparenright}\ \isanewline
\ \ \ \ \isamarkupfalse%
\isacommand{apply}{\isacharparenleft}simp\ add{\isacharcolon}\ commute{\isacharunderscore}dsef{\isacharbrackleft}of\ {\isachardoublequote}{\isasymDown}\ b{\isachardoublequote}\ {\isachardoublequote}{\isasymDown}\ a{\isachardoublequote}{\isacharbrackright}{\isacharparenright}\ \isanewline
\ \ \ \ \isamarkupfalse%
\isacommand{apply}{\isacharparenleft}simp\ add{\isacharcolon}\ cp{\isacharunderscore}arb\ dsef{\isacharunderscore}cp{\isacharbrackleft}OF\ dsef{\isacharunderscore}Rep{\isacharunderscore}Dsef{\isacharbrackright}\ del{\isacharcolon}\ bind{\isacharunderscore}assoc{\isacharparenright}\ \isanewline
\ \ \ \ \isamarkupfalse%
\isacommand{apply}\ {\isacharparenleft}simp\ add{\isacharcolon}\ dsef{\isacharunderscore}dis{\isacharbrackleft}OF\ dsef{\isacharunderscore}Rep{\isacharunderscore}Dsef{\isacharbrackright}\ dis{\isacharunderscore}left{\isadigit{2}}{\isacharparenright}\isanewline
\ \ \ \ \isamarkupfalse%
\isacommand{apply}{\isacharparenleft}subst\ Ret{\isacharunderscore}def{\isacharbrackleft}symmetric{\isacharbrackright}{\isacharparenright}\isanewline
\ \ \ \ \isamarkupfalse%
\isacommand{by}\ simp\isanewline
\ \ \isamarkupfalse%
\isacommand{from}\ this\ ab\ bc\ \isamarkupfalse%
\isacommand{show}\ {\isacharquery}thesis\ \isamarkupfalse%
\isacommand{by}\ {\isacharparenleft}rule\ pdl{\isacharunderscore}mp{\isacharbrackleft}THEN\ pdl{\isacharunderscore}mp{\isacharbrackright}{\isacharparenright}\isanewline
\isamarkupfalse%
\isacommand{qed}\isamarkupfalse%
%
\begin{isamarkuptext}%
Reflexivity of monadic equality.%
\end{isamarkuptext}%
\isamarkuptrue%
\isacommand{lemma}\ mon{\isacharunderscore}eq{\isacharunderscore}refl{\isacharcolon}\ \ {\isachardoublequote}{\isasymturnstile}\ a\ {\isacharequal}\isactrlsub D\ a{\isachardoublequote}\isanewline
\ \ \isamarkupfalse%
\isacommand{apply}{\isacharparenleft}simp\ add{\isacharcolon}\ MonEq{\isacharunderscore}def\ liftM{\isadigit{2}}{\isacharunderscore}def{\isacharparenright}\isanewline
\ \ \isamarkupfalse%
\isacommand{apply}{\isacharparenleft}simp\ add{\isacharcolon}\ cp{\isacharunderscore}arb\ dsef{\isacharunderscore}cp{\isacharbrackleft}OF\ dsef{\isacharunderscore}Rep{\isacharunderscore}Dsef{\isacharbrackright}{\isacharparenright}\isanewline
\ \ \isamarkupfalse%
\isacommand{apply}{\isacharparenleft}simp\ add{\isacharcolon}\ dis{\isacharunderscore}left{\isadigit{2}}\ dsef{\isacharunderscore}dis{\isacharbrackleft}OF\ dsef{\isacharunderscore}Rep{\isacharunderscore}Dsef{\isacharbrackright}{\isacharparenright}\isanewline
\ \ \isamarkupfalse%
\isacommand{apply}{\isacharparenleft}subst\ Ret{\isacharunderscore}def{\isacharbrackleft}symmetric{\isacharbrackright}{\isacharparenright}\isanewline
\ \ \isamarkupfalse%
\isacommand{by}\ {\isacharparenleft}simp{\isacharparenright}\isamarkupfalse%
%
\begin{isamarkuptext}%
Auxiliary lemma, just to help the simplifier.%
\end{isamarkuptext}%
\isamarkuptrue%
\isacommand{lemma}\ sym{\isacharunderscore}subst{\isacharunderscore}seq{\isadigit{2}}{\isacharcolon}\ {\isachardoublequote}{\isasymforall}x\ y{\isachardot}\ c\ x\ y\ {\isacharequal}\ c\ y\ x\ {\isasymLongrightarrow}\ \isanewline
\ \ {\isacharparenleft}{\isasymUp}\ {\isacharparenleft}do\ {\isacharbraceleft}x{\isasymleftarrow}p{\isacharsemicolon}\ y{\isasymleftarrow}q{\isacharsemicolon}\ c\ x\ y{\isacharbraceright}{\isacharparenright}{\isacharparenright}\ {\isacharequal}\ {\isacharparenleft}{\isasymUp}\ {\isacharparenleft}do\ {\isacharbraceleft}x{\isasymleftarrow}p{\isacharsemicolon}\ y{\isasymleftarrow}q{\isacharsemicolon}\ c\ y\ x{\isacharbraceright}{\isacharparenright}{\isacharparenright}{\isachardoublequote}\isanewline
\ \ \isamarkupfalse%
\isacommand{by}\ simp\isamarkupfalse%
%
\begin{isamarkuptext}%
Symmetry of monadic equality. 
  The simplifier gets into trouble here, for it must apply symmetry
  of real equality inside the scope of lambda terms. We circumvent
  this problem by extracting the essential proof obligation through 
  \isa{sym{\isacharunderscore}subst{\isacharunderscore}seq{\isadigit{2}}} and then working by hand.%
\end{isamarkuptext}%
\isamarkuptrue%
\isacommand{lemma}\ mon{\isacharunderscore}eq{\isacharunderscore}sym{\isacharcolon}\ \ \ {\isachardoublequote}{\isacharparenleft}a\ {\isacharequal}\isactrlsub D\ b{\isacharparenright}\ {\isacharequal}\ {\isacharparenleft}b\ {\isacharequal}\isactrlsub D\ a{\isacharparenright}{\isachardoublequote}\isanewline
\ \ \isamarkupfalse%
\isacommand{apply}{\isacharparenleft}simp\ add{\isacharcolon}\ MonEq{\isacharunderscore}def\ liftM{\isadigit{2}}{\isacharunderscore}def{\isacharparenright}\isanewline
\ \ \isamarkupfalse%
\isacommand{apply}{\isacharparenleft}simp\ add{\isacharcolon}\ commute{\isacharunderscore}dsef{\isacharbrackleft}of\ {\isachardoublequote}{\isasymDown}\ a{\isachardoublequote}\ {\isachardoublequote}{\isasymDown}\ b{\isachardoublequote}{\isacharbrackright}{\isacharparenright}\isanewline
\ \ \isamarkupfalse%
\isacommand{apply}{\isacharparenleft}rule\ sym{\isacharunderscore}subst{\isacharunderscore}seq{\isadigit{2}}{\isacharparenright}\isanewline
\ \ \isamarkupfalse%
\isacommand{apply}{\isacharparenleft}clarify{\isacharparenright}\isanewline
\ \ \isamarkupfalse%
\isacommand{apply}{\isacharparenleft}rule\ arg{\isacharunderscore}cong{\isacharbrackleft}\isakeyword{where}\ f\ {\isacharequal}\ ret{\isacharbrackright}{\isacharparenright}\ \isanewline
\ \ \isamarkupfalse%
\isacommand{by}\ {\isacharparenleft}rule\ eq{\isacharunderscore}sym{\isacharunderscore}conv{\isacharparenright}\isanewline
\isanewline
\isamarkupfalse%
\isacommand{end}\isanewline
\isamarkupfalse%
\end{isabellebody}%
%%% Local Variables:
%%% mode: latex
%%% TeX-master: "root"
%%% End:


%
\begin{isabellebody}%
\def\isabellecontext{PDL}%
%
\isamarkupheader{The Proof Calculus of Monadic Dynamic Logic%
}
\isamarkuptrue%
\isacommand{theory}\ PDL\ {\isacharequal}\ MonLogic{\isacharcolon}\isamarkupfalse%
%
\label{sec:pdl-thy}
%
\isamarkupsubsection{Types, Rules and Axioms%
}
\isamarkuptrue%
%
\begin{isamarkuptext}%
Types, rules and axioms for the box and diamond operators of PDL formulas.
  \label{isa:pdl-calculus}%
\end{isamarkuptext}%
\isamarkuptrue%
\isacommand{consts}\isanewline
\ Box\ {\isacharcolon}{\isacharcolon}\ {\isachardoublequote}{\isacharprime}a\ T\ {\isasymRightarrow}\ {\isacharparenleft}{\isacharprime}a\ {\isasymRightarrow}\ bool\ D{\isacharparenright}\ {\isasymRightarrow}\ bool\ D{\isachardoublequote}\ \ \ \ \ {\isacharparenleft}{\isachardoublequote}{\isacharbrackleft}{\isacharhash}\ {\isacharunderscore}{\isacharbrackright}{\isacharunderscore}{\isachardoublequote}\ {\isacharbrackleft}{\isadigit{0}}{\isacharcomma}\ {\isadigit{1}}{\isadigit{0}}{\isadigit{0}}{\isacharbrackright}\ {\isadigit{1}}{\isadigit{0}}{\isadigit{0}}{\isacharparenright}\isanewline
\ Dmd\ {\isacharcolon}{\isacharcolon}\ {\isachardoublequote}{\isacharprime}a\ T\ {\isasymRightarrow}\ {\isacharparenleft}{\isacharprime}a\ {\isasymRightarrow}\ bool\ D{\isacharparenright}\ {\isasymRightarrow}\ bool\ D{\isachardoublequote}\ \ \ \ \ {\isacharparenleft}{\isachardoublequote}{\isasymlangle}{\isacharunderscore}{\isasymrangle}{\isacharunderscore}{\isachardoublequote}\ {\isacharbrackleft}{\isadigit{0}}{\isacharcomma}\ {\isadigit{1}}{\isadigit{0}}{\isadigit{0}}{\isacharbrackright}\ {\isadigit{1}}{\isadigit{0}}{\isadigit{0}}{\isacharparenright}\isamarkupfalse%
%
\begin{isamarkuptext}%
Syntax translations that let you write e.g.
    \isa{{\isacharbrackleft}{\isacharhash}\ x{\isasymleftarrow}p{\isacharsemicolon}\ y{\isasymleftarrow}q{\isacharbrackright}{\isacharparenleft}ret\ {\isacharparenleft}x{\isacharequal}y{\isacharparenright}{\isacharparenright}}
  for 
    \isa{Box\ {\isacharparenleft}do\ {\isacharbraceleft}x{\isasymleftarrow}p{\isacharsemicolon}\ y{\isasymleftarrow}q{\isacharsemicolon}\ ret\ {\isacharparenleft}x{\isacharcomma}y{\isacharparenright}{\isacharbraceright}{\isacharparenright}\ {\isacharparenleft}{\isasymlambda}{\isacharparenleft}x{\isacharcomma}y{\isacharparenright}{\isachardot}\ ret\ {\isacharparenleft}x{\isacharequal}y{\isacharparenright}{\isacharparenright}}
  Essentially, these translations collect all bound variables inside the 
  boxes and return them as a tuple. The lambda term that constitutes the 
  second argument of Box will then also take a tuple pattern as its sole
  argument.%
\end{isamarkuptext}%
\isamarkuptrue%
\isacommand{nonterminals}\isanewline
\ \ bndseq\ bndstep\isanewline
\ \isanewline
\isamarkupfalse%
\isacommand{syntax}\ {\isacharparenleft}xsymbols{\isacharparenright}\isanewline
\ \ {\isachardoublequote}{\isacharunderscore}pdlbox{\isachardoublequote}\ \ {\isacharcolon}{\isacharcolon}\ {\isachardoublequote}{\isacharbrackleft}bndseq{\isacharcomma}\ bool\ D{\isacharbrackright}\ {\isasymRightarrow}\ bool\ D{\isachardoublequote}\ \ \ \ \ \ \ \ {\isacharparenleft}{\isachardoublequote}{\isacharbrackleft}{\isacharhash}\ {\isacharunderscore}{\isacharbrackright}{\isacharunderscore}{\isachardoublequote}\ {\isacharbrackleft}{\isadigit{0}}{\isacharcomma}\ {\isadigit{1}}{\isadigit{0}}{\isadigit{0}}{\isacharbrackright}\ {\isadigit{1}}{\isadigit{0}}{\isadigit{0}}{\isacharparenright}\ \isanewline
\ \ {\isachardoublequote}{\isacharunderscore}pdldmd{\isachardoublequote}\ \ {\isacharcolon}{\isacharcolon}\ {\isachardoublequote}{\isacharbrackleft}bndseq{\isacharcomma}\ bool\ D{\isacharbrackright}\ {\isasymRightarrow}\ bool\ D{\isachardoublequote}\ \ \ \ \ \ \ \ {\isacharparenleft}{\isachardoublequote}{\isasymlangle}{\isacharunderscore}{\isasymrangle}{\isacharunderscore}{\isachardoublequote}\ {\isacharbrackleft}{\isadigit{0}}{\isacharcomma}\ {\isadigit{1}}{\isadigit{0}}{\isadigit{0}}{\isacharbrackright}\ {\isadigit{1}}{\isadigit{0}}{\isadigit{0}}{\isacharparenright}\isanewline
\ \ {\isachardoublequote}{\isacharunderscore}pdlbnd{\isachardoublequote}\ \ {\isacharcolon}{\isacharcolon}\ {\isachardoublequote}{\isacharbrackleft}idt{\isacharcomma}\ {\isacharprime}a\ T{\isacharbrackright}\ {\isasymRightarrow}\ bndstep{\isachardoublequote}\ \ \ \ \ \ \ \ \ \ \ {\isacharparenleft}{\isachardoublequote}{\isacharunderscore}{\isasymleftarrow}{\isacharunderscore}{\isachardoublequote}{\isacharparenright}\isanewline
\ \ {\isachardoublequote}{\isacharunderscore}pdlseq{\isachardoublequote}\ \ {\isacharcolon}{\isacharcolon}\ {\isachardoublequote}{\isacharbrackleft}bndstep{\isacharcomma}\ bndseq{\isacharbrackright}\ {\isasymRightarrow}\ bndseq{\isachardoublequote}\ \ \ \ \ {\isacharparenleft}{\isachardoublequote}{\isacharunderscore}{\isacharsemicolon}{\isacharslash}\ {\isacharunderscore}{\isachardoublequote}{\isacharparenright}\isanewline
\ \ {\isachardoublequote}{\isachardoublequote}\ \ \ \ \ \ \ \ \ {\isacharcolon}{\isacharcolon}\ {\isachardoublequote}bndstep\ {\isasymRightarrow}\ bndseq{\isachardoublequote}\ \ \ \ \ \ \ \ \ \ \ \ \ \ \ {\isacharparenleft}{\isachardoublequote}{\isacharunderscore}{\isachardoublequote}{\isacharparenright}\isanewline
\ \ {\isachardoublequote}{\isacharunderscore}pdlin{\isachardoublequote}\ \ \ {\isacharcolon}{\isacharcolon}\ {\isachardoublequote}{\isacharbrackleft}pttrn{\isacharcomma}\ bndseq{\isacharbrackright}\ {\isasymRightarrow}\ bndseq{\isachardoublequote}\ \ \ \ \ \ \ \isanewline
\ \ {\isachardoublequote}{\isacharunderscore}pdlout{\isachardoublequote}\ \ {\isacharcolon}{\isacharcolon}\ {\isachardoublequote}{\isacharbrackleft}pttrn{\isacharcomma}\ bndseq{\isacharbrackright}\ {\isasymRightarrow}\ bndseq{\isachardoublequote}\ \ \ \ \ \ \isanewline
\isanewline
\isanewline
\isanewline
\isamarkupfalse%
\isacommand{translations}\ \isanewline
\ \ {\isachardoublequote}{\isacharunderscore}pdlbox\ {\isacharparenleft}{\isacharunderscore}pdlseq\ {\isacharparenleft}{\isacharunderscore}pdlbnd\ x\ p{\isacharparenright}\ r{\isacharparenright}\ phi{\isachardoublequote}\ \ \isanewline
\ \ \ \ \ \ \ \ \ \ {\isasymrightharpoonup}\ \ {\isachardoublequote}Box\ {\isacharparenleft}{\isacharunderscore}pdlseq\ {\isacharparenleft}{\isacharunderscore}pdlbnd\ x\ p{\isacharparenright}\ {\isacharparenleft}{\isacharunderscore}pdlin\ x\ r{\isacharparenright}{\isacharparenright}\ phi{\isachardoublequote}\isanewline
\ \ {\isachardoublequote}{\isacharunderscore}pdlbox\ {\isacharparenleft}{\isacharunderscore}pdlbnd\ x\ p{\isacharparenright}\ phi{\isachardoublequote}\ \ {\isasymrightharpoonup}\ {\isachardoublequote}Box\ p\ {\isacharparenleft}{\isasymlambda}x{\isachardot}\ phi{\isacharparenright}{\isachardoublequote}\isanewline
\ \ {\isachardoublequote}{\isacharunderscore}pdldmd\ {\isacharparenleft}{\isacharunderscore}pdlseq\ {\isacharparenleft}{\isacharunderscore}pdlbnd\ x\ p{\isacharparenright}\ r{\isacharparenright}\ phi{\isachardoublequote}\ \ \isanewline
\ \ \ \ \ \ \ \ \ \ {\isasymrightharpoonup}\ \ {\isachardoublequote}Dmd\ {\isacharparenleft}{\isacharunderscore}pdlseq\ {\isacharparenleft}{\isacharunderscore}pdlbnd\ x\ p{\isacharparenright}\ {\isacharparenleft}{\isacharunderscore}pdlin\ x\ r{\isacharparenright}{\isacharparenright}\ phi{\isachardoublequote}\isanewline
\ \ {\isachardoublequote}{\isacharunderscore}pdldmd\ {\isacharparenleft}{\isacharunderscore}pdlbnd\ x\ p{\isacharparenright}\ phi{\isachardoublequote}\ \ {\isasymrightharpoonup}\ {\isachardoublequote}Dmd\ p\ {\isacharparenleft}{\isasymlambda}x{\isachardot}\ phi{\isacharparenright}{\isachardoublequote}\isanewline
\ \ {\isachardoublequote}{\isacharunderscore}pdlin\ tpl\ {\isacharparenleft}{\isacharunderscore}pdlseq\ {\isacharparenleft}{\isacharunderscore}pdlbnd\ x\ p{\isacharparenright}\ r{\isacharparenright}{\isachardoublequote}\isanewline
\ \ \ \ \ \ \ \ \ \ {\isasymrightharpoonup}\ \ {\isachardoublequote}{\isacharunderscore}pdlseq\ {\isacharparenleft}{\isacharunderscore}pdlbnd\ x\ p{\isacharparenright}\ {\isacharparenleft}{\isacharunderscore}pdlin\ {\isacharparenleft}tpl{\isacharcomma}\ x{\isacharparenright}\ r{\isacharparenright}{\isachardoublequote}\isanewline
\ \ {\isachardoublequote}{\isacharunderscore}pdlin\ tpl\ {\isacharparenleft}{\isacharunderscore}pdlbnd\ x\ p{\isacharparenright}{\isachardoublequote}\isanewline
\ \ \ \ \ \ \ \ \ \ {\isasymrightharpoonup}\ \ {\isachardoublequote}{\isacharunderscore}pdlout\ {\isacharparenleft}tpl{\isacharcomma}x{\isacharparenright}\ {\isacharparenleft}do\ {\isacharbraceleft}x{\isasymleftarrow}p{\isacharsemicolon}\ ret{\isacharparenleft}tpl{\isacharcomma}x{\isacharparenright}{\isacharbraceright}{\isacharparenright}{\isachardoublequote}\isanewline
\ \ {\isachardoublequote}{\isacharunderscore}pdlseq\ {\isacharparenleft}{\isacharunderscore}pdlbnd\ x\ p{\isacharparenright}\ {\isacharparenleft}{\isacharunderscore}pdlout\ tpl\ r{\isacharparenright}{\isachardoublequote}\isanewline
\ \ \ \ \ \ \ \ \ \ {\isasymrightharpoonup}\ \ {\isachardoublequote}{\isacharunderscore}pdlout\ tpl\ {\isacharparenleft}do\ {\isacharbraceleft}x{\isasymleftarrow}p{\isacharsemicolon}\ r{\isacharbraceright}{\isacharparenright}{\isachardoublequote}\isanewline
\ \ {\isachardoublequote}Box\ {\isacharparenleft}{\isacharunderscore}pdlout\ tpl\ r{\isacharparenright}\ phi{\isachardoublequote}\ \isanewline
\ \ \ \ \ \ \ \ \ \ {\isasymrightharpoonup}\ \ {\isachardoublequote}Box\ r\ {\isacharparenleft}{\isasymlambda}tpl{\isachardot}\ phi{\isacharparenright}{\isachardoublequote}\isanewline
\ \ {\isachardoublequote}Dmd\ {\isacharparenleft}{\isacharunderscore}pdlout\ tpl\ r{\isacharparenright}\ phi{\isachardoublequote}\ \isanewline
\ \ \ \ \ \ \ \ \ \ {\isasymrightharpoonup}\ \ {\isachardoublequote}Dmd\ r\ {\isacharparenleft}{\isasymlambda}tpl{\isachardot}\ phi{\isacharparenright}{\isachardoublequote}\isamarkupfalse%
%
\begin{isamarkuptext}%
The axioms of the proof calculus for propositional dynamic logic.%
\end{isamarkuptext}%
\isamarkuptrue%
\isacommand{axioms}\isanewline
\ \ pdl{\isacharunderscore}nec{\isacharcolon}\ \ \ {\isachardoublequote}{\isacharparenleft}{\isasymforall}x{\isachardot}\ {\isasymturnstile}\ P\ x{\isacharparenright}\ {\isasymLongrightarrow}\ {\isasymturnstile}\ {\isacharbrackleft}{\isacharhash}\ x{\isasymleftarrow}p{\isacharbrackright}{\isacharparenleft}P\ x{\isacharparenright}{\isachardoublequote}\isanewline
\ \ pdl{\isacharunderscore}mp{\isacharunderscore}{\isacharcolon}\ \ \ \ {\isachardoublequote}{\isasymlbrakk}{\isasymturnstile}\ {\isacharparenleft}P\ {\isasymlongrightarrow}\isactrlsub D\ \ Q{\isacharparenright}{\isacharsemicolon}\ {\isasymturnstile}\ P{\isasymrbrakk}\ {\isasymLongrightarrow}\ {\isasymturnstile}\ Q{\isachardoublequote}\ \ %
\isamarkupcmt{Only repeated here for completeness.%
}
\isanewline
\ \isanewline
\ \ pdl{\isacharunderscore}k{\isadigit{1}}{\isacharcolon}\ \ \ \ {\isachardoublequote}{\isasymturnstile}\ {\isacharbrackleft}{\isacharhash}\ x{\isasymleftarrow}p{\isacharbrackright}{\isacharparenleft}P\ x\ {\isasymlongrightarrow}\isactrlsub D\ Q\ x{\isacharparenright}\ {\isasymlongrightarrow}\isactrlsub D\ {\isacharbrackleft}{\isacharhash}\ x{\isasymleftarrow}p{\isacharbrackright}{\isacharparenleft}P\ x{\isacharparenright}\ {\isasymlongrightarrow}\isactrlsub D\ {\isacharbrackleft}{\isacharhash}\ x{\isasymleftarrow}p{\isacharbrackright}{\isacharparenleft}Q\ x{\isacharparenright}{\isachardoublequote}\isanewline
\ \ pdl{\isacharunderscore}k{\isadigit{2}}{\isacharcolon}\ \ \ \ {\isachardoublequote}{\isasymturnstile}\ {\isacharbrackleft}{\isacharhash}\ x{\isasymleftarrow}p{\isacharbrackright}{\isacharparenleft}P\ x\ {\isasymlongrightarrow}\isactrlsub D\ Q\ x{\isacharparenright}\ {\isasymlongrightarrow}\isactrlsub D\ {\isasymlangle}x{\isasymleftarrow}p{\isasymrangle}{\isacharparenleft}P\ x{\isacharparenright}\ {\isasymlongrightarrow}\isactrlsub D\ {\isasymlangle}x{\isasymleftarrow}p{\isasymrangle}{\isacharparenleft}Q\ x{\isacharparenright}{\isachardoublequote}\isanewline
\ \ pdl{\isacharunderscore}k{\isadigit{3}}B{\isacharcolon}\ \ \ {\isachardoublequote}{\isasymturnstile}\ Ret\ P\ {\isasymlongrightarrow}\isactrlsub D\ {\isacharbrackleft}{\isacharhash}\ x{\isasymleftarrow}p{\isacharbrackright}{\isacharparenleft}Ret\ P{\isacharparenright}{\isachardoublequote}\isanewline
\ \ pdl{\isacharunderscore}k{\isadigit{3}}D{\isacharcolon}\ \ \ {\isachardoublequote}{\isasymturnstile}\ {\isasymlangle}x{\isasymleftarrow}p{\isasymrangle}{\isacharparenleft}Ret\ P{\isacharparenright}\ {\isasymlongrightarrow}\isactrlsub D\ Ret\ P{\isachardoublequote}\isanewline
\ \ pdl{\isacharunderscore}k{\isadigit{4}}{\isacharcolon}\ \ \ \ {\isachardoublequote}{\isasymturnstile}\ {\isasymlangle}x{\isasymleftarrow}p{\isasymrangle}{\isacharparenleft}P\ x\ {\isasymor}\isactrlsub D\ Q\ x{\isacharparenright}\ {\isasymlongrightarrow}\isactrlsub D\ {\isacharparenleft}{\isasymlangle}x{\isasymleftarrow}p{\isasymrangle}{\isacharparenleft}P\ x{\isacharparenright}\ {\isasymor}\isactrlsub D\ {\isasymlangle}x{\isasymleftarrow}p{\isasymrangle}{\isacharparenleft}Q\ x{\isacharparenright}{\isacharparenright}{\isachardoublequote}\isanewline
\ \ pdl{\isacharunderscore}k{\isadigit{5}}{\isacharcolon}\ \ \ \ {\isachardoublequote}{\isasymturnstile}\ {\isacharparenleft}{\isasymlangle}x{\isasymleftarrow}p{\isasymrangle}{\isacharparenleft}P\ x{\isacharparenright}\ {\isasymlongrightarrow}\isactrlsub D\ {\isacharbrackleft}{\isacharhash}\ x{\isasymleftarrow}p{\isacharbrackright}{\isacharparenleft}Q\ x{\isacharparenright}{\isacharparenright}\ {\isasymlongrightarrow}\isactrlsub D\ {\isacharbrackleft}{\isacharhash}\ x{\isasymleftarrow}p{\isacharbrackright}{\isacharparenleft}P\ x\ {\isasymlongrightarrow}\isactrlsub D\ Q\ x{\isacharparenright}{\isachardoublequote}\isanewline
\ \ pdl{\isacharunderscore}seqB{\isacharcolon}\ \ {\isachardoublequote}{\isasymturnstile}\ {\isacharbrackleft}{\isacharhash}\ x{\isasymleftarrow}p{\isacharsemicolon}\ y{\isasymleftarrow}q\ x{\isacharbrackright}{\isacharparenleft}P\ x\ y{\isacharparenright}\ {\isasymlongleftrightarrow}\isactrlsub D\ {\isacharbrackleft}{\isacharhash}\ x{\isasymleftarrow}p{\isacharbrackright}{\isacharbrackleft}{\isacharhash}\ y{\isasymleftarrow}q\ x{\isacharbrackright}{\isacharparenleft}P\ x\ y{\isacharparenright}{\isachardoublequote}\isanewline
\ \ pdl{\isacharunderscore}seqD{\isacharcolon}\ \ {\isachardoublequote}{\isasymturnstile}\ {\isasymlangle}x{\isasymleftarrow}p{\isacharsemicolon}\ y{\isasymleftarrow}q\ x{\isasymrangle}{\isacharparenleft}P\ x\ y{\isacharparenright}\ {\isasymlongleftrightarrow}\isactrlsub D\ {\isasymlangle}x{\isasymleftarrow}p{\isasymrangle}{\isasymlangle}y{\isasymleftarrow}q\ x{\isasymrangle}{\isacharparenleft}P\ x\ y{\isacharparenright}{\isachardoublequote}\isanewline
\ \ pdl{\isacharunderscore}ctrB{\isacharcolon}\ \ {\isachardoublequote}{\isasymturnstile}\ {\isacharbrackleft}{\isacharhash}\ x{\isasymleftarrow}p{\isacharsemicolon}\ y{\isasymleftarrow}q\ x{\isacharbrackright}{\isacharparenleft}P\ y{\isacharparenright}\ {\isasymlongrightarrow}\isactrlsub D\ {\isacharbrackleft}{\isacharhash}\ y{\isasymleftarrow}do\ {\isacharbraceleft}x{\isasymleftarrow}p{\isacharsemicolon}\ q\ x{\isacharbraceright}{\isacharbrackright}{\isacharparenleft}P\ y{\isacharparenright}{\isachardoublequote}\isanewline
\ \ pdl{\isacharunderscore}ctrD{\isacharcolon}\ \ {\isachardoublequote}{\isasymturnstile}\ {\isasymlangle}y{\isasymleftarrow}do\ {\isacharbraceleft}x{\isasymleftarrow}p{\isacharsemicolon}\ q\ x{\isacharbraceright}{\isasymrangle}{\isacharparenleft}P\ y{\isacharparenright}\ {\isasymlongrightarrow}\isactrlsub D\ \ {\isasymlangle}x{\isasymleftarrow}p{\isacharsemicolon}\ y{\isasymleftarrow}q\ x{\isasymrangle}{\isacharparenleft}P\ y{\isacharparenright}{\isachardoublequote}\isanewline
\ \ pdl{\isacharunderscore}retB{\isacharcolon}\ \ {\isachardoublequote}{\isasymturnstile}\ {\isacharbrackleft}{\isacharhash}\ x{\isasymleftarrow}ret\ a{\isacharbrackright}{\isacharparenleft}P\ x{\isacharparenright}\ {\isasymlongleftrightarrow}\isactrlsub D\ P\ a{\isachardoublequote}\isanewline
\ \ pdl{\isacharunderscore}retD{\isacharcolon}\ \ {\isachardoublequote}{\isasymturnstile}\ {\isasymlangle}x{\isasymleftarrow}ret\ a{\isasymrangle}{\isacharparenleft}P\ x{\isacharparenright}\ {\isasymlongleftrightarrow}\isactrlsub D\ P\ a{\isachardoublequote}\isanewline
\ \ pdl{\isacharunderscore}dsefB{\isacharcolon}\ {\isachardoublequote}dsef\ p\ {\isasymLongrightarrow}\ {\isasymturnstile}\ {\isasymUp}\ {\isacharparenleft}do\ {\isacharbraceleft}a{\isasymleftarrow}p{\isacharsemicolon}\ {\isasymDown}\ {\isacharparenleft}P\ a{\isacharparenright}{\isacharbraceright}{\isacharparenright}\ {\isasymlongleftrightarrow}\isactrlsub D\ {\isacharbrackleft}{\isacharhash}\ a{\isasymleftarrow}p{\isacharbrackright}{\isacharparenleft}P\ a{\isacharparenright}{\isachardoublequote}\isanewline
\ \ pdl{\isacharunderscore}dsefD{\isacharcolon}\ {\isachardoublequote}dsef\ p\ {\isasymLongrightarrow}\ {\isasymturnstile}\ {\isasymUp}\ {\isacharparenleft}do\ {\isacharbraceleft}a{\isasymleftarrow}p{\isacharsemicolon}\ {\isasymDown}\ {\isacharparenleft}P\ a{\isacharparenright}{\isacharbraceright}{\isacharparenright}\ {\isasymlongleftrightarrow}\isactrlsub D\ {\isasymlangle}a{\isasymleftarrow}p{\isasymrangle}{\isacharparenleft}P\ a{\isacharparenright}{\isachardoublequote}\isamarkupfalse%
%
\begin{isamarkuptext}%
A simpler notion of sequencing is often more practical in real programs.
  Essentially this boils down to admitting just one binding within the modal
  operators.%
\end{isamarkuptext}%
\isamarkuptrue%
\isacommand{axioms}\isanewline
pdl{\isacharunderscore}seqB{\isacharunderscore}simp{\isacharcolon}\ {\isachardoublequote}{\isasymturnstile}\ {\isacharparenleft}\ {\isacharbrackleft}{\isacharhash}\ x{\isasymleftarrow}p{\isacharbrackright}{\isacharbrackleft}{\isacharhash}\ y{\isasymleftarrow}q\ x{\isacharbrackright}{\isacharparenleft}P\ y{\isacharparenright}\ {\isacharparenright}\ {\isasymlongleftrightarrow}\isactrlsub D\ {\isacharparenleft}\ {\isacharbrackleft}{\isacharhash}\ y{\isasymleftarrow}do\ {\isacharbraceleft}x{\isasymleftarrow}p{\isacharsemicolon}\ q\ x{\isacharbraceright}{\isacharbrackright}{\isacharparenleft}P\ y{\isacharparenright}\ {\isacharparenright}{\isachardoublequote}\isanewline
pdl{\isacharunderscore}seqD{\isacharunderscore}simp{\isacharcolon}\ {\isachardoublequote}{\isasymturnstile}\ {\isacharparenleft}\ {\isasymlangle}x{\isasymleftarrow}p{\isasymrangle}{\isasymlangle}y{\isasymleftarrow}q\ x{\isasymrangle}{\isacharparenleft}P\ y{\isacharparenright}\ {\isacharparenright}\ {\isasymlongleftrightarrow}\isactrlsub D\ {\isacharparenleft}\ {\isasymlangle}y{\isasymleftarrow}do\ {\isacharbraceleft}x{\isasymleftarrow}p{\isacharsemicolon}\ q\ x{\isacharbraceright}{\isasymrangle}{\isacharparenleft}P\ y{\isacharparenright}\ {\isacharparenright}{\isachardoublequote}\isamarkupfalse%
%
\begin{isamarkuptext}%
For simple monads \cite{SchroederMossakowski:PDL} both rules can be derived from
  axiom \isa{pdl{\isacharunderscore}seqB} (or \isa{pdl{\isacharunderscore}seqD}). Simplicity is
  exploited through the use of the converse rule of \isa{pdl{\isacharunderscore}ctrB}.%
\end{isamarkuptext}%
\isamarkuptrue%
\isacommand{lemma}\ {\isachardoublequote}{\isasymturnstile}\ {\isacharbrackleft}{\isacharhash}\ y{\isasymleftarrow}do\ {\isacharbraceleft}x{\isasymleftarrow}p{\isacharsemicolon}\ q\ x{\isacharbraceright}{\isacharbrackright}{\isacharparenleft}P\ y{\isacharparenright}\ {\isasymlongrightarrow}\isactrlsub D\ {\isacharbrackleft}{\isacharhash}\ x{\isasymleftarrow}p{\isacharsemicolon}\ y{\isasymleftarrow}q\ x{\isacharbrackright}{\isacharparenleft}P\ y{\isacharparenright}\ {\isasymLongrightarrow}\isanewline
\ \ \ \ \ \ \ {\isasymturnstile}\ {\isacharparenleft}\ {\isacharbrackleft}{\isacharhash}\ p{\isacharbrackright}{\isacharparenleft}{\isasymlambda}x{\isachardot}\ {\isacharbrackleft}{\isacharhash}\ q\ x{\isacharbrackright}P{\isacharparenright}\ {\isacharparenright}\ {\isasymlongleftrightarrow}\isactrlsub D\ {\isacharparenleft}\ {\isacharbrackleft}{\isacharhash}\ do\ {\isacharbraceleft}x{\isasymleftarrow}p{\isacharsemicolon}\ q\ x{\isacharbraceright}{\isacharbrackright}P\ {\isacharparenright}{\isachardoublequote}\isanewline
\ \ \isamarkupfalse%
\isacommand{apply}{\isacharparenleft}rule\ pdl{\isacharunderscore}iffI{\isacharparenright}\isanewline
\ \ \isamarkupfalse%
\isacommand{apply}{\isacharparenleft}rule\ pdl{\isacharunderscore}imp{\isacharunderscore}trans{\isacharparenright}\isanewline
\ \ \ \ \isamarkupfalse%
\isacommand{apply}{\isacharparenleft}rule\ pdl{\isacharunderscore}iffD{\isadigit{2}}{\isacharbrackleft}OF\ pdl{\isacharunderscore}seqB{\isacharbrackright}{\isacharparenright}\isanewline
\ \ \ \ \isamarkupfalse%
\isacommand{apply}{\isacharparenleft}rule\ pdl{\isacharunderscore}ctrB{\isacharparenright}\ \ %
\isamarkupcmt{dispose of the trailing ret expression%
}
\isanewline
\ \ \isamarkupfalse%
\isacommand{apply}{\isacharparenleft}rule\ pdl{\isacharunderscore}imp{\isacharunderscore}trans{\isacharparenright}\isanewline
\ \ \ \ \isamarkupfalse%
\isacommand{apply}{\isacharparenleft}assumption{\isacharparenright}\ \ \ \ \ %
\isamarkupcmt{this time dispose by the converse of \isa{pdl{\isacharunderscore}ctrB}%
}
\isanewline
\ \ \ \ \isamarkupfalse%
\isacommand{apply}{\isacharparenleft}rule\ pdl{\isacharunderscore}iffD{\isadigit{1}}{\isacharbrackleft}OF\ pdl{\isacharunderscore}seqB{\isacharbrackright}{\isacharparenright}\isanewline
\isamarkupfalse%
\isacommand{done}\isamarkupfalse%
%
\begin{isamarkuptext}%
Further axioms satisfied by logically regular monads (which we deal with here).
  Cf. \cite[Page 601]{SchroederMossakowski:PDL}%
\end{isamarkuptext}%
\isamarkuptrue%
\isacommand{axioms}\isanewline
\ \ pdl{\isacharunderscore}eqB{\isacharcolon}\ {\isachardoublequote}{\isasymturnstile}\ Ret\ {\isacharparenleft}p\ {\isacharequal}\ q{\isacharparenright}\ {\isasymlongrightarrow}\isactrlsub D\ {\isacharbrackleft}{\isacharhash}\ x{\isasymleftarrow}p{\isacharbrackright}{\isacharparenleft}P\ x{\isacharparenright}\ {\isasymlongrightarrow}\isactrlsub D\ {\isacharbrackleft}{\isacharhash}\ x{\isasymleftarrow}q{\isacharbrackright}{\isacharparenleft}P\ x{\isacharparenright}{\isachardoublequote}\isanewline
\ \ pdl{\isacharunderscore}eqD{\isacharcolon}\ {\isachardoublequote}{\isasymturnstile}\ Ret\ {\isacharparenleft}p\ {\isacharequal}\ q{\isacharparenright}\ {\isasymlongrightarrow}\isactrlsub D\ {\isasymlangle}x{\isasymleftarrow}p{\isasymrangle}{\isacharparenleft}P\ x{\isacharparenright}\ {\isasymlongrightarrow}\isactrlsub D\ {\isasymlangle}x{\isasymleftarrow}q{\isasymrangle}{\isacharparenleft}P\ x{\isacharparenright}{\isachardoublequote}\isamarkupfalse%
%
\isamarkupsubsection{Derived Rules of Inference%
}
\isamarkuptrue%
%
\begin{isamarkuptext}%
`Multiple' modus ponens, provided for convenience.%
\end{isamarkuptext}%
\isamarkuptrue%
\isacommand{lemmas}\ \isanewline
pdl{\isacharunderscore}mp{\isacharunderscore}{\isadigit{2}}x\ {\isacharequal}\ pdl{\isacharunderscore}mp{\isacharbrackleft}THEN\ pdl{\isacharunderscore}mp{\isacharbrackright}\ \isakeyword{and}\isanewline
pdl{\isacharunderscore}mp{\isacharunderscore}{\isadigit{3}}x\ {\isacharequal}\ pdl{\isacharunderscore}mp{\isacharbrackleft}THEN\ pdl{\isacharunderscore}mp{\isacharcomma}\ THEN\ pdl{\isacharunderscore}mp{\isacharbrackright}\isamarkupfalse%
%
\begin{isamarkuptext}%
First half of the classical relationship between diamond and box.%
\end{isamarkuptext}%
\isamarkuptrue%
\isacommand{lemma}\ dmd{\isacharunderscore}box{\isacharunderscore}rel{\isadigit{1}}{\isacharcolon}\ {\isachardoublequote}{\isasymturnstile}\ {\isacharparenleft}{\isacharbrackleft}{\isacharhash}\ x{\isasymleftarrow}p{\isacharbrackright}{\isacharparenleft}P\ x\ {\isasymlongrightarrow}\isactrlsub D\ Ret\ False{\isacharparenright}\ {\isasymlongrightarrow}\isactrlsub D\ Ret\ False{\isacharparenright}\ {\isasymlongrightarrow}\isactrlsub D\ {\isasymlangle}x{\isasymleftarrow}p{\isasymrangle}{\isacharparenleft}P\ x{\isacharparenright}{\isachardoublequote}\ \isanewline
\ \ {\isacharparenleft}\isakeyword{is}\ {\isachardoublequote}{\isasymturnstile}\ {\isacharparenleft}{\isacharquery}b\ {\isasymlongrightarrow}\isactrlsub D\ Ret\ False{\isacharparenright}\ {\isasymlongrightarrow}\isactrlsub D\ {\isacharquery}d{\isachardoublequote}{\isacharparenright}\isanewline
\isamarkupfalse%
\isacommand{proof}\ {\isacharminus}\isanewline
\ \ %
\isamarkupcmt{Show a classically equivalent statement%
}
\isanewline
\ \ \isamarkupfalse%
\isacommand{have}\ {\isachardoublequote}{\isasymturnstile}\ {\isacharparenleft}{\isacharquery}d\ {\isasymlongrightarrow}\isactrlsub D\ Ret\ False{\isacharparenright}\ {\isasymlongrightarrow}\isactrlsub D\ {\isacharquery}b{\isachardoublequote}\ \isanewline
\ \ \isamarkupfalse%
\isacommand{proof}\ {\isacharminus}\isanewline
\ \ \ \ %
\isamarkupcmt{The `usual' axiomatic proof method%
}
\isanewline
\ \ \ \ \isamarkupfalse%
\isacommand{have}\ f{\isadigit{1}}{\isacharcolon}\ {\isachardoublequote}{\isasymturnstile}\ {\isacharparenleft}{\isacharparenleft}{\isacharquery}d\ {\isasymlongrightarrow}\isactrlsub D\ {\isacharbrackleft}{\isacharhash}\ x{\isasymleftarrow}p{\isacharbrackright}{\isacharparenleft}Ret\ False{\isacharparenright}{\isacharparenright}\ {\isasymlongrightarrow}\isactrlsub D\ {\isacharquery}b{\isacharparenright}\ {\isasymlongrightarrow}\isactrlsub D\ \isanewline
\ \ \ \ \ \ \ \ \ \ \ \ \ \ \ \ {\isacharparenleft}{\isacharquery}d\ {\isasymlongrightarrow}\isactrlsub D\ Ret\ False{\isacharparenright}\ {\isasymlongrightarrow}\isactrlsub D\ {\isacharquery}b{\isachardoublequote}\isanewline
\ \ \ \ \ \ \isamarkupfalse%
\isacommand{by}\ {\isacharparenleft}simp\ add{\isacharcolon}\ pdl{\isacharunderscore}taut{\isacharparenright}\isanewline
\ \ \ \ \isamarkupfalse%
\isacommand{have}\ f{\isadigit{2}}{\isacharcolon}\ {\isachardoublequote}{\isasymturnstile}\ {\isacharparenleft}{\isacharquery}d\ {\isasymlongrightarrow}\isactrlsub D\ {\isacharbrackleft}{\isacharhash}\ x{\isasymleftarrow}p{\isacharbrackright}{\isacharparenleft}Ret\ False{\isacharparenright}{\isacharparenright}\ {\isasymlongrightarrow}\isactrlsub D\ {\isacharquery}b{\isachardoublequote}\ \isanewline
\ \ \ \ \ \ \isamarkupfalse%
\isacommand{by}\ {\isacharparenleft}rule\ pdl{\isacharunderscore}k{\isadigit{5}}{\isacharparenright}\isanewline
\ \ \ \ \isamarkupfalse%
\isacommand{from}\ f{\isadigit{1}}\ f{\isadigit{2}}\ \isamarkupfalse%
\isacommand{show}\ {\isacharquery}thesis\ \isamarkupfalse%
\isacommand{by}\ {\isacharparenleft}rule\ pdl{\isacharunderscore}mp{\isacharparenright}\isanewline
\ \ \isamarkupfalse%
\isacommand{qed}\isanewline
\ \ \isamarkupfalse%
\isacommand{thus}\ {\isacharquery}thesis\ \isamarkupfalse%
\isacommand{by}\ {\isacharparenleft}simp\ add{\isacharcolon}\ pdl{\isacharunderscore}taut{\isacharparenright}\isanewline
\isamarkupfalse%
\isacommand{qed}\isamarkupfalse%
%
\begin{isamarkuptext}%
\dots and the second half.%
\end{isamarkuptext}%
\isamarkuptrue%
\isacommand{lemma}\ dmd{\isacharunderscore}box{\isacharunderscore}rel{\isadigit{2}}{\isacharcolon}\ {\isachardoublequote}{\isasymturnstile}\ {\isasymlangle}x{\isasymleftarrow}p{\isasymrangle}{\isacharparenleft}P\ x{\isacharparenright}\ {\isasymlongrightarrow}\isactrlsub D\ {\isacharbrackleft}{\isacharhash}\ x{\isasymleftarrow}p{\isacharbrackright}{\isacharparenleft}P\ x\ {\isasymlongrightarrow}\isactrlsub D\ Ret\ False{\isacharparenright}\ {\isasymlongrightarrow}\isactrlsub D\ Ret\ False{\isachardoublequote}\isanewline
\isamarkupfalse%
\isacommand{proof}\ {\isacharminus}\isanewline
\ \ \isamarkupfalse%
\isacommand{have}\ {\isachardoublequote}{\isasymturnstile}\ {\isacharparenleft}{\isasymlangle}x{\isasymleftarrow}p{\isasymrangle}{\isacharparenleft}Ret\ False{\isacharparenright}\ {\isasymlongrightarrow}\isactrlsub D\ Ret\ False{\isacharparenright}\ {\isasymlongrightarrow}\isactrlsub D\ \isanewline
\ \ \ \ \ \ \ \ \ \ {\isacharparenleft}{\isacharbrackleft}{\isacharhash}\ x{\isasymleftarrow}p{\isacharbrackright}{\isacharparenleft}P\ x\ {\isasymlongrightarrow}\isactrlsub D\ Ret\ False{\isacharparenright}\ {\isasymlongrightarrow}\isactrlsub D\ {\isasymlangle}x{\isasymleftarrow}p{\isasymrangle}{\isacharparenleft}P\ x{\isacharparenright}\ {\isasymlongrightarrow}\isactrlsub D\ {\isasymlangle}x{\isasymleftarrow}p{\isasymrangle}{\isacharparenleft}Ret\ False{\isacharparenright}{\isacharparenright}\ {\isasymlongrightarrow}\isactrlsub D\ \ \isanewline
\ \ \ \ \ \ \ \ \ \ \ {\isasymlangle}x{\isasymleftarrow}p{\isasymrangle}{\isacharparenleft}P\ x{\isacharparenright}\ {\isasymlongrightarrow}\isactrlsub D\ {\isacharbrackleft}{\isacharhash}\ x{\isasymleftarrow}p{\isacharbrackright}{\isacharparenleft}P\ x\ {\isasymlongrightarrow}\isactrlsub D\ Ret\ False{\isacharparenright}\ {\isasymlongrightarrow}\isactrlsub D\ Ret\ False{\isachardoublequote}\isanewline
\ \ \ \ \isamarkupfalse%
\isacommand{by}\ {\isacharparenleft}simp\ add{\isacharcolon}\ pdl{\isacharunderscore}taut{\isacharparenright}\isanewline
\ \ \isamarkupfalse%
\isacommand{from}\ this\ pdl{\isacharunderscore}k{\isadigit{3}}D\ pdl{\isacharunderscore}k{\isadigit{2}}\ \isamarkupfalse%
\isacommand{show}\ {\isacharquery}thesis\ \isamarkupfalse%
\isacommand{by}\ {\isacharparenleft}rule\ pdl{\isacharunderscore}mp{\isacharunderscore}{\isadigit{2}}x{\isacharparenright}\isanewline
\isamarkupfalse%
\isacommand{qed}\isamarkupfalse%
%
\begin{isamarkuptext}%
Inheriting the classical theorems from Isabelle/HOL, one also obtains the classical equivalence
  between the diamond and box operator.

  The proofs of \isa{dmd{\isacharunderscore}box{\isacharunderscore}rel{\isadigit{1}}} and \isa{dmd{\isacharunderscore}box{\isacharunderscore}rel{\isadigit{2}}} implicitly employ
  classical arguments through the use of the simplifier, since the algebraization of propositional
  logic behaves classically.
  \label{isa:dmd-box-rel}%
\end{isamarkuptext}%
\isamarkuptrue%
\isacommand{theorem}\ dmd{\isacharunderscore}box{\isacharunderscore}rel{\isacharcolon}\ {\isachardoublequote}{\isasymturnstile}\ {\isasymlangle}x{\isasymleftarrow}p{\isasymrangle}{\isacharparenleft}P\ x{\isacharparenright}\ {\isasymlongleftrightarrow}\isactrlsub D\ {\isasymnot}\isactrlsub D\ {\isacharbrackleft}{\isacharhash}\ x{\isasymleftarrow}p{\isacharbrackright}{\isacharparenleft}{\isasymnot}\isactrlsub D\ P\ x{\isacharparenright}{\isachardoublequote}\isanewline
\ \ \isamarkupfalse%
\isacommand{apply}{\isacharparenleft}rule\ pdl{\isacharunderscore}iffI{\isacharparenright}\isanewline
\ \ \isamarkupfalse%
\isacommand{apply}{\isacharparenleft}unfold\ NotD{\isacharunderscore}def{\isacharparenright}\isanewline
\ \ \isamarkupfalse%
\isacommand{apply}{\isacharparenleft}rule\ dmd{\isacharunderscore}box{\isacharunderscore}rel{\isadigit{2}}{\isacharparenright}\isanewline
\ \ \isamarkupfalse%
\isacommand{apply}{\isacharparenleft}rule\ dmd{\isacharunderscore}box{\isacharunderscore}rel{\isadigit{1}}{\isacharparenright}\isanewline
\isamarkupfalse%
\isacommand{done}\isamarkupfalse%
%
\begin{isamarkuptext}%
Given \isa{dmd{\isacharunderscore}box{\isacharunderscore}rel}, one easily obtains a dual one.%
\end{isamarkuptext}%
\isamarkuptrue%
\isacommand{theorem}\ box{\isacharunderscore}dmd{\isacharunderscore}rel{\isacharcolon}\ {\isachardoublequote}{\isasymturnstile}\ {\isacharbrackleft}{\isacharhash}\ x{\isasymleftarrow}p{\isacharbrackright}{\isacharparenleft}P\ x{\isacharparenright}\ {\isasymlongleftrightarrow}\isactrlsub D\ {\isasymnot}\isactrlsub D\ {\isasymlangle}x{\isasymleftarrow}p{\isasymrangle}{\isacharparenleft}{\isasymnot}\isactrlsub D\ P\ x{\isacharparenright}{\isachardoublequote}\isanewline
\isamarkupfalse%
\isacommand{proof}\ {\isacharminus}\isanewline
\ \ \isamarkupfalse%
\isacommand{have}\ {\isachardoublequote}{\isasymturnstile}\ {\isacharparenleft}\ {\isasymlangle}x{\isasymleftarrow}p{\isasymrangle}{\isacharparenleft}{\isasymnot}\isactrlsub D\ P\ x{\isacharparenright}\ {\isasymlongleftrightarrow}\isactrlsub D\ {\isasymnot}\isactrlsub D\ {\isacharbrackleft}{\isacharhash}\ x{\isasymleftarrow}p{\isacharbrackright}{\isacharparenleft}{\isasymnot}\isactrlsub D\ {\isasymnot}\isactrlsub D\ P\ x{\isacharparenright}\ {\isacharparenright}\ {\isasymlongrightarrow}\isactrlsub D\ \isanewline
\ \ \ \ \ \ \ \ \ \ {\isacharparenleft}\ {\isacharbrackleft}{\isacharhash}\ x{\isasymleftarrow}p{\isacharbrackright}{\isacharparenleft}P\ x{\isacharparenright}\ {\isasymlongleftrightarrow}\isactrlsub D\ {\isasymnot}\isactrlsub D\ {\isasymnot}\isactrlsub D\ {\isacharbrackleft}{\isacharhash}\ x{\isasymleftarrow}p{\isacharbrackright}{\isacharparenleft}{\isasymnot}\isactrlsub D\ {\isasymnot}\isactrlsub D\ P\ x{\isacharparenright}\ {\isacharparenright}\ {\isasymlongrightarrow}\isactrlsub D\ \isanewline
\ \ \ \ \ \ \ \ \ \ {\isacharparenleft}\ {\isacharbrackleft}{\isacharhash}\ x{\isasymleftarrow}p{\isacharbrackright}{\isacharparenleft}P\ x{\isacharparenright}\ {\isasymlongleftrightarrow}\isactrlsub D\ {\isasymnot}\isactrlsub D\ {\isasymlangle}x{\isasymleftarrow}p{\isasymrangle}{\isacharparenleft}{\isasymnot}\isactrlsub D\ P\ x{\isacharparenright}\ {\isacharparenright}\ {\isachardoublequote}\ \isanewline
\ \ \ \ \isamarkupfalse%
\isacommand{by}\ {\isacharparenleft}simp\ add{\isacharcolon}\ pdl{\isacharunderscore}taut{\isacharparenright}\isanewline
\ \ \isamarkupfalse%
\isacommand{moreover}\ \isanewline
\ \ \isamarkupfalse%
\isacommand{have}\ {\isachardoublequote}{\isasymturnstile}\ \ {\isasymlangle}x{\isasymleftarrow}p{\isasymrangle}{\isacharparenleft}{\isasymnot}\isactrlsub D\ P\ x{\isacharparenright}\ {\isasymlongleftrightarrow}\isactrlsub D\ {\isasymnot}\isactrlsub D\ {\isacharbrackleft}{\isacharhash}\ x{\isasymleftarrow}p{\isacharbrackright}{\isacharparenleft}{\isasymnot}\isactrlsub D\ {\isasymnot}\isactrlsub D\ P\ x{\isacharparenright}{\isachardoublequote}\isanewline
\ \ \ \ \isamarkupfalse%
\isacommand{by}\ {\isacharparenleft}rule\ dmd{\isacharunderscore}box{\isacharunderscore}rel{\isacharparenright}\isanewline
\ \ \isamarkupfalse%
\isacommand{moreover}\isanewline
\ \ \isamarkupfalse%
\isacommand{have}\ {\isachardoublequote}{\isasymturnstile}\ {\isacharbrackleft}{\isacharhash}\ x{\isasymleftarrow}p{\isacharbrackright}{\isacharparenleft}P\ x{\isacharparenright}\ {\isasymlongleftrightarrow}\isactrlsub D\ {\isasymnot}\isactrlsub D\ {\isasymnot}\isactrlsub D\ {\isacharbrackleft}{\isacharhash}\ x{\isasymleftarrow}p{\isacharbrackright}{\isacharparenleft}{\isasymnot}\isactrlsub D\ {\isasymnot}\isactrlsub D\ P\ x{\isacharparenright}{\isachardoublequote}\isanewline
\ \ \ \ \isamarkupfalse%
\isacommand{by}\ {\isacharparenleft}simp\ add{\isacharcolon}\ pdl{\isacharunderscore}taut{\isacharparenright}\isanewline
\ \ \isamarkupfalse%
\isacommand{ultimately}\ \isanewline
\ \ \isamarkupfalse%
\isacommand{show}\ {\isacharquery}thesis\isanewline
\ \ \ \ \isamarkupfalse%
\isacommand{by}\ {\isacharparenleft}rule\ pdl{\isacharunderscore}mp{\isacharunderscore}{\isadigit{2}}x{\isacharparenright}\isanewline
\isamarkupfalse%
\isacommand{qed}\isamarkupfalse%
%
\begin{isamarkuptext}%
A specialized form of the equality rule \isa{pdl{\isacharunderscore}eqD} that only requires the arguments
  of a program \isa{p} to be equal.%
\end{isamarkuptext}%
\isamarkuptrue%
\isacommand{theorem}\ pdl{\isacharunderscore}eqD{\isacharunderscore}ext{\isacharcolon}\ {\isachardoublequote}{\isasymturnstile}\ Ret\ {\isacharparenleft}a\ {\isacharequal}\ b{\isacharparenright}\ {\isasymlongrightarrow}\isactrlsub D\ {\isasymlangle}p\ a{\isasymrangle}P\ {\isasymlongrightarrow}\isactrlsub D\ {\isasymlangle}p\ b{\isasymrangle}P{\isachardoublequote}\ {\isacharparenleft}\isakeyword{is}\ {\isachardoublequote}{\isasymturnstile}\ {\isacharquery}ab\ {\isasymlongrightarrow}\isactrlsub D\ {\isacharquery}pa\ {\isasymlongrightarrow}\isactrlsub D\ {\isacharquery}pb{\isachardoublequote}{\isacharparenright}\isanewline
\isamarkupfalse%
\isacommand{proof}\ {\isacharminus}\isanewline
\ \ \isamarkupfalse%
\isacommand{have}\ {\isachardoublequote}{\isasymturnstile}\ {\isacharparenleft}Ret\ {\isacharparenleft}a\ {\isacharequal}\ b{\isacharparenright}\ {\isasymlongrightarrow}\isactrlsub D\ Ret\ {\isacharparenleft}p\ a\ {\isacharequal}\ p\ b{\isacharparenright}{\isacharparenright}\ {\isasymlongrightarrow}\isactrlsub D\isanewline
\ \ \ \ \ \ \ \ \ \ {\isacharparenleft}Ret\ {\isacharparenleft}p\ a\ {\isacharequal}\ p\ b{\isacharparenright}\ {\isasymlongrightarrow}\isactrlsub D\ {\isacharquery}pa\ {\isasymlongrightarrow}\isactrlsub D\ {\isacharquery}pb{\isacharparenright}\ {\isasymlongrightarrow}\isactrlsub D\isanewline
\ \ \ \ \ \ \ \ \ \ {\isacharparenleft}{\isacharquery}ab\ {\isasymlongrightarrow}\isactrlsub D\ {\isacharquery}pa\ {\isasymlongrightarrow}\isactrlsub D\ {\isacharquery}pb{\isacharparenright}{\isachardoublequote}\ \isamarkupfalse%
\isacommand{by}\ {\isacharparenleft}simp\ add{\isacharcolon}\ pdl{\isacharunderscore}taut{\isacharparenright}\isanewline
\ \ \isamarkupfalse%
\isacommand{moreover}\ \isanewline
\ \ \isamarkupfalse%
\isacommand{have}\ {\isachardoublequote}{\isasymturnstile}\ Ret\ {\isacharparenleft}a\ {\isacharequal}\ b{\isacharparenright}\ {\isasymlongrightarrow}\isactrlsub D\ Ret\ {\isacharparenleft}p\ a\ {\isacharequal}\ p\ b{\isacharparenright}{\isachardoublequote}\isanewline
\ \ \isamarkupfalse%
\isacommand{proof}\ {\isacharparenleft}subst\ impD{\isacharunderscore}Ret{\isacharunderscore}hom{\isacharbrackleft}symmetric{\isacharbrackright}{\isacharparenright}\isanewline
\ \ \ \ \isamarkupfalse%
\isacommand{show}\ {\isachardoublequote}{\isasymturnstile}\ Ret\ {\isacharparenleft}a\ {\isacharequal}\ b\ {\isasymlongrightarrow}\ p\ a\ {\isacharequal}\ p\ b{\isacharparenright}{\isachardoublequote}\isanewline
\ \ \ \ \isamarkupfalse%
\isacommand{proof}\ {\isacharparenleft}rule\ iffD{\isadigit{2}}{\isacharbrackleft}OF\ Valid{\isacharunderscore}Ret{\isacharbrackright}{\isacharparenright}\isanewline
\ \ \ \ \ \ \isamarkupfalse%
\isacommand{show}\ {\isachardoublequote}a\ {\isacharequal}\ b\ {\isasymlongrightarrow}\ p\ a\ {\isacharequal}\ p\ b{\isachardoublequote}\ \isamarkupfalse%
\isacommand{by}\ blast\isanewline
\ \ \ \ \isamarkupfalse%
\isacommand{qed}\isanewline
\ \ \isamarkupfalse%
\isacommand{qed}\isanewline
\ \ \isamarkupfalse%
\isacommand{moreover}\isanewline
\ \ \isamarkupfalse%
\isacommand{have}\ {\isachardoublequote}{\isasymturnstile}\ Ret\ {\isacharparenleft}p\ a\ {\isacharequal}\ p\ b{\isacharparenright}\ {\isasymlongrightarrow}\isactrlsub D\ {\isacharquery}pa\ {\isasymlongrightarrow}\isactrlsub D\ {\isacharquery}pb{\isachardoublequote}\isanewline
\ \ \ \ \isamarkupfalse%
\isacommand{by}\ {\isacharparenleft}rule\ pdl{\isacharunderscore}eqD{\isacharparenright}\isanewline
\ \ \isamarkupfalse%
\isacommand{ultimately}\ \isanewline
\ \ \isamarkupfalse%
\isacommand{show}\ {\isacharquery}thesis\ \isamarkupfalse%
\isacommand{by}\ {\isacharparenleft}rule\ pdl{\isacharunderscore}mp{\isacharunderscore}{\isadigit{2}}x{\isacharparenright}\isanewline
\isamarkupfalse%
\isacommand{qed}\isamarkupfalse%
%
\begin{isamarkuptext}%
The following are simple consequences of the axioms above;
  rather than monadic implication, they use Isabelle's meta implication
  (and hence represent rules).
  \label{isa:pdl-derived-rules}%
\end{isamarkuptext}%
\isamarkuptrue%
\isacommand{lemma}\ box{\isacharunderscore}imp{\isacharunderscore}distrib{\isacharcolon}\ {\isachardoublequote}{\isasymturnstile}\ {\isacharbrackleft}{\isacharhash}\ x{\isasymleftarrow}p{\isacharbrackright}{\isacharparenleft}P\ x\ {\isasymlongrightarrow}\isactrlsub D\ Q\ x{\isacharparenright}\ {\isasymLongrightarrow}\ {\isasymturnstile}\ {\isacharbrackleft}{\isacharhash}\ x{\isasymleftarrow}p{\isacharbrackright}{\isacharparenleft}P\ x{\isacharparenright}\ {\isasymlongrightarrow}\isactrlsub D\ {\isacharbrackleft}{\isacharhash}\ x{\isasymleftarrow}p{\isacharbrackright}{\isacharparenleft}Q\ x{\isacharparenright}{\isachardoublequote}\isanewline
\ \isamarkupfalse%
\isacommand{by}{\isacharparenleft}rule\ pdl{\isacharunderscore}k{\isadigit{1}}{\isacharbrackleft}THEN\ pdl{\isacharunderscore}mp{\isacharbrackright}{\isacharparenright}\isanewline
\isanewline
\isamarkupfalse%
\isacommand{lemma}\ dmd{\isacharunderscore}imp{\isacharunderscore}distrib{\isacharcolon}\ {\isachardoublequote}{\isasymturnstile}\ {\isacharbrackleft}{\isacharhash}\ x{\isasymleftarrow}p{\isacharbrackright}{\isacharparenleft}P\ x\ {\isasymlongrightarrow}\isactrlsub D\ Q\ x{\isacharparenright}\ {\isasymLongrightarrow}\ {\isasymturnstile}\ {\isasymlangle}x{\isasymleftarrow}p{\isasymrangle}{\isacharparenleft}P\ x{\isacharparenright}\ {\isasymlongrightarrow}\isactrlsub D\ {\isasymlangle}x{\isasymleftarrow}p{\isasymrangle}{\isacharparenleft}Q\ x{\isacharparenright}{\isachardoublequote}\isanewline
\ \ \isamarkupfalse%
\isacommand{by}\ {\isacharparenleft}rule\ pdl{\isacharunderscore}mp{\isacharbrackleft}OF\ pdl{\isacharunderscore}k{\isadigit{2}}{\isacharbrackright}{\isacharparenright}\isanewline
\isanewline
\isamarkupfalse%
\isacommand{lemma}\ pdl{\isacharunderscore}box{\isacharunderscore}reg{\isacharcolon}\ {\isachardoublequote}\ {\isasymforall}x{\isachardot}\ {\isasymturnstile}\ P\ x\ {\isasymlongrightarrow}\isactrlsub D\ Q\ x\ {\isasymLongrightarrow}\ {\isasymturnstile}\ {\isacharbrackleft}{\isacharhash}\ x{\isasymleftarrow}p{\isacharbrackright}{\isacharparenleft}P\ x{\isacharparenright}\ {\isasymlongrightarrow}\isactrlsub D\ {\isacharbrackleft}{\isacharhash}\ x{\isasymleftarrow}p{\isacharbrackright}{\isacharparenleft}Q\ x{\isacharparenright}{\isachardoublequote}\isanewline
\ \ \isamarkupfalse%
\isacommand{apply}{\isacharparenleft}rule\ box{\isacharunderscore}imp{\isacharunderscore}distrib{\isacharparenright}\isanewline
\ \ \isamarkupfalse%
\isacommand{apply}{\isacharparenleft}rule\ pdl{\isacharunderscore}nec{\isacharparenright}\isanewline
\ \ \isamarkupfalse%
\isacommand{apply}\ assumption\isanewline
\isamarkupfalse%
\isacommand{done}\isanewline
\isanewline
\isamarkupfalse%
\isacommand{lemma}\ pdl{\isacharunderscore}dmd{\isacharunderscore}reg{\isacharcolon}\ {\isachardoublequote}\ {\isasymforall}x{\isachardot}\ {\isasymturnstile}\ P\ x\ {\isasymlongrightarrow}\isactrlsub D\ Q\ x\ {\isasymLongrightarrow}\ {\isasymturnstile}\ {\isasymlangle}x{\isasymleftarrow}p{\isasymrangle}{\isacharparenleft}P\ x{\isacharparenright}\ {\isasymlongrightarrow}\isactrlsub D\ {\isasymlangle}x{\isasymleftarrow}p{\isasymrangle}{\isacharparenleft}Q\ x{\isacharparenright}{\isachardoublequote}\isanewline
\ \ \isamarkupfalse%
\isacommand{apply}{\isacharparenleft}rule\ dmd{\isacharunderscore}imp{\isacharunderscore}distrib{\isacharparenright}\isanewline
\ \ \isamarkupfalse%
\isacommand{apply}{\isacharparenleft}rule\ pdl{\isacharunderscore}nec{\isacharparenright}\isanewline
\ \ \isamarkupfalse%
\isacommand{apply}\ assumption\isanewline
\isamarkupfalse%
\isacommand{done}\isanewline
\isanewline
\isanewline
\isamarkupfalse%
\isacommand{theorem}\ pdl{\isacharunderscore}wkB{\isacharcolon}\ {\isachardoublequote}{\isasymlbrakk}{\isasymturnstile}\ {\isacharbrackleft}{\isacharhash}\ x{\isasymleftarrow}p{\isacharbrackright}{\isacharparenleft}P\ x{\isacharparenright}{\isacharsemicolon}\ {\isasymforall}x{\isachardot}\ {\isasymturnstile}\ P\ x\ {\isasymlongrightarrow}\isactrlsub D\ Q\ x{\isasymrbrakk}\ {\isasymLongrightarrow}\ {\isasymturnstile}\ {\isacharbrackleft}{\isacharhash}\ x{\isasymleftarrow}p{\isacharbrackright}{\isacharparenleft}Q\ x{\isacharparenright}{\isachardoublequote}\isanewline
\ \isamarkupfalse%
\isacommand{apply}{\isacharparenleft}rule\ pdl{\isacharunderscore}mp{\isacharparenright}\isanewline
\ \isamarkupfalse%
\isacommand{apply}{\isacharparenleft}rule\ box{\isacharunderscore}imp{\isacharunderscore}distrib{\isacharparenright}\isanewline
\isamarkupfalse%
\isacommand{by}{\isacharparenleft}rule\ pdl{\isacharunderscore}nec{\isacharparenright}\isanewline
\isanewline
\isanewline
\isamarkupfalse%
\isacommand{theorem}\ pdl{\isacharunderscore}wkD{\isacharcolon}\ {\isachardoublequote}{\isasymlbrakk}{\isasymturnstile}\ {\isasymlangle}x{\isasymleftarrow}p{\isasymrangle}{\isacharparenleft}P\ x{\isacharparenright}{\isacharsemicolon}\ {\isasymforall}x{\isachardot}\ {\isasymturnstile}\ P\ x\ {\isasymlongrightarrow}\isactrlsub D\ Q\ x{\isasymrbrakk}\ {\isasymLongrightarrow}\ {\isasymturnstile}\ {\isasymlangle}x{\isasymleftarrow}p{\isasymrangle}{\isacharparenleft}Q\ x{\isacharparenright}{\isachardoublequote}\isanewline
\isamarkupfalse%
\isacommand{proof}\ {\isacharminus}\isanewline
\ \ \isamarkupfalse%
\isacommand{assume}\ a{\isacharcolon}\ {\isachardoublequote}{\isasymturnstile}\ {\isasymlangle}x{\isasymleftarrow}p{\isasymrangle}{\isacharparenleft}P\ x{\isacharparenright}{\isachardoublequote}\ \isakeyword{and}\ b{\isacharcolon}\ {\isachardoublequote}{\isasymforall}x{\isachardot}\ {\isasymturnstile}\ P\ x\ {\isasymlongrightarrow}\isactrlsub D\ Q\ x{\isachardoublequote}\isanewline
\ \ \isamarkupfalse%
\isacommand{from}\ b\ \isamarkupfalse%
\isacommand{have}\ {\isachardoublequote}{\isasymturnstile}\ {\isacharbrackleft}{\isacharhash}\ x{\isasymleftarrow}p{\isacharbrackright}{\isacharparenleft}P\ x\ \ {\isasymlongrightarrow}\isactrlsub D\ Q\ x{\isacharparenright}{\isachardoublequote}\ \isamarkupfalse%
\isacommand{by}\ {\isacharparenleft}rule\ pdl{\isacharunderscore}nec{\isacharparenright}\isanewline
\ \ \isamarkupfalse%
\isacommand{hence}\ {\isachardoublequote}{\isasymturnstile}\ {\isasymlangle}x{\isasymleftarrow}p{\isasymrangle}{\isacharparenleft}P\ x{\isacharparenright}\ \ {\isasymlongrightarrow}\isactrlsub D\ {\isasymlangle}x{\isasymleftarrow}p{\isasymrangle}{\isacharparenleft}Q\ x{\isacharparenright}{\isachardoublequote}\ \isamarkupfalse%
\isacommand{by}\ {\isacharparenleft}rule\ pdl{\isacharunderscore}k{\isadigit{2}}{\isacharbrackleft}THEN\ pdl{\isacharunderscore}mp{\isacharbrackright}{\isacharparenright}\isanewline
\ \ \isamarkupfalse%
\isacommand{from}\ this\ a\ \isamarkupfalse%
\isacommand{show}\ \ {\isachardoublequote}{\isasymturnstile}\ {\isasymlangle}x{\isasymleftarrow}p{\isasymrangle}{\isacharparenleft}Q\ x{\isacharparenright}{\isachardoublequote}\ \isamarkupfalse%
\isacommand{by}\ {\isacharparenleft}rule\ pdl{\isacharunderscore}mp{\isacharparenright}\isanewline
\isamarkupfalse%
\isacommand{qed}\isamarkupfalse%
%
\begin{isamarkuptext}%
The following rule comes in handy when program sequences occur inside the box.%
\end{isamarkuptext}%
\isamarkuptrue%
\isacommand{theorem}\ pdl{\isacharunderscore}plugB{\isacharcolon}\ {\isachardoublequote}{\isasymlbrakk}{\isasymturnstile}\ {\isacharbrackleft}{\isacharhash}\ x{\isasymleftarrow}p{\isacharbrackright}{\isacharparenleft}P\ x{\isacharparenright}{\isacharsemicolon}\ {\isasymforall}x{\isachardot}\ {\isasymturnstile}\ P\ x\ {\isasymlongrightarrow}\isactrlsub D\ {\isacharbrackleft}{\isacharhash}\ y{\isasymleftarrow}q\ x{\isacharbrackright}{\isacharparenleft}C\ y{\isacharparenright}{\isasymrbrakk}\ {\isasymLongrightarrow}\ {\isasymturnstile}\ {\isacharbrackleft}{\isacharhash}\ do\ {\isacharbraceleft}x{\isasymleftarrow}p{\isacharsemicolon}\ q\ x{\isacharbraceright}{\isacharbrackright}C{\isachardoublequote}\isanewline
\ \ \isamarkupfalse%
\isacommand{apply}{\isacharparenleft}drule\ pdl{\isacharunderscore}wkB{\isacharcomma}\ assumption{\isacharparenright}\isanewline
\ \ \isamarkupfalse%
\isacommand{by}\ {\isacharparenleft}rule\ pdl{\isacharunderscore}iffD{\isadigit{1}}{\isacharbrackleft}OF\ pdl{\isacharunderscore}seqB{\isacharunderscore}simp{\isacharcomma}\ THEN\ pdl{\isacharunderscore}mp{\isacharbrackright}{\isacharparenright}\isanewline
\isanewline
\isamarkupfalse%
\isacommand{theorem}\ pdl{\isacharunderscore}plugD{\isacharcolon}\ {\isachardoublequote}{\isasymlbrakk}{\isasymturnstile}\ {\isasymlangle}x{\isasymleftarrow}p{\isasymrangle}{\isacharparenleft}P\ x{\isacharparenright}{\isacharsemicolon}\ {\isasymforall}x{\isachardot}\ {\isasymturnstile}\ P\ x\ {\isasymlongrightarrow}\isactrlsub D\ {\isasymlangle}y{\isasymleftarrow}q\ x{\isasymrangle}{\isacharparenleft}C\ y{\isacharparenright}{\isasymrbrakk}\ {\isasymLongrightarrow}\ {\isasymturnstile}\ {\isasymlangle}do\ {\isacharbraceleft}x{\isasymleftarrow}p{\isacharsemicolon}\ q\ x{\isacharbraceright}{\isasymrangle}C{\isachardoublequote}\isanewline
\ \ \isamarkupfalse%
\isacommand{apply}{\isacharparenleft}drule\ pdl{\isacharunderscore}wkD{\isacharcomma}\ assumption{\isacharparenright}\isanewline
\ \ \isamarkupfalse%
\isacommand{by}\ {\isacharparenleft}rule\ pdl{\isacharunderscore}iffD{\isadigit{1}}{\isacharbrackleft}OF\ pdl{\isacharunderscore}seqD{\isacharunderscore}simp{\isacharcomma}\ THEN\ pdl{\isacharunderscore}mp{\isacharbrackright}{\isacharparenright}\isanewline
\isanewline
\isamarkupfalse%
\isacommand{lemma}\ box{\isacharunderscore}conj{\isacharunderscore}distrib{\isadigit{1}}{\isacharcolon}\ {\isachardoublequote}{\isasymturnstile}\ {\isacharbrackleft}{\isacharhash}\ x{\isasymleftarrow}p{\isacharbrackright}{\isacharparenleft}P\ x{\isacharparenright}\ {\isasymand}\isactrlsub D\ {\isacharbrackleft}{\isacharhash}\ x{\isasymleftarrow}p{\isacharbrackright}{\isacharparenleft}Q\ x{\isacharparenright}\ {\isasymlongrightarrow}\isactrlsub D\ {\isacharbrackleft}{\isacharhash}\ x{\isasymleftarrow}p{\isacharbrackright}{\isacharparenleft}P\ x\ {\isasymand}\isactrlsub D\ Q\ x{\isacharparenright}{\isachardoublequote}\isanewline
\isamarkupfalse%
\isacommand{proof}\ {\isacharminus}\isanewline
\ \ \isamarkupfalse%
\isacommand{have}\ {\isachardoublequote}{\isasymforall}x{\isachardot}\ {\isasymturnstile}\ P\ x\ {\isasymlongrightarrow}\isactrlsub D\ Q\ x\ {\isasymlongrightarrow}\isactrlsub D\ P\ x\ {\isasymand}\isactrlsub D\ Q\ x{\isachardoublequote}\isanewline
\ \ \isamarkupfalse%
\isacommand{proof}\isanewline
\ \ \ \ \isamarkupfalse%
\isacommand{fix}\ x\ \isamarkupfalse%
\isacommand{show}\ {\isachardoublequote}{\isasymturnstile}\ P\ x\ {\isasymlongrightarrow}\isactrlsub D\ Q\ x\ {\isasymlongrightarrow}\isactrlsub D\ P\ x\ {\isasymand}\isactrlsub D\ Q\ x{\isachardoublequote}\isanewline
\ \ \ \ \ \ \isamarkupfalse%
\isacommand{by}\ {\isacharparenleft}simp\ only{\isacharcolon}\ pdl{\isacharunderscore}taut\ Valid{\isacharunderscore}Ret{\isacharparenright}\isanewline
\ \ \isamarkupfalse%
\isacommand{qed}\isanewline
\ \ \isamarkupfalse%
\isacommand{hence}\ a{\isadigit{2}}{\isacharcolon}\ {\isachardoublequote}{\isasymturnstile}\ {\isacharbrackleft}{\isacharhash}\ x{\isasymleftarrow}p{\isacharbrackright}{\isacharparenleft}P\ x{\isacharparenright}\ {\isasymlongrightarrow}\isactrlsub D\ {\isacharbrackleft}{\isacharhash}\ x{\isasymleftarrow}p{\isacharbrackright}{\isacharparenleft}Q\ x\ {\isasymlongrightarrow}\isactrlsub D\ {\isacharparenleft}P\ x\ {\isasymand}\isactrlsub D\ Q\ x{\isacharparenright}{\isacharparenright}{\isachardoublequote}\isanewline
\ \ \ \ \isamarkupfalse%
\isacommand{by}\ {\isacharparenleft}rule\ pdl{\isacharunderscore}box{\isacharunderscore}reg{\isacharparenright}\isanewline
\ \ \isamarkupfalse%
\isacommand{from}\ this\ pdl{\isacharunderscore}k{\isadigit{1}}\ \isamarkupfalse%
\isacommand{have}\ {\isachardoublequote}{\isasymturnstile}\ {\isacharbrackleft}{\isacharhash}\ x{\isasymleftarrow}p{\isacharbrackright}{\isacharparenleft}P\ x{\isacharparenright}\ {\isasymlongrightarrow}\isactrlsub D\ {\isacharbrackleft}{\isacharhash}\ x{\isasymleftarrow}p{\isacharbrackright}{\isacharparenleft}Q\ x{\isacharparenright}\ {\isasymlongrightarrow}\isactrlsub D\ {\isacharbrackleft}{\isacharhash}\ x{\isasymleftarrow}p{\isacharbrackright}{\isacharparenleft}P\ x\ {\isasymand}\isactrlsub D\ Q\ x{\isacharparenright}{\isachardoublequote}\isanewline
\ \ \ \ \isamarkupfalse%
\isacommand{by}\ {\isacharparenleft}rule\ pdl{\isacharunderscore}imp{\isacharunderscore}trans{\isacharparenright}\isanewline
\ \ \isamarkupfalse%
\isacommand{thus}\ {\isacharquery}thesis\ \isamarkupfalse%
\isacommand{by}\ {\isacharparenleft}simp\ only{\isacharcolon}\ pdl{\isacharunderscore}taut{\isacharparenright}\isanewline
\isamarkupfalse%
\isacommand{qed}\isanewline
\ \ \isanewline
\isanewline
\isamarkupfalse%
\isacommand{lemma}\ box{\isacharunderscore}conj{\isacharunderscore}distrib{\isadigit{2}}{\isacharcolon}\ {\isachardoublequote}{\isasymturnstile}\ {\isacharbrackleft}{\isacharhash}\ x{\isasymleftarrow}p{\isacharbrackright}{\isacharparenleft}P\ x\ {\isasymand}\isactrlsub D\ Q\ x{\isacharparenright}\ {\isasymlongrightarrow}\isactrlsub D\ {\isacharbrackleft}{\isacharhash}\ x{\isasymleftarrow}p{\isacharbrackright}{\isacharparenleft}P\ x{\isacharparenright}\ {\isasymand}\isactrlsub D\ {\isacharbrackleft}{\isacharhash}\ x{\isasymleftarrow}p{\isacharbrackright}{\isacharparenleft}Q\ x{\isacharparenright}{\isachardoublequote}\isanewline
\isamarkupfalse%
\isacommand{proof}\ {\isacharminus}\isanewline
\ \ \isamarkupfalse%
\isacommand{have}\ {\isachardoublequote}\ {\isasymforall}x{\isachardot}\ {\isasymturnstile}\ P\ x\ {\isasymand}\isactrlsub D\ Q\ x\ {\isasymlongrightarrow}\isactrlsub D\ P\ x{\isachardoublequote}\ \isamarkupfalse%
\isacommand{by}\ {\isacharparenleft}simp\ add{\isacharcolon}\ pdl{\isacharunderscore}taut{\isacharparenright}\isanewline
\ \ \isamarkupfalse%
\isacommand{hence}\ a{\isadigit{1}}{\isacharcolon}\ {\isachardoublequote}{\isasymturnstile}\ {\isacharbrackleft}{\isacharhash}\ x{\isasymleftarrow}p{\isacharbrackright}\ {\isacharparenleft}P\ x\ {\isasymand}\isactrlsub D\ Q\ x{\isacharparenright}\ {\isasymlongrightarrow}\isactrlsub D\ {\isacharbrackleft}{\isacharhash}\ x{\isasymleftarrow}p{\isacharbrackright}{\isacharparenleft}P\ x{\isacharparenright}{\isachardoublequote}\ \isamarkupfalse%
\isacommand{by}\ {\isacharparenleft}rule\ pdl{\isacharunderscore}box{\isacharunderscore}reg{\isacharparenright}\isanewline
\ \ \isamarkupfalse%
\isacommand{have}\ {\isachardoublequote}\ {\isasymforall}x{\isachardot}\ {\isasymturnstile}\ P\ x\ {\isasymand}\isactrlsub D\ Q\ x\ {\isasymlongrightarrow}\isactrlsub D\ Q\ x{\isachardoublequote}\ \ \ \isamarkupfalse%
\isacommand{by}\ {\isacharparenleft}simp\ add{\isacharcolon}\ pdl{\isacharunderscore}taut{\isacharparenright}\isanewline
\ \ \isamarkupfalse%
\isacommand{hence}\ a{\isadigit{2}}{\isacharcolon}\ {\isachardoublequote}{\isasymturnstile}\ {\isacharbrackleft}{\isacharhash}\ x{\isasymleftarrow}p{\isacharbrackright}\ {\isacharparenleft}P\ x\ {\isasymand}\isactrlsub D\ Q\ x{\isacharparenright}\ {\isasymlongrightarrow}\isactrlsub D\ {\isacharbrackleft}{\isacharhash}\ x{\isasymleftarrow}p{\isacharbrackright}{\isacharparenleft}Q\ x{\isacharparenright}{\isachardoublequote}\ \isamarkupfalse%
\isacommand{by}\ {\isacharparenleft}rule\ pdl{\isacharunderscore}box{\isacharunderscore}reg{\isacharparenright}\isanewline
\ \ \isamarkupfalse%
\isacommand{let}\ {\isacharquery}P\ {\isacharequal}\ {\isachardoublequote}{\isacharbrackleft}{\isacharhash}\ x{\isasymleftarrow}p{\isacharbrackright}{\isacharparenleft}P\ x{\isacharparenright}{\isachardoublequote}\ \isakeyword{and}\ {\isacharquery}Q\ {\isacharequal}\ {\isachardoublequote}{\isacharbrackleft}{\isacharhash}\ x{\isasymleftarrow}p{\isacharbrackright}{\isacharparenleft}Q\ x{\isacharparenright}{\isachardoublequote}\ \isakeyword{and}\ {\isacharquery}PQ\ {\isacharequal}\ {\isachardoublequote}{\isacharbrackleft}{\isacharhash}\ x{\isasymleftarrow}p{\isacharbrackright}{\isacharparenleft}P\ x\ {\isasymand}\isactrlsub D\ Q\ x{\isacharparenright}{\isachardoublequote}\isanewline
\ \ \isamarkupfalse%
\isacommand{have}\ {\isachardoublequote}{\isasymturnstile}\ {\isacharparenleft}{\isacharquery}PQ\ {\isasymlongrightarrow}\isactrlsub D\ {\isacharquery}P{\isacharparenright}\ {\isasymlongrightarrow}\isactrlsub D\ {\isacharparenleft}{\isacharquery}PQ\ {\isasymlongrightarrow}\isactrlsub D\ {\isacharquery}Q{\isacharparenright}\ {\isasymlongrightarrow}\isactrlsub D\ {\isacharparenleft}{\isacharquery}PQ\ {\isasymlongrightarrow}\isactrlsub D\ {\isacharquery}P\ {\isasymand}\isactrlsub D\ {\isacharquery}Q{\isacharparenright}{\isachardoublequote}\isanewline
\ \ \ \ \isamarkupfalse%
\isacommand{by}\ {\isacharparenleft}simp\ only{\isacharcolon}\ pdl{\isacharunderscore}taut\ Valid{\isacharunderscore}Ret{\isacharparenright}\isanewline
\ \ \isamarkupfalse%
\isacommand{from}\ this\ a{\isadigit{1}}\ \isamarkupfalse%
\isacommand{have}\ {\isachardoublequote}{\isasymturnstile}\ {\isacharparenleft}{\isacharquery}PQ\ {\isasymlongrightarrow}\isactrlsub D\ {\isacharquery}Q{\isacharparenright}\ {\isasymlongrightarrow}\isactrlsub D\ {\isacharparenleft}{\isacharquery}PQ\ {\isasymlongrightarrow}\isactrlsub D\ {\isacharquery}P\ {\isasymand}\isactrlsub D\ {\isacharquery}Q{\isacharparenright}{\isachardoublequote}\ \isamarkupfalse%
\isacommand{by}\ {\isacharparenleft}rule\ pdl{\isacharunderscore}mp{\isacharparenright}\isanewline
\ \ \isamarkupfalse%
\isacommand{from}\ this\ a{\isadigit{2}}\ \isamarkupfalse%
\isacommand{show}\ {\isacharquery}thesis\ \isamarkupfalse%
\isacommand{by}\ {\isacharparenleft}rule\ pdl{\isacharunderscore}mp{\isacharparenright}\isanewline
\isamarkupfalse%
\isacommand{qed}\isamarkupfalse%
%
\begin{isamarkuptext}%
The box operator distributes over (finite) conjunction.%
\end{isamarkuptext}%
\isamarkuptrue%
\isacommand{theorem}\ box{\isacharunderscore}conj{\isacharunderscore}distrib{\isacharcolon}\ {\isachardoublequote}{\isasymturnstile}\ {\isacharbrackleft}{\isacharhash}\ x{\isasymleftarrow}p{\isacharbrackright}{\isacharparenleft}P\ x\ {\isasymand}\isactrlsub D\ Q\ x{\isacharparenright}\ {\isasymlongleftrightarrow}\isactrlsub D\ {\isacharbrackleft}{\isacharhash}\ x{\isasymleftarrow}p{\isacharbrackright}{\isacharparenleft}P\ x{\isacharparenright}\ {\isasymand}\isactrlsub D\ {\isacharbrackleft}{\isacharhash}\ x{\isasymleftarrow}p{\isacharbrackright}{\isacharparenleft}Q\ x{\isacharparenright}{\isachardoublequote}\isanewline
\ \ \isamarkupfalse%
\isacommand{apply}\ {\isacharparenleft}rule\ pdl{\isacharunderscore}iffI{\isacharparenright}\isanewline
\ \ \isamarkupfalse%
\isacommand{apply}\ {\isacharparenleft}rule\ box{\isacharunderscore}conj{\isacharunderscore}distrib{\isadigit{2}}{\isacharparenright}\isanewline
\ \ \isamarkupfalse%
\isacommand{apply}\ {\isacharparenleft}rule\ box{\isacharunderscore}conj{\isacharunderscore}distrib{\isadigit{1}}{\isacharparenright}\isanewline
\isamarkupfalse%
\isacommand{done}\isamarkupfalse%
%
\begin{isamarkuptext}%
Split and join rules for boxes and diamonds.%
\end{isamarkuptext}%
\isamarkuptrue%
\isacommand{lemma}\ pdl{\isacharunderscore}seqB{\isacharunderscore}split{\isacharcolon}\ {\isachardoublequote}{\isasymturnstile}\ {\isacharbrackleft}{\isacharhash}\ do\ {\isacharbraceleft}x{\isasymleftarrow}p{\isacharsemicolon}\ y{\isasymleftarrow}q\ x{\isacharsemicolon}\ ret\ {\isacharparenleft}x{\isacharcomma}\ y{\isacharparenright}{\isacharbraceright}{\isacharbrackright}{\isacharparenleft}{\isasymlambda}{\isacharparenleft}x{\isacharcomma}\ y{\isacharparenright}{\isachardot}\ P\ x\ y{\isacharparenright}\ \isanewline
\ \ \ \ \ \ \ \ \ \ \ \ \ \ \ \ \ \ \ \ \ \ \ \ \ {\isasymLongrightarrow}\ {\isasymturnstile}\ {\isacharbrackleft}{\isacharhash}\ p{\isacharbrackright}{\isacharparenleft}{\isasymlambda}x{\isachardot}\ {\isacharbrackleft}{\isacharhash}\ q\ x{\isacharbrackright}P\ x{\isacharparenright}{\isachardoublequote}\isanewline
\ \ \isamarkupfalse%
\isacommand{by}\ {\isacharparenleft}rule\ pdl{\isacharunderscore}seqB{\isacharbrackleft}THEN\ pdl{\isacharunderscore}iffD{\isadigit{1}}{\isacharcomma}\ THEN\ pdl{\isacharunderscore}mp{\isacharbrackright}{\isacharparenright}\isanewline
\isanewline
\isamarkupfalse%
\isacommand{lemma}\ pdl{\isacharunderscore}seqB{\isacharunderscore}join{\isacharcolon}\ {\isachardoublequote}{\isasymturnstile}\ {\isacharbrackleft}{\isacharhash}\ p{\isacharbrackright}{\isacharparenleft}{\isasymlambda}x{\isachardot}\ {\isacharbrackleft}{\isacharhash}\ q\ x{\isacharbrackright}P\ x{\isacharparenright}\ \isanewline
\ \ \ \ \ \ \ \ \ \ \ \ \ \ \ \ \ \ \ \ \ \ \ \ \ {\isasymLongrightarrow}\ {\isasymturnstile}\ {\isacharbrackleft}{\isacharhash}\ do\ {\isacharbraceleft}x{\isasymleftarrow}p{\isacharsemicolon}\ y{\isasymleftarrow}q\ x{\isacharsemicolon}\ ret\ {\isacharparenleft}x{\isacharcomma}\ y{\isacharparenright}{\isacharbraceright}{\isacharbrackright}{\isacharparenleft}{\isasymlambda}{\isacharparenleft}x{\isacharcomma}\ y{\isacharparenright}{\isachardot}\ P\ x\ y{\isacharparenright}{\isachardoublequote}\isanewline
\ \ \isamarkupfalse%
\isacommand{by}\ {\isacharparenleft}rule\ pdl{\isacharunderscore}seqB{\isacharbrackleft}THEN\ pdl{\isacharunderscore}iffD{\isadigit{2}}{\isacharcomma}\ THEN\ pdl{\isacharunderscore}mp{\isacharbrackright}{\isacharparenright}\isanewline
\isanewline
\isamarkupfalse%
\isacommand{lemma}\ pdl{\isacharunderscore}seqD{\isacharunderscore}split{\isacharcolon}\ {\isachardoublequote}{\isasymturnstile}\ {\isasymlangle}do\ {\isacharbraceleft}x{\isasymleftarrow}p{\isacharsemicolon}\ y{\isasymleftarrow}q\ x{\isacharsemicolon}\ ret\ {\isacharparenleft}x{\isacharcomma}\ y{\isacharparenright}{\isacharbraceright}{\isasymrangle}{\isacharparenleft}{\isasymlambda}{\isacharparenleft}x{\isacharcomma}\ y{\isacharparenright}{\isachardot}\ P\ x\ y{\isacharparenright}\ \isanewline
\ \ \ \ \ \ \ \ \ \ \ \ \ \ \ \ \ \ \ \ \ \ \ \ \ {\isasymLongrightarrow}\ {\isasymturnstile}\ {\isasymlangle}p{\isasymrangle}{\isacharparenleft}{\isasymlambda}x{\isachardot}\ {\isasymlangle}q\ x{\isasymrangle}P\ x{\isacharparenright}{\isachardoublequote}\isanewline
\ \ \isamarkupfalse%
\isacommand{by}\ {\isacharparenleft}rule\ pdl{\isacharunderscore}seqD{\isacharbrackleft}THEN\ pdl{\isacharunderscore}iffD{\isadigit{1}}{\isacharcomma}\ THEN\ pdl{\isacharunderscore}mp{\isacharbrackright}{\isacharparenright}\isanewline
\isanewline
\isamarkupfalse%
\isacommand{lemma}\ pdl{\isacharunderscore}seqD{\isacharunderscore}join{\isacharcolon}\ {\isachardoublequote}{\isasymturnstile}\ {\isasymlangle}p{\isasymrangle}{\isacharparenleft}{\isasymlambda}x{\isachardot}\ {\isasymlangle}q\ x{\isasymrangle}P\ x{\isacharparenright}\ \isanewline
\ \ \ \ \ \ \ \ \ \ \ \ \ \ \ \ \ \ \ \ \ \ \ \ \ {\isasymLongrightarrow}\ {\isasymturnstile}\ {\isasymlangle}do\ {\isacharbraceleft}x{\isasymleftarrow}p{\isacharsemicolon}\ y{\isasymleftarrow}q\ x{\isacharsemicolon}\ ret\ {\isacharparenleft}x{\isacharcomma}\ y{\isacharparenright}{\isacharbraceright}{\isasymrangle}{\isacharparenleft}{\isasymlambda}{\isacharparenleft}x{\isacharcomma}\ y{\isacharparenright}{\isachardot}\ P\ x\ y{\isacharparenright}{\isachardoublequote}\isanewline
\ \ \isamarkupfalse%
\isacommand{by}\ {\isacharparenleft}rule\ pdl{\isacharunderscore}seqD{\isacharbrackleft}THEN\ pdl{\isacharunderscore}iffD{\isadigit{2}}{\isacharcomma}\ THEN\ pdl{\isacharunderscore}mp{\isacharbrackright}{\isacharparenright}\isamarkupfalse%
%
\begin{isamarkuptext}%
Working in an axiomatic proof system requires a lot of auxiliary 
  rules; especially the lack of an implication introduction rule 
  (\isa{{\isacharparenleft}P\ {\isasymLongrightarrow}\ Q{\isacharparenright}\ {\isasymLongrightarrow}\ P\ {\isasymlongrightarrow}\ Q}) cries for lots of lemmas that are essentially just
  basic lemmas lifted over some premiss.
  \label{isa:pdl-lifted-lemmas}%
\end{isamarkuptext}%
\isamarkuptrue%
\isacommand{lemma}\ pdl{\isacharunderscore}wkB{\isacharunderscore}lifted{\isadigit{1}}{\isacharcolon}\ {\isachardoublequote}{\isasymlbrakk}\ {\isasymturnstile}\ A\ {\isasymlongrightarrow}\isactrlsub D\ {\isacharbrackleft}{\isacharhash}\ p{\isacharbrackright}B{\isacharsemicolon}\ {\isasymforall}x{\isachardot}\ {\isasymturnstile}\ B\ x\ {\isasymlongrightarrow}\isactrlsub D\ C\ x{\isasymrbrakk}\ {\isasymLongrightarrow}\ {\isasymturnstile}\ A\ {\isasymlongrightarrow}\isactrlsub D\ {\isacharbrackleft}{\isacharhash}\ p{\isacharbrackright}C{\isachardoublequote}\isanewline
\isamarkupfalse%
\isacommand{proof}\ {\isacharminus}\isanewline
\ \ \isamarkupfalse%
\isacommand{assume}\ a{\isadigit{1}}{\isacharcolon}\ {\isachardoublequote}{\isasymturnstile}\ A\ {\isasymlongrightarrow}\isactrlsub D\ {\isacharbrackleft}{\isacharhash}\ p{\isacharbrackright}B{\isachardoublequote}\ \isakeyword{and}\ a{\isadigit{2}}{\isacharcolon}\ {\isachardoublequote}{\isasymforall}x{\isachardot}\ {\isasymturnstile}\ B\ x\ {\isasymlongrightarrow}\isactrlsub D\ C\ x{\isachardoublequote}\isanewline
\ \ \isamarkupfalse%
\isacommand{from}\ a{\isadigit{2}}\ \isamarkupfalse%
\isacommand{have}\ {\isachardoublequote}{\isasymturnstile}\ {\isacharbrackleft}{\isacharhash}\ p{\isacharbrackright}B\ {\isasymlongrightarrow}\isactrlsub D\ {\isacharbrackleft}{\isacharhash}\ p{\isacharbrackright}C{\isachardoublequote}\ \isamarkupfalse%
\isacommand{by}\ {\isacharparenleft}rule\ pdl{\isacharunderscore}box{\isacharunderscore}reg{\isacharparenright}\isanewline
\ \ \isamarkupfalse%
\isacommand{with}\ a{\isadigit{1}}\ \isamarkupfalse%
\isacommand{show}\ {\isacharquery}thesis\ \isamarkupfalse%
\isacommand{by}\ {\isacharparenleft}rule\ pdl{\isacharunderscore}imp{\isacharunderscore}trans{\isacharparenright}\isanewline
\isamarkupfalse%
\isacommand{qed}\isanewline
\isanewline
\isamarkupfalse%
\isacommand{lemma}\ pdl{\isacharunderscore}wkD{\isacharunderscore}lifted{\isadigit{1}}{\isacharcolon}\ {\isachardoublequote}{\isasymlbrakk}\ {\isasymturnstile}\ A\ {\isasymlongrightarrow}\isactrlsub D\ {\isasymlangle}p{\isasymrangle}B{\isacharsemicolon}\ {\isasymforall}x{\isachardot}\ {\isasymturnstile}\ B\ x\ {\isasymlongrightarrow}\isactrlsub D\ C\ x{\isasymrbrakk}\ {\isasymLongrightarrow}\ {\isasymturnstile}\ A\ {\isasymlongrightarrow}\isactrlsub D\ {\isasymlangle}p{\isasymrangle}C{\isachardoublequote}\isanewline
\isamarkupfalse%
\isacommand{proof}\ {\isacharminus}\ \ \isanewline
\ \ \isamarkupfalse%
\isacommand{assume}\ a{\isadigit{1}}{\isacharcolon}\ {\isachardoublequote}{\isasymturnstile}\ A\ {\isasymlongrightarrow}\isactrlsub D\ {\isasymlangle}p{\isasymrangle}B{\isachardoublequote}\ \isakeyword{and}\ a{\isadigit{2}}{\isacharcolon}\ {\isachardoublequote}{\isasymforall}x{\isachardot}\ {\isasymturnstile}\ B\ x\ {\isasymlongrightarrow}\isactrlsub D\ C\ x{\isachardoublequote}\isanewline
\ \ \isamarkupfalse%
\isacommand{from}\ a{\isadigit{2}}\ \isamarkupfalse%
\isacommand{have}\ {\isachardoublequote}{\isasymturnstile}\ {\isasymlangle}p{\isasymrangle}B\ {\isasymlongrightarrow}\isactrlsub D\ {\isasymlangle}p{\isasymrangle}C{\isachardoublequote}\ \isamarkupfalse%
\isacommand{by}\ {\isacharparenleft}rule\ pdl{\isacharunderscore}dmd{\isacharunderscore}reg{\isacharparenright}\isanewline
\ \ \isamarkupfalse%
\isacommand{with}\ a{\isadigit{1}}\ \isamarkupfalse%
\isacommand{show}\ {\isacharquery}thesis\ \isamarkupfalse%
\isacommand{by}\ {\isacharparenleft}rule\ pdl{\isacharunderscore}imp{\isacharunderscore}trans{\isacharparenright}\isanewline
\isamarkupfalse%
\isacommand{qed}\isanewline
\isanewline
\isamarkupfalse%
\isacommand{lemma}\ box{\isacharunderscore}conj{\isacharunderscore}distrib{\isacharunderscore}lifted{\isadigit{1}}{\isacharcolon}\ {\isachardoublequote}{\isasymturnstile}\ {\isacharparenleft}A\ {\isasymlongrightarrow}\isactrlsub D\ {\isacharbrackleft}{\isacharhash}\ p{\isacharbrackright}{\isacharparenleft}{\isasymlambda}x{\isachardot}\ P\ x\ {\isasymand}\isactrlsub D\ Q\ x{\isacharparenright}{\isacharparenright}\ {\isasymlongleftrightarrow}\isactrlsub D\ {\isacharparenleft}{\isacharparenleft}A\ {\isasymlongrightarrow}\isactrlsub D\ {\isacharbrackleft}{\isacharhash}\ p{\isacharbrackright}P{\isacharparenright}\ {\isasymand}\isactrlsub D\ {\isacharparenleft}A\ {\isasymlongrightarrow}\isactrlsub D\ {\isacharbrackleft}{\isacharhash}\ p{\isacharbrackright}Q{\isacharparenright}{\isacharparenright}{\isachardoublequote}\isanewline
\isamarkupfalse%
\isacommand{proof}\ {\isacharparenleft}rule\ pdl{\isacharunderscore}iffI{\isacharparenright}\isanewline
\ \ \isamarkupfalse%
\isacommand{show}\ {\isachardoublequote}{\isasymturnstile}\ {\isacharparenleft}A\ {\isasymlongrightarrow}\isactrlsub D\ {\isacharbrackleft}{\isacharhash}\ p{\isacharbrackright}{\isacharparenleft}{\isasymlambda}x{\isachardot}\ P\ x\ {\isasymand}\isactrlsub D\ Q\ x{\isacharparenright}{\isacharparenright}\ {\isasymlongrightarrow}\isactrlsub D\ {\isacharparenleft}A\ {\isasymlongrightarrow}\isactrlsub D\ {\isacharbrackleft}{\isacharhash}\ p{\isacharbrackright}P{\isacharparenright}\ {\isasymand}\isactrlsub D\ {\isacharparenleft}A\ {\isasymlongrightarrow}\isactrlsub D\ {\isacharbrackleft}{\isacharhash}\ p{\isacharbrackright}Q{\isacharparenright}{\isachardoublequote}\isanewline
\ \ \isamarkupfalse%
\isacommand{proof}\ {\isacharminus}\isanewline
\ \ \ \ \isamarkupfalse%
\isacommand{have}\ {\isachardoublequote}{\isasymturnstile}\ {\isacharparenleft}{\isacharbrackleft}{\isacharhash}\ p{\isacharbrackright}{\isacharparenleft}{\isasymlambda}x{\isachardot}\ P\ x\ {\isasymand}\isactrlsub D\ Q\ x{\isacharparenright}\ {\isasymlongrightarrow}\isactrlsub D\ {\isacharbrackleft}{\isacharhash}\ p{\isacharbrackright}P\ {\isasymand}\isactrlsub D\ {\isacharbrackleft}{\isacharhash}\ p{\isacharbrackright}Q{\isacharparenright}\ {\isasymlongrightarrow}\isactrlsub D\isanewline
\ \ \ \ \ \ \ \ \ \ \ \ {\isacharparenleft}A\ {\isasymlongrightarrow}\isactrlsub D\ {\isacharbrackleft}{\isacharhash}\ p{\isacharbrackright}{\isacharparenleft}{\isasymlambda}x{\isachardot}\ P\ x\ {\isasymand}\isactrlsub D\ Q\ x{\isacharparenright}{\isacharparenright}\ {\isasymlongrightarrow}\isactrlsub D\ \isanewline
\ \ \ \ \ \ \ \ \ \ \ \ \ {\isacharparenleft}A\ {\isasymlongrightarrow}\isactrlsub D\ {\isacharbrackleft}{\isacharhash}\ p{\isacharbrackright}P{\isacharparenright}\ {\isasymand}\isactrlsub D\ {\isacharparenleft}A\ {\isasymlongrightarrow}\isactrlsub D\ {\isacharbrackleft}{\isacharhash}\ p{\isacharbrackright}Q{\isacharparenright}{\isachardoublequote}\isanewline
\ \ \ \ \ \ \isamarkupfalse%
\isacommand{by}\ {\isacharparenleft}simp\ add{\isacharcolon}\ pdl{\isacharunderscore}taut{\isacharparenright}\isanewline
\ \ \ \ \isamarkupfalse%
\isacommand{from}\ this\ box{\isacharunderscore}conj{\isacharunderscore}distrib{\isadigit{2}}\ \isamarkupfalse%
\isacommand{show}\ {\isacharquery}thesis\ \isamarkupfalse%
\isacommand{by}\ {\isacharparenleft}rule\ pdl{\isacharunderscore}mp{\isacharparenright}\isanewline
\ \ \isamarkupfalse%
\isacommand{qed}\isanewline
\isamarkupfalse%
\isacommand{next}\isanewline
\ \ \isamarkupfalse%
\isacommand{show}\ {\isachardoublequote}{\isasymturnstile}\ {\isacharparenleft}{\isacharparenleft}A\ {\isasymlongrightarrow}\isactrlsub D\ {\isacharbrackleft}{\isacharhash}\ p{\isacharbrackright}P{\isacharparenright}\ {\isasymand}\isactrlsub D\ {\isacharparenleft}A\ {\isasymlongrightarrow}\isactrlsub D\ {\isacharbrackleft}{\isacharhash}\ p{\isacharbrackright}Q{\isacharparenright}{\isacharparenright}\ {\isasymlongrightarrow}\isactrlsub D\ A\ {\isasymlongrightarrow}\isactrlsub D\ {\isacharbrackleft}{\isacharhash}\ p{\isacharbrackright}{\isacharparenleft}{\isasymlambda}x{\isachardot}\ P\ x\ {\isasymand}\isactrlsub D\ Q\ x{\isacharparenright}{\isachardoublequote}\isanewline
\ \ \isamarkupfalse%
\isacommand{proof}\ {\isacharminus}\isanewline
\ \ \ \ \isamarkupfalse%
\isacommand{have}\ {\isachardoublequote}{\isasymturnstile}\ {\isacharparenleft}{\isacharbrackleft}{\isacharhash}\ p{\isacharbrackright}P\ {\isasymand}\isactrlsub D\ {\isacharbrackleft}{\isacharhash}\ p{\isacharbrackright}Q\ {\isasymlongrightarrow}\isactrlsub D\ {\isacharbrackleft}{\isacharhash}\ p{\isacharbrackright}{\isacharparenleft}{\isasymlambda}x{\isachardot}\ P\ x\ {\isasymand}\isactrlsub D\ Q\ x{\isacharparenright}{\isacharparenright}\ {\isasymlongrightarrow}\isactrlsub D\isanewline
\ \ \ \ \ \ \ \ \ \ \ \ {\isacharparenleft}{\isacharparenleft}A\ {\isasymlongrightarrow}\isactrlsub D\ {\isacharbrackleft}{\isacharhash}\ p{\isacharbrackright}P{\isacharparenright}\ {\isasymand}\isactrlsub D\ {\isacharparenleft}A\ {\isasymlongrightarrow}\isactrlsub D\ {\isacharbrackleft}{\isacharhash}\ p{\isacharbrackright}Q{\isacharparenright}{\isacharparenright}\ {\isasymlongrightarrow}\isactrlsub D\ \isanewline
\ \ \ \ \ \ \ \ \ \ \ \ \ A\ {\isasymlongrightarrow}\isactrlsub D\ {\isacharbrackleft}{\isacharhash}\ p{\isacharbrackright}{\isacharparenleft}{\isasymlambda}x{\isachardot}\ P\ x\ {\isasymand}\isactrlsub D\ Q\ x{\isacharparenright}{\isachardoublequote}\isanewline
\ \ \ \ \ \ \isamarkupfalse%
\isacommand{by}\ {\isacharparenleft}simp\ add{\isacharcolon}\ pdl{\isacharunderscore}taut{\isacharparenright}\isanewline
\ \ \ \ \isamarkupfalse%
\isacommand{from}\ this\ box{\isacharunderscore}conj{\isacharunderscore}distrib{\isadigit{1}}\ \isamarkupfalse%
\isacommand{show}\ {\isacharquery}thesis\ \isamarkupfalse%
\isacommand{by}\ {\isacharparenleft}rule\ pdl{\isacharunderscore}mp{\isacharparenright}\isanewline
\ \ \isamarkupfalse%
\isacommand{qed}\isanewline
\isamarkupfalse%
\isacommand{qed}\isanewline
\isanewline
\isamarkupfalse%
\isacommand{lemma}\ pdl{\isacharunderscore}seqB{\isacharunderscore}lifted{\isadigit{1}}{\isacharcolon}\ {\isachardoublequote}{\isasymturnstile}\ {\isacharparenleft}\ A\ {\isasymlongrightarrow}\isactrlsub D\ {\isacharbrackleft}{\isacharhash}\ p{\isacharbrackright}{\isacharparenleft}{\isasymlambda}x{\isachardot}\ {\isacharbrackleft}{\isacharhash}\ q\ x{\isacharbrackright}P{\isacharparenright}\ {\isacharparenright}\ {\isasymlongleftrightarrow}\isactrlsub D\ {\isacharparenleft}\ A\ {\isasymlongrightarrow}\isactrlsub D\ {\isacharbrackleft}{\isacharhash}\ do\ {\isacharbraceleft}x{\isasymleftarrow}p{\isacharsemicolon}\ q\ x{\isacharbraceright}{\isacharbrackright}P\ {\isacharparenright}{\isachardoublequote}\isanewline
\isamarkupfalse%
\isacommand{proof}\ {\isacharparenleft}rule\ pdl{\isacharunderscore}iffI{\isacharparenright}\isanewline
\ \ \isamarkupfalse%
\isacommand{show}\ {\isachardoublequote}{\isasymturnstile}\ {\isacharparenleft}A\ {\isasymlongrightarrow}\isactrlsub D\ {\isacharbrackleft}{\isacharhash}\ p{\isacharbrackright}{\isacharparenleft}{\isasymlambda}x{\isachardot}\ {\isacharbrackleft}{\isacharhash}\ q\ x{\isacharbrackright}P{\isacharparenright}{\isacharparenright}\ {\isasymlongrightarrow}\isactrlsub D\ A\ {\isasymlongrightarrow}\isactrlsub D\ {\isacharbrackleft}{\isacharhash}\ do\ {\isacharbraceleft}x{\isasymleftarrow}p{\isacharsemicolon}\ q\ x{\isacharbraceright}{\isacharbrackright}P{\isachardoublequote}\isanewline
\ \ \isamarkupfalse%
\isacommand{proof}\ {\isacharminus}\isanewline
\ \ \ \ \isamarkupfalse%
\isacommand{have}\ {\isachardoublequote}{\isasymturnstile}\ {\isacharparenleft}{\isacharbrackleft}{\isacharhash}\ p{\isacharbrackright}{\isacharparenleft}{\isasymlambda}x{\isachardot}\ {\isacharbrackleft}{\isacharhash}\ q\ x{\isacharbrackright}P{\isacharparenright}\ {\isasymlongrightarrow}\isactrlsub D\ {\isacharbrackleft}{\isacharhash}\ do\ {\isacharbraceleft}x{\isasymleftarrow}p{\isacharsemicolon}\ q\ x{\isacharbraceright}{\isacharbrackright}P{\isacharparenright}\ {\isasymlongrightarrow}\isactrlsub D\isanewline
\ \ \ \ \ \ \ \ \ \ \ \ \ {\isacharparenleft}A\ {\isasymlongrightarrow}\isactrlsub D\ {\isacharbrackleft}{\isacharhash}\ p{\isacharbrackright}{\isacharparenleft}{\isasymlambda}x{\isachardot}\ {\isacharbrackleft}{\isacharhash}\ q\ x{\isacharbrackright}P{\isacharparenright}{\isacharparenright}\ {\isasymlongrightarrow}\isactrlsub D\isanewline
\ \ \ \ \ \ \ \ \ \ \ \ \ {\isacharparenleft}A\ {\isasymlongrightarrow}\isactrlsub D\ {\isacharbrackleft}{\isacharhash}\ do\ {\isacharbraceleft}x{\isasymleftarrow}p{\isacharsemicolon}\ q\ x{\isacharbraceright}{\isacharbrackright}P{\isacharparenright}{\isachardoublequote}\isanewline
\ \ \ \ \ \ \isamarkupfalse%
\isacommand{by}\ {\isacharparenleft}simp\ add{\isacharcolon}\ pdl{\isacharunderscore}taut{\isacharparenright}\isanewline
\ \ \ \ \isamarkupfalse%
\isacommand{from}\ this\ pdl{\isacharunderscore}iffD{\isadigit{1}}{\isacharbrackleft}OF\ pdl{\isacharunderscore}seqB{\isacharunderscore}simp{\isacharbrackright}\ \isamarkupfalse%
\isacommand{show}\ {\isacharquery}thesis\ \isamarkupfalse%
\isacommand{by}\ {\isacharparenleft}rule\ pdl{\isacharunderscore}mp{\isacharparenright}\isanewline
\ \ \isamarkupfalse%
\isacommand{qed}\isanewline
\isamarkupfalse%
\isacommand{next}\isanewline
\ \ \isamarkupfalse%
\isacommand{show}\ {\isachardoublequote}{\isasymturnstile}\ {\isacharparenleft}A\ {\isasymlongrightarrow}\isactrlsub D\ {\isacharbrackleft}{\isacharhash}\ do\ {\isacharbraceleft}x{\isasymleftarrow}p{\isacharsemicolon}\ q\ x{\isacharbraceright}{\isacharbrackright}P{\isacharparenright}\ {\isasymlongrightarrow}\isactrlsub D\ A\ {\isasymlongrightarrow}\isactrlsub D\ {\isacharbrackleft}{\isacharhash}\ p{\isacharbrackright}{\isacharparenleft}{\isasymlambda}x{\isachardot}\ {\isacharbrackleft}{\isacharhash}\ q\ x{\isacharbrackright}P{\isacharparenright}{\isachardoublequote}\isanewline
\ \ \isamarkupfalse%
\isacommand{proof}\ {\isacharminus}\isanewline
\ \ \ \ \isamarkupfalse%
\isacommand{have}\ {\isachardoublequote}{\isasymturnstile}\ {\isacharparenleft}{\isacharbrackleft}{\isacharhash}\ do\ {\isacharbraceleft}x{\isasymleftarrow}p{\isacharsemicolon}\ q\ x{\isacharbraceright}{\isacharbrackright}P\ {\isasymlongrightarrow}\isactrlsub D\ {\isacharbrackleft}{\isacharhash}\ p{\isacharbrackright}{\isacharparenleft}{\isasymlambda}x{\isachardot}\ {\isacharbrackleft}{\isacharhash}\ q\ x{\isacharbrackright}P{\isacharparenright}{\isacharparenright}\ {\isasymlongrightarrow}\isactrlsub D\isanewline
\ \ \ \ \ \ \ \ \ \ \ \ {\isacharparenleft}A\ {\isasymlongrightarrow}\isactrlsub D\ {\isacharbrackleft}{\isacharhash}\ do\ {\isacharbraceleft}x{\isasymleftarrow}p{\isacharsemicolon}\ q\ x{\isacharbraceright}{\isacharbrackright}P{\isacharparenright}\ {\isasymlongrightarrow}\isactrlsub D\isanewline
\ \ \ \ \ \ \ \ \ \ \ \ {\isacharparenleft}A\ {\isasymlongrightarrow}\isactrlsub D\ {\isacharbrackleft}{\isacharhash}\ p{\isacharbrackright}{\isacharparenleft}{\isasymlambda}x{\isachardot}\ {\isacharbrackleft}{\isacharhash}\ q\ x{\isacharbrackright}P{\isacharparenright}{\isacharparenright}{\isachardoublequote}\isanewline
\ \ \ \ \ \ \isamarkupfalse%
\isacommand{by}\ {\isacharparenleft}simp\ add{\isacharcolon}\ pdl{\isacharunderscore}taut{\isacharparenright}\isanewline
\ \ \ \ \isamarkupfalse%
\isacommand{from}\ this\ pdl{\isacharunderscore}iffD{\isadigit{2}}{\isacharbrackleft}OF\ pdl{\isacharunderscore}seqB{\isacharunderscore}simp{\isacharbrackright}\ \isamarkupfalse%
\isacommand{show}\ {\isacharquery}thesis\ \isamarkupfalse%
\isacommand{by}\ {\isacharparenleft}rule\ pdl{\isacharunderscore}mp{\isacharparenright}\isanewline
\ \ \isamarkupfalse%
\isacommand{qed}\isanewline
\isamarkupfalse%
\isacommand{qed}\isanewline
\isanewline
\isamarkupfalse%
\isacommand{lemma}\ pdl{\isacharunderscore}seqD{\isacharunderscore}lifted{\isadigit{1}}{\isacharcolon}\ {\isachardoublequote}{\isasymturnstile}\ {\isacharparenleft}\ A{\isasymlongrightarrow}\isactrlsub D\ {\isasymlangle}x{\isasymleftarrow}p{\isasymrangle}{\isasymlangle}q\ x{\isasymrangle}P\ {\isacharparenright}\ {\isasymlongleftrightarrow}\isactrlsub D\ {\isacharparenleft}A\ {\isasymlongrightarrow}\isactrlsub D\ {\isasymlangle}do\ {\isacharbraceleft}x{\isasymleftarrow}p{\isacharsemicolon}\ q\ x{\isacharbraceright}{\isasymrangle}P\ {\isacharparenright}{\isachardoublequote}\isanewline
\isamarkupfalse%
\isacommand{proof}\ {\isacharparenleft}rule\ pdl{\isacharunderscore}iffI{\isacharparenright}\isanewline
\ \ \isamarkupfalse%
\isacommand{show}\ {\isachardoublequote}{\isasymturnstile}\ {\isacharparenleft}A\ {\isasymlongrightarrow}\isactrlsub D\ {\isasymlangle}p{\isasymrangle}{\isacharparenleft}{\isasymlambda}x{\isachardot}\ {\isasymlangle}q\ x{\isasymrangle}P{\isacharparenright}{\isacharparenright}\ {\isasymlongrightarrow}\isactrlsub D\ A\ {\isasymlongrightarrow}\isactrlsub D\ {\isasymlangle}do\ {\isacharbraceleft}x{\isasymleftarrow}p{\isacharsemicolon}\ q\ x{\isacharbraceright}{\isasymrangle}P{\isachardoublequote}\isanewline
\ \ \isamarkupfalse%
\isacommand{proof}\ {\isacharminus}\isanewline
\ \ \ \ \isamarkupfalse%
\isacommand{have}\ {\isachardoublequote}{\isasymturnstile}\ {\isacharparenleft}{\isasymlangle}p{\isasymrangle}{\isacharparenleft}{\isasymlambda}x{\isachardot}\ {\isasymlangle}q\ x{\isasymrangle}P{\isacharparenright}\ {\isasymlongrightarrow}\isactrlsub D\ {\isasymlangle}do\ {\isacharbraceleft}x{\isasymleftarrow}p{\isacharsemicolon}\ q\ x{\isacharbraceright}{\isasymrangle}P{\isacharparenright}\ {\isasymlongrightarrow}\isactrlsub D\isanewline
\ \ \ \ \ \ \ \ \ \ \ \ \ {\isacharparenleft}A\ {\isasymlongrightarrow}\isactrlsub D\ {\isasymlangle}p{\isasymrangle}{\isacharparenleft}{\isasymlambda}x{\isachardot}\ {\isasymlangle}q\ x{\isasymrangle}P{\isacharparenright}{\isacharparenright}\ {\isasymlongrightarrow}\isactrlsub D\isanewline
\ \ \ \ \ \ \ \ \ \ \ \ \ {\isacharparenleft}A\ {\isasymlongrightarrow}\isactrlsub D\ {\isasymlangle}do\ {\isacharbraceleft}x{\isasymleftarrow}p{\isacharsemicolon}\ q\ x{\isacharbraceright}{\isasymrangle}P{\isacharparenright}{\isachardoublequote}\isanewline
\ \ \ \ \ \ \isamarkupfalse%
\isacommand{by}\ {\isacharparenleft}simp\ add{\isacharcolon}\ pdl{\isacharunderscore}taut{\isacharparenright}\isanewline
\ \ \ \ \isamarkupfalse%
\isacommand{from}\ this\ pdl{\isacharunderscore}iffD{\isadigit{1}}{\isacharbrackleft}OF\ pdl{\isacharunderscore}seqD{\isacharunderscore}simp{\isacharbrackright}\ \isamarkupfalse%
\isacommand{show}\ {\isacharquery}thesis\ \isamarkupfalse%
\isacommand{by}\ {\isacharparenleft}rule\ pdl{\isacharunderscore}mp{\isacharparenright}\isanewline
\ \ \isamarkupfalse%
\isacommand{qed}\isanewline
\isamarkupfalse%
\isacommand{next}\isanewline
\ \ \isamarkupfalse%
\isacommand{show}\ {\isachardoublequote}{\isasymturnstile}\ {\isacharparenleft}A\ {\isasymlongrightarrow}\isactrlsub D\ {\isasymlangle}do\ {\isacharbraceleft}x{\isasymleftarrow}p{\isacharsemicolon}\ q\ x{\isacharbraceright}{\isasymrangle}P{\isacharparenright}\ {\isasymlongrightarrow}\isactrlsub D\ A\ {\isasymlongrightarrow}\isactrlsub D\ {\isasymlangle}p{\isasymrangle}{\isacharparenleft}{\isasymlambda}x{\isachardot}\ {\isasymlangle}q\ x{\isasymrangle}P{\isacharparenright}{\isachardoublequote}\isanewline
\ \ \isamarkupfalse%
\isacommand{proof}\ {\isacharminus}\isanewline
\ \ \ \ \isamarkupfalse%
\isacommand{have}\ {\isachardoublequote}{\isasymturnstile}\ {\isacharparenleft}{\isasymlangle}do\ {\isacharbraceleft}x{\isasymleftarrow}p{\isacharsemicolon}\ q\ x{\isacharbraceright}{\isasymrangle}P\ {\isasymlongrightarrow}\isactrlsub D\ {\isasymlangle}p{\isasymrangle}{\isacharparenleft}{\isasymlambda}x{\isachardot}\ {\isasymlangle}q\ x{\isasymrangle}P{\isacharparenright}{\isacharparenright}\ {\isasymlongrightarrow}\isactrlsub D\isanewline
\ \ \ \ \ \ \ \ \ \ \ \ {\isacharparenleft}A\ {\isasymlongrightarrow}\isactrlsub D\ {\isasymlangle}do\ {\isacharbraceleft}x{\isasymleftarrow}p{\isacharsemicolon}\ q\ x{\isacharbraceright}{\isasymrangle}P{\isacharparenright}\ {\isasymlongrightarrow}\isactrlsub D\isanewline
\ \ \ \ \ \ \ \ \ \ \ \ {\isacharparenleft}A\ {\isasymlongrightarrow}\isactrlsub D\ {\isasymlangle}p{\isasymrangle}{\isacharparenleft}{\isasymlambda}x{\isachardot}\ {\isasymlangle}q\ x{\isasymrangle}P{\isacharparenright}{\isacharparenright}{\isachardoublequote}\isanewline
\ \ \ \ \ \ \isamarkupfalse%
\isacommand{by}\ {\isacharparenleft}simp\ add{\isacharcolon}\ pdl{\isacharunderscore}taut{\isacharparenright}\isanewline
\ \ \ \ \isamarkupfalse%
\isacommand{from}\ this\ pdl{\isacharunderscore}iffD{\isadigit{2}}{\isacharbrackleft}OF\ pdl{\isacharunderscore}seqD{\isacharunderscore}simp{\isacharbrackright}\ \isamarkupfalse%
\isacommand{show}\ {\isacharquery}thesis\ \isamarkupfalse%
\isacommand{by}\ {\isacharparenleft}rule\ pdl{\isacharunderscore}mp{\isacharparenright}\isanewline
\ \ \isamarkupfalse%
\isacommand{qed}\isanewline
\isamarkupfalse%
\isacommand{qed}\isanewline
\isanewline
\ \ \isanewline
\isanewline
\isamarkupfalse%
\isacommand{lemma}\ pdl{\isacharunderscore}plugB{\isacharunderscore}lifted{\isadigit{1}}{\isacharcolon}\ {\isachardoublequote}{\isasymlbrakk}\ {\isasymturnstile}\ A\ {\isasymlongrightarrow}\isactrlsub D\ {\isacharbrackleft}{\isacharhash}\ p{\isacharbrackright}B{\isacharsemicolon}\ {\isasymforall}x{\isachardot}\ {\isasymturnstile}\ B\ x\ {\isasymlongrightarrow}\isactrlsub D\ {\isacharbrackleft}{\isacharhash}\ q\ x{\isacharbrackright}C{\isasymrbrakk}\ {\isasymLongrightarrow}\ {\isasymturnstile}\ A\ {\isasymlongrightarrow}\isactrlsub D\ {\isacharbrackleft}{\isacharhash}\ do\ {\isacharbraceleft}x{\isasymleftarrow}p{\isacharsemicolon}\ q\ x{\isacharbraceright}{\isacharbrackright}C{\isachardoublequote}\isanewline
\isamarkupfalse%
\isacommand{proof}\ {\isacharminus}\isanewline
\ \ \isamarkupfalse%
\isacommand{assume}\ a{\isadigit{1}}{\isacharcolon}\ {\isachardoublequote}{\isasymturnstile}\ A\ {\isasymlongrightarrow}\isactrlsub D\ {\isacharbrackleft}{\isacharhash}\ p{\isacharbrackright}B{\isachardoublequote}\ \isakeyword{and}\ a{\isadigit{2}}{\isacharcolon}\ {\isachardoublequote}{\isasymforall}x{\isachardot}\ {\isasymturnstile}\ B\ x\ {\isasymlongrightarrow}\isactrlsub D\ {\isacharbrackleft}{\isacharhash}\ q\ x{\isacharbrackright}C{\isachardoublequote}\isanewline
\ \ \isamarkupfalse%
\isacommand{from}\ a{\isadigit{1}}\ a{\isadigit{2}}\ \isamarkupfalse%
\isacommand{have}\ {\isachardoublequote}{\isasymturnstile}\ A\ {\isasymlongrightarrow}\isactrlsub D\ {\isacharbrackleft}{\isacharhash}\ p{\isacharbrackright}{\isacharparenleft}{\isasymlambda}x{\isachardot}\ {\isacharbrackleft}{\isacharhash}\ q\ x{\isacharbrackright}C{\isacharparenright}{\isachardoublequote}\ \isamarkupfalse%
\isacommand{by}\ {\isacharparenleft}rule\ pdl{\isacharunderscore}wkB{\isacharunderscore}lifted{\isadigit{1}}{\isacharparenright}\isanewline
\ \ \isamarkupfalse%
\isacommand{thus}\ {\isacharquery}thesis\ \isamarkupfalse%
\isacommand{by}\ {\isacharparenleft}rule\ pdl{\isacharunderscore}iffD{\isadigit{1}}{\isacharbrackleft}OF\ pdl{\isacharunderscore}seqB{\isacharunderscore}lifted{\isadigit{1}}{\isacharcomma}\ THEN\ pdl{\isacharunderscore}mp{\isacharbrackright}{\isacharparenright}\isanewline
\isamarkupfalse%
\isacommand{qed}\isanewline
\isanewline
\isamarkupfalse%
\isacommand{lemma}\ pdl{\isacharunderscore}plugD{\isacharunderscore}lifted{\isadigit{1}}{\isacharcolon}\ {\isachardoublequote}{\isasymlbrakk}\ {\isasymturnstile}\ A\ {\isasymlongrightarrow}\isactrlsub D\ {\isasymlangle}p{\isasymrangle}B{\isacharsemicolon}\ {\isasymforall}x{\isachardot}\ {\isasymturnstile}\ B\ x\ {\isasymlongrightarrow}\isactrlsub D\ {\isasymlangle}q\ x{\isasymrangle}C{\isasymrbrakk}\ {\isasymLongrightarrow}\ {\isasymturnstile}\ A\ {\isasymlongrightarrow}\isactrlsub D\ {\isasymlangle}do\ {\isacharbraceleft}x{\isasymleftarrow}p{\isacharsemicolon}\ q\ x{\isacharbraceright}{\isasymrangle}C{\isachardoublequote}\isanewline
\isamarkupfalse%
\isacommand{proof}\ {\isacharminus}\isanewline
\ \ \isamarkupfalse%
\isacommand{assume}\ a{\isadigit{1}}{\isacharcolon}\ {\isachardoublequote}{\isasymturnstile}\ A\ {\isasymlongrightarrow}\isactrlsub D\ {\isasymlangle}p{\isasymrangle}B{\isachardoublequote}\ \isakeyword{and}\ a{\isadigit{2}}{\isacharcolon}\ {\isachardoublequote}{\isasymforall}x{\isachardot}\ {\isasymturnstile}\ B\ x\ {\isasymlongrightarrow}\isactrlsub D\ {\isasymlangle}q\ x{\isasymrangle}C{\isachardoublequote}\isanewline
\ \ \isamarkupfalse%
\isacommand{from}\ a{\isadigit{1}}\ a{\isadigit{2}}\ \isamarkupfalse%
\isacommand{have}\ {\isachardoublequote}{\isasymturnstile}\ A\ {\isasymlongrightarrow}\isactrlsub D\ {\isasymlangle}x{\isasymleftarrow}p{\isasymrangle}{\isasymlangle}q\ x{\isasymrangle}C{\isachardoublequote}\ \isamarkupfalse%
\isacommand{by}\ {\isacharparenleft}rule\ pdl{\isacharunderscore}wkD{\isacharunderscore}lifted{\isadigit{1}}{\isacharparenright}\isanewline
\ \ \isamarkupfalse%
\isacommand{thus}\ {\isacharquery}thesis\ \isamarkupfalse%
\isacommand{by}\ {\isacharparenleft}rule\ pdl{\isacharunderscore}iffD{\isadigit{1}}{\isacharbrackleft}OF\ pdl{\isacharunderscore}seqD{\isacharunderscore}lifted{\isadigit{1}}{\isacharcomma}\ THEN\ pdl{\isacharunderscore}mp{\isacharbrackright}{\isacharparenright}\isanewline
\isamarkupfalse%
\isacommand{qed}\isanewline
\isanewline
\isanewline
\isamarkupfalse%
\isacommand{lemma}\ imp{\isacharunderscore}box{\isacharunderscore}conj{\isadigit{1}}{\isacharcolon}\ {\isachardoublequote}{\isasymturnstile}\ A\ {\isasymlongrightarrow}\isactrlsub D\ {\isacharbrackleft}{\isacharhash}\ p{\isacharbrackright}{\isacharparenleft}{\isasymlambda}x{\isachardot}\ B\ x\ {\isasymand}\isactrlsub D\ C\ x{\isacharparenright}\ {\isasymLongrightarrow}\ {\isasymturnstile}\ A\ {\isasymlongrightarrow}\isactrlsub D\ {\isacharbrackleft}{\isacharhash}\ p{\isacharbrackright}B{\isachardoublequote}\isanewline
\isamarkupfalse%
\isacommand{proof}\ {\isacharparenleft}rule\ pdl{\isacharunderscore}wkB{\isacharunderscore}lifted{\isadigit{1}}{\isacharparenright}\isanewline
\ \ \isamarkupfalse%
\isacommand{assume}\ {\isachardoublequote}{\isasymturnstile}\ A\ {\isasymlongrightarrow}\isactrlsub D\ {\isacharbrackleft}{\isacharhash}\ p{\isacharbrackright}{\isacharparenleft}{\isasymlambda}x{\isachardot}\ B\ x\ {\isasymand}\isactrlsub D\ C\ x{\isacharparenright}{\isachardoublequote}\isanewline
\ \ \isamarkupfalse%
\isacommand{show}\ {\isachardoublequote}{\isasymturnstile}\ A\ {\isasymlongrightarrow}\isactrlsub D\ {\isacharbrackleft}{\isacharhash}\ p{\isacharbrackright}{\isacharparenleft}{\isasymlambda}x{\isachardot}\ B\ x\ {\isasymand}\isactrlsub D\ C\ x{\isacharparenright}{\isachardoublequote}\ \isamarkupfalse%
\isacommand{{\isachardot}}\isanewline
\isamarkupfalse%
\isacommand{next}\isanewline
\ \ \isamarkupfalse%
\isacommand{assume}\ {\isachardoublequote}{\isasymturnstile}\ A\ {\isasymlongrightarrow}\isactrlsub D\ {\isacharbrackleft}{\isacharhash}\ p{\isacharbrackright}{\isacharparenleft}{\isasymlambda}x{\isachardot}\ B\ x\ {\isasymand}\isactrlsub D\ C\ x{\isacharparenright}{\isachardoublequote}\isanewline
\ \ \isamarkupfalse%
\isacommand{show}\ {\isachardoublequote}{\isasymforall}x{\isachardot}\ {\isasymturnstile}\ B\ x\ {\isasymand}\isactrlsub D\ C\ x\ {\isasymlongrightarrow}\isactrlsub D\ B\ x{\isachardoublequote}\isanewline
\ \ \isamarkupfalse%
\isacommand{proof}\ \isanewline
\ \ \ \ \isamarkupfalse%
\isacommand{fix}\ x\ \isamarkupfalse%
\isacommand{show}\ {\isachardoublequote}{\isasymturnstile}\ B\ x\ {\isasymand}\isactrlsub D\ C\ x\ {\isasymlongrightarrow}\isactrlsub D\ B\ x{\isachardoublequote}\ \isamarkupfalse%
\isacommand{by}\ {\isacharparenleft}simp\ add{\isacharcolon}\ pdl{\isacharunderscore}taut{\isacharparenright}\isanewline
\ \ \isamarkupfalse%
\isacommand{qed}\isanewline
\isamarkupfalse%
\isacommand{qed}\isanewline
\isanewline
\isanewline
\isamarkupfalse%
\isacommand{lemma}\ imp{\isacharunderscore}box{\isacharunderscore}conj{\isadigit{2}}{\isacharcolon}\ {\isachardoublequote}{\isasymturnstile}\ A\ {\isasymlongrightarrow}\isactrlsub D\ {\isacharbrackleft}{\isacharhash}\ p{\isacharbrackright}{\isacharparenleft}{\isasymlambda}x{\isachardot}\ B\ x\ {\isasymand}\isactrlsub D\ C\ x{\isacharparenright}\ {\isasymLongrightarrow}\ {\isasymturnstile}\ A\ {\isasymlongrightarrow}\isactrlsub D\ {\isacharbrackleft}{\isacharhash}\ p{\isacharbrackright}C{\isachardoublequote}\isanewline
\isamarkupfalse%
\isacommand{proof}\ {\isacharparenleft}rule\ pdl{\isacharunderscore}wkB{\isacharunderscore}lifted{\isadigit{1}}{\isacharparenright}\isanewline
\ \ \isamarkupfalse%
\isacommand{assume}\ {\isachardoublequote}{\isasymturnstile}\ A\ {\isasymlongrightarrow}\isactrlsub D\ {\isacharbrackleft}{\isacharhash}\ p{\isacharbrackright}{\isacharparenleft}{\isasymlambda}x{\isachardot}\ B\ x\ {\isasymand}\isactrlsub D\ C\ x{\isacharparenright}{\isachardoublequote}\isanewline
\ \ \isamarkupfalse%
\isacommand{show}\ {\isachardoublequote}{\isasymturnstile}\ A\ {\isasymlongrightarrow}\isactrlsub D\ {\isacharbrackleft}{\isacharhash}\ p{\isacharbrackright}{\isacharparenleft}{\isasymlambda}x{\isachardot}\ B\ x\ {\isasymand}\isactrlsub D\ C\ x{\isacharparenright}{\isachardoublequote}\ \isamarkupfalse%
\isacommand{{\isachardot}}\isanewline
\isamarkupfalse%
\isacommand{next}\isanewline
\ \ \isamarkupfalse%
\isacommand{assume}\ {\isachardoublequote}{\isasymturnstile}\ A\ {\isasymlongrightarrow}\isactrlsub D\ {\isacharbrackleft}{\isacharhash}\ p{\isacharbrackright}{\isacharparenleft}{\isasymlambda}x{\isachardot}\ B\ x\ {\isasymand}\isactrlsub D\ C\ x{\isacharparenright}{\isachardoublequote}\isanewline
\ \ \isamarkupfalse%
\isacommand{show}\ {\isachardoublequote}{\isasymforall}x{\isachardot}\ {\isasymturnstile}\ B\ x\ {\isasymand}\isactrlsub D\ C\ x\ {\isasymlongrightarrow}\isactrlsub D\ C\ x{\isachardoublequote}\isanewline
\ \ \isamarkupfalse%
\isacommand{proof}\ \isanewline
\ \ \ \ \isamarkupfalse%
\isacommand{fix}\ x\ \isamarkupfalse%
\isacommand{show}\ {\isachardoublequote}{\isasymturnstile}\ B\ x\ {\isasymand}\isactrlsub D\ C\ x\ {\isasymlongrightarrow}\isactrlsub D\ C\ x{\isachardoublequote}\ \isamarkupfalse%
\isacommand{by}\ {\isacharparenleft}simp\ add{\isacharcolon}\ pdl{\isacharunderscore}taut{\isacharparenright}\isanewline
\ \ \isamarkupfalse%
\isacommand{qed}\isanewline
\isamarkupfalse%
\isacommand{qed}\isamarkupfalse%
%
\begin{isamarkuptext}%
The following lemmas show how one can split and join boxes freely with the help
  of axiom \isa{pdl{\isacharunderscore}seqB{\isacharunderscore}simp}.%
\end{isamarkuptext}%
\isamarkuptrue%
\isacommand{lemma}\ pdl{\isacharunderscore}imp{\isacharunderscore}id{\isacharcolon}\ {\isachardoublequote}{\isasymturnstile}\ A\ {\isasymlongrightarrow}\isactrlsub D\ A{\isachardoublequote}\isanewline
\ \ \isamarkupfalse%
\isacommand{by}\ {\isacharparenleft}simp\ add{\isacharcolon}\ pdl{\isacharunderscore}taut{\isacharparenright}\isanewline
\isanewline
\isamarkupfalse%
\isacommand{lemma}\ {\isachardoublequote}{\isasymturnstile}\ {\isacharbrackleft}{\isacharhash}\ do\ {\isacharbraceleft}x{\isadigit{1}}{\isasymleftarrow}p{\isadigit{1}}{\isacharsemicolon}\ x{\isadigit{2}}{\isasymleftarrow}p{\isadigit{2}}{\isacharsemicolon}\ x{\isadigit{3}}{\isasymleftarrow}p{\isadigit{3}}{\isacharsemicolon}\ r\ x{\isadigit{1}}\ x{\isadigit{2}}\ x{\isadigit{3}}{\isacharbraceright}{\isacharbrackright}P\ {\isasymlongrightarrow}\isactrlsub D\isanewline
\ \ \ \ \ \ \ \ \ {\isacharbrackleft}{\isacharhash}\ x{\isadigit{1}}{\isasymleftarrow}p{\isadigit{1}}{\isacharbrackright}{\isacharbrackleft}{\isacharhash}\ x{\isadigit{2}}{\isasymleftarrow}p{\isadigit{2}}{\isacharbrackright}{\isacharbrackleft}{\isacharhash}\ x{\isadigit{3}}{\isasymleftarrow}p{\isadigit{3}}{\isacharbrackright}{\isacharbrackleft}{\isacharhash}\ r\ x{\isadigit{1}}\ x{\isadigit{2}}\ x{\isadigit{3}}{\isacharbrackright}P{\isachardoublequote}\isanewline
\ \ \isamarkupfalse%
\isacommand{apply}{\isacharparenleft}rule\ pdl{\isacharunderscore}imp{\isacharunderscore}trans{\isacharcomma}\ rule\ pdl{\isacharunderscore}iffD{\isadigit{2}}{\isacharbrackleft}OF\ pdl{\isacharunderscore}seqB{\isacharunderscore}simp{\isacharbrackright}{\isacharcomma}\ rule\ pdl{\isacharunderscore}box{\isacharunderscore}reg\ {\isacharcomma}rule\ allI{\isacharparenright}{\isacharplus}\isanewline
\ \ \isamarkupfalse%
\isacommand{by}\ {\isacharparenleft}simp\ add{\isacharcolon}\ pdl{\isacharunderscore}taut{\isacharparenright}\isanewline
\isanewline
\isanewline
\isamarkupfalse%
\isacommand{lemma}\ {\isachardoublequote}{\isasymturnstile}\ {\isacharbrackleft}{\isacharhash}\ x{\isadigit{1}}{\isasymleftarrow}p{\isadigit{1}}{\isacharbrackright}{\isacharbrackleft}{\isacharhash}\ x{\isadigit{2}}{\isasymleftarrow}p{\isadigit{2}}{\isacharbrackright}{\isacharbrackleft}{\isacharhash}\ x{\isadigit{3}}{\isasymleftarrow}p{\isadigit{3}}{\isacharbrackright}{\isacharbrackleft}{\isacharhash}\ x{\isadigit{4}}{\isasymleftarrow}p{\isadigit{4}}{\isacharbrackright}{\isacharbrackleft}{\isacharhash}\ r\ x{\isadigit{1}}\ x{\isadigit{2}}\ x{\isadigit{3}}\ x{\isadigit{4}}{\isacharbrackright}P\ {\isasymlongrightarrow}\isactrlsub D\isanewline
\ \ \ \ \ \ \ \ \ \ {\isacharbrackleft}{\isacharhash}\ do\ {\isacharbraceleft}x{\isadigit{1}}{\isasymleftarrow}p{\isadigit{1}}{\isacharsemicolon}\ x{\isadigit{2}}{\isasymleftarrow}p{\isadigit{2}}{\isacharsemicolon}\ x{\isadigit{3}}{\isasymleftarrow}p{\isadigit{3}}{\isacharsemicolon}\ x{\isadigit{4}}{\isasymleftarrow}p{\isadigit{4}}{\isacharsemicolon}\ r\ x{\isadigit{1}}\ x{\isadigit{2}}\ x{\isadigit{3}}\ x{\isadigit{4}}{\isacharbraceright}{\isacharbrackright}P{\isachardoublequote}\isanewline
\ \ \isamarkupfalse%
\isacommand{apply}{\isacharparenleft}rule\ pdl{\isacharunderscore}plugB{\isacharunderscore}lifted{\isadigit{1}}{\isacharcomma}\ rule\ pdl{\isacharunderscore}imp{\isacharunderscore}id{\isacharcomma}\ rule\ allI{\isacharparenright}{\isacharplus}\isanewline
\ \ \isamarkupfalse%
\isacommand{by}\ {\isacharparenleft}simp\ add{\isacharcolon}\ pdl{\isacharunderscore}taut{\isacharparenright}\isanewline
\isanewline
\isamarkupfalse%
\isamarkupfalse%
\isamarkupfalse%
\isamarkupfalse%
\isamarkupfalse%
\isamarkupfalse%
\isamarkupfalse%
\isamarkupfalse%
\isamarkupfalse%
\isamarkupfalse%
\isamarkupfalse%
\isamarkupfalse%
\isamarkupfalse%
\isamarkupfalse%
\isamarkupfalse%
\isamarkupfalse%
\isamarkupfalse%
\isamarkupfalse%
\isamarkupfalse%
\isamarkupfalse%
\isamarkupfalse%
\isamarkupfalse%
\isamarkupfalse%
\isamarkupfalse%
\isamarkupfalse%
\isamarkupfalse%
\isamarkuptrue%
\isamarkupfalse%
\isamarkupfalse%
\isamarkupfalse%
\isamarkupfalse%
\isamarkupfalse%
\isamarkupfalse%
\isamarkupfalse%
%
\isamarkupsubsection{Examples%
}
\isamarkuptrue%
%
\begin{isamarkuptext}%
Examples from \cite[Theorem 6]{HarelKozen02}.
  \label{isa:harel-kozen}%
\end{isamarkuptext}%
\isamarkuptrue%
\isacommand{lemma}\ {\isachardoublequote}{\isasymturnstile}\ {\isasymlangle}x{\isasymleftarrow}p{\isasymrangle}{\isacharparenleft}P\ x{\isacharparenright}\ {\isasymor}\isactrlsub D\ {\isasymlangle}x{\isasymleftarrow}p{\isasymrangle}{\isacharparenleft}Q\ x{\isacharparenright}\ {\isasymlongrightarrow}\isactrlsub D\ {\isasymlangle}x{\isasymleftarrow}p{\isasymrangle}{\isacharparenleft}P\ x\ {\isasymor}\isactrlsub D\ Q\ x{\isacharparenright}{\isachardoublequote}\isanewline
\isamarkupfalse%
\isacommand{proof}\ {\isacharminus}\isanewline
\ \ \isamarkupfalse%
\isacommand{have}\ {\isachardoublequote}\ {\isasymforall}x{\isachardot}\ {\isasymturnstile}\ P\ x\ {\isasymlongrightarrow}\isactrlsub D\ P\ x\ \ {\isasymor}\isactrlsub D\ Q\ x{\isachardoublequote}\ \isamarkupfalse%
\isacommand{by}\ {\isacharparenleft}simp\ add{\isacharcolon}\ pdl{\isacharunderscore}taut{\isacharparenright}\ \isanewline
\ \ \isamarkupfalse%
\isacommand{hence}\ a{\isadigit{1}}{\isacharcolon}\ {\isachardoublequote}{\isasymturnstile}\ {\isasymlangle}x{\isasymleftarrow}p{\isasymrangle}{\isacharparenleft}P\ x{\isacharparenright}\ {\isasymlongrightarrow}\isactrlsub D\ {\isasymlangle}x{\isasymleftarrow}p{\isasymrangle}{\isacharparenleft}P\ x\ \ {\isasymor}\isactrlsub D\ Q\ x{\isacharparenright}{\isachardoublequote}\ \isamarkupfalse%
\isacommand{by}\ {\isacharparenleft}rule\ pdl{\isacharunderscore}dmd{\isacharunderscore}reg{\isacharparenright}\isanewline
\ \ \isamarkupfalse%
\isacommand{have}\ {\isachardoublequote}\ {\isasymforall}x{\isachardot}\ {\isasymturnstile}\ Q\ x\ {\isasymlongrightarrow}\isactrlsub D\ P\ x\ \ {\isasymor}\isactrlsub D\ Q\ x{\isachardoublequote}\ \isamarkupfalse%
\isacommand{by}\ {\isacharparenleft}simp\ add{\isacharcolon}\ pdl{\isacharunderscore}taut{\isacharparenright}\isanewline
\ \ \isamarkupfalse%
\isacommand{hence}\ a{\isadigit{2}}{\isacharcolon}\ {\isachardoublequote}{\isasymturnstile}\ {\isasymlangle}x{\isasymleftarrow}p{\isasymrangle}{\isacharparenleft}Q\ x{\isacharparenright}\ {\isasymlongrightarrow}\isactrlsub D\ {\isasymlangle}x{\isasymleftarrow}p{\isasymrangle}{\isacharparenleft}P\ x\ \ {\isasymor}\isactrlsub D\ Q\ x{\isacharparenright}{\isachardoublequote}\ \isamarkupfalse%
\isacommand{by}\ {\isacharparenleft}rule\ pdl{\isacharunderscore}dmd{\isacharunderscore}reg{\isacharparenright}\isanewline
\ \ \isamarkupfalse%
\isacommand{let}\ {\isacharquery}P\ {\isacharequal}\ {\isachardoublequote}{\isasymlangle}x{\isasymleftarrow}p{\isasymrangle}{\isacharparenleft}P\ x{\isacharparenright}{\isachardoublequote}\ \isakeyword{and}\ {\isacharquery}Q\ {\isacharequal}\ {\isachardoublequote}{\isasymlangle}x{\isasymleftarrow}p{\isasymrangle}{\isacharparenleft}Q\ x{\isacharparenright}{\isachardoublequote}\ \isakeyword{and}\ {\isacharquery}PQ\ {\isacharequal}\ {\isachardoublequote}{\isasymlangle}x{\isasymleftarrow}p{\isasymrangle}{\isacharparenleft}P\ x\ {\isasymor}\isactrlsub D\ Q\ x{\isacharparenright}{\isachardoublequote}\isanewline
\ \ \isamarkupfalse%
\isacommand{have}\ {\isachardoublequote}{\isasymturnstile}\ {\isacharparenleft}{\isacharquery}P\ {\isasymlongrightarrow}\isactrlsub D\ {\isacharquery}PQ{\isacharparenright}\ {\isasymlongrightarrow}\isactrlsub D\ {\isacharparenleft}{\isacharquery}Q\ {\isasymlongrightarrow}\isactrlsub D\ {\isacharquery}PQ{\isacharparenright}\ {\isasymlongrightarrow}\isactrlsub D\ {\isacharparenleft}{\isacharquery}P\ \ {\isasymor}\isactrlsub D\ {\isacharquery}Q\ {\isasymlongrightarrow}\isactrlsub D\ {\isacharquery}PQ{\isacharparenright}{\isachardoublequote}\isanewline
\ \ \ \ \isamarkupfalse%
\isacommand{by}\ {\isacharparenleft}simp\ only{\isacharcolon}\ pdl{\isacharunderscore}taut\ Valid{\isacharunderscore}Ret{\isacharparenright}\isanewline
\ \ \isamarkupfalse%
\isacommand{from}\ this\ a{\isadigit{1}}\ \isamarkupfalse%
\isacommand{have}\ {\isachardoublequote}{\isasymturnstile}\ {\isacharparenleft}{\isacharquery}Q\ {\isasymlongrightarrow}\isactrlsub D\ {\isacharquery}PQ{\isacharparenright}\ {\isasymlongrightarrow}\isactrlsub D\ {\isacharparenleft}{\isacharquery}P\ \ {\isasymor}\isactrlsub D\ {\isacharquery}Q\ {\isasymlongrightarrow}\isactrlsub D\ {\isacharquery}PQ{\isacharparenright}{\isachardoublequote}\ \isamarkupfalse%
\isacommand{by}\ {\isacharparenleft}rule\ pdl{\isacharunderscore}mp{\isacharparenright}\isanewline
\ \ \isamarkupfalse%
\isacommand{from}\ this\ a{\isadigit{2}}\ \isamarkupfalse%
\isacommand{show}\ {\isacharquery}thesis\ \ \isamarkupfalse%
\isacommand{by}\ {\isacharparenleft}rule\ pdl{\isacharunderscore}mp{\isacharparenright}\isanewline
\isamarkupfalse%
\isacommand{qed}\isanewline
\isanewline
\isamarkupfalse%
\isacommand{lemma}\ {\isachardoublequote}{\isasymturnstile}\ {\isasymlangle}x{\isasymleftarrow}p{\isasymrangle}{\isacharparenleft}P\ x{\isacharparenright}\ {\isasymand}\isactrlsub D\ {\isacharbrackleft}{\isacharhash}\ x{\isasymleftarrow}p{\isacharbrackright}{\isacharparenleft}Q\ x{\isacharparenright}\ \ {\isasymlongrightarrow}\isactrlsub D\ {\isasymlangle}x{\isasymleftarrow}p{\isasymrangle}{\isacharparenleft}P\ x\ {\isasymand}\isactrlsub D\ Q\ x{\isacharparenright}{\isachardoublequote}\isanewline
\isamarkupfalse%
\isacommand{proof}\ {\isacharminus}\isanewline
\ \ \isamarkupfalse%
\isacommand{have}\ {\isachardoublequote}\ {\isasymforall}x{\isachardot}\ \ {\isasymturnstile}\ Q\ x\ {\isasymlongrightarrow}\isactrlsub D\ P\ x\ {\isasymlongrightarrow}\isactrlsub D\ P\ x\ {\isasymand}\isactrlsub D\ Q\ x{\isachardoublequote}\ \isamarkupfalse%
\isacommand{by}\ {\isacharparenleft}simp\ add{\isacharcolon}\ pdl{\isacharunderscore}taut{\isacharparenright}\isanewline
\ \ \isamarkupfalse%
\isacommand{hence}\ {\isachardoublequote}{\isasymturnstile}\ {\isacharbrackleft}{\isacharhash}\ x{\isasymleftarrow}p{\isacharbrackright}{\isacharparenleft}Q\ x{\isacharparenright}\ \ {\isasymlongrightarrow}\isactrlsub D\ {\isacharbrackleft}{\isacharhash}\ x{\isasymleftarrow}p{\isacharbrackright}{\isacharparenleft}P\ x\ \ {\isasymlongrightarrow}\isactrlsub D\ P\ x\ {\isasymand}\isactrlsub D\ Q\ x{\isacharparenright}{\isachardoublequote}\isanewline
\ \ \ \ \isamarkupfalse%
\isacommand{by}\ {\isacharparenleft}rule\ pdl{\isacharunderscore}box{\isacharunderscore}reg{\isacharparenright}\isanewline
\ \ \isamarkupfalse%
\isacommand{moreover}\ \isamarkupfalse%
\isacommand{have}\ {\isachardoublequote}{\isasymturnstile}\ {\isacharbrackleft}{\isacharhash}\ x{\isasymleftarrow}p{\isacharbrackright}{\isacharparenleft}P\ x\ {\isasymlongrightarrow}\isactrlsub D\ P\ x\ {\isasymand}\isactrlsub D\ Q\ x{\isacharparenright}\ {\isasymlongrightarrow}\isactrlsub D\ {\isasymlangle}x{\isasymleftarrow}p{\isasymrangle}{\isacharparenleft}P\ x{\isacharparenright}\ {\isasymlongrightarrow}\isactrlsub D\ {\isasymlangle}x{\isasymleftarrow}p{\isasymrangle}{\isacharparenleft}P\ x\ {\isasymand}\isactrlsub D\ Q\ x{\isacharparenright}{\isachardoublequote}\isanewline
\ \ \ \ \isamarkupfalse%
\isacommand{by}\ {\isacharparenleft}rule\ pdl{\isacharunderscore}k{\isadigit{2}}{\isacharparenright}\isanewline
\ \ \isamarkupfalse%
\isacommand{ultimately}\ \isamarkupfalse%
\isacommand{have}\ {\isachardoublequote}{\isasymturnstile}\ {\isacharbrackleft}{\isacharhash}\ x{\isasymleftarrow}p{\isacharbrackright}{\isacharparenleft}Q\ x{\isacharparenright}\ {\isasymlongrightarrow}\isactrlsub D\ {\isasymlangle}x{\isasymleftarrow}p{\isasymrangle}{\isacharparenleft}P\ x{\isacharparenright}\ {\isasymlongrightarrow}\isactrlsub D\ {\isasymlangle}x{\isasymleftarrow}p{\isasymrangle}{\isacharparenleft}P\ x\ {\isasymand}\isactrlsub D\ Q\ x{\isacharparenright}{\isachardoublequote}\isanewline
\ \ \ \ \isamarkupfalse%
\isacommand{by}\ {\isacharparenleft}rule\ pdl{\isacharunderscore}imp{\isacharunderscore}trans{\isacharparenright}\ \ %
\isamarkupcmt{transitivity of implication%
}
\isanewline
\ \ \isamarkupfalse%
\isacommand{thus}\ {\isacharquery}thesis\ \isamarkupfalse%
\isacommand{by}\ {\isacharparenleft}simp\ only{\isacharcolon}\ pdl{\isacharunderscore}taut{\isacharparenright}\isanewline
\isamarkupfalse%
\isacommand{qed}\isanewline
\isanewline
\isamarkupfalse%
\isacommand{lemma}\ pdl{\isacharunderscore}conj{\isacharunderscore}dmd{\isacharcolon}\ {\isachardoublequote}{\isasymturnstile}\ {\isasymlangle}x{\isasymleftarrow}p{\isasymrangle}{\isacharparenleft}P\ x\ {\isasymand}\isactrlsub D\ Q\ x{\isacharparenright}\ {\isasymlongrightarrow}\isactrlsub D\ {\isasymlangle}x{\isasymleftarrow}p{\isasymrangle}{\isacharparenleft}P\ x{\isacharparenright}\ {\isasymand}\isactrlsub D\ {\isasymlangle}x{\isasymleftarrow}p{\isasymrangle}{\isacharparenleft}Q\ x{\isacharparenright}{\isachardoublequote}\isanewline
\isamarkupfalse%
\isacommand{proof}\ {\isacharminus}\isanewline
\ \ %
\isamarkupcmt{first proving the `P-part'%
}
\isanewline
\ \ \isamarkupfalse%
\isacommand{have}\ dp{\isacharcolon}\ {\isachardoublequote}{\isasymturnstile}\ {\isasymlangle}x{\isasymleftarrow}p{\isasymrangle}{\isacharparenleft}P\ x\ {\isasymand}\isactrlsub D\ Q\ x{\isacharparenright}\ {\isasymlongrightarrow}\isactrlsub D\ {\isasymlangle}x{\isasymleftarrow}p{\isasymrangle}{\isacharparenleft}P\ x{\isacharparenright}{\isachardoublequote}\isanewline
\ \ \isamarkupfalse%
\isacommand{proof}\ {\isacharminus}\isanewline
\ \ \ \ \isamarkupfalse%
\isacommand{have}\ fa{\isacharcolon}\ {\isachardoublequote}{\isasymforall}x{\isachardot}\ {\isasymturnstile}\ P\ x\ {\isasymand}\isactrlsub D\ Q\ x\ {\isasymlongrightarrow}\isactrlsub D\ P\ x{\isachardoublequote}\ \isamarkupfalse%
\isacommand{by}\ {\isacharparenleft}simp\ add{\isacharcolon}\ pdl{\isacharunderscore}taut{\isacharparenright}\isanewline
\ \ \ \ \isamarkupfalse%
\isacommand{thus}\ {\isacharquery}thesis\isanewline
\ \ \ \ \isamarkupfalse%
\isacommand{proof}\ {\isacharminus}\ \isanewline
\ \ \ \ \ \ \isamarkupfalse%
\isacommand{assume}\ {\isachardoublequote}{\isasymforall}x{\isachardot}\ {\isasymturnstile}\ P\ x\ {\isasymand}\isactrlsub D\ Q\ x\ {\isasymlongrightarrow}\isactrlsub D\ P\ x{\isachardoublequote}\isanewline
\ \ \ \ \ \ \isamarkupfalse%
\isacommand{thus}\ {\isachardoublequote}{\isasymturnstile}\ {\isasymlangle}x{\isasymleftarrow}p{\isasymrangle}{\isacharparenleft}P\ x\ {\isasymand}\isactrlsub D\ Q\ x{\isacharparenright}\ {\isasymlongrightarrow}\isactrlsub D\ {\isasymlangle}x{\isasymleftarrow}p{\isasymrangle}{\isacharparenleft}P\ x{\isacharparenright}{\isachardoublequote}\ \isamarkupfalse%
\isacommand{by}\ {\isacharparenleft}rule\ pdl{\isacharunderscore}dmd{\isacharunderscore}reg{\isacharparenright}\isanewline
\ \ \ \ \isamarkupfalse%
\isacommand{qed}\isanewline
\ \ \isamarkupfalse%
\isacommand{qed}\isanewline
\ \ %
\isamarkupcmt{the same for Q%
}
\isanewline
\ \ \isamarkupfalse%
\isacommand{moreover}\ \isanewline
\ \ \isamarkupfalse%
\isacommand{have}\ dq{\isacharcolon}\ {\isachardoublequote}{\isasymturnstile}\ {\isasymlangle}x{\isasymleftarrow}p{\isasymrangle}{\isacharparenleft}P\ x\ {\isasymand}\isactrlsub D\ Q\ x{\isacharparenright}\ {\isasymlongrightarrow}\isactrlsub D\ {\isasymlangle}x{\isasymleftarrow}p{\isasymrangle}{\isacharparenleft}Q\ x{\isacharparenright}{\isachardoublequote}\isanewline
\ \ \isamarkupfalse%
\isacommand{proof}\ {\isacharminus}\isanewline
\ \ \ \ \isamarkupfalse%
\isacommand{have}\ fa{\isacharcolon}\ {\isachardoublequote}{\isasymforall}x{\isachardot}\ {\isasymturnstile}\ P\ x\ {\isasymand}\isactrlsub D\ Q\ x\ {\isasymlongrightarrow}\isactrlsub D\ Q\ x{\isachardoublequote}\ \ \isamarkupfalse%
\isacommand{by}\ {\isacharparenleft}simp\ add{\isacharcolon}\ pdl{\isacharunderscore}taut{\isacharparenright}\isanewline
\ \ \ \ \isamarkupfalse%
\isacommand{thus}\ {\isacharquery}thesis\isanewline
\ \ \ \ \isamarkupfalse%
\isacommand{proof}\ {\isacharminus}\ \isanewline
\ \ \ \ \ \ \isamarkupfalse%
\isacommand{assume}\ {\isachardoublequote}{\isasymforall}x{\isachardot}\ {\isasymturnstile}\ P\ x\ {\isasymand}\isactrlsub D\ Q\ x\ {\isasymlongrightarrow}\isactrlsub D\ Q\ x{\isachardoublequote}\isanewline
\ \ \ \ \ \ \isamarkupfalse%
\isacommand{thus}\ {\isachardoublequote}{\isasymturnstile}\ {\isasymlangle}x{\isasymleftarrow}p{\isasymrangle}{\isacharparenleft}P\ x\ {\isasymand}\isactrlsub D\ Q\ x{\isacharparenright}\ {\isasymlongrightarrow}\isactrlsub D\ {\isasymlangle}x{\isasymleftarrow}p{\isasymrangle}{\isacharparenleft}Q\ x{\isacharparenright}{\isachardoublequote}\ \isamarkupfalse%
\isacommand{by}\ {\isacharparenleft}rule\ pdl{\isacharunderscore}dmd{\isacharunderscore}reg{\isacharparenright}\isanewline
\ \ \ \ \isamarkupfalse%
\isacommand{qed}\isanewline
\ \ \isamarkupfalse%
\isacommand{qed}\isanewline
\ \ %
\isamarkupcmt{Now assemble the results to arrive at the main thesis%
}
\isanewline
\ \ \isamarkupfalse%
\isacommand{ultimately}\ \isamarkupfalse%
\isacommand{show}\ {\isacharquery}thesis\ \isamarkupfalse%
\isacommand{by}\ {\isacharparenleft}rule\ \ pdl{\isacharunderscore}conjI{\isacharunderscore}lifted{\isacharparenright}\isanewline
\isamarkupfalse%
\isacommand{qed}\isanewline
\isanewline
\isamarkupfalse%
\isacommand{lemma}\ {\isachardoublequote}{\isasymturnstile}\ {\isacharbrackleft}{\isacharhash}\ x{\isasymleftarrow}p{\isacharbrackright}{\isacharparenleft}P\ x{\isacharparenright}\ {\isasymor}\isactrlsub D\ {\isacharbrackleft}{\isacharhash}\ x{\isasymleftarrow}p{\isacharbrackright}{\isacharparenleft}Q\ x{\isacharparenright}\ {\isasymlongrightarrow}\isactrlsub D\ {\isacharbrackleft}{\isacharhash}\ x{\isasymleftarrow}p{\isacharbrackright}{\isacharparenleft}P\ x\ {\isasymor}\isactrlsub D\ Q\ x{\isacharparenright}{\isachardoublequote}\isanewline
\isamarkupfalse%
\isacommand{proof}\ {\isacharminus}\isanewline
\ \ \isamarkupfalse%
\isacommand{have}\ {\isachardoublequote}\ {\isasymforall}x{\isachardot}\ {\isasymturnstile}\ P\ x\ {\isasymlongrightarrow}\isactrlsub D\ P\ x\ \ {\isasymor}\isactrlsub D\ Q\ x{\isachardoublequote}\ \ \isamarkupfalse%
\isacommand{by}\ {\isacharparenleft}simp\ add{\isacharcolon}\ pdl{\isacharunderscore}taut{\isacharparenright}\isanewline
\ \ \isamarkupfalse%
\isacommand{hence}\ a{\isadigit{1}}{\isacharcolon}\ {\isachardoublequote}{\isasymturnstile}\ {\isacharbrackleft}{\isacharhash}\ x{\isasymleftarrow}p{\isacharbrackright}{\isacharparenleft}P\ x{\isacharparenright}\ {\isasymlongrightarrow}\isactrlsub D\ {\isacharbrackleft}{\isacharhash}\ x{\isasymleftarrow}p{\isacharbrackright}{\isacharparenleft}P\ x\ \ {\isasymor}\isactrlsub D\ Q\ x{\isacharparenright}{\isachardoublequote}\ \isamarkupfalse%
\isacommand{by}\ {\isacharparenleft}rule\ pdl{\isacharunderscore}box{\isacharunderscore}reg{\isacharparenright}\isanewline
\ \ \isamarkupfalse%
\isacommand{have}\ {\isachardoublequote}\ {\isasymforall}x{\isachardot}\ {\isasymturnstile}\ Q\ x\ {\isasymlongrightarrow}\isactrlsub D\ P\ x\ \ {\isasymor}\isactrlsub D\ Q\ x{\isachardoublequote}\ \isamarkupfalse%
\isacommand{by}\ {\isacharparenleft}simp\ add{\isacharcolon}\ pdl{\isacharunderscore}taut{\isacharparenright}\isanewline
\ \ \isamarkupfalse%
\isacommand{hence}\ a{\isadigit{2}}{\isacharcolon}\ {\isachardoublequote}{\isasymturnstile}\ {\isacharbrackleft}{\isacharhash}\ x{\isasymleftarrow}p{\isacharbrackright}{\isacharparenleft}Q\ x{\isacharparenright}\ {\isasymlongrightarrow}\isactrlsub D\ {\isacharbrackleft}{\isacharhash}\ x{\isasymleftarrow}p{\isacharbrackright}{\isacharparenleft}P\ x\ \ {\isasymor}\isactrlsub D\ Q\ x{\isacharparenright}{\isachardoublequote}\ \isamarkupfalse%
\isacommand{by}\ {\isacharparenleft}rule\ pdl{\isacharunderscore}box{\isacharunderscore}reg{\isacharparenright}\isanewline
\ \ \isamarkupfalse%
\isacommand{let}\ {\isacharquery}P\ {\isacharequal}\ {\isachardoublequote}{\isacharbrackleft}{\isacharhash}\ x{\isasymleftarrow}p{\isacharbrackright}{\isacharparenleft}P\ x{\isacharparenright}{\isachardoublequote}\ \isakeyword{and}\ {\isacharquery}Q\ {\isacharequal}\ {\isachardoublequote}{\isacharbrackleft}{\isacharhash}\ x{\isasymleftarrow}p{\isacharbrackright}{\isacharparenleft}Q\ x{\isacharparenright}{\isachardoublequote}\ \isakeyword{and}\ {\isacharquery}PQ\ {\isacharequal}\ {\isachardoublequote}{\isacharbrackleft}{\isacharhash}\ x{\isasymleftarrow}p{\isacharbrackright}{\isacharparenleft}P\ x\ \ {\isasymor}\isactrlsub D\ Q\ x{\isacharparenright}{\isachardoublequote}\isanewline
\ \ \isamarkupfalse%
\isacommand{have}\ {\isachardoublequote}{\isasymturnstile}\ {\isacharparenleft}{\isacharquery}P\ {\isasymlongrightarrow}\isactrlsub D\ {\isacharquery}PQ{\isacharparenright}\ {\isasymlongrightarrow}\isactrlsub D\ {\isacharparenleft}{\isacharquery}Q\ {\isasymlongrightarrow}\isactrlsub D\ {\isacharquery}PQ{\isacharparenright}\ {\isasymlongrightarrow}\isactrlsub D\ {\isacharparenleft}{\isacharquery}P\ \ {\isasymor}\isactrlsub D\ {\isacharquery}Q\ {\isasymlongrightarrow}\isactrlsub D\ {\isacharquery}PQ{\isacharparenright}{\isachardoublequote}\isanewline
\ \ \ \ \isamarkupfalse%
\isacommand{by}\ {\isacharparenleft}simp\ only{\isacharcolon}\ pdl{\isacharunderscore}taut\ Valid{\isacharunderscore}Ret{\isacharparenright}\isanewline
\ \ \isamarkupfalse%
\isacommand{from}\ this\ a{\isadigit{1}}\ a{\isadigit{2}}\ \isamarkupfalse%
\isacommand{show}\ {\isacharquery}thesis\ \isamarkupfalse%
\isacommand{by}\ {\isacharparenleft}rule\ pdl{\isacharunderscore}mp{\isacharunderscore}{\isadigit{2}}x{\isacharparenright}\isanewline
\isamarkupfalse%
\isacommand{qed}\isanewline
\isanewline
\isamarkupfalse%
\isacommand{end}\isanewline
\isamarkupfalse%
\end{isabellebody}%
%%% Local Variables:
%%% mode: latex
%%% TeX-master: "root"
%%% End:


%
\begin{isabellebody}%
\def\isabellecontext{Parsec}%
%
\isamarkupheader{A Deterministic Parser Monad with Fall Back Alternatives%
}
\isamarkuptrue%
\isacommand{theory}\ Parsec\ {\isacharequal}\ PDL\ {\isacharplus}\ MonEq{\isacharcolon}\isamarkupfalse%
%
\label{sec:parsec-thy}
%
\begin{isamarkuptext}%
In a typical implementation of this parser monad, \isa{T} would have the 
  form \isa{T\ A\ {\isacharequal}\ {\isacharparenleft}S\ {\isasymRightarrow}\ {\isacharparenleft}E\ {\isacharplus}\ A{\isacharparenright}\ {\isasymtimes}\ S{\isacharparenright}}, i.e. it would be a state monad (over states
  $S$) with exceptions of type $E$. The fall back alternative \isa{q} in
  \isa{p{\isasymparallel}q} would then only be used if \isa{p} failed to terminate.
  \label{isa:parsec-spec}%
\end{isamarkuptext}%
\isamarkuptrue%
\isacommand{consts}\isanewline
\ \ item\ \ \ \ \ \ \ {\isacharcolon}{\isacharcolon}\ {\isachardoublequote}nat\ T{\isachardoublequote}\ \ \ \ \ \ %
\isamarkupcmt{Parses exactly one character (natural number)%
}
\isanewline
\ \ fail\ \ \ \ \ \ \ {\isacharcolon}{\isacharcolon}\ {\isachardoublequote}{\isacharprime}a\ T{\isachardoublequote}\ \ \ \ \ \ \ %
\isamarkupcmt{Always fails%
}
\isanewline
\ \ alt\ \ \ \ \ \ \ \ {\isacharcolon}{\isacharcolon}\ {\isachardoublequote}{\isacharprime}a\ T\ {\isasymRightarrow}\ {\isacharprime}a\ T\ {\isasymRightarrow}\ {\isacharprime}a\ T{\isachardoublequote}\ {\isacharparenleft}\isakeyword{infixl}\ {\isachardoublequote}{\isasymparallel}{\isachardoublequote}\ {\isadigit{1}}{\isadigit{4}}{\isadigit{0}}{\isacharparenright}\ %
\isamarkupcmt{Prefer first parser, but fall back on second if necessary%
}
\isanewline
\ \ getInput\ \ \ {\isacharcolon}{\isacharcolon}\ {\isachardoublequote}nat\ list\ T{\isachardoublequote}\ %
\isamarkupcmt{read the current state%
}
\isanewline
\ \ setInput\ \ \ {\isacharcolon}{\isacharcolon}\ {\isachardoublequote}nat\ list\ {\isasymRightarrow}\ unit\ T{\isachardoublequote}\ \isanewline
\isanewline
\isanewline
\isamarkupfalse%
\isacommand{constdefs}\ \isanewline
\ \ eot\ {\isacharcolon}{\isacharcolon}\ {\isachardoublequote}bool\ T{\isachardoublequote}\isanewline
\ \ {\isachardoublequote}eot\ {\isasymequiv}\ {\isacharparenleft}do\ {\isacharbraceleft}i\ {\isasymleftarrow}\ getInput{\isacharsemicolon}\ ret\ {\isacharparenleft}null\ i{\isacharparenright}{\isacharbraceright}{\isacharparenright}{\isachardoublequote}\isanewline
\ \ Eot\ {\isacharcolon}{\isacharcolon}\ {\isachardoublequote}bool\ D{\isachardoublequote}\isanewline
\ \ {\isachardoublequote}Eot\ {\isasymequiv}\ {\isasymUp}\ eot{\isachardoublequote}\isanewline
\ \ GetInput\ {\isacharcolon}{\isacharcolon}\ {\isachardoublequote}nat\ list\ D{\isachardoublequote}\isanewline
\ \ {\isachardoublequote}GetInput\ {\isasymequiv}\ {\isasymUp}\ getInput{\isachardoublequote}\isamarkupfalse%
%
\begin{isamarkuptext}%
\isa{GetInput} and \isa{Eot} are the abstractions in \isa{{\isacharprime}a\ D} of the
resp. lower case terms in \isa{{\isacharprime}a\ T}.%
\end{isamarkuptext}%
\isamarkuptrue%
\isacommand{axioms}\isanewline
\ \ dsef{\isacharunderscore}getInput{\isacharcolon}\ {\isachardoublequote}dsef\ getInput{\isachardoublequote}\isanewline
\ \ fail{\isacharunderscore}bot{\isacharcolon}\ {\isachardoublequote}{\isasymturnstile}\ {\isacharbrackleft}{\isacharhash}\ fail{\isacharbrackright}{\isacharparenleft}{\isasymlambda}x{\isachardot}\ Ret\ False{\isacharparenright}{\isachardoublequote}\isanewline
\ \ eot{\isacharunderscore}item{\isacharcolon}\ {\isachardoublequote}{\isasymturnstile}\ Eot\ {\isasymlongrightarrow}\isactrlsub D\ {\isacharbrackleft}{\isacharhash}\ x{\isasymleftarrow}item{\isacharbrackright}{\isacharparenleft}Ret\ False{\isacharparenright}{\isachardoublequote}\isanewline
\ \ set{\isacharunderscore}get{\isacharcolon}\ \ {\isachardoublequote}{\isasymturnstile}\ {\isasymlangle}setInput\ x{\isasymrangle}{\isacharparenleft}{\isasymlambda}u{\isachardot}\ GetInput\ {\isacharequal}\isactrlsub D\ Ret\ x{\isacharparenright}{\isachardoublequote}\isanewline
\ \ get{\isacharunderscore}item{\isacharcolon}\ {\isachardoublequote}{\isasymturnstile}\ GetInput\ {\isacharequal}\isactrlsub D\ Ret\ {\isacharparenleft}y{\isacharhash}ys{\isacharparenright}\ {\isasymlongrightarrow}\isactrlsub D\ {\isasymlangle}x{\isasymleftarrow}item{\isasymrangle}{\isacharparenleft}Ret\ {\isacharparenleft}x\ {\isacharequal}\ y{\isacharparenright}\ {\isasymand}\isactrlsub D\ GetInput\ {\isacharequal}\isactrlsub D\ Ret\ ys{\isacharparenright}{\isachardoublequote}\isanewline
\ \ altB{\isacharunderscore}iff{\isacharcolon}\ {\isachardoublequote}{\isasymturnstile}\ {\isacharbrackleft}{\isacharhash}\ x{\isasymleftarrow}p{\isasymparallel}q{\isacharbrackright}{\isacharparenleft}P\ x{\isacharparenright}\ {\isasymlongleftrightarrow}\isactrlsub D\ {\isacharparenleft}\ {\isacharbrackleft}{\isacharhash}\ x{\isasymleftarrow}p{\isacharbrackright}{\isacharparenleft}P\ x{\isacharparenright}\ {\isasymand}\isactrlsub D\ {\isasymlangle}x{\isasymleftarrow}p{\isasymrangle}{\isacharparenleft}Ret\ True{\isacharparenright}\ {\isacharparenright}\ {\isasymor}\isactrlsub D\ \isanewline
\ \ \ \ \ \ \ \ \ \ \ \ \ \ \ \ \ \ \ \ \ \ \ \ \ \ \ \ \ \ \ \ \ \ \ \ \ {\isacharparenleft}\ {\isacharbrackleft}{\isacharhash}\ x{\isasymleftarrow}q{\isacharbrackright}{\isacharparenleft}P\ x{\isacharparenright}\ {\isasymand}\isactrlsub D\ {\isacharbrackleft}{\isacharhash}\ x{\isasymleftarrow}p{\isacharbrackright}{\isacharparenleft}Ret\ False{\isacharparenright}\ {\isacharparenright}{\isachardoublequote}\isanewline
\ \ altD{\isacharunderscore}iff{\isacharcolon}\ {\isachardoublequote}{\isasymturnstile}\ {\isasymlangle}x{\isasymleftarrow}p{\isasymparallel}q{\isasymrangle}{\isacharparenleft}P\ x{\isacharparenright}\ {\isasymlongleftrightarrow}\isactrlsub D\ {\isasymlangle}x{\isasymleftarrow}p{\isasymrangle}{\isacharparenleft}P\ x{\isacharparenright}\ {\isasymor}\isactrlsub D\ {\isacharparenleft}{\isasymlangle}x{\isasymleftarrow}q{\isasymrangle}{\isacharparenleft}P\ x{\isacharparenright}\ {\isasymand}\isactrlsub D\ {\isacharbrackleft}{\isacharhash}\ x{\isasymleftarrow}p{\isacharbrackright}{\isacharparenleft}Ret\ False{\isacharparenright}{\isacharparenright}{\isachardoublequote}\isanewline
\ \ determ{\isacharcolon}\ \ \ {\isachardoublequote}{\isasymturnstile}\ {\isasymlangle}x{\isasymleftarrow}p{\isasymrangle}{\isacharparenleft}P\ x{\isacharparenright}\ {\isasymlongleftrightarrow}\isactrlsub D\ {\isacharbrackleft}{\isacharhash}\ x{\isasymleftarrow}p{\isacharbrackright}{\isacharparenleft}P\ x{\isacharparenright}\ {\isasymand}\isactrlsub D\ {\isasymlangle}x{\isasymleftarrow}p{\isasymrangle}{\isacharparenleft}Ret\ True{\isacharparenright}{\isachardoublequote}\isamarkupfalse%
%
\begin{isamarkuptext}%
Axiom \isa{{\isachardoublequote}determ{\isachardoublequote}} is the typical relationship between \isa{{\isasymlangle}p{\isasymrangle}P} and \isa{{\isacharbrackleft}{\isacharhash}\ p{\isacharbrackright}P} 
  when no non-determinism is involved. Axioms \isa{{\isachardoublequote}altB{\isacharunderscore}iff{\isachardoublequote}\ {\isachardoublequote}altD{\isacharunderscore}iff{\isachardoublequote}} describe the 
  fall back behaviour of the alternative operation.%
\end{isamarkuptext}%
\isamarkuptrue%
%
\begin{isamarkuptext}%
\isa{dsef\ getInput} implies \isa{dsef\ eot}.%
\end{isamarkuptext}%
\isamarkuptrue%
\isacommand{lemma}\ dsef{\isacharunderscore}eot{\isacharcolon}\ {\isachardoublequote}dsef\ eot{\isachardoublequote}\isanewline
\ \ \isamarkupfalse%
\isacommand{by}\ {\isacharparenleft}simp\ add{\isacharcolon}\ eot{\isacharunderscore}def\ dsef{\isacharunderscore}seq\ dsef{\isacharunderscore}ret\ dsef{\isacharunderscore}getInput{\isacharparenright}\isamarkupfalse%
%
\begin{isamarkuptext}%
Another way to state the properties of alternation (for the diamond operator).%
\end{isamarkuptext}%
\isamarkuptrue%
\isacommand{axioms}\isanewline
altD{\isacharunderscore}left{\isacharcolon}\ {\isachardoublequote}{\isasymturnstile}\ {\isasymlangle}p{\isasymrangle}P\ {\isasymlongrightarrow}\isactrlsub D\ {\isasymlangle}p{\isasymparallel}q{\isasymrangle}P{\isachardoublequote}\isanewline
altD{\isacharunderscore}right{\isacharcolon}\ {\isachardoublequote}{\isasymturnstile}\ {\isasymlangle}q{\isasymrangle}P\ {\isasymlongrightarrow}\isactrlsub D\ {\isasymlangle}p{\isasymrangle}{\isacharparenleft}{\isasymlambda}x{\isachardot}\ Ret\ True{\isacharparenright}\ {\isasymor}\isactrlsub D\ {\isasymlangle}p{\isasymparallel}q{\isasymrangle}P{\isachardoublequote}\isamarkupfalse%
%
\begin{isamarkuptext}%
Proof that \isa{Eot} actually is just an abbreviation.%
\end{isamarkuptext}%
\isamarkuptrue%
\isacommand{lemma}\ Eot{\isacharunderscore}GetInput{\isacharcolon}\ {\isachardoublequote}Eot\ {\isacharequal}\ {\isacharparenleft}GetInput\ {\isacharequal}\isactrlsub D\ Ret\ {\isacharbrackleft}{\isacharbrackright}{\isacharparenright}{\isachardoublequote}\isanewline
\isamarkupfalse%
\isacommand{proof}\ {\isacharminus}\isanewline
\ \ \isamarkupfalse%
\isacommand{have}\ null{\isacharunderscore}eq{\isacharunderscore}nil{\isacharcolon}\ {\isachardoublequote}\ {\isasymforall}x{\isachardot}\ null\ x\ {\isacharequal}\ {\isacharparenleft}x\ {\isacharequal}\ {\isacharbrackleft}{\isacharbrackright}{\isacharparenright}{\isachardoublequote}\isanewline
\ \ \isamarkupfalse%
\isacommand{proof}\isanewline
\ \ \ \ \isamarkupfalse%
\isacommand{fix}\ x\ \isamarkupfalse%
\isacommand{show}\ {\isachardoublequote}null\ x\ {\isacharequal}\ {\isacharparenleft}x\ {\isacharequal}\ {\isacharbrackleft}{\isacharbrackright}{\isacharparenright}{\isachardoublequote}\isanewline
\ \ \ \ \isamarkupfalse%
\isacommand{proof}\ {\isacharparenleft}cases\ x{\isacharparenright}\ \isanewline
\ \ \ \ \ \ \isamarkupfalse%
\isacommand{assume}\ {\isachardoublequote}x\ {\isacharequal}\ {\isacharbrackleft}{\isacharbrackright}{\isachardoublequote}\ \isamarkupfalse%
\isacommand{thus}\ {\isachardoublequote}null\ x\ {\isacharequal}\ {\isacharparenleft}x\ {\isacharequal}\ {\isacharbrackleft}{\isacharbrackright}{\isacharparenright}{\isachardoublequote}\ \isamarkupfalse%
\isacommand{by}\ simp\isanewline
\ \ \ \ \isamarkupfalse%
\isacommand{next}\isanewline
\ \ \ \ \ \ \isamarkupfalse%
\isacommand{fix}\ a\ list\ \isamarkupfalse%
\isacommand{assume}\ {\isachardoublequote}x\ {\isacharequal}\ {\isacharparenleft}a{\isacharhash}list{\isacharparenright}{\isachardoublequote}\ \isamarkupfalse%
\isacommand{thus}\ {\isachardoublequote}null\ x\ {\isacharequal}\ {\isacharparenleft}x\ {\isacharequal}\ {\isacharbrackleft}{\isacharbrackright}{\isacharparenright}{\isachardoublequote}\ \isamarkupfalse%
\isacommand{by}\ simp\isanewline
\ \ \ \ \isamarkupfalse%
\isacommand{qed}\isanewline
\ \ \isamarkupfalse%
\isacommand{qed}\isanewline
\ \ \isamarkupfalse%
\isacommand{show}\ {\isacharquery}thesis\isanewline
\ \ \isamarkupfalse%
\isacommand{by}{\isacharparenleft}simp\ add{\isacharcolon}\ Eot{\isacharunderscore}def\ eot{\isacharunderscore}def\ GetInput{\isacharunderscore}def\ MonEq{\isacharunderscore}def\ liftM{\isadigit{2}}{\isacharunderscore}def\ \isanewline
\ \ \ \ \ \ \ \ \ \ \ \ \ \ \ \ \ \ dsef{\isacharunderscore}getInput\ Abs{\isacharunderscore}Dsef{\isacharunderscore}inverse\ Dsef{\isacharunderscore}def\ Ret{\isacharunderscore}def\ null{\isacharunderscore}eq{\isacharunderscore}nil{\isacharparenright}\isanewline
\isamarkupfalse%
\isacommand{qed}\isanewline
\isanewline
\isamarkupfalse%
\isacommand{lemma}\ GetInput{\isacharunderscore}item{\isacharunderscore}fail{\isacharcolon}\ {\isachardoublequote}{\isasymturnstile}\ GetInput\ {\isacharequal}\isactrlsub D\ Ret\ {\isacharbrackleft}{\isacharbrackright}\ {\isasymlongrightarrow}\isactrlsub D\ {\isacharbrackleft}{\isacharhash}\ item{\isacharbrackright}{\isacharparenleft}{\isasymlambda}x{\isachardot}\ Ret\ False{\isacharparenright}{\isachardoublequote}\isanewline
\ \ \isamarkupfalse%
\isacommand{apply}{\isacharparenleft}rule\ subst{\isacharbrackleft}OF\ Eot{\isacharunderscore}GetInput{\isacharbrackright}{\isacharparenright}\isanewline
\ \ \isamarkupfalse%
\isacommand{by}\ {\isacharparenleft}rule\ eot{\isacharunderscore}item{\isacharparenright}\isamarkupfalse%
%
\begin{isamarkuptext}%
We can show that an alternative parser terminates iff one of its constituent
  parsers does.%
\end{isamarkuptext}%
\isamarkuptrue%
\isacommand{lemma}\ par{\isacharunderscore}term{\isacharcolon}\ {\isachardoublequote}{\isasymturnstile}\ {\isasymlangle}x\ {\isasymleftarrow}\ p{\isasymparallel}q{\isasymrangle}{\isacharparenleft}Ret\ True{\isacharparenright}\ {\isasymlongleftrightarrow}\isactrlsub D\ {\isasymlangle}x{\isasymleftarrow}p{\isasymrangle}{\isacharparenleft}Ret\ True{\isacharparenright}\ {\isasymor}\isactrlsub D\ {\isasymlangle}x{\isasymleftarrow}q{\isasymrangle}{\isacharparenleft}Ret\ True{\isacharparenright}{\isachardoublequote}\isanewline
\isamarkupfalse%
\isacommand{proof}\ {\isacharparenleft}rule\ pdl{\isacharunderscore}iffI{\isacharparenright}\isanewline
\ \ \isamarkupfalse%
\isacommand{have}\ {\isachardoublequote}{\isasymturnstile}\ {\isacharparenleft}\ {\isasymlangle}x{\isasymleftarrow}p{\isasymparallel}q{\isasymrangle}{\isacharparenleft}Ret\ True{\isacharparenright}\ {\isasymlongrightarrow}\isactrlsub D\ {\isasymlangle}x{\isasymleftarrow}p{\isasymrangle}{\isacharparenleft}Ret\ True{\isacharparenright}\ {\isasymor}\isactrlsub D\ {\isasymlangle}x{\isasymleftarrow}q{\isasymrangle}{\isacharparenleft}Ret\ True{\isacharparenright}\ {\isasymand}\isactrlsub D\ {\isacharbrackleft}{\isacharhash}\ x{\isasymleftarrow}p{\isacharbrackright}{\isacharparenleft}Ret\ False{\isacharparenright}\ {\isacharparenright}\ {\isasymlongrightarrow}\isactrlsub D\ \isanewline
\ \ \ \ \ \ \ \ \ \ {\isasymlangle}x{\isasymleftarrow}p{\isasymparallel}q{\isasymrangle}{\isacharparenleft}Ret\ True{\isacharparenright}\ {\isasymlongrightarrow}\isactrlsub D\ {\isasymlangle}x{\isasymleftarrow}p{\isasymrangle}{\isacharparenleft}Ret\ True{\isacharparenright}\ {\isasymor}\isactrlsub D\ {\isasymlangle}x{\isasymleftarrow}q{\isasymrangle}{\isacharparenleft}Ret\ True{\isacharparenright}{\isachardoublequote}\isanewline
\ \ \ \ \isamarkupfalse%
\isacommand{by}\ {\isacharparenleft}simp\ add{\isacharcolon}\ pdl{\isacharunderscore}taut{\isacharparenright}\isanewline
\ \ \isamarkupfalse%
\isacommand{moreover}\ \isamarkupfalse%
\isacommand{note}\ pdl{\isacharunderscore}iffD{\isadigit{1}}{\isacharbrackleft}OF\ altD{\isacharunderscore}iff{\isacharbrackright}\isanewline
\ \ \isamarkupfalse%
\isacommand{ultimately}\ \isamarkupfalse%
\isacommand{show}\ \ {\isachardoublequote}{\isasymturnstile}\ {\isasymlangle}p\ {\isasymparallel}\ q{\isasymrangle}{\isacharparenleft}{\isasymlambda}x{\isachardot}\ Ret\ True{\isacharparenright}\ {\isasymlongrightarrow}\isactrlsub D\ {\isasymlangle}p{\isasymrangle}{\isacharparenleft}{\isasymlambda}x{\isachardot}\ Ret\ True{\isacharparenright}\ {\isasymor}\isactrlsub D\ {\isasymlangle}q{\isasymrangle}{\isacharparenleft}{\isasymlambda}x{\isachardot}\ Ret\ True{\isacharparenright}{\isachardoublequote}\ \isanewline
\ \ \ \ \isamarkupfalse%
\isacommand{by}\ {\isacharparenleft}rule\ pdl{\isacharunderscore}mp{\isacharparenright}\isanewline
\isamarkupfalse%
\isacommand{next}\isanewline
\ \ \isamarkupfalse%
\isacommand{have}\ {\isachardoublequote}{\isasymturnstile}\ {\isacharparenleft}\ {\isasymlangle}x{\isasymleftarrow}p{\isasymrangle}{\isacharparenleft}Ret\ True{\isacharparenright}\ {\isasymor}\isactrlsub D\ {\isasymlangle}x{\isasymleftarrow}q{\isasymrangle}{\isacharparenleft}Ret\ True{\isacharparenright}\ {\isasymand}\isactrlsub D\ {\isacharbrackleft}{\isacharhash}\ x{\isasymleftarrow}p{\isacharbrackright}{\isacharparenleft}Ret\ False{\isacharparenright}\ {\isasymlongrightarrow}\isactrlsub D\ {\isasymlangle}x{\isasymleftarrow}\ p\ {\isasymparallel}\ q{\isasymrangle}{\isacharparenleft}Ret\ True{\isacharparenright}\ {\isacharparenright}\ {\isasymlongrightarrow}\isactrlsub D\ \isanewline
\ \ \ \ \ \ \ \ \ \ {\isacharparenleft}\ {\isacharbrackleft}{\isacharhash}\ x{\isasymleftarrow}p{\isacharbrackright}{\isacharparenleft}Ret\ False{\isacharparenright}\ {\isasymlongleftrightarrow}\isactrlsub D\ \ {\isasymnot}\isactrlsub D\ {\isasymlangle}x{\isasymleftarrow}p{\isasymrangle}{\isacharparenleft}{\isasymnot}\isactrlsub D\ Ret\ False{\isacharparenright}\ {\isacharparenright}\ {\isasymlongrightarrow}\isactrlsub D\ \isanewline
\ \ \ \ \ \ \ \ \ \ \ {\isasymlangle}x{\isasymleftarrow}p{\isasymrangle}{\isacharparenleft}Ret\ True{\isacharparenright}\ {\isasymor}\isactrlsub D\ {\isasymlangle}x{\isasymleftarrow}q{\isasymrangle}{\isacharparenleft}Ret\ True{\isacharparenright}\ {\isasymlongrightarrow}\isactrlsub D\ {\isasymlangle}x{\isasymleftarrow}\ p\ {\isasymparallel}\ q{\isasymrangle}{\isacharparenleft}Ret\ True{\isacharparenright}{\isachardoublequote}\isanewline
\ \ \ \ \isamarkupfalse%
\isacommand{by}\ {\isacharparenleft}simp\ add{\isacharcolon}\ pdl{\isacharunderscore}taut{\isacharparenright}\isanewline
\ \ \isamarkupfalse%
\isacommand{moreover}\ \isanewline
\ \ \isamarkupfalse%
\isacommand{note}\ pdl{\isacharunderscore}iffD{\isadigit{2}}{\isacharbrackleft}OF\ altD{\isacharunderscore}iff{\isacharbrackright}\isanewline
\ \ \isamarkupfalse%
\isacommand{moreover}\ \isanewline
\ \ \isamarkupfalse%
\isacommand{note}\ box{\isacharunderscore}dmd{\isacharunderscore}rel\isanewline
\ \ \isamarkupfalse%
\isacommand{ultimately}\isanewline
\ \ \isamarkupfalse%
\isacommand{show}\ {\isachardoublequote}{\isasymturnstile}\ {\isasymlangle}x{\isasymleftarrow}p{\isasymrangle}{\isacharparenleft}Ret\ True{\isacharparenright}\ {\isasymor}\isactrlsub D\ {\isasymlangle}x{\isasymleftarrow}q{\isasymrangle}{\isacharparenleft}Ret\ True{\isacharparenright}\ {\isasymlongrightarrow}\isactrlsub D\ {\isasymlangle}x{\isasymleftarrow}\ p\ {\isasymparallel}\ q{\isasymrangle}{\isacharparenleft}Ret\ True{\isacharparenright}{\isachardoublequote}\isanewline
\ \ \ \ \isamarkupfalse%
\isacommand{by}\ {\isacharparenleft}rule\ pdl{\isacharunderscore}mp{\isacharunderscore}{\isadigit{2}}x{\isacharparenright}\isanewline
\isamarkupfalse%
\isacommand{qed}\isamarkupfalse%
%
\begin{isamarkuptext}%
The following two lemmas are immediate from the axioms.%
\end{isamarkuptext}%
\isamarkuptrue%
\isacommand{lemma}\ parI{\isadigit{1}}{\isacharcolon}\ {\isachardoublequote}{\isasymturnstile}\ \ {\isacharbrackleft}{\isacharhash}\ x{\isasymleftarrow}p{\isacharbrackright}{\isacharparenleft}P\ x{\isacharparenright}\ {\isasymand}\isactrlsub D\ {\isasymlangle}x{\isasymleftarrow}p{\isasymrangle}{\isacharparenleft}Ret\ True{\isacharparenright}\ {\isasymlongrightarrow}\isactrlsub D\ {\isacharbrackleft}{\isacharhash}\ x{\isasymleftarrow}p{\isasymparallel}q{\isacharbrackright}{\isacharparenleft}P\ x{\isacharparenright}{\isachardoublequote}\isamarkupfalse%
\isamarkupfalse%
\isamarkupfalse%
\isamarkupfalse%
\isamarkupfalse%
\isamarkupfalse%
\isamarkupfalse%
\isamarkupfalse%
\isamarkupfalse%
\isamarkupfalse%
\isanewline
\isanewline
\isamarkupfalse%
\isacommand{lemma}\ parI{\isadigit{2}}{\isacharcolon}\ {\isachardoublequote}{\isasymturnstile}\ {\isacharbrackleft}{\isacharhash}\ x{\isasymleftarrow}p{\isacharbrackright}{\isacharparenleft}Ret\ False{\isacharparenright}\ {\isasymand}\isactrlsub D\ {\isacharbrackleft}{\isacharhash}\ x{\isasymleftarrow}q{\isacharbrackright}{\isacharparenleft}P\ x{\isacharparenright}\ {\isasymlongrightarrow}\isactrlsub D\ {\isacharbrackleft}{\isacharhash}\ x{\isasymleftarrow}\ p{\isasymparallel}q{\isacharbrackright}{\isacharparenleft}P\ x{\isacharparenright}{\isachardoublequote}\isamarkupfalse%
\isamarkupfalse%
\isamarkupfalse%
\isamarkupfalse%
\isamarkupfalse%
\isamarkupfalse%
\isamarkupfalse%
\isamarkupfalse%
\isamarkupfalse%
\isamarkupfalse%
\isamarkupfalse%
%
\isamarkupsubsection{Specifying Simple Parsers in Terms of the Basic Ones%
}
\isamarkuptrue%
%
\label{isa:defined-parsers}
\isacommand{constdefs}\isanewline
\ \ sat\ \ \ \ \ \ \ \ {\isacharcolon}{\isacharcolon}\ {\isachardoublequote}{\isacharparenleft}nat\ {\isasymRightarrow}\ bool{\isacharparenright}\ {\isasymRightarrow}\ nat\ T{\isachardoublequote}\isanewline
\ \ {\isachardoublequote}sat\ p\ {\isasymequiv}\ do\ {\isacharbraceleft}x{\isasymleftarrow}item{\isacharsemicolon}\ if\ p\ x\ then\ ret\ x\ else\ fail{\isacharbraceright}{\isachardoublequote}\isanewline
\ \ digitp\ \ \ \ \ \ \ {\isacharcolon}{\isacharcolon}\ {\isachardoublequote}nat\ T{\isachardoublequote}\isanewline
\ \ {\isachardoublequote}digitp\ {\isasymequiv}\ sat\ {\isacharparenleft}{\isasymlambda}x{\isachardot}\ x\ {\isacharless}\ {\isadigit{1}}{\isadigit{0}}{\isacharparenright}{\isachardoublequote}\isamarkupfalse%
%
\begin{isamarkuptext}%
The intended semantics of \isa{many} is that it maps a parser $p$ into one
  that applies $p$ as often as possible and collects the results (which may be 
  none). \isa{many{\isadigit{1}}} requires at least one successful run of $p$.%
\end{isamarkuptext}%
\isamarkuptrue%
\isacommand{consts}\isanewline
many\ \ {\isacharcolon}{\isacharcolon}\ {\isachardoublequote}{\isacharprime}a\ T\ {\isasymRightarrow}\ {\isacharprime}a\ list\ T{\isachardoublequote}\isanewline
many{\isadigit{1}}\ {\isacharcolon}{\isacharcolon}\ {\isachardoublequote}{\isacharprime}a\ T\ {\isasymRightarrow}\ {\isacharprime}a\ list\ T{\isachardoublequote}\isamarkupfalse%
%
\begin{isamarkuptext}%
We cannot define \isa{many}, since it is not primitive recursive 
  and there is no termination measure. 
  \label{isa:many-unfold}%
\end{isamarkuptext}%
\isamarkuptrue%
\isacommand{axioms}\isanewline
many{\isacharunderscore}unfold{\isacharcolon}\ {\isachardoublequote}many\ p\ {\isacharequal}\ {\isacharparenleft}{\isacharparenleft}do\ {\isacharbraceleft}x\ {\isasymleftarrow}\ p{\isacharsemicolon}\ xs\ {\isasymleftarrow}\ many\ p{\isacharsemicolon}\ ret\ {\isacharparenleft}x{\isacharhash}xs{\isacharparenright}{\isacharbraceright}{\isacharparenright}\ {\isasymparallel}\ ret\ {\isacharbrackleft}{\isacharbrackright}{\isacharparenright}{\isachardoublequote}\isanewline
\isanewline
\isamarkupfalse%
\isacommand{defs}\isanewline
many{\isadigit{1}}{\isacharunderscore}def{\isacharcolon}\ {\isachardoublequote}many{\isadigit{1}}\ p\ {\isasymequiv}\ {\isacharparenleft}do\ {\isacharbraceleft}x\ {\isasymleftarrow}\ p{\isacharsemicolon}\ xs\ {\isasymleftarrow}\ many\ p{\isacharsemicolon}\ ret\ {\isacharparenleft}x{\isacharhash}xs{\isacharparenright}{\isacharbraceright}{\isacharparenright}{\isachardoublequote}\isamarkupfalse%
%
\begin{isamarkuptext}%
This is the most convenient and expressive rule we can hope for at the
   moment.%
\end{isamarkuptext}%
\isamarkuptrue%
\isacommand{lemma}\ many{\isacharunderscore}step{\isacharcolon}\ {\isachardoublequote}{\isasymlbrakk}\ {\isasymturnstile}\ {\isasymlangle}{\isacharparenleft}do\ {\isacharbraceleft}x\ {\isasymleftarrow}\ p{\isacharsemicolon}\ xs\ {\isasymleftarrow}\ many\ p{\isacharsemicolon}\ ret\ {\isacharparenleft}x{\isacharhash}xs{\isacharparenright}{\isacharbraceright}{\isacharparenright}{\isasymrangle}P\ {\isasymor}\isactrlsub D\ \isanewline
\ \ \ \ \ \ \ \ \ \ \ \ \ \ \ \ \ \ \ \ \ \ {\isasymlangle}ret\ {\isacharbrackleft}{\isacharbrackright}{\isasymrangle}P\ {\isasymand}\isactrlsub D\ {\isacharbrackleft}{\isacharhash}\ x{\isasymleftarrow}p{\isacharbrackright}{\isacharparenleft}Ret\ False{\isacharparenright}\ {\isasymrbrakk}\ {\isasymLongrightarrow}\ {\isasymturnstile}\ {\isasymlangle}many\ p{\isasymrangle}P{\isachardoublequote}\isamarkupfalse%
\isamarkupfalse%
\isamarkupfalse%
\isamarkupfalse%
\isamarkupfalse%
\isamarkupfalse%
\isamarkupfalse%
\isamarkupfalse%
\isamarkupfalse%
\isamarkupfalse%
\isamarkupfalse%
\isamarkupfalse%
\isamarkupfalse%
\isamarkupfalse%
\isamarkupfalse%
\isamarkupfalse%
\isamarkupfalse%
\isamarkupfalse%
\isamarkupfalse%
\isamarkupfalse%
\isamarkupfalse%
\isamarkupfalse%
\isamarkupfalse%
\isamarkupfalse%
\isamarkupfalse%
\isamarkupfalse%
\isamarkupfalse%
\isamarkupfalse%
\isamarkupfalse%
\isamarkupfalse%
\isamarkupfalse%
\isanewline
\isamarkupfalse%
\isacommand{constdefs}\isanewline
natp\ {\isacharcolon}{\isacharcolon}\ {\isachardoublequote}nat\ T{\isachardoublequote}\isanewline
{\isachardoublequote}natp\ {\isasymequiv}\ do\ {\isacharbraceleft}ns\ {\isasymleftarrow}\ many{\isadigit{1}}\ digitp{\isacharsemicolon}\ ret\ {\isacharparenleft}foldl\ {\isacharparenleft}{\isasymlambda}r\ n{\isachardot}\ {\isadigit{1}}{\isadigit{0}}\ {\isacharasterisk}\ r\ {\isacharplus}\ n{\isacharparenright}\ {\isadigit{0}}\ ns{\isacharparenright}{\isacharbraceright}{\isachardoublequote}\isamarkupfalse%
%
\begin{isamarkuptext}%
The parser for natural numbers \isa{natp} works on an input stream
  that conists of natural numbers and reads numbers between 0 and 9 (inclusive) until 
  no such number can be read. Then it transforms its result list into a number
  by folding an appropriate function into the list. Of course, one might just as
  well consider an input stream of bounded numbers (e.g. ASCII characters in their
  numeric representation) and then read `0' to `9', but this would not 
  provide any interesting further insight.%
\end{isamarkuptext}%
\isamarkuptrue%
%
\isamarkupsubsection{Auxiliary Lemmas%
}
\isamarkuptrue%
%
\begin{isamarkuptext}%
A convenient rendition of axiom \isa{altD{\isacharunderscore}iff} as a rule.%
\end{isamarkuptext}%
\isamarkuptrue%
\isacommand{lemma}\ altD{\isacharunderscore}iff{\isacharunderscore}lifted{\isadigit{1}}{\isacharcolon}\ {\isachardoublequote}{\isasymlbrakk}{\isasymturnstile}\ A\ {\isasymlongrightarrow}\isactrlsub D\ {\isasymlangle}x{\isasymleftarrow}q{\isasymrangle}{\isacharparenleft}P\ x{\isacharparenright}{\isacharsemicolon}\ {\isasymturnstile}\ A\ {\isasymlongrightarrow}\isactrlsub D\ {\isacharbrackleft}{\isacharhash}\ x{\isasymleftarrow}p{\isacharbrackright}{\isacharparenleft}Ret\ False{\isacharparenright}{\isasymrbrakk}\ {\isasymLongrightarrow}\ {\isasymturnstile}\ A\ {\isasymlongrightarrow}\isactrlsub D\ {\isasymlangle}x{\isasymleftarrow}\ p{\isasymparallel}q{\isasymrangle}{\isacharparenleft}P\ x{\isacharparenright}{\isachardoublequote}\isanewline
\isamarkupfalse%
\isacommand{proof}\ {\isacharminus}\ \isanewline
\ \ \isamarkupfalse%
\isacommand{have}\ {\isachardoublequote}{\isasymturnstile}\ {\isacharparenleft}{\isasymlangle}x{\isasymleftarrow}p{\isasymparallel}q{\isasymrangle}{\isacharparenleft}P\ x{\isacharparenright}\ {\isasymlongleftrightarrow}\isactrlsub D\ {\isasymlangle}x{\isasymleftarrow}p{\isasymrangle}{\isacharparenleft}P\ x{\isacharparenright}\ {\isasymor}\isactrlsub D\ {\isasymlangle}x{\isasymleftarrow}q{\isasymrangle}{\isacharparenleft}P\ x{\isacharparenright}\ {\isasymand}\isactrlsub D\ {\isacharbrackleft}{\isacharhash}\ x{\isasymleftarrow}p{\isacharbrackright}{\isacharparenleft}Ret\ False{\isacharparenright}{\isacharparenright}\ {\isasymlongrightarrow}\isactrlsub D\isanewline
\ \ \ \ \ \ \ \ \ \ {\isacharparenleft}A\ {\isasymlongrightarrow}\isactrlsub D\ {\isasymlangle}x{\isasymleftarrow}q{\isasymrangle}{\isacharparenleft}P\ x{\isacharparenright}{\isacharparenright}\ {\isasymlongrightarrow}\isactrlsub D\ {\isacharparenleft}A\ {\isasymlongrightarrow}\isactrlsub D\ {\isacharbrackleft}{\isacharhash}\ x{\isasymleftarrow}p{\isacharbrackright}{\isacharparenleft}Ret\ False{\isacharparenright}{\isacharparenright}\ {\isasymlongrightarrow}\isactrlsub D\isanewline
\ \ \ \ \ \ \ \ \ \ \ A\ {\isasymlongrightarrow}\isactrlsub D\ {\isasymlangle}x{\isasymleftarrow}\ p{\isasymparallel}q{\isasymrangle}{\isacharparenleft}P\ x{\isacharparenright}{\isachardoublequote}\isanewline
\ \ \ \ \isamarkupfalse%
\isacommand{by}\ {\isacharparenleft}simp\ add{\isacharcolon}\ pdl{\isacharunderscore}taut{\isacharparenright}\isanewline
\ \ \isamarkupfalse%
\isacommand{moreover}\ \isanewline
\ \ \isamarkupfalse%
\isacommand{note}\ altD{\isacharunderscore}iff\isanewline
\ \ \isamarkupfalse%
\isacommand{moreover}\isanewline
\ \ \isamarkupfalse%
\isacommand{assume}\ {\isachardoublequote}{\isasymturnstile}\ A\ {\isasymlongrightarrow}\isactrlsub D\ {\isasymlangle}x{\isasymleftarrow}q{\isasymrangle}{\isacharparenleft}P\ x{\isacharparenright}{\isachardoublequote}\isanewline
\ \ \isamarkupfalse%
\isacommand{moreover}\isanewline
\ \ \isamarkupfalse%
\isacommand{assume}\ {\isachardoublequote}{\isasymturnstile}\ A\ {\isasymlongrightarrow}\isactrlsub D\ {\isacharbrackleft}{\isacharhash}\ x{\isasymleftarrow}p{\isacharbrackright}{\isacharparenleft}Ret\ False{\isacharparenright}{\isachardoublequote}\isanewline
\ \ \isamarkupfalse%
\isacommand{ultimately}\isanewline
\ \ \isamarkupfalse%
\isacommand{show}\ {\isacharquery}thesis\ \isamarkupfalse%
\isacommand{by}\ {\isacharparenleft}rule\ pdl{\isacharunderscore}mp{\isacharunderscore}{\isadigit{3}}x{\isacharparenright}\isanewline
\isamarkupfalse%
\isacommand{qed}\isamarkupfalse%
%
\begin{isamarkuptext}%
The correctness of \isa{natp} obviously relies on the correctness of \isa{digitp}, 
  which is proved first.%
\end{isamarkuptext}%
\isamarkuptrue%
\isacommand{theorem}\ digitp{\isacharunderscore}nat{\isacharcolon}\ {\isachardoublequote}{\isasymturnstile}\ GetInput\ {\isacharequal}\isactrlsub D\ Ret\ {\isacharparenleft}{\isadigit{1}}{\isacharhash}ys{\isacharparenright}\ {\isasymlongrightarrow}\isactrlsub D\ {\isasymlangle}x{\isasymleftarrow}digitp{\isasymrangle}{\isacharparenleft}Ret\ {\isacharparenleft}x\ {\isacharequal}\ {\isadigit{1}}{\isacharparenright}\ {\isasymand}\isactrlsub D\ GetInput\ {\isacharequal}\isactrlsub D\ Ret\ ys{\isacharparenright}{\isachardoublequote}\isanewline
\ \ {\isacharparenleft}\isakeyword{is}\ {\isachardoublequote}{\isasymturnstile}\ {\isacharquery}A\ {\isasymlongrightarrow}\isactrlsub D\ {\isasymlangle}digitp{\isasymrangle}{\isacharparenleft}{\isasymlambda}x{\isachardot}\ {\isacharquery}C\ x\ {\isasymand}\isactrlsub D\ {\isacharquery}D{\isacharparenright}{\isachardoublequote}{\isacharparenright}\isanewline
\ \ \isamarkupfalse%
\isacommand{apply}{\isacharparenleft}unfold\ digitp{\isacharunderscore}def\ sat{\isacharunderscore}def{\isacharparenright}\isanewline
\ \ \isamarkupfalse%
\isacommand{apply}{\isacharparenleft}rule\ pdl{\isacharunderscore}plugD{\isacharunderscore}lifted{\isadigit{1}}{\isacharparenright}\isanewline
\ \ \isamarkupfalse%
\isacommand{apply}{\isacharparenleft}rule\ get{\isacharunderscore}item{\isacharparenright}\isanewline
\ \ \isamarkupfalse%
\isacommand{apply}{\isacharparenleft}rule\ allI{\isacharparenright}\isanewline
\ \ \isamarkupfalse%
\isacommand{apply}{\isacharparenleft}simp\ add{\isacharcolon}\ split{\isacharunderscore}if{\isacharparenright}\isanewline
\ \ \isamarkupfalse%
\isacommand{apply}{\isacharparenleft}safe{\isacharparenright}\ \isanewline
\ \ \isamarkupfalse%
\isacommand{apply}{\isacharparenleft}rule\ pdl{\isacharunderscore}iffD{\isadigit{2}}{\isacharbrackleft}OF\ pdl{\isacharunderscore}retD{\isacharbrackright}{\isacharparenright}\isanewline
\ \ \isamarkupfalse%
\isacommand{by}\ {\isacharparenleft}simp\ add{\isacharcolon}\ pdl{\isacharunderscore}taut{\isacharparenright}\ %
\isamarkupcmt{For the else-branch we obtain a contradiction, since the input was 1%
}
\isamarkupfalse%
%
\begin{isamarkuptext}%
On empty input, \isa{digitp} will fail.%
\end{isamarkuptext}%
\isamarkuptrue%
\isacommand{theorem}\ digitp{\isacharunderscore}fail{\isacharcolon}\ {\isachardoublequote}{\isasymturnstile}\ GetInput\ {\isacharequal}\isactrlsub D\ Ret\ {\isacharbrackleft}{\isacharbrackright}\ {\isasymlongrightarrow}\isactrlsub D\ {\isacharbrackleft}{\isacharhash}\ digitp{\isacharbrackright}{\isacharparenleft}{\isasymlambda}x{\isachardot}\ Ret\ False{\isacharparenright}{\isachardoublequote}\isanewline
\ \ \isamarkupfalse%
\isacommand{apply}{\isacharparenleft}simp\ add{\isacharcolon}\ digitp{\isacharunderscore}def\ sat{\isacharunderscore}def{\isacharparenright}\isanewline
\ \ \isamarkupfalse%
\isacommand{apply}{\isacharparenleft}rule\ pdl{\isacharunderscore}plugB{\isacharunderscore}lifted{\isadigit{1}}{\isacharparenright}\isanewline
\ \ \isamarkupfalse%
\isacommand{apply}{\isacharparenleft}rule\ GetInput{\isacharunderscore}item{\isacharunderscore}fail{\isacharparenright}\isanewline
\ \ \isamarkupfalse%
\isacommand{apply}{\isacharparenleft}rule\ allI{\isacharparenright}\isanewline
\ \ \isamarkupfalse%
\isacommand{apply}{\isacharparenleft}rule\ pdl{\isacharunderscore}False{\isacharunderscore}imp{\isacharparenright}\isanewline
\isamarkupfalse%
\isacommand{done}\isanewline
\isanewline
\isanewline
\isamarkupfalse%
\isacommand{lemma}\ ret{\isacharunderscore}nil{\isacharunderscore}aux{\isacharcolon}\ {\isachardoublequote}\ {\isasymturnstile}\ A\ {\isasymand}\isactrlsub D\ B\ {\isasymlongrightarrow}\isactrlsub D\isanewline
\ \ {\isasymlangle}ret\ {\isacharbrackleft}{\isacharbrackright}{\isasymrangle}{\isacharparenleft}{\isasymlambda}xs{\isachardot}\ A\ {\isasymand}\isactrlsub D\ B\ {\isasymand}\isactrlsub D\ Ret\ {\isacharparenleft}xs\ {\isacharequal}\ {\isacharbrackleft}{\isacharbrackright}{\isacharparenright}{\isacharparenright}{\isachardoublequote}\isamarkupfalse%
\isamarkupfalse%
\isamarkupfalse%
\isanewline
\isanewline
\isamarkupfalse%
\isacommand{lemma}\ ret{\isacharunderscore}one{\isacharunderscore}aux{\isacharcolon}\ {\isachardoublequote}{\isasymturnstile}\ A\ {\isasymlongrightarrow}\isactrlsub D\ \isanewline
\ \ \ \ \ \ \ \ \ \ \ \ \ \ \ \ \ \ \ \ \ \ {\isasymlangle}ret\ {\isacharparenleft}Suc\ {\isadigit{0}}{\isacharparenright}{\isasymrangle}{\isacharparenleft}{\isasymlambda}n{\isachardot}\ Ret\ {\isacharparenleft}n\ {\isacharequal}\ Suc\ {\isadigit{0}}{\isacharparenright}\ {\isasymand}\isactrlsub D\ A{\isacharparenright}{\isachardoublequote}\isamarkupfalse%
\isamarkupfalse%
\isamarkupfalse%
\isanewline
\isanewline
\isanewline
\isamarkupfalse%
\isacommand{lemma}\ pdl{\isacharunderscore}eqD{\isacharunderscore}aux{\isadigit{1}}{\isacharcolon}\ {\isachardoublequote}{\isasymturnstile}\ {\isacharparenleft}B\ {\isasymand}\isactrlsub D\ C\ {\isasymlongrightarrow}\isactrlsub D\ {\isasymlangle}p\ b{\isasymrangle}P{\isacharparenright}\ {\isasymlongrightarrow}\isactrlsub D\ Ret\ {\isacharparenleft}a\ {\isacharequal}\ b{\isacharparenright}\ {\isasymand}\isactrlsub D\ B\ {\isasymand}\isactrlsub D\ C\ {\isasymlongrightarrow}\isactrlsub D\ {\isasymlangle}p\ a{\isasymrangle}P{\isachardoublequote}\isamarkupfalse%
\isamarkupfalse%
\isamarkupfalse%
\isamarkupfalse%
\isamarkupfalse%
\isamarkupfalse%
\isamarkupfalse%
\isamarkupfalse%
\isamarkupfalse%
\isamarkupfalse%
\isamarkupfalse%
\isamarkupfalse%
\isamarkupfalse%
\isamarkupfalse%
\isamarkupfalse%
\isanewline
\isamarkupfalse%
\isacommand{lemma}\ pdl{\isacharunderscore}eqD{\isacharunderscore}aux{\isadigit{2}}{\isacharcolon}\ {\isachardoublequote}{\isasymturnstile}\ {\isacharparenleft}A\ {\isasymlongrightarrow}\isactrlsub D\ {\isasymlangle}\ p\ b{\isasymrangle}P{\isacharparenright}\ {\isasymlongrightarrow}\isactrlsub D\ A\ {\isasymand}\isactrlsub D\ Ret\ {\isacharparenleft}a\ {\isacharequal}\ b{\isacharparenright}\ {\isasymlongrightarrow}\isactrlsub D\ {\isasymlangle}\ p\ a{\isasymrangle}P{\isachardoublequote}\isamarkupfalse%
\isamarkupfalse%
\isamarkupfalse%
\isamarkupfalse%
\isamarkupfalse%
\isamarkupfalse%
\isamarkupfalse%
\isamarkupfalse%
\isamarkupfalse%
\isamarkupfalse%
\isamarkupfalse%
\isamarkupfalse%
\isamarkupfalse%
\isamarkupfalse%
\isamarkupfalse%
\isanewline
\isanewline
\isamarkupfalse%
\isacommand{lemma}\ pdl{\isacharunderscore}imp{\isacharunderscore}strg{\isadigit{1}}{\isacharcolon}\ {\isachardoublequote}{\isasymturnstile}\ A\ {\isasymlongrightarrow}\isactrlsub D\ C\ {\isasymLongrightarrow}\ {\isasymturnstile}\ A\ {\isasymand}\isactrlsub D\ B\ {\isasymlongrightarrow}\isactrlsub D\ C{\isachardoublequote}\isamarkupfalse%
\isamarkupfalse%
\isamarkupfalse%
\isamarkupfalse%
\isamarkupfalse%
\isamarkupfalse%
\isamarkupfalse%
\isanewline
\isamarkupfalse%
\isacommand{lemma}\ pdl{\isacharunderscore}imp{\isacharunderscore}strg{\isadigit{2}}{\isacharcolon}\ {\isachardoublequote}{\isasymturnstile}\ B\ {\isasymlongrightarrow}\isactrlsub D\ C\ {\isasymLongrightarrow}\ {\isasymturnstile}\ A\ {\isasymand}\isactrlsub D\ B\ {\isasymlongrightarrow}\isactrlsub D\ C{\isachardoublequote}\isamarkupfalse%
\isamarkupfalse%
\isamarkupfalse%
\isamarkupfalse%
\isamarkupfalse%
\isamarkupfalse%
\isamarkupfalse%
\isamarkupfalse%
%
\isamarkupsubsection{Correctness of the Monadic Parser%
}
\isamarkuptrue%
%
\begin{isamarkuptext}%
The following is a major theorem, more because of its complexity and since it 
  involves most of the axioms given for the monad, than because of its
  theoretical insight. Essentially, it states that \isa{natp} behaves
  totally correct for a given input.
  \label{isa:natp-proof}%
\end{isamarkuptext}%
\isamarkuptrue%
\isacommand{theorem}\ natp{\isacharunderscore}corr{\isacharcolon}\ {\isachardoublequote}{\isasymturnstile}\ {\isasymlangle}do\ {\isacharbraceleft}uu{\isasymleftarrow}setInput\ {\isacharbrackleft}{\isadigit{1}}{\isacharbrackright}{\isacharsemicolon}\ natp{\isacharbraceright}{\isasymrangle}{\isacharparenleft}{\isasymlambda}n{\isachardot}\ Ret\ {\isacharparenleft}n\ {\isacharequal}\ {\isadigit{1}}{\isacharparenright}\ {\isasymand}\isactrlsub D\ Eot{\isacharparenright}{\isachardoublequote}\isanewline
\isamarkupfalse%
\isacommand{proof}\ {\isacharminus}\isanewline
\ \ \isamarkupfalse%
\isacommand{have}\ {\isachardoublequote}{\isasymturnstile}\ {\isasymlangle}uu{\isasymleftarrow}setInput\ {\isacharbrackleft}{\isadigit{1}}{\isacharbrackright}{\isasymrangle}{\isacharparenleft}GetInput\ {\isacharequal}\isactrlsub D\ Ret\ {\isacharbrackleft}{\isadigit{1}}{\isacharbrackright}{\isacharparenright}{\isachardoublequote}\isanewline
\ \ \ \ \isamarkupfalse%
\isacommand{by}\ {\isacharparenleft}rule\ set{\isacharunderscore}get{\isacharparenright}\isanewline
\ \ \isamarkupfalse%
\isacommand{moreover}\isanewline
\ \ \isamarkupfalse%
\isacommand{have}\ {\isachardoublequote}{\isasymforall}uu{\isacharcolon}{\isacharcolon}unit{\isachardot}\ {\isasymturnstile}\ GetInput\ {\isacharequal}\isactrlsub D\ Ret\ {\isacharbrackleft}{\isadigit{1}}{\isacharbrackright}\ {\isasymlongrightarrow}\isactrlsub D\ {\isasymlangle}n{\isasymleftarrow}natp{\isasymrangle}{\isacharparenleft}Ret\ {\isacharparenleft}n\ {\isacharequal}\ {\isadigit{1}}{\isacharparenright}\ {\isasymand}\isactrlsub D\ Eot{\isacharparenright}{\isachardoublequote}\isanewline
\ \ \isamarkupfalse%
\isacommand{proof}\isanewline
\ \ \ \ \isamarkupfalse%
\isacommand{fix}\ uu\isanewline
\ \ \ \ %
\isamarkupcmt{The actual proof starts here: from a given input, show that \isa{natp} is correct%
}
\isanewline
\ \ \ \ \isamarkupfalse%
\isacommand{show}\ {\isachardoublequote}{\isasymturnstile}\ GetInput\ {\isacharequal}\isactrlsub D\ Ret\ {\isacharbrackleft}{\isadigit{1}}{\isacharbrackright}\ {\isasymlongrightarrow}\isactrlsub D\ {\isasymlangle}natp{\isasymrangle}{\isacharparenleft}{\isasymlambda}n{\isachardot}\ Ret\ {\isacharparenleft}n\ {\isacharequal}\ {\isadigit{1}}{\isacharparenright}\ {\isasymand}\isactrlsub D\ Eot{\isacharparenright}{\isachardoublequote}\isanewline
\ \ \ \ \isamarkupfalse%
\isacommand{proof}\ {\isacharminus}\isanewline
\ \ \ \ \ \ %
\isamarkupcmt{Prove the formula with defn. of \isa{natp} unfolded%
}
\isanewline
\ \ \ \ \ \ \isamarkupfalse%
\isacommand{have}\ {\isachardoublequote}{\isasymturnstile}\ GetInput\ {\isacharequal}\isactrlsub D\ Ret\ {\isacharbrackleft}{\isadigit{1}}{\isacharbrackright}\ {\isasymlongrightarrow}\isactrlsub D\ {\isasymlangle}do\ {\isacharbraceleft}x{\isasymleftarrow}digitp{\isacharsemicolon}\ xs{\isasymleftarrow}many\ digitp{\isacharsemicolon}\ ret\ {\isacharparenleft}foldl\ {\isacharparenleft}{\isasymlambda}r{\isachardot}\ op\ {\isacharplus}\ {\isacharparenleft}{\isadigit{1}}{\isadigit{0}}\ {\isacharasterisk}\ r{\isacharparenright}{\isacharparenright}\ x\ xs{\isacharparenright}{\isacharbraceright}{\isasymrangle}{\isacharparenleft}{\isasymlambda}n{\isachardot}\ Ret\ {\isacharparenleft}n\ {\isacharequal}\ {\isadigit{1}}{\isacharparenright}\ {\isasymand}\isactrlsub D\ Eot{\isacharparenright}{\isachardoublequote}\ {\isacharparenleft}\isakeyword{is}\ {\isachardoublequote}{\isasymturnstile}\ {\isacharquery}a\ {\isasymlongrightarrow}\isactrlsub D\ {\isacharquery}b{\isachardoublequote}{\isacharparenright}\isanewline
\ \ \ \ \ \ \isamarkupfalse%
\isacommand{proof}\ {\isacharminus}\ %
\isamarkupcmt{Work out each atomic program separately%
}
\isanewline
\ \ \ \ \ \ \ \ \isamarkupfalse%
\isacommand{have}\ {\isachardoublequote}{\isasymturnstile}\ GetInput\ {\isacharequal}\isactrlsub D\ Ret\ {\isacharbrackleft}{\isadigit{1}}{\isacharbrackright}\ {\isasymlongrightarrow}\isactrlsub D\ {\isasymlangle}x{\isasymleftarrow}digitp{\isasymrangle}{\isacharparenleft}Ret\ {\isacharparenleft}x{\isacharequal}{\isadigit{1}}{\isacharparenright}\ {\isasymand}\isactrlsub D\ GetInput\ {\isacharequal}\isactrlsub D\ Ret\ {\isacharbrackleft}{\isacharbrackright}{\isacharparenright}{\isachardoublequote}\isanewline
\ \ \ \ \ \ \ \ \ \ \isamarkupfalse%
\isacommand{by}\ {\isacharparenleft}rule\ digitp{\isacharunderscore}nat{\isacharparenright}\isanewline
\ \ \ \ \ \ \ \ \isamarkupfalse%
\isacommand{moreover}\isanewline
\ \ \ \ \ \ \ \ \isamarkupfalse%
\isacommand{have}\ {\isachardoublequote}{\isasymforall}x{\isachardot}\ {\isasymturnstile}\ {\isacharparenleft}Ret\ {\isacharparenleft}x{\isacharequal}{\isacharparenleft}{\isadigit{1}}{\isacharcolon}{\isacharcolon}nat{\isacharparenright}{\isacharparenright}\ {\isasymand}\isactrlsub D\ GetInput\ {\isacharequal}\isactrlsub D\ Ret\ {\isacharbrackleft}{\isacharbrackright}{\isacharparenright}\ {\isasymlongrightarrow}\isactrlsub D\isanewline
\ \ \ \ \ \ \ \ \ \ \ \ {\isacharparenleft}{\isasymlangle}do\ {\isacharbraceleft}xs{\isasymleftarrow}many\ digitp{\isacharsemicolon}\ ret\ {\isacharparenleft}foldl\ {\isacharparenleft}{\isasymlambda}r{\isachardot}\ op\ {\isacharplus}\ {\isacharparenleft}{\isadigit{1}}{\isadigit{0}}\ {\isacharasterisk}\ r{\isacharparenright}{\isacharparenright}\ x\ xs{\isacharparenright}{\isacharbraceright}{\isasymrangle}{\isacharparenleft}{\isasymlambda}n{\isachardot}\ Ret{\isacharparenleft}n{\isacharequal}{\isadigit{1}}{\isacharparenright}\ {\isasymand}\isactrlsub D\ Eot{\isacharparenright}{\isacharparenright}{\isachardoublequote}\isanewline
\ \ \ \ \ \ \ \ \isamarkupfalse%
\isacommand{proof}\ %
\isamarkupcmt{Here, \isa{digitp} will fail, ie. \isa{many} will return \isa{{\isacharbrackleft}{\isacharbrackright}}%
}
\isanewline
\ \ \ \ \ \ \ \ \ \ \isamarkupfalse%
\isacommand{fix}\ x\isanewline
\ \ \ \ \ \ \ \ \ \ \isamarkupfalse%
\isacommand{show}\ {\isachardoublequote}{\isasymturnstile}\ Ret\ {\isacharparenleft}x\ {\isacharequal}\ {\isadigit{1}}{\isacharparenright}\ {\isasymand}\isactrlsub D\ GetInput\ {\isacharequal}\isactrlsub D\ Ret\ {\isacharbrackleft}{\isacharbrackright}\ {\isasymlongrightarrow}\isactrlsub D\isanewline
\ \ \ \ \ \ \ \ \ \ \ \ \ {\isasymlangle}do\ {\isacharbraceleft}xs{\isasymleftarrow}many\ digitp{\isacharsemicolon}\ ret\ {\isacharparenleft}foldl\ {\isacharparenleft}{\isasymlambda}r{\isachardot}\ op\ {\isacharplus}\ {\isacharparenleft}{\isadigit{1}}{\isadigit{0}}\ {\isacharasterisk}\ r{\isacharparenright}{\isacharparenright}\ x\ xs{\isacharparenright}{\isacharbraceright}{\isasymrangle}{\isacharparenleft}{\isasymlambda}n{\isachardot}\ Ret\ {\isacharparenleft}n\ {\isacharequal}\ {\isadigit{1}}{\isacharparenright}\ {\isasymand}\isactrlsub D\ Eot{\isacharparenright}{\isachardoublequote}\isanewline
\ \ \ \ \ \ \ \ \ \ \isamarkupfalse%
\isacommand{proof}\ {\isacharparenleft}rule\ pdl{\isacharunderscore}plugD{\isacharunderscore}lifted{\isadigit{1}}{\isacharbrackleft}\isakeyword{where}\ B\ {\isacharequal}\ {\isachardoublequote}{\isasymlambda}xs{\isachardot}\ Ret\ {\isacharparenleft}x\ {\isacharequal}\ {\isadigit{1}}{\isacharparenright}\ {\isasymand}\isactrlsub D\ GetInput\ {\isacharequal}\isactrlsub D\ Ret\ {\isacharbrackleft}{\isacharbrackright}\ {\isasymand}\isactrlsub D\isanewline
\ \ \ \ \ \ \ \ \ \ \ \ \ \ \ \ \ \ \ \ \ \ \ \ \ \ \ \ \ \ \ \ \ \ \ \ \ \ \ \ \ \ \ \ \ \ \ \ \ \ \ \ \ \ \ \ Ret\ {\isacharparenleft}xs\ {\isacharequal}\ {\isacharbrackleft}{\isacharbrackright}{\isacharparenright}{\isachardoublequote}{\isacharbrackright}{\isacharparenright}\isanewline
\ \ \ \ \ \ \ \ \ \ \ \ \isamarkupfalse%
\isacommand{show}\ {\isachardoublequote}{\isasymturnstile}\ Ret\ {\isacharparenleft}x\ {\isacharequal}\ {\isadigit{1}}{\isacharparenright}\ {\isasymand}\isactrlsub D\ GetInput\ {\isacharequal}\isactrlsub D\ Ret\ {\isacharbrackleft}{\isacharbrackright}\ {\isasymlongrightarrow}\isactrlsub D\isanewline
\ \ \ \ \ \ \ \ \ \ \ \ \ \ {\isasymlangle}many\ digitp{\isasymrangle}{\isacharparenleft}{\isasymlambda}xs{\isachardot}\ Ret\ {\isacharparenleft}x\ {\isacharequal}\ {\isadigit{1}}{\isacharparenright}\ {\isasymand}\isactrlsub D\ GetInput\ {\isacharequal}\isactrlsub D\ Ret\ {\isacharbrackleft}{\isacharbrackright}\ {\isasymand}\isactrlsub D\ Ret\ {\isacharparenleft}xs\ {\isacharequal}\ {\isacharbrackleft}{\isacharbrackright}{\isacharparenright}{\isacharparenright}{\isachardoublequote}\isanewline
\ \ \ \ \ \ \ \ \ \ \ \ \ \ \isamarkupfalse%
\isacommand{apply}{\isacharparenleft}subst\ many{\isacharunderscore}unfold{\isacharparenright}\isanewline
\ \ \ \ \ \ \ \ \ \ \ \ \ \ \isamarkupfalse%
\isacommand{apply}{\isacharparenleft}rule\ altD{\isacharunderscore}iff{\isacharunderscore}lifted{\isadigit{1}}{\isacharparenright}\isanewline
\ \ \ \ \ \ \ \ \ \ \ \ \ \ \isamarkupfalse%
\isacommand{apply}{\isacharparenleft}rule\ ret{\isacharunderscore}nil{\isacharunderscore}aux{\isacharparenright}\isanewline
\ \ \ \ \ \ \ \ \ \ \ \ \ \ \isamarkupfalse%
\isacommand{apply}{\isacharparenleft}rule\ pdl{\isacharunderscore}plugB{\isacharunderscore}lifted{\isadigit{1}}{\isacharparenright}\isanewline
\ \ \ \ \ \ \ \ \ \ \ \ \ \ \isamarkupfalse%
\isacommand{apply}{\isacharparenleft}rule\ pdl{\isacharunderscore}imp{\isacharunderscore}strg{\isadigit{2}}{\isacharparenright}\isanewline
\ \ \ \ \ \ \ \ \ \ \ \ \ \ \isamarkupfalse%
\isacommand{apply}{\isacharparenleft}rule\ digitp{\isacharunderscore}fail{\isacharparenright}\isanewline
\ \ \ \ \ \ \ \ \ \ \ \ \ \ \isamarkupfalse%
\isacommand{apply}{\isacharparenleft}rule\ allI{\isacharparenright}\ \isanewline
\ \ \ \ \ \ \ \ \ \ \ \ \ \ \isamarkupfalse%
\isacommand{by}\ {\isacharparenleft}simp\ add{\isacharcolon}\ pdl{\isacharunderscore}taut{\isacharparenright}\isanewline
\ \ \ \ \ \ \ \ \ \ \isamarkupfalse%
\isacommand{next}\isanewline
\ \ \ \ \ \ \ \ \ \ \ \ \isamarkupfalse%
\isacommand{show}\ {\isachardoublequote}{\isasymforall}xs{\isachardot}\ {\isasymturnstile}\ Ret\ {\isacharparenleft}x\ {\isacharequal}\ {\isadigit{1}}{\isacharparenright}\ {\isasymand}\isactrlsub D\ GetInput\ {\isacharequal}\isactrlsub D\ Ret\ {\isacharbrackleft}{\isacharbrackright}\ {\isasymand}\isactrlsub D\ Ret\ {\isacharparenleft}xs\ {\isacharequal}\ {\isacharbrackleft}{\isacharbrackright}{\isacharparenright}\ {\isasymlongrightarrow}\isactrlsub D\isanewline
\ \ \ \ \ \ \ \ \ \ \ \ \ \ \ \ \ \ \ \ \ \ \ \ \ {\isasymlangle}ret\ {\isacharparenleft}foldl\ {\isacharparenleft}{\isasymlambda}r{\isachardot}\ op\ {\isacharplus}\ {\isacharparenleft}{\isadigit{1}}{\isadigit{0}}\ {\isacharasterisk}\ r{\isacharparenright}{\isacharparenright}\ x\ xs{\isacharparenright}{\isasymrangle}{\isacharparenleft}{\isasymlambda}n{\isachardot}\ Ret\ {\isacharparenleft}n\ {\isacharequal}\ {\isadigit{1}}{\isacharparenright}\ {\isasymand}\isactrlsub D\ Eot{\isacharparenright}{\isachardoublequote}\isanewline
\ \ \ \ \ \ \ \ \ \ \ \ \ \ \isamarkupfalse%
\isacommand{apply}{\isacharparenleft}rule\ allI{\isacharparenright}\isanewline
\ \ \ \ \ \ \ \ \ \ \ \ \ \ \isamarkupfalse%
\isacommand{apply}{\isacharparenleft}rule\ pdl{\isacharunderscore}eqD{\isacharunderscore}aux{\isadigit{1}}{\isacharbrackleft}THEN\ pdl{\isacharunderscore}mp{\isacharbrackright}{\isacharparenright}\isanewline
\ \ \ \ \ \ \ \ \ \ \ \ \ \ \isamarkupfalse%
\isacommand{apply}{\isacharparenleft}rule\ pdl{\isacharunderscore}eqD{\isacharunderscore}aux{\isadigit{2}}{\isacharbrackleft}THEN\ pdl{\isacharunderscore}mp{\isacharbrackright}{\isacharparenright}\isanewline
\ \ \ \ \ \ \ \ \ \ \ \ \ \ \isamarkupfalse%
\isacommand{apply}{\isacharparenleft}simp{\isacharparenright}\isanewline
\ \ \ \ \ \ \ \ \ \ \ \ \ \ \isamarkupfalse%
\isacommand{apply}{\isacharparenleft}subst\ Eot{\isacharunderscore}GetInput{\isacharparenright}\isanewline
\ \ \ \ \ \ \ \ \ \ \ \ \ \ \isamarkupfalse%
\isacommand{by}\ {\isacharparenleft}rule\ ret{\isacharunderscore}one{\isacharunderscore}aux{\isacharparenright}\isanewline
\ \ \ \ \ \ \ \ \ \ \isamarkupfalse%
\isacommand{qed}\isanewline
\ \ \ \ \ \ \ \ \isamarkupfalse%
\isacommand{qed}\isanewline
\ \ \ \ \ \ \ \ \isamarkupfalse%
\isacommand{ultimately}\isanewline
\ \ \ \ \ \ \ \ \isamarkupfalse%
\isacommand{show}\ {\isacharquery}thesis\ \isamarkupfalse%
\isacommand{by}\ {\isacharparenleft}rule\ pdl{\isacharunderscore}plugD{\isacharunderscore}lifted{\isadigit{1}}{\isacharparenright}\isanewline
\ \ \ \ \ \ \isamarkupfalse%
\isacommand{qed}\isanewline
\ \ \ \ \ \ \isamarkupfalse%
\isacommand{thus}\ {\isacharquery}thesis\ \isamarkupfalse%
\isacommand{by}\ {\isacharparenleft}simp\ add{\isacharcolon}\ natp{\isacharunderscore}def\ many{\isadigit{1}}{\isacharunderscore}def\ mon{\isacharunderscore}ctr\ del{\isacharcolon}\ bind{\isacharunderscore}assoc{\isacharparenright}\isanewline
\ \ \ \ \isamarkupfalse%
\isacommand{qed}\isanewline
\ \ \isamarkupfalse%
\isacommand{qed}\isanewline
\ \ \isamarkupfalse%
\isacommand{ultimately}\ \isamarkupfalse%
\isacommand{show}\ {\isacharquery}thesis\ \isamarkupfalse%
\isacommand{by}\ {\isacharparenleft}rule\ pdl{\isacharunderscore}plugD{\isacharparenright}\isanewline
\isamarkupfalse%
\isacommand{qed}\isanewline
\isamarkupfalse%
\isamarkupfalse%
\isamarkuptrue%
\isamarkupfalse%
\isamarkupfalse%
\isamarkupfalse%
\isamarkupfalse%
\isamarkupfalse%
\isamarkupfalse%
\isamarkupfalse%
\isamarkupfalse%
\isamarkupfalse%
\isamarkupfalse%
\isamarkupfalse%
\isamarkupfalse%
\isamarkupfalse%
\isamarkupfalse%
\isamarkupfalse%
\isamarkupfalse%
\isamarkuptrue%
\isamarkupfalse%
\isamarkupfalse%
\isamarkupfalse%
\isamarkupfalse%
\isamarkupfalse%
\isamarkupfalse%
\isamarkupfalse%
\isamarkupfalse%
\isacommand{end}\isanewline
\isamarkupfalse%
\end{isabellebody}%
%%% Local Variables:
%%% mode: latex
%%% TeX-master: "root"
%%% End:


%
\begin{isabellebody}%
\def\isabellecontext{State}%
%
\isamarkupheader{A Simple Reference Monad with \texttt{while} and \texttt{if}%
}
\isamarkuptrue%
\isacommand{theory}\ State\ {\isacharequal}\ PDL\ {\isacharplus}\ MonEq{\isacharcolon}\isamarkupfalse%
%
\label{sec:state-thy}
%
\begin{isamarkuptext}%
Read/write operations on references of arbitrary type, and a while loop.
  \label{isa:ref-spec}%
\end{isamarkuptext}%
\isamarkuptrue%
\isacommand{typedecl}\ {\isacharprime}a\ ref\isanewline
\isanewline
\isamarkupfalse%
\isacommand{consts}\isanewline
\ \ newRef\ \ \ \ \ {\isacharcolon}{\isacharcolon}\ {\isachardoublequote}{\isacharprime}a\ {\isasymRightarrow}\ {\isacharprime}a\ ref\ T{\isachardoublequote}\isanewline
\ \ readRef\ \ \ \ {\isacharcolon}{\isacharcolon}\ {\isachardoublequote}{\isacharprime}a\ ref\ {\isasymRightarrow}\ {\isacharprime}a\ T{\isachardoublequote}\ \ \ \ \ \ \ \ \ \ \ \ \ \ \ \isanewline
\ \ writeRef\ \ \ {\isacharcolon}{\isacharcolon}\ {\isachardoublequote}{\isacharprime}a\ ref\ {\isasymRightarrow}\ {\isacharprime}a\ {\isasymRightarrow}\ unit\ T{\isachardoublequote}\ \ \ \ \ \ \ \ \ \ \ {\isacharparenleft}{\isachardoublequote}{\isacharparenleft}{\isacharunderscore}\ {\isacharcolon}{\isacharequal}\ {\isacharunderscore}{\isacharparenright}{\isachardoublequote}\ {\isacharbrackleft}{\isadigit{1}}{\isadigit{0}}{\isadigit{0}}{\isacharcomma}\ {\isadigit{1}}{\isadigit{0}}{\isacharbrackright}\ {\isadigit{1}}{\isadigit{0}}{\isacharparenright}\isanewline
\ \ monWhile\ \ \ {\isacharcolon}{\isacharcolon}\ {\isachardoublequote}bool\ D\ {\isasymRightarrow}\ unit\ T\ {\isasymRightarrow}\ unit\ T{\isachardoublequote}\ \ \ \ \ \ \ {\isacharparenleft}{\isachardoublequote}WHILE\ {\isacharparenleft}{\isadigit{4}}{\isacharunderscore}{\isacharparenright}\ {\isacharslash}DO\ {\isacharparenleft}{\isadigit{4}}{\isacharunderscore}{\isacharparenright}\ {\isacharslash}END{\isachardoublequote}{\isacharparenright}\isamarkupfalse%
%
\begin{isamarkuptext}%
To make the dsef operation of reading a reference more readable (pun unintended),
  we introduce syntactical sugar: \isa{{\isacharasterisk}r} stands for \isa{{\isasymUp}\ readRef\ r}.%
\end{isamarkuptext}%
\isamarkuptrue%
\isacommand{syntax}\ \isanewline
\ \ {\isachardoublequote}{\isacharunderscore}readRefD{\isachardoublequote}\ \ {\isacharcolon}{\isacharcolon}\ {\isachardoublequote}{\isacharprime}a\ ref\ {\isasymRightarrow}\ {\isacharprime}a\ D{\isachardoublequote}\ \ \ \ \ \ \ \ \ \ \ \ \ \ \ \ {\isacharparenleft}{\isachardoublequote}{\isacharasterisk}{\isacharunderscore}{\isachardoublequote}\ {\isacharbrackleft}{\isadigit{1}}{\isadigit{0}}{\isadigit{0}}{\isacharbrackright}\ {\isadigit{1}}{\isadigit{0}}{\isadigit{0}}{\isacharparenright}\isanewline
\isanewline
\isamarkupfalse%
\isacommand{translations}\isanewline
\ \ {\isachardoublequote}{\isacharunderscore}readRefD\ r{\isachardoublequote}\ \ \ \ \ \ \ \ \ {\isasymrightleftharpoons}\ \ \ \ {\isachardoublequote}{\isasymUp}\ {\isacharparenleft}readRef\ r{\isacharparenright}{\isachardoublequote}\isamarkupfalse%
%
\begin{isamarkuptext}%
This definition is rather useless as it stands, since one actually wants
  \isa{oldref\ r} to be a formula in \isa{bool\ D}. The quantifier is necessary to
  avoid introducing a fresh variable \isa{a} on the right hand side of the
  definition.
  
  The idea is appealing however, since it would provide a statement about the
  existence of \isa{r} as a reference.%
\end{isamarkuptext}%
\isamarkuptrue%
\isacommand{constdefs}\isanewline
\ \ oldref\ \ \ \ \ {\isacharcolon}{\isacharcolon}\ {\isachardoublequote}{\isacharprime}a\ ref\ {\isasymRightarrow}\ bool{\isachardoublequote}\isanewline
\ \ {\isachardoublequote}oldref\ r\ \ {\isasymequiv}\ \ {\isasymforall}a{\isachardot}\ {\isasymturnstile}\ {\isacharbrackleft}{\isacharhash}\ s{\isasymleftarrow}newRef\ a{\isacharbrackright}{\isacharparenleft}Ret\ {\isacharparenleft}{\isasymnot}{\isacharparenleft}r{\isacharequal}s{\isacharparenright}{\isacharparenright}{\isacharparenright}{\isachardoublequote}\isamarkupfalse%
%
\begin{isamarkuptext}%
The basic axioms of a simple while language with references. In the following we will not
  make use of operation \isa{newRef} and hence neither of its axioms.%
\end{isamarkuptext}%
\isamarkuptrue%
\isacommand{axioms}\isanewline
dsef{\isacharunderscore}read{\isacharcolon}\ \ \ \ \ {\isachardoublequote}dsef\ {\isacharparenleft}readRef\ r{\isacharparenright}{\isachardoublequote}\isanewline
read{\isacharunderscore}write{\isacharcolon}\ \ \ \ {\isachardoublequote}{\isasymturnstile}\ {\isacharbrackleft}{\isacharhash}\ r\ {\isacharcolon}{\isacharequal}\ x{\isacharbrackright}{\isacharparenleft}{\isasymlambda}uu{\isachardot}\ {\isacharasterisk}r\ {\isacharequal}\isactrlsub D\ Ret\ x{\isacharparenright}{\isachardoublequote}\isanewline
read{\isacharunderscore}write{\isacharunderscore}other{\isacharunderscore}gen{\isacharcolon}\ {\isachardoublequote}{\isasymturnstile}\ {\isasymUp}\ {\isacharparenleft}do\ {\isacharbraceleft}u{\isasymleftarrow}readRef\ r{\isacharsemicolon}\ ret\ {\isacharparenleft}f\ u{\isacharparenright}{\isacharbraceright}{\isacharparenright}\ {\isasymlongrightarrow}\isactrlsub D\ \isanewline
\ \ \ \ \ \ \ \ \ \ \ \ \ \ \ \ \ \ \ \ \ \ \ \ \ \ \ \ {\isacharbrackleft}{\isacharhash}\ s\ {\isacharcolon}{\isacharequal}\ y{\isacharbrackright}{\isacharparenleft}{\isasymlambda}uu{\isachardot}\ Ret\ {\isacharparenleft}r{\isasymnoteq}s{\isacharparenright}\ {\isasymlongrightarrow}\isactrlsub D\ {\isasymUp}\ {\isacharparenleft}do\ {\isacharbraceleft}u{\isasymleftarrow}readRef\ r{\isacharsemicolon}\ ret\ {\isacharparenleft}f\ u{\isacharparenright}{\isacharbraceright}{\isacharparenright}{\isacharparenright}{\isachardoublequote}\isanewline
while{\isacharunderscore}par{\isacharcolon}\ \ \ \ \ {\isachardoublequote}{\isasymturnstile}\ P\ {\isasymand}\isactrlsub D\ b\ {\isasymlongrightarrow}\isactrlsub D\ {\isacharbrackleft}{\isacharhash}\ p{\isacharbrackright}{\isacharparenleft}{\isasymlambda}u{\isachardot}\ P{\isacharparenright}\ {\isasymLongrightarrow}\ {\isasymturnstile}\ P\ {\isasymlongrightarrow}\isactrlsub D\ {\isacharbrackleft}{\isacharhash}\ WHILE\ b\ DO\ p\ END{\isacharbrackright}{\isacharparenleft}{\isasymlambda}x{\isachardot}\ P\ {\isasymand}\isactrlsub D\ {\isasymnot}\isactrlsub D\ b{\isacharparenright}{\isachardoublequote}\isanewline
read{\isacharunderscore}new{\isacharcolon}\ \ \ \ \ \ {\isachardoublequote}{\isasymturnstile}\ {\isacharbrackleft}{\isacharhash}\ r{\isasymleftarrow}newRef\ a{\isacharbrackright}{\isacharparenleft}\ Ret\ a\ {\isacharequal}\isactrlsub D\ {\isacharasterisk}r{\isacharparenright}{\isachardoublequote}\isanewline
read{\isacharunderscore}new{\isacharunderscore}other{\isacharcolon}\ {\isachardoublequote}{\isasymturnstile}\ {\isacharparenleft}Ret\ x\ {\isacharequal}\isactrlsub D\ {\isacharasterisk}r{\isacharparenright}\ {\isasymlongrightarrow}\isactrlsub D\ {\isacharbrackleft}{\isacharhash}\ s\ {\isasymleftarrow}\ newRef\ y{\isacharbrackright}{\isacharparenleft}{\isacharparenleft}Ret\ x\ {\isacharequal}\isactrlsub D\ {\isacharasterisk}r{\isacharparenright}\ {\isasymor}\isactrlsub D\ Ret\ {\isacharparenleft}r{\isacharequal}s{\isacharparenright}{\isacharparenright}{\isachardoublequote}\isanewline
\isanewline
\isanewline
\isanewline
\isamarkupfalse%
\isacommand{lemma}\ read{\isacharunderscore}write{\isacharunderscore}other{\isacharcolon}\ {\isachardoublequote}{\isasymturnstile}\ {\isacharparenleft}\ {\isacharasterisk}r\ {\isacharequal}\isactrlsub D\ Ret\ x{\isacharparenright}\ {\isasymlongrightarrow}\isactrlsub D\ {\isacharbrackleft}{\isacharhash}\ s\ {\isacharcolon}{\isacharequal}\ y{\isacharbrackright}{\isacharparenleft}{\isasymlambda}uu{\isachardot}\ Ret\ {\isacharparenleft}r{\isasymnoteq}s{\isacharparenright}\ {\isasymlongrightarrow}\isactrlsub D\ {\isacharparenleft}\ {\isacharasterisk}r\ {\isacharequal}\isactrlsub D\ Ret\ x{\isacharparenright}{\isacharparenright}{\isachardoublequote}\isanewline
\isamarkupfalse%
\isacommand{proof}\ {\isacharminus}\isanewline
\ \ \isamarkupfalse%
\isacommand{have}\ {\isachardoublequote}{\isasymturnstile}\ {\isasymUp}\ {\isacharparenleft}do\ {\isacharbraceleft}u{\isasymleftarrow}readRef\ r{\isacharsemicolon}\ ret\ {\isacharparenleft}u\ {\isacharequal}\ x{\isacharparenright}{\isacharbraceright}{\isacharparenright}\ {\isasymlongrightarrow}\isactrlsub D\isanewline
\ \ \ \ \ \ \ \ \ \ \ \ {\isacharbrackleft}{\isacharhash}\ s\ {\isacharcolon}{\isacharequal}\ y{\isacharbrackright}{\isacharparenleft}{\isasymlambda}uu{\isachardot}\ Ret\ {\isacharparenleft}r{\isasymnoteq}s{\isacharparenright}\ {\isasymlongrightarrow}\isactrlsub D\ {\isasymUp}\ {\isacharparenleft}do\ {\isacharbraceleft}u{\isasymleftarrow}readRef\ r{\isacharsemicolon}\ ret\ {\isacharparenleft}u\ {\isacharequal}\ x{\isacharparenright}{\isacharbraceright}{\isacharparenright}{\isacharparenright}{\isachardoublequote}\isanewline
\ \ \ \ \isamarkupfalse%
\isacommand{by}\ {\isacharparenleft}rule\ read{\isacharunderscore}write{\isacharunderscore}other{\isacharunderscore}gen{\isacharparenright}\isanewline
\ \ \isamarkupfalse%
\isacommand{thus}\ {\isacharquery}thesis\isanewline
\ \ \ \ \isamarkupfalse%
\isacommand{by}\ {\isacharparenleft}simp\ add{\isacharcolon}\ MonEq{\isacharunderscore}def\ liftM{\isadigit{2}}{\isacharunderscore}def\ Dsef{\isacharunderscore}def\ Ret{\isacharunderscore}def\ Abs{\isacharunderscore}Dsef{\isacharunderscore}inverse\ dsef{\isacharunderscore}read{\isacharparenright}\isanewline
\isamarkupfalse%
\isacommand{qed}\isamarkupfalse%
%
\begin{isamarkuptext}%
It is not really necessary to step back to the do-notation for 
  \isa{read{\isacharunderscore}write{\isacharunderscore}other{\isacharunderscore}gen}.%
\end{isamarkuptext}%
\isamarkuptrue%
\isacommand{lemma}\ {\isachardoublequote}{\isasymturnstile}\ {\isacharasterisk}r\ {\isacharequal}\isactrlsub D\ Ret\ b\ {\isasymand}\isactrlsub D\ Ret\ {\isacharparenleft}f\ b{\isacharparenright}\ {\isasymlongrightarrow}\isactrlsub D\ {\isasymUp}\ {\isacharparenleft}do\ {\isacharbraceleft}a{\isasymleftarrow}readRef\ r{\isacharsemicolon}\ ret\ {\isacharparenleft}f\ a\ {\isasymand}\ a\ {\isacharequal}\ b{\isacharparenright}{\isacharbraceright}{\isacharparenright}{\isachardoublequote}\isamarkupfalse%
\isamarkupfalse%
\isamarkupfalse%
\isamarkupfalse%
\isamarkupfalse%
%
\begin{isamarkuptext}%
Definitions of oddity and evenness of natural numbers, as well as an algorithm
  for computing Russian multiplication \isa{rumult}.
  \label{isa:rumult-spec}%
\end{isamarkuptext}%
\isamarkuptrue%
\isacommand{constdefs}\isanewline
\ \ nat{\isacharunderscore}even\ \ {\isacharcolon}{\isacharcolon}\ {\isachardoublequote}nat\ {\isasymRightarrow}\ bool{\isachardoublequote}\isanewline
\ \ {\isachardoublequote}nat{\isacharunderscore}even\ n\ {\isasymequiv}\ {\isadigit{2}}\ dvd\ n{\isachardoublequote}\isanewline
\ \ nat{\isacharunderscore}odd\ \ \ {\isacharcolon}{\isacharcolon}\ {\isachardoublequote}nat\ {\isasymRightarrow}\ bool{\isachardoublequote}\isanewline
\ \ {\isachardoublequote}nat{\isacharunderscore}odd\ n\ {\isasymequiv}\ {\isasymnot}\ nat{\isacharunderscore}even\ n{\isachardoublequote}\isanewline
\ \ rumult\ \ \ \ \ {\isacharcolon}{\isacharcolon}\ {\isachardoublequote}nat\ {\isasymRightarrow}\ nat\ {\isasymRightarrow}\ nat\ ref\ {\isasymRightarrow}\ nat\ ref\ {\isasymRightarrow}\ nat\ ref\ {\isasymRightarrow}\ nat\ T{\isachardoublequote}\isanewline
\ \ {\isachardoublequote}rumult\ a\ b\ x\ y\ r\ {\isasymequiv}\ do\ {\isacharbraceleft}x{\isacharcolon}{\isacharequal}a{\isacharsemicolon}\ y{\isacharcolon}{\isacharequal}b{\isacharsemicolon}\ r{\isacharcolon}{\isacharequal}{\isadigit{0}}{\isacharsemicolon}\isanewline
\ \ \ \ \ \ \ \ \ \ \ \ \ \ \ \ \ \ \ \ \ \ \ \ \ \ WHILE\ {\isacharparenleft}{\isasymUp}\ {\isacharparenleft}do\ {\isacharbraceleft}u{\isasymleftarrow}readRef\ x{\isacharsemicolon}\ ret\ {\isacharparenleft}{\isadigit{0}}\ {\isacharless}\ u{\isacharparenright}{\isacharbraceright}{\isacharparenright}{\isacharparenright}\isanewline
\ \ \ \ \ \ \ \ \ \ \ \ \ \ \ \ \ \ \ \ \ \ \ \ \ \ \ \ \ DO\ do\ {\isacharbraceleft}u{\isasymleftarrow}readRef\ x{\isacharsemicolon}\ v{\isasymleftarrow}readRef\ y{\isacharsemicolon}\ w{\isasymleftarrow}readRef\ r{\isacharsemicolon}\isanewline
\ \ \ \ \ \ \ \ \ \ \ \ \ \ \ \ \ \ \ \ \ \ \ \ \ \ \ \ \ \ \ \ \ \ \ \ if\ {\isacharparenleft}nat{\isacharunderscore}odd\ u{\isacharparenright}\ then\ {\isacharparenleft}r\ {\isacharcolon}{\isacharequal}\ w\ {\isacharplus}\ v{\isacharparenright}\ else\ ret\ {\isacharparenleft}{\isacharparenright}{\isacharsemicolon}\isanewline
\ \ \ \ \ \ \ \ \ \ \ \ \ \ \ \ \ \ \ \ \ \ \ \ \ \ \ \ \ \ \ \ \ \ \ \ x\ {\isacharcolon}{\isacharequal}\ u\ div\ {\isadigit{2}}{\isacharsemicolon}\ y\ {\isacharcolon}{\isacharequal}\ v\ {\isacharasterisk}\ {\isadigit{2}}{\isacharbraceright}\ END{\isacharsemicolon}\ readRef\ r{\isacharbraceright}{\isachardoublequote}\isamarkupfalse%
%
\isamarkupsubsection{General Auxiliary Lemmas%
}
\isamarkuptrue%
%
\begin{isamarkuptext}%
Following are several auxiliary lemmas which are not general enough to be placed
  inside the general theory files, but which are used more than once below -- and thus
  justify their mere existence.%
\end{isamarkuptext}%
\isamarkuptrue%
%
\begin{isamarkuptext}%
Some weakening rules.%
\end{isamarkuptext}%
\isamarkuptrue%
\isacommand{lemma}\ pdl{\isacharunderscore}conj{\isacharunderscore}imp{\isacharunderscore}wk{\isadigit{1}}{\isacharcolon}\ {\isachardoublequote}{\isasymturnstile}\ A\ {\isasymlongrightarrow}\isactrlsub D\ C\ {\isasymLongrightarrow}\ {\isasymturnstile}\ A\ {\isasymand}\isactrlsub D\ B\ {\isasymlongrightarrow}\isactrlsub D\ C{\isachardoublequote}\isanewline
\isamarkupfalse%
\isacommand{proof}\ {\isacharminus}\isanewline
\ \ \isamarkupfalse%
\isacommand{assume}\ {\isachardoublequote}{\isasymturnstile}\ A\ {\isasymlongrightarrow}\isactrlsub D\ C{\isachardoublequote}\isanewline
\ \ \isamarkupfalse%
\isacommand{have}\ {\isachardoublequote}{\isasymturnstile}\ {\isacharparenleft}A\ {\isasymlongrightarrow}\isactrlsub D\ C{\isacharparenright}\ {\isasymlongrightarrow}\isactrlsub D\ A\ {\isasymand}\isactrlsub D\ B\ {\isasymlongrightarrow}\isactrlsub D\ C{\isachardoublequote}\isanewline
\ \ \ \ \isamarkupfalse%
\isacommand{by}\ {\isacharparenleft}simp\ add{\isacharcolon}\ pdl{\isacharunderscore}taut{\isacharparenright}\isanewline
\ \ \isamarkupfalse%
\isacommand{thus}\ {\isacharquery}thesis\ \isamarkupfalse%
\isacommand{by}\ {\isacharparenleft}rule\ pdl{\isacharunderscore}mp{\isacharparenright}\isanewline
\isamarkupfalse%
\isacommand{qed}\isanewline
\isanewline
\isamarkupfalse%
\isacommand{lemma}\ pdl{\isacharunderscore}conj{\isacharunderscore}imp{\isacharunderscore}wk{\isadigit{2}}{\isacharcolon}\ {\isachardoublequote}{\isasymturnstile}\ B\ {\isasymlongrightarrow}\isactrlsub D\ C\ {\isasymLongrightarrow}\ {\isasymturnstile}\ A\ {\isasymand}\isactrlsub D\ B\ {\isasymlongrightarrow}\isactrlsub D\ C{\isachardoublequote}\isanewline
\isamarkupfalse%
\isacommand{proof}\ {\isacharminus}\isanewline
\ \ \isamarkupfalse%
\isacommand{assume}\ {\isachardoublequote}{\isasymturnstile}\ B\ {\isasymlongrightarrow}\isactrlsub D\ C{\isachardoublequote}\isanewline
\ \ \isamarkupfalse%
\isacommand{have}\ {\isachardoublequote}{\isasymturnstile}\ {\isacharparenleft}B\ {\isasymlongrightarrow}\isactrlsub D\ C{\isacharparenright}\ {\isasymlongrightarrow}\isactrlsub D\ A\ {\isasymand}\isactrlsub D\ B\ {\isasymlongrightarrow}\isactrlsub D\ C{\isachardoublequote}\isanewline
\ \ \ \ \isamarkupfalse%
\isacommand{by}\ {\isacharparenleft}simp\ add{\isacharcolon}\ pdl{\isacharunderscore}taut{\isacharparenright}\isanewline
\ \ \isamarkupfalse%
\isacommand{thus}\ {\isacharquery}thesis\ \isamarkupfalse%
\isacommand{by}\ {\isacharparenleft}rule\ pdl{\isacharunderscore}mp{\isacharparenright}\isanewline
\isamarkupfalse%
\isacommand{qed}\isamarkupfalse%
%
\begin{isamarkuptext}%
The following can be used to prove a specific goal by proving two parts separately. It is
  similar to \isa{pdl{\isacharunderscore}iffD{\isadigit{2}}\ {\isacharbrackleft}\ OF\ box{\isacharunderscore}conj{\isacharunderscore}distrib{\isacharunderscore}lifted{\isadigit{1}}\ {\isacharcomma}\ THEN\ pdl{\isacharunderscore}mp\ {\isacharbrackright}}, which is

  \begin{isabelle}%
{\isasymturnstile}\ {\isacharparenleft}A{\isacharunderscore}{\isadigit{2}}\ {\isasymlongrightarrow}\isactrlsub D\ {\isacharbrackleft}{\isacharhash}\ p{\isacharunderscore}{\isadigit{2}}{\isacharbrackright}P{\isacharunderscore}{\isadigit{2}}{\isacharparenright}\ {\isasymand}\isactrlsub D\ {\isacharparenleft}A{\isacharunderscore}{\isadigit{2}}\ {\isasymlongrightarrow}\isactrlsub D\ {\isacharbrackleft}{\isacharhash}\ p{\isacharunderscore}{\isadigit{2}}{\isacharbrackright}Q{\isacharunderscore}{\isadigit{2}}{\isacharparenright}\ {\isasymLongrightarrow}\isanewline
{\isasymturnstile}\ A{\isacharunderscore}{\isadigit{2}}\ {\isasymlongrightarrow}\isactrlsub D\ {\isacharbrackleft}{\isacharhash}\ p{\isacharunderscore}{\isadigit{2}}{\isacharbrackright}{\isacharparenleft}{\isasymlambda}x{\isachardot}\ P{\isacharunderscore}{\isadigit{2}}\ x\ {\isasymand}\isactrlsub D\ Q{\isacharunderscore}{\isadigit{2}}\ x{\isacharparenright}%
\end{isabelle}.%
\end{isamarkuptext}%
\isamarkuptrue%
\isacommand{lemma}\ pdl{\isacharunderscore}conj{\isacharunderscore}imp{\isacharunderscore}box{\isacharunderscore}split{\isacharcolon}\ {\isachardoublequote}{\isasymlbrakk}{\isasymturnstile}\ A\ {\isasymlongrightarrow}\isactrlsub D\ {\isacharbrackleft}{\isacharhash}\ p{\isacharbrackright}C{\isacharsemicolon}\ {\isasymturnstile}\ B\ {\isasymlongrightarrow}\isactrlsub D\ {\isacharbrackleft}{\isacharhash}\ p{\isacharbrackright}D{\isasymrbrakk}\ {\isasymLongrightarrow}\ {\isasymturnstile}\ A\ {\isasymand}\isactrlsub D\ B\ {\isasymlongrightarrow}\isactrlsub D\ {\isacharbrackleft}{\isacharhash}\ x{\isasymleftarrow}p{\isacharbrackright}{\isacharparenleft}C\ x\ {\isasymand}\isactrlsub D\ D\ x{\isacharparenright}{\isachardoublequote}\isanewline
\isamarkupfalse%
\isacommand{proof}\ {\isacharparenleft}rule\ pdl{\isacharunderscore}iffD{\isadigit{2}}{\isacharbrackleft}OF\ box{\isacharunderscore}conj{\isacharunderscore}distrib{\isacharunderscore}lifted{\isadigit{1}}{\isacharcomma}\ THEN\ pdl{\isacharunderscore}mp{\isacharbrackright}{\isacharparenright}\isanewline
\ \ \isamarkupfalse%
\isacommand{assume}\ a{\isadigit{1}}{\isacharcolon}\ {\isachardoublequote}{\isasymturnstile}\ A\ {\isasymlongrightarrow}\isactrlsub D\ {\isacharbrackleft}{\isacharhash}\ p{\isacharbrackright}C{\isachardoublequote}\ \isakeyword{and}\ a{\isadigit{2}}{\isacharcolon}\ {\isachardoublequote}{\isasymturnstile}\ B\ {\isasymlongrightarrow}\isactrlsub D\ {\isacharbrackleft}{\isacharhash}\ p{\isacharbrackright}D{\isachardoublequote}\isanewline
\ \ \isamarkupfalse%
\isacommand{show}\ {\isachardoublequote}{\isasymturnstile}\ {\isacharparenleft}A\ {\isasymand}\isactrlsub D\ B\ {\isasymlongrightarrow}\isactrlsub D\ {\isacharbrackleft}{\isacharhash}\ p{\isacharbrackright}C{\isacharparenright}\ {\isasymand}\isactrlsub D\ {\isacharparenleft}A\ {\isasymand}\isactrlsub D\ B\ {\isasymlongrightarrow}\isactrlsub D\ {\isacharbrackleft}{\isacharhash}\ p{\isacharbrackright}D{\isacharparenright}{\isachardoublequote}\isanewline
\ \ \isamarkupfalse%
\isacommand{proof}\ {\isacharparenleft}rule\ pdl{\isacharunderscore}conjI{\isacharparenright}\isanewline
\ \ \ \ \isamarkupfalse%
\isacommand{show}\ {\isachardoublequote}{\isasymturnstile}\ A\ {\isasymand}\isactrlsub D\ B\ {\isasymlongrightarrow}\isactrlsub D\ {\isacharbrackleft}{\isacharhash}\ p{\isacharbrackright}C{\isachardoublequote}\isanewline
\ \ \ \ \isamarkupfalse%
\isacommand{proof}\ {\isacharparenleft}rule\ pdl{\isacharunderscore}conj{\isacharunderscore}imp{\isacharunderscore}wk{\isadigit{1}}{\isacharparenright}\ \isanewline
\ \ \ \ \ \ \isamarkupfalse%
\isacommand{show}\ {\isachardoublequote}{\isasymturnstile}\ A\ {\isasymlongrightarrow}\isactrlsub D\ {\isacharbrackleft}{\isacharhash}\ p{\isacharbrackright}C{\isachardoublequote}\ \isamarkupfalse%
\isacommand{{\isachardot}}\isanewline
\ \ \ \ \isamarkupfalse%
\isacommand{qed}\isanewline
\ \ \isamarkupfalse%
\isacommand{next}\isanewline
\ \ \ \ \isamarkupfalse%
\isacommand{show}\ {\isachardoublequote}{\isasymturnstile}\ A\ {\isasymand}\isactrlsub D\ B\ {\isasymlongrightarrow}\isactrlsub D\ {\isacharbrackleft}{\isacharhash}\ p{\isacharbrackright}D{\isachardoublequote}\isanewline
\ \ \ \ \isamarkupfalse%
\isacommand{proof}\ {\isacharparenleft}rule\ pdl{\isacharunderscore}conj{\isacharunderscore}imp{\isacharunderscore}wk{\isadigit{2}}{\isacharparenright}\isanewline
\ \ \ \ \ \ \isamarkupfalse%
\isacommand{show}\ {\isachardoublequote}{\isasymturnstile}\ B\ {\isasymlongrightarrow}\isactrlsub D\ {\isacharbrackleft}{\isacharhash}\ p{\isacharbrackright}D{\isachardoublequote}\ \isamarkupfalse%
\isacommand{{\isachardot}}\isanewline
\ \ \ \ \isamarkupfalse%
\isacommand{qed}\isanewline
\ \ \isamarkupfalse%
\isacommand{qed}\isanewline
\isamarkupfalse%
\isacommand{qed}\isamarkupfalse%
%
\begin{isamarkuptext}%
Since dsef programs may be discarded, a formula is equal to itself prefixed
  by such a program.%
\end{isamarkuptext}%
\isamarkuptrue%
\isacommand{lemma}\ dsef{\isacharunderscore}form{\isacharunderscore}eq{\isacharcolon}\ {\isachardoublequote}dsef\ p\ {\isasymLongrightarrow}\ P\ {\isacharequal}\ {\isasymUp}\ {\isacharparenleft}do\ {\isacharbraceleft}a{\isasymleftarrow}p{\isacharsemicolon}\ {\isasymDown}\ P{\isacharbraceright}{\isacharparenright}{\isachardoublequote}\isanewline
\isamarkupfalse%
\isacommand{proof}\ {\isacharminus}\isanewline
\ \ \isamarkupfalse%
\isacommand{assume}\ a{\isadigit{1}}{\isacharcolon}\ {\isachardoublequote}dsef\ p{\isachardoublequote}\isanewline
\ \ \isamarkupfalse%
\isacommand{have}\ f{\isadigit{1}}{\isacharcolon}\ {\isachardoublequote}do\ {\isacharbraceleft}a{\isasymleftarrow}p{\isacharsemicolon}\ {\isasymDown}\ P{\isacharbraceright}\ {\isacharequal}\ {\isasymDown}\ P{\isachardoublequote}\isanewline
\ \ \isamarkupfalse%
\isacommand{proof}\ {\isacharparenleft}rule\ dis{\isacharunderscore}left{\isadigit{2}}{\isacharparenright}\isanewline
\ \ \ \ \isamarkupfalse%
\isacommand{show}\ {\isachardoublequote}dis\ p{\isachardoublequote}\isanewline
\ \ \ \ \ \ \isamarkupfalse%
\isacommand{by}\ {\isacharparenleft}rule\ dsef{\isacharunderscore}dis{\isacharbrackleft}OF\ a{\isadigit{1}}{\isacharbrackright}{\isacharparenright}\isanewline
\ \ \isamarkupfalse%
\isacommand{qed}\isanewline
\ \ \isamarkupfalse%
\isacommand{thus}\ {\isacharquery}thesis\ \isanewline
\ \ \isamarkupfalse%
\isacommand{proof}\ {\isacharminus}\isanewline
\ \ \ \ \isamarkupfalse%
\isacommand{have}\ {\isachardoublequote}P\ \ {\isacharequal}\ {\isasymUp}\ {\isacharparenleft}{\isasymDown}\ P{\isacharparenright}{\isachardoublequote}\isanewline
\ \ \ \ \ \ \isamarkupfalse%
\isacommand{by}\ {\isacharparenleft}rule\ Rep{\isacharunderscore}Dsef{\isacharunderscore}inverse{\isacharbrackleft}symmetric{\isacharbrackright}{\isacharparenright}\isanewline
\ \ \ \ \isamarkupfalse%
\isacommand{with}\ f{\isadigit{1}}\ \isamarkupfalse%
\isacommand{show}\ {\isacharquery}thesis\ \isamarkupfalse%
\isacommand{by}\ simp\isanewline
\ \ \isamarkupfalse%
\isacommand{qed}\isanewline
\isamarkupfalse%
\isacommand{qed}\isamarkupfalse%
%
\begin{isamarkuptext}%
A rendition of \isa{pdl{\isacharunderscore}dsefB}.%
\end{isamarkuptext}%
\isamarkuptrue%
\isacommand{lemma}\ dsefB{\isacharunderscore}D{\isacharcolon}\ {\isachardoublequote}dsef\ p\ {\isasymLongrightarrow}\ {\isasymturnstile}\ P\ {\isasymlongrightarrow}\isactrlsub D\ {\isacharbrackleft}{\isacharhash}\ x{\isasymleftarrow}p{\isacharbrackright}P{\isachardoublequote}\isanewline
\isamarkupfalse%
\isacommand{by}{\isacharparenleft}subst\ dsef{\isacharunderscore}form{\isacharunderscore}eq{\isacharbrackleft}of\ p\ P{\isacharbrackright}{\isacharcomma}\ assumption{\isacharcomma}\ rule\ pdl{\isacharunderscore}iffD{\isadigit{1}}{\isacharbrackleft}OF\ pdl{\isacharunderscore}dsefB{\isacharbrackright}{\isacharparenright}\isamarkupfalse%
%
\begin{isamarkuptext}%
An even number is equal to the sum of its div-halves.%
\end{isamarkuptext}%
\isamarkuptrue%
\isacommand{lemma}\ even{\isacharunderscore}div{\isacharunderscore}eq{\isacharcolon}\ {\isachardoublequote}nat{\isacharunderscore}even\ n\ {\isacharequal}\ {\isacharparenleft}n\ div\ {\isadigit{2}}\ {\isacharplus}\ n\ div\ {\isadigit{2}}\ {\isacharequal}\ n{\isacharparenright}{\isachardoublequote}\isanewline
\ \ \isamarkupfalse%
\isacommand{apply}{\isacharparenleft}unfold\ nat{\isacharunderscore}even{\isacharunderscore}def{\isacharparenright}\isanewline
\ \ \isamarkupfalse%
\isacommand{by}\ arith\isamarkupfalse%
%
\begin{isamarkuptext}%
Dividing $n$ by two and adding the result to itself yields a number one less
  than $n$.%
\end{isamarkuptext}%
\isamarkuptrue%
\isacommand{lemma}\ odd{\isacharunderscore}div{\isacharunderscore}eq{\isacharcolon}\ {\isachardoublequote}nat{\isacharunderscore}odd\ {\isacharparenleft}x{\isacharcolon}{\isacharcolon}nat{\isacharparenright}\ {\isacharequal}\ {\isacharparenleft}x\ div\ {\isadigit{2}}\ {\isacharplus}\ x\ div\ {\isadigit{2}}\ {\isacharplus}\ {\isadigit{1}}\ {\isacharequal}\ x{\isacharparenright}{\isachardoublequote}\isanewline
\ \ \isamarkupfalse%
\isacommand{apply}{\isacharparenleft}simp\ add{\isacharcolon}\ nat{\isacharunderscore}odd{\isacharunderscore}def\ nat{\isacharunderscore}even{\isacharunderscore}def{\isacharparenright}\isanewline
\ \ \isamarkupfalse%
\isacommand{by}\ {\isacharparenleft}arith{\isacharparenright}\isamarkupfalse%
%
\begin{isamarkuptext}%
A slight variant of \isa{pdl{\isacharunderscore}dsefB} for stateless formulas.%
\end{isamarkuptext}%
\isamarkuptrue%
\isacommand{lemma}\ pdl{\isacharunderscore}dsefB{\isacharunderscore}ret{\isacharcolon}\ {\isachardoublequote}dsef\ p\ {\isasymLongrightarrow}\ {\isasymturnstile}\ {\isasymUp}\ {\isacharparenleft}do\ {\isacharbraceleft}a{\isasymleftarrow}p{\isacharsemicolon}\ ret\ {\isacharparenleft}P\ a{\isacharparenright}{\isacharbraceright}{\isacharparenright}\ {\isasymlongleftrightarrow}\isactrlsub D\ {\isacharbrackleft}{\isacharhash}\ a{\isasymleftarrow}p{\isacharbrackright}{\isacharparenleft}Ret\ {\isacharparenleft}P\ a{\isacharparenright}{\isacharparenright}{\isachardoublequote}\isanewline
\ \ \isamarkupfalse%
\isacommand{apply}{\isacharparenleft}subgoal{\isacharunderscore}tac\ {\isachardoublequote}{\isasymforall}a{\isachardot}\ ret\ {\isacharparenleft}P\ a{\isacharparenright}\ {\isacharequal}\ {\isasymDown}\ Ret\ {\isacharparenleft}P\ a{\isacharparenright}{\isachardoublequote}{\isacharparenright}\isanewline
\ \ \isamarkupfalse%
\isacommand{apply}{\isacharparenleft}simp{\isacharparenright}\isanewline
\ \ \isamarkupfalse%
\isacommand{apply}{\isacharparenleft}rule\ pdl{\isacharunderscore}dsefB{\isacharparenright}\isanewline
\ \ \isamarkupfalse%
\isacommand{apply}{\isacharparenleft}assumption{\isacharparenright}\isanewline
\ \ \isamarkupfalse%
\isacommand{apply}{\isacharparenleft}simp\ add{\isacharcolon}\ Ret{\isacharunderscore}ret{\isacharparenright}\isanewline
\isamarkupfalse%
\isacommand{done}\isamarkupfalse%
%
\isamarkupsubsection{Problem-Specific Auxiliary Lemmas%
}
\isamarkuptrue%
%
\begin{isamarkuptext}%
The following lemmas are required for the final correctness proof to go through, but
  are of rather limited interest in general.%
\end{isamarkuptext}%
\isamarkuptrue%
\isacommand{lemma}\ var{\isacharunderscore}aux{\isadigit{1}}{\isacharcolon}\ {\isachardoublequote}{\isasymturnstile}\ {\isacharparenleft}\ {\isacharasterisk}y\ {\isacharequal}\isactrlsub D\ Ret\ b\ {\isasymand}\isactrlsub D\ Ret\ {\isacharparenleft}x\ {\isasymnoteq}\ y\ {\isasymand}\ y\ {\isasymnoteq}\ r\ {\isasymand}\ x\ {\isasymnoteq}\ r{\isacharparenright}\ {\isasymand}\isactrlsub D\ {\isacharparenleft}Ret\ {\isacharparenleft}x\ {\isasymnoteq}\ y{\isacharparenright}\ {\isasymlongrightarrow}\isactrlsub D\ {\isacharasterisk}x\ {\isacharequal}\isactrlsub D\ Ret\ a{\isacharparenright}\ {\isacharparenright}\ {\isasymlongrightarrow}\isactrlsub D\isanewline
\ \ \ \ \ \ \ \ \ \ \ \ \ \ \ {\isacharparenleft}\ {\isacharasterisk}x\ {\isacharequal}\isactrlsub D\ Ret\ a\ {\isasymand}\isactrlsub D\ {\isacharasterisk}y\ {\isacharequal}\isactrlsub D\ Ret\ b\ {\isasymand}\isactrlsub D\ Ret\ {\isacharparenleft}x\ {\isasymnoteq}\ y\ {\isasymand}\ y\ {\isasymnoteq}\ r\ {\isasymand}\ x\ {\isasymnoteq}\ r{\isacharparenright}\ {\isacharparenright}{\isachardoublequote}\isanewline
\ \ \isamarkupfalse%
\isacommand{by}\ {\isacharparenleft}simp\ add{\isacharcolon}\ conjD{\isacharunderscore}Ret{\isacharunderscore}hom\ pdl{\isacharunderscore}taut{\isacharparenright}\isanewline
\isanewline
\isanewline
\isamarkupfalse%
\isacommand{lemma}\ var{\isacharunderscore}aux{\isadigit{2}}{\isacharcolon}\ {\isachardoublequote}{\isasymturnstile}\ {\isacharparenleft}{\isacharparenleft}\ {\isacharasterisk}r\ {\isacharequal}\isactrlsub D\ Ret\ {\isadigit{0}}\ {\isasymand}\isactrlsub D\ Ret\ {\isacharparenleft}x\ {\isasymnoteq}\ y\ {\isasymand}\ y\ {\isasymnoteq}\ r\ {\isasymand}\ x\ {\isasymnoteq}\ r{\isacharparenright}{\isacharparenright}\ {\isasymand}\isactrlsub D\ {\isacharparenleft}Ret\ {\isacharparenleft}x\ {\isasymnoteq}\ r{\isacharparenright}\ {\isasymlongrightarrow}\isactrlsub D\ {\isacharasterisk}x\ {\isacharequal}\isactrlsub D\ Ret\ a{\isacharparenright}{\isacharparenright}\ {\isasymand}\isactrlsub D\isanewline
\ \ \ \ \ \ \ \ \ \ \ \ \ \ \ \ \ \ \ {\isacharparenleft}Ret\ {\isacharparenleft}y\ {\isasymnoteq}\ r{\isacharparenright}\ {\isasymlongrightarrow}\isactrlsub D\ {\isacharasterisk}y\ {\isacharequal}\isactrlsub D\ Ret\ b{\isacharparenright}\ {\isasymlongrightarrow}\isactrlsub D\isanewline
\ \ \ \ \ \ \ \ \ \ \ \ \ \ \ \ \ \ \ {\isacharparenleft}\ {\isacharasterisk}x\ {\isacharequal}\isactrlsub D\ Ret\ a\ {\isasymand}\isactrlsub D\ {\isacharasterisk}y\ {\isacharequal}\isactrlsub D\ Ret\ b\ {\isasymand}\isactrlsub D\ {\isacharasterisk}r\ {\isacharequal}\isactrlsub D\ Ret\ {\isacharparenleft}{\isadigit{0}}{\isacharcolon}{\isacharcolon}nat{\isacharparenright}\ {\isasymand}\isactrlsub D\ Ret\ {\isacharparenleft}x\ {\isasymnoteq}\ y\ {\isasymand}\ y\ {\isasymnoteq}\ r\ {\isasymand}\ x\ {\isasymnoteq}\ r{\isacharparenright}{\isacharparenright}{\isachardoublequote}\isanewline
\ \ \isamarkupfalse%
\isacommand{by}\ {\isacharparenleft}simp\ add{\isacharcolon}\ conjD{\isacharunderscore}Ret{\isacharunderscore}hom\ pdl{\isacharunderscore}taut{\isacharparenright}\isamarkupfalse%
%
\begin{isamarkuptext}%
The following proof it typical: since some formulas are built from do-terms and then lifted
  into \isa{bool\ D}, the usual proof rules will not get us far. The standard scheme in this 
  case is to proceed as documented in the following side remarks.%
\end{isamarkuptext}%
\isamarkuptrue%
\isacommand{lemma}\ derive{\isacharunderscore}inv{\isacharunderscore}aux{\isacharcolon}\ {\isachardoublequote}\ {\isasymturnstile}\ {\isacharasterisk}x\ {\isacharequal}\isactrlsub D\ Ret\ a\ {\isasymand}\isactrlsub D\ {\isacharasterisk}y\ {\isacharequal}\isactrlsub D\ Ret\ b\ {\isasymand}\isactrlsub D\ {\isacharasterisk}r\ {\isacharequal}\isactrlsub D\ Ret\ {\isacharparenleft}{\isadigit{0}}{\isacharcolon}{\isacharcolon}nat{\isacharparenright}\ {\isasymand}\isactrlsub D\ Ret\ {\isacharparenleft}x\ {\isasymnoteq}\ y\ {\isasymand}\ y\ {\isasymnoteq}\ r\ {\isasymand}\ x\ {\isasymnoteq}\ r{\isacharparenright}\ \isanewline
\ \ \ \ \ \ \ \ \ \ \ \ \ \ \ \ \ \ \ \ \ \ \ \ \ {\isasymlongrightarrow}\isactrlsub D\ Ret\ {\isacharparenleft}x\ {\isasymnoteq}\ y\ {\isasymand}\ y\ {\isasymnoteq}\ r\ {\isasymand}\ x\ {\isasymnoteq}\ r{\isacharparenright}\ {\isasymand}\isactrlsub D\ \isanewline
\ \ \ \ \ \ \ \ \ \ \ \ \ \ \ \ \ \ \ \ \ \ \ \ \ \ \ \ \ \ \ {\isasymUp}\ {\isacharparenleft}do\ {\isacharbraceleft}u{\isasymleftarrow}readRef\ x{\isacharsemicolon}\ v{\isasymleftarrow}readRef\ y{\isacharsemicolon}\ w{\isasymleftarrow}readRef\ r{\isacharsemicolon}\ ret\ {\isacharparenleft}u{\isacharasterisk}v{\isacharplus}w\ {\isacharequal}\ a{\isacharasterisk}b{\isacharparenright}{\isacharbraceright}{\isacharparenright}{\isachardoublequote}\isanewline
\ \ {\isacharparenleft}\isakeyword{is}\ {\isachardoublequote}{\isasymturnstile}\ {\isacharquery}x\ {\isasymand}\isactrlsub D\ {\isacharquery}y\ {\isasymand}\isactrlsub D\ {\isacharquery}r\ {\isasymand}\isactrlsub D\ {\isacharquery}diff\ {\isasymlongrightarrow}\isactrlsub D\ {\isacharquery}diff\ {\isasymand}\isactrlsub D\ {\isacharquery}seq{\isachardoublequote}{\isacharparenright}\isanewline
\isamarkupfalse%
\isacommand{proof}\ {\isacharminus}\isanewline
\ \ %
\isamarkupcmt{Simplify the goal by proving something tautologously equivalent.%
}
\isanewline
\ \ \isamarkupfalse%
\isacommand{have}\ {\isachardoublequote}{\isasymturnstile}\ {\isacharparenleft}{\isacharquery}x\ {\isasymand}\isactrlsub D\ {\isacharquery}y\ {\isasymand}\isactrlsub D\ {\isacharquery}r\ {\isasymlongrightarrow}\isactrlsub D\ {\isacharquery}seq{\isacharparenright}\ {\isasymlongrightarrow}\isactrlsub D\isanewline
\ \ \ \ \ \ \ \ \ \ {\isacharparenleft}{\isacharquery}x\ {\isasymand}\isactrlsub D\ {\isacharquery}y\ {\isasymand}\isactrlsub D\ {\isacharquery}r\ {\isasymand}\isactrlsub D\ {\isacharquery}diff\ {\isasymlongrightarrow}\isactrlsub D\ {\isacharquery}diff\ {\isasymand}\isactrlsub D\ {\isacharquery}seq{\isacharparenright}{\isachardoublequote}\ \isamarkupfalse%
\isacommand{by}\ {\isacharparenleft}simp\ add{\isacharcolon}\ pdl{\isacharunderscore}taut{\isacharparenright}\isanewline
\ \ \isamarkupfalse%
\isacommand{moreover}\isanewline
\ \ \isamarkupfalse%
\isacommand{have}\ {\isachardoublequote}{\isasymturnstile}\ {\isacharquery}x\ {\isasymand}\isactrlsub D\ {\isacharquery}y\ {\isasymand}\isactrlsub D\ {\isacharquery}r\ {\isasymlongrightarrow}\isactrlsub D\ {\isacharquery}seq{\isachardoublequote}\isanewline
\ \ \ \ %
\isamarkupcmt{Turn the formula into a straight program sequence%
}
\isanewline
\ \ \ \ \isamarkupfalse%
\isacommand{apply}{\isacharparenleft}simp\ add{\isacharcolon}\ liftM{\isadigit{2}}{\isacharunderscore}def\ impD{\isacharunderscore}def\ conjD{\isacharunderscore}def\ MonEq{\isacharunderscore}def\ dsef{\isacharunderscore}read\ Abs{\isacharunderscore}Dsef{\isacharunderscore}inverse\ Dsef{\isacharunderscore}def\ Ret{\isacharunderscore}ret{\isacharparenright}\isanewline
\ \ \ \ \isamarkupfalse%
\isacommand{apply}{\isacharparenleft}simp\ add{\isacharcolon}\ dsef{\isacharunderscore}read\ Abs{\isacharunderscore}Dsef{\isacharunderscore}inverse\ Dsef{\isacharunderscore}def\ dsef{\isacharunderscore}seq{\isacharparenright}\isanewline
\ \ \ \ \isamarkupfalse%
\isacommand{apply}{\isacharparenleft}simp\ add{\isacharcolon}\ mon{\isacharunderscore}ctr\ del{\isacharcolon}\ bind{\isacharunderscore}assoc{\isacharparenright}\isanewline
\ \ \ \ %
\isamarkupcmt{Sort programs so that equal ones are next to each other%
}
\isanewline
\ \ \ \ \isamarkupfalse%
\isacommand{apply}{\isacharparenleft}simp\ del{\isacharcolon}\ dsef{\isacharunderscore}ret\ add{\isacharcolon}\ commute{\isacharunderscore}dsef{\isacharbrackleft}of\ {\isachardoublequote}readRef\ r{\isachardoublequote}\ {\isachardoublequote}readRef\ x{\isachardoublequote}{\isacharbrackright}\ dsef{\isacharunderscore}read{\isacharparenright}\isanewline
\ \ \ \ \isamarkupfalse%
\isacommand{apply}{\isacharparenleft}simp\ del{\isacharcolon}\ dsef{\isacharunderscore}ret\ add{\isacharcolon}\ commute{\isacharunderscore}dsef{\isacharbrackleft}of\ {\isachardoublequote}readRef\ y{\isachardoublequote}\ {\isachardoublequote}readRef\ x{\isachardoublequote}{\isacharbrackright}\ dsef{\isacharunderscore}read{\isacharparenright}\isanewline
\ \ \ \ \isamarkupfalse%
\isacommand{apply}{\isacharparenleft}simp\ del{\isacharcolon}\ dsef{\isacharunderscore}ret\ add{\isacharcolon}\ commute{\isacharunderscore}dsef{\isacharbrackleft}of\ {\isachardoublequote}readRef\ r{\isachardoublequote}\ {\isachardoublequote}readRef\ y{\isachardoublequote}{\isacharbrackright}\ dsef{\isacharunderscore}read{\isacharparenright}\isanewline
\ \ \ \ %
\isamarkupcmt{Remove duplicate occurrences of all programs%
}
\isanewline
\ \ \ \ \isamarkupfalse%
\isacommand{apply}{\isacharparenleft}simp\ add{\isacharcolon}\ dsef{\isacharunderscore}cp{\isacharbrackleft}OF\ dsef{\isacharunderscore}read{\isacharbrackleft}of\ {\isachardoublequote}x{\isachardoublequote}{\isacharbrackright}{\isacharbrackright}\ cp{\isacharunderscore}arb{\isacharparenright}\isanewline
\ \ \ \ \isamarkupfalse%
\isacommand{apply}{\isacharparenleft}simp\ add{\isacharcolon}\ dsef{\isacharunderscore}cp{\isacharbrackleft}OF\ dsef{\isacharunderscore}read{\isacharbrackleft}of\ {\isachardoublequote}y{\isachardoublequote}{\isacharbrackright}{\isacharbrackright}\ cp{\isacharunderscore}arb{\isacharparenright}\isanewline
\ \ \ \ \isamarkupfalse%
\isacommand{apply}{\isacharparenleft}simp\ add{\isacharcolon}\ dsef{\isacharunderscore}cp{\isacharbrackleft}OF\ dsef{\isacharunderscore}read{\isacharbrackleft}of\ {\isachardoublequote}r{\isachardoublequote}{\isacharbrackright}{\isacharbrackright}\ cp{\isacharunderscore}arb{\isacharparenright}\isanewline
\ \ \ \ %
\isamarkupcmt{Finally prove the returned stateless formula and conclude by reducing 
          the program to \isa{ret\ True}%
}
\isanewline
\ \ \ \ \isamarkupfalse%
\isacommand{apply}{\isacharparenleft}simp\ add{\isacharcolon}\ dsef{\isacharunderscore}dis{\isacharbrackleft}OF\ dsef{\isacharunderscore}read{\isacharbrackright}\ dis{\isacharunderscore}left{\isadigit{2}}{\isacharparenright}\isanewline
\ \ \ \ \isamarkupfalse%
\isacommand{apply}{\isacharparenleft}simp\ add{\isacharcolon}\ Valid{\isacharunderscore}simp\ Abs{\isacharunderscore}Dsef{\isacharunderscore}inverse\ Dsef{\isacharunderscore}def{\isacharparenright}\isanewline
\ \ \ \ \isamarkupfalse%
\isacommand{done}\isanewline
\ \ \isamarkupfalse%
\isacommand{ultimately}\ \isamarkupfalse%
\isacommand{show}\ {\isacharquery}thesis\ \isamarkupfalse%
\isacommand{by}\ {\isacharparenleft}rule\ pdl{\isacharunderscore}mp{\isacharparenright}\isanewline
\isamarkupfalse%
\isacommand{qed}\isanewline
\isanewline
\isanewline
\isamarkupfalse%
\isacommand{lemma}\ doterm{\isacharunderscore}eq{\isadigit{1}}{\isacharunderscore}aux{\isacharcolon}\ {\isachardoublequote}do\ {\isacharbraceleft}u{\isasymleftarrow}readRef\ x{\isacharsemicolon}\ v{\isasymleftarrow}readRef\ y{\isacharsemicolon}\ w{\isasymleftarrow}readRef\ r{\isacharsemicolon}\ ret\ {\isacharparenleft}u\ {\isacharasterisk}\ v\ {\isacharplus}\ w\ {\isacharequal}\ a\ {\isacharasterisk}\ b{\isacharparenright}{\isacharbraceright}\ {\isacharequal}\isanewline
\ \ \ \ \ \ \ \ \ \ \ \ \ \ \ \ \ \ \ \ \ \ \ do\ {\isacharbraceleft}u{\isasymleftarrow}readRef\ x{\isacharsemicolon}\ {\isasymDown}\ {\isacharparenleft}{\isasymUp}\ {\isacharparenleft}do\ {\isacharbraceleft}v{\isasymleftarrow}readRef\ y{\isacharsemicolon}\ w{\isasymleftarrow}readRef\ r{\isacharsemicolon}\ ret\ {\isacharparenleft}u\ {\isacharasterisk}\ v\ {\isacharplus}\ w\ {\isacharequal}\ a\ {\isacharasterisk}\ b{\isacharparenright}{\isacharbraceright}{\isacharparenright}{\isacharparenright}{\isacharbraceright}{\isachardoublequote}\isamarkupfalse%
\isamarkupfalse%
\isamarkupfalse%
\isamarkupfalse%
\isamarkupfalse%
\isamarkupfalse%
\isamarkupfalse%
\isamarkupfalse%
\isamarkupfalse%
\isamarkupfalse%
\isamarkupfalse%
\isanewline
\isanewline
\isamarkupfalse%
\isacommand{lemma}\ doterm{\isacharunderscore}eq{\isadigit{2}}{\isacharunderscore}aux{\isacharcolon}\ {\isachardoublequote}do\ {\isacharbraceleft}v{\isasymleftarrow}readRef\ y{\isacharsemicolon}\ w{\isasymleftarrow}readRef\ r{\isacharsemicolon}\ ret\ {\isacharparenleft}u\ {\isacharasterisk}\ v\ {\isacharplus}\ w\ {\isacharequal}\ a\ {\isacharasterisk}\ b{\isacharparenright}{\isacharbraceright}\ {\isacharequal}\isanewline
\ \ \ \ \ \ \ \ \ \ \ \ \ \ \ \ \ \ \ \ \ \ \ do\ {\isacharbraceleft}v{\isasymleftarrow}readRef\ y{\isacharsemicolon}\ {\isasymDown}\ {\isacharparenleft}{\isasymUp}\ {\isacharparenleft}do\ {\isacharbraceleft}w{\isasymleftarrow}readRef\ r{\isacharsemicolon}\ ret\ {\isacharparenleft}u\ {\isacharasterisk}\ v\ {\isacharplus}\ w\ {\isacharequal}\ a\ {\isacharasterisk}\ b{\isacharparenright}{\isacharbraceright}{\isacharparenright}{\isacharparenright}{\isacharbraceright}{\isachardoublequote}\isamarkupfalse%
\isamarkupfalse%
\isamarkupfalse%
\isamarkupfalse%
\isamarkupfalse%
\isamarkupfalse%
\isamarkupfalse%
\isamarkupfalse%
\isamarkupfalse%
\isamarkupfalse%
\isamarkupfalse%
\isanewline
\isanewline
\isamarkupfalse%
\isacommand{lemma}\ arith{\isacharunderscore}aux{\isacharcolon}\ {\isachardoublequote}{\isasymlbrakk}nat{\isacharunderscore}odd\ u{\isacharsemicolon}\ u\ {\isacharasterisk}\ v\ {\isacharplus}\ w\ {\isacharequal}\ a\ {\isacharasterisk}\ b{\isasymrbrakk}\ {\isasymLongrightarrow}\ {\isacharparenleft}u\ div\ {\isadigit{2}}\ {\isacharplus}\ u\ div\ {\isadigit{2}}{\isacharparenright}\ {\isacharasterisk}\ v\ {\isacharplus}\ {\isacharparenleft}w\ {\isacharplus}\ v{\isacharparenright}\ {\isacharequal}\ a\ {\isacharasterisk}\ b{\isachardoublequote}\isamarkupfalse%
\isamarkupfalse%
\isamarkupfalse%
\isamarkupfalse%
\isamarkupfalse%
\isamarkupfalse%
\isamarkupfalse%
\isamarkupfalse%
\isamarkupfalse%
\isamarkupfalse%
\isamarkupfalse%
\isamarkupfalse%
\isamarkupfalse%
\isamarkupfalse%
\isamarkupfalse%
\isamarkupfalse%
\isamarkupfalse%
\isamarkupfalse%
\isamarkupfalse%
\isamarkupfalse%
\isamarkupfalse%
\isanewline
\isanewline
\isamarkupfalse%
\isacommand{lemma}\ rel{\isadigit{1}}{\isacharunderscore}aux{\isacharcolon}\ {\isachardoublequote}nat{\isacharunderscore}odd\ u\ {\isasymLongrightarrow}\ {\isasymturnstile}\ \ {\isacharparenleft}\ Ret\ {\isacharparenleft}x\ {\isasymnoteq}\ y\ {\isasymand}\ y\ {\isasymnoteq}\ r\ {\isasymand}\ x\ {\isasymnoteq}\ r{\isacharparenright}\ {\isasymand}\isactrlsub D\ {\isacharasterisk}r\ {\isacharequal}\isactrlsub D\ Ret\ {\isacharparenleft}w\ {\isacharplus}\ v{\isacharparenright}\ {\isasymand}\isactrlsub D\ Ret\ {\isacharparenleft}u\ {\isacharasterisk}\ v\ {\isacharplus}\ w\ {\isacharequal}\ a\ {\isacharasterisk}\ b{\isacharparenright}\ {\isacharparenright}\ {\isasymlongrightarrow}\isactrlsub D\isanewline
\ \ \ \ \ \ \ \ \ \ \ \ \ \ \ \ \ \ \ \ Ret\ {\isacharparenleft}x{\isasymnoteq}y\ {\isasymand}\ y{\isasymnoteq}r\ {\isasymand}\ x{\isasymnoteq}r{\isacharparenright}\ {\isasymand}\isactrlsub D\ {\isasymUp}\ {\isacharparenleft}do\ {\isacharbraceleft}w{\isasymleftarrow}readRef\ r{\isacharsemicolon}\ ret\ {\isacharparenleft}{\isacharparenleft}u\ div\ {\isadigit{2}}\ {\isacharplus}\ u\ div\ {\isadigit{2}}{\isacharparenright}\ {\isacharasterisk}\ v\ {\isacharplus}\ w\ {\isacharequal}\ a\ {\isacharasterisk}\ b{\isacharparenright}{\isacharbraceright}{\isacharparenright}{\isachardoublequote}\isanewline
\ \ {\isacharparenleft}\isakeyword{is}\ {\isachardoublequote}{\isacharquery}odd\ {\isasymLongrightarrow}\ {\isasymturnstile}\ {\isacharparenleft}{\isacharquery}diff\ {\isasymand}\isactrlsub D\ {\isacharquery}r\ {\isasymand}\isactrlsub D\ {\isacharquery}ar{\isacharparenright}\ {\isasymlongrightarrow}\isactrlsub D\ {\isacharquery}diff\ {\isasymand}\isactrlsub D\ {\isacharquery}seq{\isachardoublequote}{\isacharparenright}\isamarkupfalse%
\isamarkupfalse%
\isamarkupfalse%
\isamarkupfalse%
\isamarkupfalse%
\isamarkupfalse%
\isamarkupfalse%
\isamarkupfalse%
\isamarkupfalse%
\isamarkupfalse%
\isamarkupfalse%
\isamarkupfalse%
\isamarkupfalse%
\isamarkupfalse%
\isamarkupfalse%
\isamarkupfalse%
\isamarkupfalse%
\isamarkupfalse%
\isamarkupfalse%
\isamarkupfalse%
\isamarkupfalse%
\isamarkupfalse%
\isanewline
\isanewline
\isamarkupfalse%
\isacommand{lemma}\ wrt{\isacharunderscore}other{\isacharunderscore}aux{\isacharcolon}\ {\isachardoublequote}{\isasymturnstile}\ Ret\ {\isacharparenleft}\ x{\isasymnoteq}y\ {\isasymand}\ y{\isasymnoteq}r\ {\isasymand}\ x{\isasymnoteq}r\ {\isacharparenright}\ {\isasymand}\isactrlsub D\ {\isasymUp}\ {\isacharparenleft}do\ {\isacharbraceleft}w{\isasymleftarrow}readRef\ r{\isacharsemicolon}\ ret\ {\isacharparenleft}f\ w{\isacharparenright}{\isacharbraceright}{\isacharparenright}\ {\isasymlongrightarrow}\isactrlsub D\ \isanewline
\ \ \ \ \ \ \ \ \ \ \ \ \ \ \ \ \ \ \ \ \ \ \ \ {\isacharbrackleft}{\isacharhash}\ x\ {\isacharcolon}{\isacharequal}\ a{\isacharbrackright}{\isacharparenleft}{\isasymlambda}uu{\isachardot}\ Ret\ {\isacharparenleft}x{\isasymnoteq}y\ {\isasymand}\ y{\isasymnoteq}r\ {\isasymand}\ x{\isasymnoteq}r{\isacharparenright}\ {\isasymand}\isactrlsub D\ {\isasymUp}\ {\isacharparenleft}do\ {\isacharbraceleft}w{\isasymleftarrow}readRef\ r{\isacharsemicolon}\ ret\ {\isacharparenleft}f\ w{\isacharparenright}{\isacharbraceright}{\isacharparenright}{\isacharparenright}{\isachardoublequote}\isamarkupfalse%
\isamarkupfalse%
\isamarkupfalse%
\isamarkupfalse%
\isamarkupfalse%
\isamarkupfalse%
\isamarkupfalse%
\isanewline
\isanewline
\isamarkupfalse%
\isacommand{lemma}\ wrt{\isacharunderscore}other{\isadigit{2}}{\isacharunderscore}aux{\isacharcolon}\ \ {\isachardoublequote}{\isasymturnstile}\ Ret\ {\isacharparenleft}\ x{\isasymnoteq}y\ {\isasymand}\ y{\isasymnoteq}r\ {\isasymand}\ x{\isasymnoteq}r\ {\isacharparenright}\ {\isasymand}\isactrlsub D\ {\isasymUp}\ {\isacharparenleft}do\ {\isacharbraceleft}w{\isasymleftarrow}readRef\ r{\isacharsemicolon}\ ret\ {\isacharparenleft}f\ w{\isacharparenright}{\isacharbraceright}{\isacharparenright}\ {\isasymlongrightarrow}\isactrlsub D\ \isanewline
\ \ \ \ \ \ \ \ \ \ \ \ \ \ \ \ \ \ \ \ \ \ \ \ {\isacharbrackleft}{\isacharhash}\ y\ {\isacharcolon}{\isacharequal}\ b{\isacharbrackright}{\isacharparenleft}{\isasymlambda}uu{\isachardot}\ Ret\ {\isacharparenleft}x{\isasymnoteq}y\ {\isasymand}\ y{\isasymnoteq}r\ {\isasymand}\ x{\isasymnoteq}r{\isacharparenright}\ {\isasymand}\isactrlsub D\ {\isasymUp}\ {\isacharparenleft}do\ {\isacharbraceleft}w{\isasymleftarrow}readRef\ r{\isacharsemicolon}\ ret\ {\isacharparenleft}f\ w{\isacharparenright}{\isacharbraceright}{\isacharparenright}{\isacharparenright}{\isachardoublequote}\isamarkupfalse%
\isamarkupfalse%
\isamarkupfalse%
\isamarkupfalse%
\isamarkupfalse%
\isamarkupfalse%
\isamarkupfalse%
\isanewline
\isanewline
\isamarkupfalse%
\isacommand{lemma}\ rd{\isacharunderscore}seq{\isacharunderscore}aux{\isacharcolon}\ {\isachardoublequote}{\isasymturnstile}\ {\isasymUp}\ {\isacharparenleft}do\ {\isacharbraceleft}w{\isasymleftarrow}readRef\ r{\isacharsemicolon}\ ret\ {\isacharparenleft}f\ a\ w{\isacharparenright}{\isacharbraceright}{\isacharparenright}\ {\isasymand}\isactrlsub D\ {\isacharasterisk}x\ {\isacharequal}\isactrlsub D\ Ret\ a\ {\isasymlongrightarrow}\isactrlsub D\isanewline
\ \ \ \ \ \ \ \ \ \ \ \ \ \ \ \ \ \ \ \ \ {\isasymUp}\ {\isacharparenleft}do\ {\isacharbraceleft}u{\isasymleftarrow}readRef\ x{\isacharsemicolon}\ w{\isasymleftarrow}readRef\ r{\isacharsemicolon}\ ret\ {\isacharparenleft}f\ u\ w{\isacharparenright}{\isacharbraceright}{\isacharparenright}{\isachardoublequote}\isamarkupfalse%
\isamarkupfalse%
\isamarkupfalse%
\isamarkupfalse%
\isamarkupfalse%
\isamarkupfalse%
\isamarkupfalse%
\isamarkupfalse%
\isamarkupfalse%
\isanewline
\isanewline
\isamarkupfalse%
\isacommand{lemma}\ arith{\isadigit{2}}{\isacharunderscore}aux{\isacharcolon}\ {\isachardoublequote}{\isacharparenleft}u\ div\ {\isacharparenleft}{\isadigit{2}}{\isacharcolon}{\isacharcolon}nat{\isacharparenright}\ {\isacharplus}\ u\ div\ {\isadigit{2}}{\isacharparenright}\ {\isacharasterisk}\ v\ {\isacharplus}\ w\ {\isacharequal}\ a\ {\isacharasterisk}\ b\ {\isasymlongrightarrow}\ u\ div\ {\isadigit{2}}\ {\isacharasterisk}\ {\isacharparenleft}v\ {\isacharasterisk}\ {\isadigit{2}}{\isacharparenright}\ {\isacharplus}\ w\ {\isacharequal}\ a\ {\isacharasterisk}\ b{\isachardoublequote}\isamarkupfalse%
\isamarkupfalse%
\isamarkupfalse%
\isamarkupfalse%
\isamarkupfalse%
\isamarkupfalse%
\isamarkupfalse%
\isamarkupfalse%
\isamarkupfalse%
\isamarkupfalse%
\isamarkupfalse%
\isamarkupfalse%
\isamarkupfalse%
\isamarkupfalse%
\isamarkupfalse%
\isamarkupfalse%
\isamarkupfalse%
\isanewline
\isanewline
\isamarkupfalse%
\isacommand{lemma}\ asm{\isacharunderscore}results{\isacharunderscore}aux{\isacharcolon}\ {\isachardoublequote}\ {\isasymturnstile}\ {\isacharparenleft}Ret\ {\isacharparenleft}x\ {\isasymnoteq}\ y{\isacharparenright}\ {\isasymlongrightarrow}\isactrlsub D\ {\isacharasterisk}x\ {\isacharequal}\isactrlsub D\ Ret\ {\isacharparenleft}u\ div\ {\isacharparenleft}{\isadigit{2}}{\isacharcolon}{\isacharcolon}nat{\isacharparenright}{\isacharparenright}{\isacharparenright}\ {\isasymand}\isactrlsub D\isanewline
\ \ \ \ \ \ \ \ \ {\isacharasterisk}y\ {\isacharequal}\isactrlsub D\ Ret\ {\isacharparenleft}v\ {\isacharasterisk}\ {\isadigit{2}}{\isacharparenright}\ {\isasymand}\isactrlsub D\isanewline
\ \ \ \ \ \ \ \ \ Ret\ {\isacharparenleft}x\ {\isasymnoteq}\ y\ {\isasymand}\ y\ {\isasymnoteq}\ r\ {\isasymand}\ x\ {\isasymnoteq}\ r{\isacharparenright}\ {\isasymand}\isactrlsub D\ {\isasymUp}\ {\isacharparenleft}do\ {\isacharbraceleft}w{\isasymleftarrow}readRef\ r{\isacharsemicolon}\ ret\ {\isacharparenleft}{\isacharparenleft}u\ div\ {\isadigit{2}}\ {\isacharplus}\ u\ div\ {\isadigit{2}}{\isacharparenright}\ {\isacharasterisk}\ v\ {\isacharplus}\ w\ {\isacharequal}\ a\ {\isacharasterisk}\ b{\isacharparenright}{\isacharbraceright}{\isacharparenright}\ {\isasymlongrightarrow}\isactrlsub D\isanewline
\ \ \ \ \ \ \ \ \ Ret\ {\isacharparenleft}x\ {\isasymnoteq}\ y\ {\isasymand}\ y\ {\isasymnoteq}\ r\ {\isasymand}\ x\ {\isasymnoteq}\ r{\isacharparenright}\ {\isasymand}\isactrlsub D\ {\isasymUp}\ {\isacharparenleft}do\ {\isacharbraceleft}u{\isasymleftarrow}readRef\ x{\isacharsemicolon}\ v{\isasymleftarrow}readRef\ y{\isacharsemicolon}\ w{\isasymleftarrow}readRef\ r{\isacharsemicolon}\ ret\ {\isacharparenleft}u\ {\isacharasterisk}\ v\ {\isacharplus}\ w\ {\isacharequal}\ a\ {\isacharasterisk}\ b{\isacharparenright}{\isacharbraceright}{\isacharparenright}{\isachardoublequote}\isamarkupfalse%
\isamarkupfalse%
\isamarkupfalse%
\isamarkupfalse%
\isamarkupfalse%
\isamarkupfalse%
\isamarkupfalse%
\isamarkupfalse%
\isamarkupfalse%
\isamarkupfalse%
\isamarkupfalse%
\isamarkupfalse%
\isamarkupfalse%
\isamarkupfalse%
%
\begin{isamarkuptext}%
Yet another dsef formula extension.%
\end{isamarkuptext}%
\isamarkuptrue%
\isacommand{lemma}\ yadfe{\isacharcolon}\ {\isachardoublequote}\ {\isasymlbrakk}dsef\ p{\isacharsemicolon}\ dsef\ q{\isacharsemicolon}\ dsef\ r{\isacharsemicolon}\ {\isasymforall}x\ y\ z{\isachardot}\ f\ x\ y\ z{\isasymrbrakk}\ {\isasymLongrightarrow}\ {\isasymturnstile}\ {\isasymUp}\ {\isacharparenleft}do\ {\isacharbraceleft}x{\isasymleftarrow}p{\isacharsemicolon}\ y{\isasymleftarrow}q{\isacharsemicolon}\ z{\isasymleftarrow}r{\isacharsemicolon}\ ret\ {\isacharparenleft}f\ x\ y\ z{\isacharparenright}{\isacharbraceright}{\isacharparenright}{\isachardoublequote}\isanewline
\isamarkupfalse%
\isacommand{proof}\ {\isacharminus}\isanewline
\ \ \isamarkupfalse%
\isacommand{assume}\ ds{\isacharcolon}\ {\isachardoublequote}dsef\ p{\isachardoublequote}\ {\isachardoublequote}dsef\ q{\isachardoublequote}\ {\isachardoublequote}dsef\ r{\isachardoublequote}\isanewline
\ \ \isamarkupfalse%
\isacommand{assume}\ a{\isadigit{1}}{\isacharcolon}\ {\isachardoublequote}{\isasymforall}x\ y\ z{\isachardot}\ f\ x\ y\ z{\isachardoublequote}\isanewline
\ \ \isamarkupfalse%
\isacommand{hence}\ {\isachardoublequote}{\isasymDown}\ {\isacharparenleft}{\isasymUp}\ {\isacharparenleft}do\ {\isacharbraceleft}x{\isasymleftarrow}p{\isacharsemicolon}\ y{\isasymleftarrow}q{\isacharsemicolon}\ z{\isasymleftarrow}r{\isacharsemicolon}\ ret\ {\isacharparenleft}f\ x\ y\ z{\isacharparenright}{\isacharbraceright}{\isacharparenright}{\isacharparenright}\ {\isacharequal}\ \isanewline
\ \ \ \ \ \ \ \ \ {\isasymDown}\ {\isacharparenleft}{\isasymUp}\ {\isacharparenleft}do\ {\isacharbraceleft}x{\isasymleftarrow}p{\isacharsemicolon}\ y{\isasymleftarrow}q{\isacharsemicolon}\ z{\isasymleftarrow}r{\isacharsemicolon}\ ret\ True{\isacharbraceright}{\isacharparenright}{\isacharparenright}{\isachardoublequote}\isanewline
\ \ \ \ \isamarkupfalse%
\isacommand{by}\ {\isacharparenleft}simp{\isacharparenright}\isanewline
\ \ \isamarkupfalse%
\isacommand{also}\ \isamarkupfalse%
\isacommand{from}\ ds\ \isamarkupfalse%
\isacommand{have}\ {\isachardoublequote}{\isasymdots}\ {\isacharequal}\ ret\ True{\isachardoublequote}\ \isanewline
\ \ \ \ \isamarkupfalse%
\isacommand{by}\ {\isacharparenleft}simp\ add{\isacharcolon}\ Abs{\isacharunderscore}Dsef{\isacharunderscore}inverse\ Dsef{\isacharunderscore}def\ dsef{\isacharunderscore}seq\ dis{\isacharunderscore}left{\isadigit{2}}\ dsef{\isacharunderscore}dis{\isacharparenright}\isanewline
\ \ \isamarkupfalse%
\isacommand{finally}\ \isamarkupfalse%
\isacommand{show}\ {\isacharquery}thesis\ \isamarkupfalse%
\isacommand{by}\ {\isacharparenleft}simp\ add{\isacharcolon}\ Valid{\isacharunderscore}simp{\isacharparenright}\isanewline
\isamarkupfalse%
\isacommand{qed}\isanewline
\isanewline
\isanewline
\isamarkupfalse%
\isacommand{lemma}\ conclude{\isacharunderscore}aux{\isacharcolon}\ {\isachardoublequote}\ {\isasymturnstile}\ {\isacharparenleft}Ret\ {\isacharparenleft}x\ {\isasymnoteq}\ y\ {\isasymand}\ y\ {\isasymnoteq}\ r\ {\isasymand}\ x\ {\isasymnoteq}\ r{\isacharparenright}\ {\isasymand}\isactrlsub D\ \isanewline
\ \ \ \ \ \ \ \ \ {\isasymUp}\ {\isacharparenleft}do\ {\isacharbraceleft}u{\isasymleftarrow}readRef\ x{\isacharsemicolon}\ v{\isasymleftarrow}readRef\ y{\isacharsemicolon}\ w{\isasymleftarrow}readRef\ r{\isacharsemicolon}\ ret\ {\isacharparenleft}u\ {\isacharasterisk}\ v\ {\isacharplus}\ w\ {\isacharequal}\ {\isacharparenleft}a{\isacharcolon}{\isacharcolon}nat{\isacharparenright}\ {\isacharasterisk}\ b{\isacharparenright}{\isacharbraceright}{\isacharparenright}{\isacharparenright}\ {\isasymand}\isactrlsub D\isanewline
\ \ \ \ \ \ \ \ \ {\isasymnot}\isactrlsub D\ {\isasymUp}\ {\isacharparenleft}do\ {\isacharbraceleft}u{\isasymleftarrow}readRef\ x{\isacharsemicolon}\ ret\ {\isacharparenleft}{\isadigit{0}}\ {\isacharless}\ u{\isacharparenright}{\isacharbraceright}{\isacharparenright}\ {\isasymlongrightarrow}\isactrlsub D\isanewline
\ \ \ \ \ \ \ \ \ {\isacharbrackleft}{\isacharhash}\ readRef\ r{\isacharbrackright}{\isacharparenleft}{\isasymlambda}x{\isachardot}\ Ret\ {\isacharparenleft}x\ {\isacharequal}\ a\ {\isacharasterisk}\ b{\isacharparenright}{\isacharparenright}{\isachardoublequote}\isamarkupfalse%
\isamarkupfalse%
\isamarkupfalse%
\isamarkupfalse%
\isamarkupfalse%
\isamarkupfalse%
\isamarkupfalse%
\isamarkupfalse%
\isamarkupfalse%
\isamarkupfalse%
\isamarkupfalse%
\isamarkupfalse%
\isamarkupfalse%
\isamarkupfalse%
\isamarkupfalse%
\isamarkupfalse%
%
\isamarkupsubsection{Correctness of Russian Multiplication%
}
\isamarkuptrue%
%
\begin{isamarkuptext}%
Equipped with all these prerequisites, the correctness proof of Russian multiplication
  is `at your fingertips'\texttrademark. We will not display the actual rule applications but
  only the important proof goals arising in between.
  \label{isa:rumult-proof}%
\end{isamarkuptext}%
\isamarkuptrue%
\isacommand{theorem}\ russian{\isacharunderscore}mult{\isacharcolon}\ {\isachardoublequote}{\isasymturnstile}\ {\isacharparenleft}Ret\ {\isacharparenleft}\ x{\isasymnoteq}y\ {\isasymand}\ y{\isasymnoteq}r\ {\isasymand}\ x{\isasymnoteq}r{\isacharparenright}{\isacharparenright}\ {\isasymlongrightarrow}\isactrlsub D\ {\isacharbrackleft}{\isacharhash}\ rumult\ a\ b\ x\ y\ r{\isacharbrackright}{\isacharparenleft}{\isasymlambda}x{\isachardot}\ Ret\ {\isacharparenleft}x\ {\isacharequal}\ a\ {\isacharasterisk}\ b{\isacharparenright}{\isacharparenright}{\isachardoublequote}\isanewline
\ \ \isamarkupfalse%
\isacommand{apply}{\isacharparenleft}unfold\ rumult{\isacharunderscore}def{\isacharparenright}\ %
\isamarkupcmt{First, unfold the definition of \isa{rumult}%
}
\isanewline
\ \ \isamarkupfalse%
\isacommand{apply}{\isacharparenleft}simp\ only{\isacharcolon}\ seq{\isacharunderscore}def{\isacharparenright}\isanewline
\ \ \isamarkupfalse%
\isacommand{apply}{\isacharparenleft}rule\ pdl{\isacharunderscore}plugB{\isacharunderscore}lifted{\isadigit{1}}{\isacharparenright}\isamarkupfalse%
%
\begin{isamarkuptxt}%
Establish the `strongest postcondition' of the assignment to \isa{x}

    \begin{isabelle}%
{\isasymturnstile}\ Ret\ {\isacharparenleft}x\ {\isasymnoteq}\ y\ {\isasymand}\ y\ {\isasymnoteq}\ r\ {\isasymand}\ x\ {\isasymnoteq}\ r{\isacharparenright}\ {\isasymlongrightarrow}\isactrlsub D\ {\isacharbrackleft}{\isacharhash}\ rumult\ a\ b\ x\ y\ r{\isacharbrackright}{\isacharparenleft}{\isasymlambda}x{\isachardot}\ Ret\ {\isacharparenleft}x\ {\isacharequal}\ a\ {\isacharasterisk}\ b{\isacharparenright}{\isacharparenright}\isanewline
\ {\isadigit{1}}{\isachardot}\ {\isasymturnstile}\ Ret\ {\isacharparenleft}x\ {\isasymnoteq}\ y\ {\isasymand}\ y\ {\isasymnoteq}\ r\ {\isasymand}\ x\ {\isasymnoteq}\ r{\isacharparenright}\ {\isasymlongrightarrow}\isactrlsub D\ {\isacharbrackleft}{\isacharhash}\ x\ {\isacharcolon}{\isacharequal}\ a{\isacharbrackright}{\isacharquery}B%
\end{isabelle}%
\end{isamarkuptxt}%
\isamarkuptrue%
\isamarkupfalse%
\isamarkupfalse%
\isamarkupfalse%
\isamarkupfalse%
\isamarkupfalse%
\isamarkupfalse%
\isamarkupfalse%
%
\begin{isamarkuptxt}%
From this postcondition proceed with assignment to \isa{y}

      \begin{isabelle}%
{\isasymturnstile}\ Ret\ {\isacharparenleft}x\ {\isasymnoteq}\ y\ {\isasymand}\ y\ {\isasymnoteq}\ r\ {\isasymand}\ x\ {\isasymnoteq}\ r{\isacharparenright}\ {\isasymlongrightarrow}\isactrlsub D\ {\isacharbrackleft}{\isacharhash}\ rumult\ a\ b\ x\ y\ r{\isacharbrackright}{\isacharparenleft}{\isasymlambda}x{\isachardot}\ Ret\ {\isacharparenleft}x\ {\isacharequal}\ a\ {\isacharasterisk}\ b{\isacharparenright}{\isacharparenright}\isanewline
\ {\isadigit{1}}{\isachardot}\ {\isasymAnd}xa{\isachardot}\ {\isasymturnstile}\ Ret\ {\isacharparenleft}x\ {\isasymnoteq}\ y\ {\isasymand}\ y\ {\isasymnoteq}\ r\ {\isasymand}\ x\ {\isasymnoteq}\ r{\isacharparenright}\ {\isasymand}\isactrlsub D\ {\isacharasterisk}x\ {\isacharequal}\isactrlsub D\ Ret\ a\ {\isasymlongrightarrow}\isactrlsub D\ {\isacharbrackleft}{\isacharhash}\ y\ {\isacharcolon}{\isacharequal}\ b{\isacharbrackright}{\isacharquery}B{\isadigit{9}}\ xa%
\end{isabelle}%
\end{isamarkuptxt}%
\isamarkuptrue%
\isamarkupfalse%
\isamarkupfalse%
\isamarkupfalse%
\isamarkupfalse%
\isamarkupfalse%
\isamarkupfalse%
\isamarkupfalse%
\isamarkupfalse%
\isamarkupfalse%
\isamarkupfalse%
\isamarkupfalse%
\isamarkupfalse%
\isamarkupfalse%
\isamarkupfalse%
%
\begin{isamarkuptxt}%
After the final assignment to \isa{r} all variables will have their initial values

    \begin{isabelle}%
{\isasymturnstile}\ Ret\ {\isacharparenleft}x\ {\isasymnoteq}\ y\ {\isasymand}\ y\ {\isasymnoteq}\ r\ {\isasymand}\ x\ {\isasymnoteq}\ r{\isacharparenright}\ {\isasymlongrightarrow}\isactrlsub D\ {\isacharbrackleft}{\isacharhash}\ rumult\ a\ b\ x\ y\ r{\isacharbrackright}{\isacharparenleft}{\isasymlambda}x{\isachardot}\ Ret\ {\isacharparenleft}x\ {\isacharequal}\ a\ {\isacharasterisk}\ b{\isacharparenright}{\isacharparenright}\isanewline
\ {\isadigit{1}}{\isachardot}\ {\isasymAnd}xa\ xaa{\isachardot}\isanewline
\isaindent{\ {\isadigit{1}}{\isachardot}\ \ \ \ }{\isasymturnstile}\ {\isacharasterisk}x\ {\isacharequal}\isactrlsub D\ Ret\ a\ {\isasymand}\isactrlsub D\ {\isacharasterisk}y\ {\isacharequal}\isactrlsub D\ Ret\ b\ {\isasymand}\isactrlsub D\ Ret\ {\isacharparenleft}x\ {\isasymnoteq}\ y\ {\isasymand}\ y\ {\isasymnoteq}\ r\ {\isasymand}\ x\ {\isasymnoteq}\ r{\isacharparenright}\ {\isasymlongrightarrow}\isactrlsub D\isanewline
\isaindent{\ {\isadigit{1}}{\isachardot}\ \ \ \ {\isasymturnstile}\ }{\isacharbrackleft}{\isacharhash}\ r\ {\isacharcolon}{\isacharequal}\ {\isadigit{0}}{\isacharbrackright}{\isacharquery}B{\isadigit{2}}{\isadigit{7}}\ xa\ xaa%
\end{isabelle}%
\end{isamarkuptxt}%
\isamarkuptrue%
\isamarkupfalse%
\isamarkupfalse%
\isamarkupfalse%
\isamarkupfalse%
\isamarkupfalse%
\isamarkupfalse%
\isamarkupfalse%
\isamarkupfalse%
\isamarkupfalse%
\isamarkupfalse%
\isamarkupfalse%
\isamarkupfalse%
\isamarkupfalse%
\isamarkupfalse%
\isamarkupfalse%
\isamarkupfalse%
\isamarkupfalse%
\isamarkupfalse%
\isamarkupfalse%
%
\begin{isamarkuptxt}%
Now we have arrived at the while-loop, with the invariant readily established.

      \begin{isabelle}%
{\isasymturnstile}\ Ret\ {\isacharparenleft}x\ {\isasymnoteq}\ y\ {\isasymand}\ y\ {\isasymnoteq}\ r\ {\isasymand}\ x\ {\isasymnoteq}\ r{\isacharparenright}\ {\isasymlongrightarrow}\isactrlsub D\ {\isacharbrackleft}{\isacharhash}\ rumult\ a\ b\ x\ y\ r{\isacharbrackright}{\isacharparenleft}{\isasymlambda}x{\isachardot}\ Ret\ {\isacharparenleft}x\ {\isacharequal}\ a\ {\isacharasterisk}\ b{\isacharparenright}{\isacharparenright}\isanewline
\ {\isadigit{1}}{\isachardot}\ {\isasymAnd}xa\ xaa\ xb{\isachardot}\isanewline
\isaindent{\ {\isadigit{1}}{\isachardot}\ \ \ \ }{\isasymturnstile}\ Ret\ {\isacharparenleft}x\ {\isasymnoteq}\ y\ {\isasymand}\ y\ {\isasymnoteq}\ r\ {\isasymand}\ x\ {\isasymnoteq}\ r{\isacharparenright}\ {\isasymand}\isactrlsub D\isanewline
\isaindent{\ {\isadigit{1}}{\isachardot}\ \ \ \ {\isasymturnstile}\ }{\isasymUp}\ {\isacharparenleft}do\ {\isacharbraceleft}u{\isasymleftarrow}readRef\ x{\isacharsemicolon}\isanewline
\isaindent{\ {\isadigit{1}}{\isachardot}\ \ \ \ {\isasymturnstile}\ {\isasymUp}\ {\isacharparenleft}do\ {\isacharbraceleft}}v{\isasymleftarrow}readRef\ y{\isacharsemicolon}\ w{\isasymleftarrow}readRef\ r{\isacharsemicolon}\ ret\ {\isacharparenleft}u\ {\isacharasterisk}\ v\ {\isacharplus}\ w\ {\isacharequal}\ a\ {\isacharasterisk}\ b{\isacharparenright}{\isacharbraceright}{\isacharparenright}\ {\isasymlongrightarrow}\isactrlsub D\isanewline
\isaindent{\ {\isadigit{1}}{\isachardot}\ \ \ \ {\isasymturnstile}\ }{\isacharbrackleft}{\isacharhash}\ do\ {\isacharbraceleft}x{\isasymleftarrow}WHILE\ {\isasymUp}\ {\isacharparenleft}do\ {\isacharbraceleft}u{\isasymleftarrow}readRef\ x{\isacharsemicolon}\ ret\ {\isacharparenleft}{\isadigit{0}}\ {\isacharless}\ u{\isacharparenright}{\isacharbraceright}{\isacharparenright}\ \isanewline
\isaindent{\ {\isadigit{1}}{\isachardot}\ \ \ \ {\isasymturnstile}\ {\isacharbrackleft}{\isacharhash}\ do\ {\isacharbraceleft}x{\isasymleftarrow}}DO\ do\ {\isacharbraceleft}u{\isasymleftarrow}readRef\ x{\isacharsemicolon}\isanewline
\isaindent{\ {\isadigit{1}}{\isachardot}\ \ \ \ {\isasymturnstile}\ {\isacharbrackleft}{\isacharhash}\ do\ {\isacharbraceleft}x{\isasymleftarrow}DO\ do\ {\isacharbraceleft}}v{\isasymleftarrow}readRef\ y{\isacharsemicolon}\isanewline
\isaindent{\ {\isadigit{1}}{\isachardot}\ \ \ \ {\isasymturnstile}\ {\isacharbrackleft}{\isacharhash}\ do\ {\isacharbraceleft}x{\isasymleftarrow}DO\ do\ {\isacharbraceleft}}w{\isasymleftarrow}readRef\ r{\isacharsemicolon}\isanewline
\isaindent{\ {\isadigit{1}}{\isachardot}\ \ \ \ {\isasymturnstile}\ {\isacharbrackleft}{\isacharhash}\ do\ {\isacharbraceleft}x{\isasymleftarrow}DO\ do\ {\isacharbraceleft}}xa{\isasymleftarrow}if\ nat{\isacharunderscore}odd\ u\ then\ r\ {\isacharcolon}{\isacharequal}\ w\ {\isacharplus}\ v\ else\ ret\ {\isacharparenleft}{\isacharparenright}{\isacharsemicolon}\isanewline
\isaindent{\ {\isadigit{1}}{\isachardot}\ \ \ \ {\isasymturnstile}\ {\isacharbrackleft}{\isacharhash}\ do\ {\isacharbraceleft}x{\isasymleftarrow}DO\ do\ {\isacharbraceleft}}x{\isasymleftarrow}x\ {\isacharcolon}{\isacharequal}\ u\ div\ {\isadigit{2}}{\isacharsemicolon}\ y\ {\isacharcolon}{\isacharequal}\ v\ {\isacharasterisk}\ {\isadigit{2}}{\isacharbraceright}\ \isanewline
\isaindent{\ {\isadigit{1}}{\isachardot}\ \ \ \ {\isasymturnstile}\ {\isacharbrackleft}{\isacharhash}\ do\ {\isacharbraceleft}x{\isasymleftarrow}}END{\isacharsemicolon}\isanewline
\isaindent{\ {\isadigit{1}}{\isachardot}\ \ \ \ {\isasymturnstile}\ {\isacharbrackleft}{\isacharhash}\ do\ {\isacharbraceleft}}readRef\ r{\isacharbraceright}{\isacharbrackright}{\isacharparenleft}{\isasymlambda}x{\isachardot}\ Ret\ {\isacharparenleft}x\ {\isacharequal}\ a\ {\isacharasterisk}\ b{\isacharparenright}{\isacharparenright}%
\end{isabelle}%
\end{isamarkuptxt}%
\ \ \isamarkuptrue%
\isacommand{apply}{\isacharparenleft}rule\ pdl{\isacharunderscore}plugB{\isacharunderscore}lifted{\isadigit{1}}{\isacharparenright}\isanewline
\ \ \ \ \isamarkupfalse%
\isacommand{apply}{\isacharparenleft}rule\ while{\isacharunderscore}par{\isacharparenright}\ \ %
\isamarkupcmt{applied the while rule%
}
\isamarkupfalse%
%
\begin{isamarkuptxt}%
After splitting off the while-loop as a single box formula, we can apply the while
      rule, so that we obtain the following proof goal, telling us to establish the invariant after
      one run of the loop body:

      \begin{isabelle}%
{\isasymturnstile}\ Ret\ {\isacharparenleft}x\ {\isasymnoteq}\ y\ {\isasymand}\ y\ {\isasymnoteq}\ r\ {\isasymand}\ x\ {\isasymnoteq}\ r{\isacharparenright}\ {\isasymlongrightarrow}\isactrlsub D\ {\isacharbrackleft}{\isacharhash}\ rumult\ a\ b\ x\ y\ r{\isacharbrackright}{\isacharparenleft}{\isasymlambda}x{\isachardot}\ Ret\ {\isacharparenleft}x\ {\isacharequal}\ a\ {\isacharasterisk}\ b{\isacharparenright}{\isacharparenright}\isanewline
\ {\isadigit{1}}{\isachardot}\ {\isasymAnd}xa\ xaa\ xb{\isachardot}\isanewline
\isaindent{\ {\isadigit{1}}{\isachardot}\ \ \ \ }{\isasymturnstile}\ {\isacharparenleft}Ret\ {\isacharparenleft}x\ {\isasymnoteq}\ y\ {\isasymand}\ y\ {\isasymnoteq}\ r\ {\isasymand}\ x\ {\isasymnoteq}\ r{\isacharparenright}\ {\isasymand}\isactrlsub D\isanewline
\isaindent{\ {\isadigit{1}}{\isachardot}\ \ \ \ {\isasymturnstile}\ {\isacharparenleft}}{\isasymUp}\ {\isacharparenleft}do\ {\isacharbraceleft}u{\isasymleftarrow}readRef\ x{\isacharsemicolon}\isanewline
\isaindent{\ {\isadigit{1}}{\isachardot}\ \ \ \ {\isasymturnstile}\ {\isacharparenleft}{\isasymUp}\ {\isacharparenleft}do\ {\isacharbraceleft}}v{\isasymleftarrow}readRef\ y{\isacharsemicolon}\ w{\isasymleftarrow}readRef\ r{\isacharsemicolon}\ ret\ {\isacharparenleft}u\ {\isacharasterisk}\ v\ {\isacharplus}\ w\ {\isacharequal}\ a\ {\isacharasterisk}\ b{\isacharparenright}{\isacharbraceright}{\isacharparenright}{\isacharparenright}\ {\isasymand}\isactrlsub D\isanewline
\isaindent{\ {\isadigit{1}}{\isachardot}\ \ \ \ {\isasymturnstile}\ }{\isasymUp}\ {\isacharparenleft}do\ {\isacharbraceleft}u{\isasymleftarrow}readRef\ x{\isacharsemicolon}\ ret\ {\isacharparenleft}{\isadigit{0}}\ {\isacharless}\ u{\isacharparenright}{\isacharbraceright}{\isacharparenright}\ {\isasymlongrightarrow}\isactrlsub D\isanewline
\isaindent{\ {\isadigit{1}}{\isachardot}\ \ \ \ {\isasymturnstile}\ }{\isacharbrackleft}{\isacharhash}\ do\ {\isacharbraceleft}u{\isasymleftarrow}readRef\ x{\isacharsemicolon}\isanewline
\isaindent{\ {\isadigit{1}}{\isachardot}\ \ \ \ {\isasymturnstile}\ {\isacharbrackleft}{\isacharhash}\ do\ {\isacharbraceleft}}v{\isasymleftarrow}readRef\ y{\isacharsemicolon}\isanewline
\isaindent{\ {\isadigit{1}}{\isachardot}\ \ \ \ {\isasymturnstile}\ {\isacharbrackleft}{\isacharhash}\ do\ {\isacharbraceleft}}w{\isasymleftarrow}readRef\ r{\isacharsemicolon}\isanewline
\isaindent{\ {\isadigit{1}}{\isachardot}\ \ \ \ {\isasymturnstile}\ {\isacharbrackleft}{\isacharhash}\ do\ {\isacharbraceleft}}xa{\isasymleftarrow}if\ nat{\isacharunderscore}odd\ u\ then\ r\ {\isacharcolon}{\isacharequal}\ w\ {\isacharplus}\ v\ else\ ret\ {\isacharparenleft}{\isacharparenright}{\isacharsemicolon}\isanewline
\isaindent{\ {\isadigit{1}}{\isachardot}\ \ \ \ {\isasymturnstile}\ {\isacharbrackleft}{\isacharhash}\ do\ {\isacharbraceleft}}x{\isasymleftarrow}x\ {\isacharcolon}{\isacharequal}\ u\ div\ {\isadigit{2}}{\isacharsemicolon}\isanewline
\isaindent{\ {\isadigit{1}}{\isachardot}\ \ \ \ {\isasymturnstile}\ {\isacharbrackleft}{\isacharhash}\ do\ {\isacharbraceleft}}y\ {\isacharcolon}{\isacharequal}\ v\ {\isacharasterisk}\isanewline
\isaindent{\ {\isadigit{1}}{\isachardot}\ \ \ \ {\isasymturnstile}\ {\isacharbrackleft}{\isacharhash}\ do\ {\isacharbraceleft}y\ {\isacharcolon}{\isacharequal}\ }{\isadigit{2}}{\isacharbraceright}{\isacharbrackright}{\isacharparenleft}{\isasymlambda}u{\isachardot}\ Ret\ {\isacharparenleft}x\ {\isasymnoteq}\ y\ {\isasymand}\ y\ {\isasymnoteq}\ r\ {\isasymand}\ x\ {\isasymnoteq}\ r{\isacharparenright}\ {\isasymand}\isactrlsub D\isanewline
\isaindent{\ {\isadigit{1}}{\isachardot}\ \ \ \ {\isasymturnstile}\ {\isacharbrackleft}{\isacharhash}\ do\ {\isacharbraceleft}y\ {\isacharcolon}{\isacharequal}\ {\isadigit{2}}{\isacharbraceright}{\isacharbrackright}{\isacharparenleft}{\isasymlambda}u{\isachardot}\ }{\isasymUp}\ {\isacharparenleft}do\ {\isacharbraceleft}u{\isasymleftarrow}readRef\ x{\isacharsemicolon}\isanewline
\isaindent{\ {\isadigit{1}}{\isachardot}\ \ \ \ {\isasymturnstile}\ {\isacharbrackleft}{\isacharhash}\ do\ {\isacharbraceleft}y\ {\isacharcolon}{\isacharequal}\ {\isadigit{2}}{\isacharbraceright}{\isacharbrackright}{\isacharparenleft}{\isasymlambda}u{\isachardot}\ {\isasymUp}\ {\isacharparenleft}do\ {\isacharbraceleft}}v{\isasymleftarrow}readRef\ y{\isacharsemicolon}\isanewline
\isaindent{\ {\isadigit{1}}{\isachardot}\ \ \ \ {\isasymturnstile}\ {\isacharbrackleft}{\isacharhash}\ do\ {\isacharbraceleft}y\ {\isacharcolon}{\isacharequal}\ {\isadigit{2}}{\isacharbraceright}{\isacharbrackright}{\isacharparenleft}{\isasymlambda}u{\isachardot}\ {\isasymUp}\ {\isacharparenleft}do\ {\isacharbraceleft}}w{\isasymleftarrow}readRef\ r{\isacharsemicolon}\ ret\ {\isacharparenleft}u\ {\isacharasterisk}\ v\ {\isacharplus}\ w\ {\isacharequal}\ a\ {\isacharasterisk}\ b{\isacharparenright}{\isacharbraceright}{\isacharparenright}{\isacharparenright}%
\end{isabelle}%
\end{isamarkuptxt}%
\isamarkuptrue%
\isamarkupfalse%
\isamarkupfalse%
\isamarkupfalse%
\isamarkupfalse%
\isamarkupfalse%
\isamarkupfalse%
\isamarkupfalse%
\isamarkupfalse%
\isamarkupfalse%
\isamarkupfalse%
\isamarkupfalse%
\isamarkupfalse%
\isamarkupfalse%
\isamarkupfalse%
\isamarkupfalse%
\isamarkupfalse%
\isamarkupfalse%
\isamarkupfalse%
\isamarkupfalse%
\isamarkupfalse%
\isamarkupfalse%
\isamarkupfalse%
\isamarkupfalse%
\isamarkupfalse%
\isamarkupfalse%
\isamarkupfalse%
\isamarkupfalse%
\isamarkupfalse%
\isamarkupfalse%
\isamarkupfalse%
\isamarkupfalse%
\isamarkupfalse%
\isamarkupfalse%
\isamarkupfalse%
\isamarkupfalse%
\isamarkupfalse%
\isamarkupfalse%
\isamarkupfalse%
\isamarkupfalse%
\isamarkupfalse%
\isamarkupfalse%
%
\begin{isamarkuptxt}%
After having worked off all read operations, we again have to establish the strongest
      postcondition that is required after the if-statement.

      \begin{isabelle}%
{\isasymturnstile}\ Ret\ {\isacharparenleft}x\ {\isasymnoteq}\ y\ {\isasymand}\ y\ {\isasymnoteq}\ r\ {\isasymand}\ x\ {\isasymnoteq}\ r{\isacharparenright}\ {\isasymlongrightarrow}\isactrlsub D\ {\isacharbrackleft}{\isacharhash}\ rumult\ a\ b\ x\ y\ r{\isacharbrackright}{\isacharparenleft}{\isasymlambda}x{\isachardot}\ Ret\ {\isacharparenleft}x\ {\isacharequal}\ a\ {\isacharasterisk}\ b{\isacharparenright}{\isacharparenright}\isanewline
\ {\isadigit{1}}{\isachardot}\ {\isasymAnd}u\ v\ w{\isachardot}\isanewline
\isaindent{\ {\isadigit{1}}{\isachardot}\ \ \ \ }{\isasymturnstile}\ Ret\ {\isacharparenleft}{\isadigit{0}}\ {\isacharless}\ u{\isacharparenright}\ {\isasymand}\isactrlsub D\isanewline
\isaindent{\ {\isadigit{1}}{\isachardot}\ \ \ \ {\isasymturnstile}\ }Ret\ {\isacharparenleft}x\ {\isasymnoteq}\ y\ {\isasymand}\ y\ {\isasymnoteq}\ r\ {\isasymand}\ x\ {\isasymnoteq}\ r{\isacharparenright}\ {\isasymand}\isactrlsub D\isanewline
\isaindent{\ {\isadigit{1}}{\isachardot}\ \ \ \ {\isasymturnstile}\ }Ret\ {\isacharparenleft}u\ {\isacharasterisk}\ v\ {\isacharplus}\ w\ {\isacharequal}\ a\ {\isacharasterisk}\ b{\isacharparenright}\ {\isasymand}\isactrlsub D\isanewline
\isaindent{\ {\isadigit{1}}{\isachardot}\ \ \ \ {\isasymturnstile}\ }{\isasymUp}\ {\isacharparenleft}do\ {\isacharbraceleft}w{\isasymleftarrow}readRef\ r{\isacharsemicolon}\ ret\ {\isacharparenleft}u\ {\isacharasterisk}\ v\ {\isacharplus}\ w\ {\isacharequal}\ a\ {\isacharasterisk}\ b{\isacharparenright}{\isacharbraceright}{\isacharparenright}\ {\isasymlongrightarrow}\isactrlsub D\isanewline
\isaindent{\ {\isadigit{1}}{\isachardot}\ \ \ \ {\isasymturnstile}\ }{\isacharbrackleft}{\isacharhash}\ if\ nat{\isacharunderscore}odd\ u\ then\ r\ {\isacharcolon}{\isacharequal}\ w\ {\isacharplus}\ v\ else\ ret\ {\isacharparenleft}{\isacharparenright}{\isacharbrackright}{\isacharquery}B{\isadigit{1}}{\isadigit{1}}{\isadigit{1}}\ u\ v\ w%
\end{isabelle}%
\end{isamarkuptxt}%
\isamarkuptrue%
\isamarkupfalse%
\isamarkupfalse%
\isamarkupfalse%
\isamarkupfalse%
\isamarkupfalse%
\isamarkupfalse%
\isamarkupfalse%
\isamarkupfalse%
\isamarkupfalse%
\isamarkupfalse%
\isamarkupfalse%
\isamarkupfalse%
\isamarkupfalse%
\isamarkupfalse%
\isamarkupfalse%
\isamarkupfalse%
\isamarkupfalse%
\isamarkupfalse%
\isamarkupfalse%
\isamarkupfalse%
\isamarkupfalse%
\isamarkupfalse%
%
\begin{isamarkuptxt}%
Here we see what the just mentioned postcondition looks like: it says that the following
      relation (found in the premiss of the implication) holds:
      
      \begin{isabelle}%
{\isasymturnstile}\ Ret\ {\isacharparenleft}x\ {\isasymnoteq}\ y\ {\isasymand}\ y\ {\isasymnoteq}\ r\ {\isasymand}\ x\ {\isasymnoteq}\ r{\isacharparenright}\ {\isasymlongrightarrow}\isactrlsub D\ {\isacharbrackleft}{\isacharhash}\ rumult\ a\ b\ x\ y\ r{\isacharbrackright}{\isacharparenleft}{\isasymlambda}x{\isachardot}\ Ret\ {\isacharparenleft}x\ {\isacharequal}\ a\ {\isacharasterisk}\ b{\isacharparenright}{\isacharparenright}\isanewline
\ {\isadigit{1}}{\isachardot}\ {\isasymAnd}u\ v\ w\ xa{\isachardot}\isanewline
\isaindent{\ {\isadigit{1}}{\isachardot}\ \ \ \ }{\isasymturnstile}\ Ret\ {\isacharparenleft}x\ {\isasymnoteq}\ y\ {\isasymand}\ y\ {\isasymnoteq}\ r\ {\isasymand}\ x\ {\isasymnoteq}\ r{\isacharparenright}\ {\isasymand}\isactrlsub D\isanewline
\isaindent{\ {\isadigit{1}}{\isachardot}\ \ \ \ {\isasymturnstile}\ }{\isasymUp}\ {\isacharparenleft}do\ {\isacharbraceleft}w{\isasymleftarrow}readRef\ r{\isacharsemicolon}\ ret\ {\isacharparenleft}{\isacharparenleft}u\ div\ {\isadigit{2}}\ {\isacharplus}\ u\ div\ {\isadigit{2}}{\isacharparenright}\ {\isacharasterisk}\ v\ {\isacharplus}\ w\ {\isacharequal}\ a\ {\isacharasterisk}\ b{\isacharparenright}{\isacharbraceright}{\isacharparenright}\ {\isasymlongrightarrow}\isactrlsub D\isanewline
\isaindent{\ {\isadigit{1}}{\isachardot}\ \ \ \ {\isasymturnstile}\ }{\isacharbrackleft}{\isacharhash}\ x\ {\isacharcolon}{\isacharequal}\ u\ div\ {\isadigit{2}}{\isacharbrackright}{\isacharquery}B{\isadigit{1}}{\isadigit{4}}{\isadigit{2}}\ u\ v\ w\ xa%
\end{isabelle}%
\end{isamarkuptxt}%
\isamarkuptrue%
\isamarkupfalse%
\isamarkupfalse%
\isamarkupfalse%
\isamarkupfalse%
\isamarkupfalse%
\isamarkupfalse%
\isamarkupfalse%
%
\begin{isamarkuptxt}%
Now only the assignment to \isa{y} remains.

      \begin{isabelle}%
{\isasymturnstile}\ Ret\ {\isacharparenleft}x\ {\isasymnoteq}\ y\ {\isasymand}\ y\ {\isasymnoteq}\ r\ {\isasymand}\ x\ {\isasymnoteq}\ r{\isacharparenright}\ {\isasymlongrightarrow}\isactrlsub D\ {\isacharbrackleft}{\isacharhash}\ rumult\ a\ b\ x\ y\ r{\isacharbrackright}{\isacharparenleft}{\isasymlambda}x{\isachardot}\ Ret\ {\isacharparenleft}x\ {\isacharequal}\ a\ {\isacharasterisk}\ b{\isacharparenright}{\isacharparenright}\isanewline
\ {\isadigit{1}}{\isachardot}\ {\isasymAnd}u\ v\ w\ xa\ xaa{\isachardot}\isanewline
\isaindent{\ {\isadigit{1}}{\isachardot}\ \ \ \ }{\isasymturnstile}\ {\isacharasterisk}x\ {\isacharequal}\isactrlsub D\ Ret\ {\isacharparenleft}u\ div\ {\isadigit{2}}{\isacharparenright}\ {\isasymand}\isactrlsub D\isanewline
\isaindent{\ {\isadigit{1}}{\isachardot}\ \ \ \ {\isasymturnstile}\ }Ret\ {\isacharparenleft}x\ {\isasymnoteq}\ y\ {\isasymand}\ y\ {\isasymnoteq}\ r\ {\isasymand}\ x\ {\isasymnoteq}\ r{\isacharparenright}\ {\isasymand}\isactrlsub D\isanewline
\isaindent{\ {\isadigit{1}}{\isachardot}\ \ \ \ {\isasymturnstile}\ }{\isasymUp}\ {\isacharparenleft}do\ {\isacharbraceleft}w{\isasymleftarrow}readRef\ r{\isacharsemicolon}\ ret\ {\isacharparenleft}{\isacharparenleft}u\ div\ {\isadigit{2}}\ {\isacharplus}\ u\ div\ {\isadigit{2}}{\isacharparenright}\ {\isacharasterisk}\ v\ {\isacharplus}\ w\ {\isacharequal}\ a\ {\isacharasterisk}\ b{\isacharparenright}{\isacharbraceright}{\isacharparenright}\ {\isasymlongrightarrow}\isactrlsub D\isanewline
\isaindent{\ {\isadigit{1}}{\isachardot}\ \ \ \ {\isasymturnstile}\ }{\isacharbrackleft}{\isacharhash}\ y\ {\isacharcolon}{\isacharequal}\ v\ {\isacharasterisk}\ {\isadigit{2}}{\isacharbrackright}{\isacharquery}B{\isadigit{1}}{\isadigit{5}}{\isadigit{1}}\ u\ v\ w\ xa\ xaa%
\end{isabelle}%
\end{isamarkuptxt}%
\isamarkuptrue%
\isamarkupfalse%
\isamarkupfalse%
\isamarkupfalse%
\isamarkupfalse%
\isamarkupfalse%
\isamarkupfalse%
\isamarkupfalse%
\isamarkupfalse%
\isamarkupfalse%
%
\begin{isamarkuptxt}%
We finally succeeded in re-establishing the loop invariant after one
      execution of the loop
      body. The final part is just to read reference \isa{r}, which is easily done.
      
       \begin{isabelle}%
{\isasymturnstile}\ Ret\ {\isacharparenleft}x\ {\isasymnoteq}\ y\ {\isasymand}\ y\ {\isasymnoteq}\ r\ {\isasymand}\ x\ {\isasymnoteq}\ r{\isacharparenright}\ {\isasymlongrightarrow}\isactrlsub D\ {\isacharbrackleft}{\isacharhash}\ rumult\ a\ b\ x\ y\ r{\isacharbrackright}{\isacharparenleft}{\isasymlambda}x{\isachardot}\ Ret\ {\isacharparenleft}x\ {\isacharequal}\ a\ {\isacharasterisk}\ b{\isacharparenright}{\isacharparenright}\isanewline
\ {\isadigit{1}}{\isachardot}\ {\isasymAnd}xa\ xaa\ xb\ xc{\isachardot}\isanewline
\isaindent{\ {\isadigit{1}}{\isachardot}\ \ \ \ }{\isasymturnstile}\ {\isacharparenleft}Ret\ {\isacharparenleft}x\ {\isasymnoteq}\ y\ {\isasymand}\ y\ {\isasymnoteq}\ r\ {\isasymand}\ x\ {\isasymnoteq}\ r{\isacharparenright}\ {\isasymand}\isactrlsub D\isanewline
\isaindent{\ {\isadigit{1}}{\isachardot}\ \ \ \ {\isasymturnstile}\ {\isacharparenleft}}{\isasymUp}\ {\isacharparenleft}do\ {\isacharbraceleft}u{\isasymleftarrow}readRef\ x{\isacharsemicolon}\isanewline
\isaindent{\ {\isadigit{1}}{\isachardot}\ \ \ \ {\isasymturnstile}\ {\isacharparenleft}{\isasymUp}\ {\isacharparenleft}do\ {\isacharbraceleft}}v{\isasymleftarrow}readRef\ y{\isacharsemicolon}\ w{\isasymleftarrow}readRef\ r{\isacharsemicolon}\ ret\ {\isacharparenleft}u\ {\isacharasterisk}\ v\ {\isacharplus}\ w\ {\isacharequal}\ a\ {\isacharasterisk}\ b{\isacharparenright}{\isacharbraceright}{\isacharparenright}{\isacharparenright}\ {\isasymand}\isactrlsub D\isanewline
\isaindent{\ {\isadigit{1}}{\isachardot}\ \ \ \ {\isasymturnstile}\ }{\isasymnot}\isactrlsub D\ {\isasymUp}\ {\isacharparenleft}do\ {\isacharbraceleft}u{\isasymleftarrow}readRef\ x{\isacharsemicolon}\ ret\ {\isacharparenleft}{\isadigit{0}}\ {\isacharless}\ u{\isacharparenright}{\isacharbraceright}{\isacharparenright}\ {\isasymlongrightarrow}\isactrlsub D\isanewline
\isaindent{\ {\isadigit{1}}{\isachardot}\ \ \ \ {\isasymturnstile}\ }{\isacharbrackleft}{\isacharhash}\ readRef\ r{\isacharbrackright}{\isacharparenleft}{\isasymlambda}x{\isachardot}\ Ret\ {\isacharparenleft}x\ {\isacharequal}\ a\ {\isacharasterisk}\ b{\isacharparenright}{\isacharparenright}%
\end{isabelle}%
\end{isamarkuptxt}%
\ \ \isamarkuptrue%
\isacommand{apply}{\isacharparenleft}rule\ conclude{\isacharunderscore}aux{\isacharparenright}\ \ %
\isamarkupcmt{\dots Just 124 straightforward proof steps later%
}
\isanewline
\isamarkupfalse%
\isacommand{done}\isanewline
\isanewline
\isanewline
\isamarkupfalse%
\isacommand{end}\isanewline
\isanewline
\isamarkupfalse%
\end{isabellebody}%
%%% Local Variables:
%%% mode: latex
%%% TeX-master: "root"
%%% End:


%%% Local Variables:
%%% mode: latex
%%% TeX-master: "root"
%%% End:




%%% Local Variables: 
%%% mode: latex
%%% TeX-master: "main"
%%% End: 


\hyphenation{well-be-haved-ness}
\hyphenation{mon-adic}
\hyphenation{assign-able}
% Mathematical ``environments'' for definitions, proofs, etc.

\theoremstyle{plain}
\newtheorem{thm}{Theorem}[chapter]
\newtheorem{lem}[thm]{Lemma}
\newtheorem{prop}[thm]{Proposition}
\newtheorem{cor}[thm]{Corollary}

\theoremstyle{definition}
\newtheorem{defn}[thm]{Definition}
\newtheorem{expl}[thm]{Example}

\theoremstyle{remark}
\newtheorem{rem}[thm]{Remark}


\numberwithin{equation}{chapter}


\newenvironment{ttscript}[1]
  {\begin{list}{}{%
        \settowidth{\labelwidth}{\texttt{#1}}
        \setlength{\leftmargin}{\labelwidth}
        \addtolength{\leftmargin}{\labelsep}
        \setlength{\parsep}{0.5ex plus0.2ex minus0.2ex}
        \setlength{\itemsep}{0.3ex}
        \renewcommand{\makelabel}[1]{$\bullet$ \texttt{##1}\hfill}}}
    {\end{list}}

               

% figures, separated below by a rule
\newenvironment{myfigure}
    {\begin{figure}\newcommand{\mylinesep}{\centerline{\rule{0.9\textwidth}{0.4pt}}}}
    {\end{figure}}


% An environment for case distinctions (to be used mainly in proofs, defns,
% etc.)
\newenvironment{listcase}
               {\begin{list}{}
                   {%
                     \renewcommand{\makelabel}[1]{[ ##1 ]\hfill}}}
                 {\end{list}}


\newenvironment{isabelleenv} %
               {\begin{displaymath}\begin{array}{l}\rule{\textwidth}{0mm}\\} %
               {\end{array}\end{displaymath}}

% standard phrases

\newcommand{\IE}{\mbox{i.\,e.}\xspace}
\newcommand{\EG}{\mbox{e.\,g.}\xspace}
\newcommand{\TM}{\texttrademark\xspace}

\newcommand{\comment}[1]{\marginpar{\color{red}#1}}

% unknown whether it really makes sense .. .
\newcommand{\Eat}[1]{}

% Isabelle stuff
\newcommand{\Isabelle}{\textrm{Isabelle}\xspace}
\newcommand{\IsabelleHOL}{\textrm{Isabelle/HOL}\xspace}
\newcommand{\IsabelleIsar}{\textrm{Isabelle/Isar}\xspace}
\newcommand{\IsaMap}[2]{::\, #1 \Rightarrow #2}
\newcommand{\IsaMapAB}[3]{::\, #1 \Rightarrow #2 \Rightarrow #3}
\newcommand{\irule}[1]{\mbox{\emph{#1}}\xspace}
\newcommand{\ruleref}[1]{\hyperlink{isathm:#1}{\mbox{\emph{#1}}}}
\newcommand{\ivar}[1]{{'\!\mathit{#1}}}
\newcommand{\ifun}[1]{\mbox{\emph{#1}}}

% Highlight an initial paragraph segment
\newcommand{\parhilite}[1]{\noindent \textbf{#1}}


% PDL symbols
%\newcommand{\Gdj}[2]{\boldsymbol{[}\hspace{-2.75pt}\boldsymbol{[}#1\boldsymbol{]}%
%  \hspace{-2.75pt}\boldsymbol{]}\,#2}
\newcommand{\Gdj}[2]{[#1]_G\,#2}
\newcommand{\PDLBox}[1]{[#1]\,}
\newcommand{\PDLSBox}[1]{[\#\;#1]\,}
\newcommand{\PDLDmd}[1]{\langle#1\rangle\,}
\newcommand{\xpbox}[1]{\PDLBox{x\leteq p}{#1}}
\newcommand{\xpdmd}[1]{\PDLDmd{x\leteq p}{#1}}
\newcommand{\gbox}{\square\hspace{-6.5pt}\raisebox{2pt}{\tiny{G}}\;}
\newcommand{\PHTriple}[3]{\{#1\}\:#2\:\{#3\}}
\newcommand{\THTriple}[3]{[#1]\:#2\:[#3]}
\newcommand{\EPHTriple}[4]{\{#1\}\:#2\:\{#3\:\|\:#4\}}
\newcommand{\ETHTriple}[4]{[#1]\:#2\:[#3\:\|\:#4]}
\newcommand{\xp}{x \leteq p}

\newcommand{\wasmathsf}[1]{\mathsf{#1}}
% mathematical / functional notation:
%\newcommand{\defeq}{\stackrel{\mathrm{def}}{\equiv}}
\newcommand{\defeq}{\equiv_{\mathrm{def}}}
\newcommand{\datop}[2]{\genfrac{}{}{0pt}{0}{#1}{#2}}
\newcommand{\True}{\wasmathsf{True}}
\newcommand{\False}{\wasmathsf{False}}
\newcommand{\Let}{\wasmathsf{let}}
\newcommand{\In}{\wasmathsf{in}}
\newcommand{\IfTerm}[3]{\wasmathsf{if}\ #1\ \wasmathsf{then}\ #2\
  \wasmathsf{else}\ #3}
\newcommand{\WhileTerm}[2]{\mathit{while}\ #1\ #2}
\newcommand{\If}{\wasmathsf{if}}
\newcommand{\Then}{\wasmathsf{then}}
\newcommand{\Else}{\wasmathsf{else}}
\newcommand{\While}{\mathit{while}}
\newcommand{\Case}{\wasmathsf{case}}
\newcommand{\Of}{\wasmathsf{of}}
\newcommand{\DO}{\wasmathsf{do}}
\newcommand{\DoStmt}[1]{\wasmathsf{do}\: \{#1\}}
\newcommand{\Type}[1]{\mathit{#1}}
\newcommand{\leteq}{\!\gets\!}
\newcommand{\NT}[1]{\mathit{#1}}
\newcommand{\Alt}{\;|\;}
\newcommand{\Rule}[2]{\frac{\displaystyle #1}{\displaystyle #2}}
\newcommand{\NRule}[3]{(\mathbf{#1})\quad\Rule{#2}{#3}}
\newcommand{\GapFrac}[2]{\frac{\rule{0em}{2.3ex}#1}{\rule{0mm}{2.3ex}#2}}
\newcommand{\LambdaTerm}[2]{{\lambda #1.\,#2}}
\newcommand{\Arg}{\_\!\_}
\newcommand{\Cat}[1]{\mathbf{#1}}
\newcommand{\Fun}[1]{\mathit{#1}}
\newcommand{\Map}[2]{:\, #1 \to #2}
\newcommand{\MapAB}[3]{:\, #1 \to #2 \to #3}
\newcommand{\Id}{{id}}
\newcommand{\bnd}{\ensuremath{.\;}}
\newcommand{\bdot}{.\:}
\newcommand{\unit}{\ast}
\newcommand{\rd}[1]{{\ast#1}}

%% \newcommand{\defn}[1]{\textsc{#1}}
\newcommand{\ibox}[1]{\mbox{\itshape #1}}

\newcommand{\code}[1]{\texttt{#1}}
\newcommand{\AssertDsef}[1]{\mathrm{dsef}(#1)}


% Functions
\newcommand{\bindOp}{\gg=}
\newcommand{\seqOp}{\gg}
\newcommand{\op}[1]{\mathit{#1}}
\newcommand{\inl}{\mathit{inl}}
\newcommand{\inr}{\mathit{inr}}
\newcommand{\Suc}{\mathit{Suc}}
\newcommand{\Ret}{\mathit{Ret}}
\newcommand{\chld}{\mathit{chld}}
\newcommand{\enq}{\mathit{enq}}
\newcommand{\enqAll}{\mathit{enqAll}}
\newcommand{\deqAll}{\mathit{deqAll}}
\newcommand{\deq}{\mathit{deq}}
\newcommand{\ret}{\mathit{ret}}
\newcommand{\get}{\mathit{get}}
\newcommand{\last}{\mathit{last}}
\newcommand{\relq}{\mathit{relq}}
\newcommand{\inq}{\mathit{inq}}
\newcommand{\qmt}{\mathit{empty}}
\newcommand{\raiseEx}{\mathit{raise}}
\newcommand{\catchEx}{\mathit{catch}}
\newcommand{\mbody}{\mathit{mbody}}
% defining some colors
%\definecolor{red}{rgb}{1.0,0,0}


%%% Local Variables: 
%%% mode: latex
%%% TeX-master: "main"
%%% End: 

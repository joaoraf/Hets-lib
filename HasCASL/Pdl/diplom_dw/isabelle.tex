
\chapter{The Theorem Prover Isabelle}
\label{cha:isabelle}

\Isabelle is an interactive theorem proving environment, \IE an assistant for
performing formal proofs. The fact that \Isabelle is generic in the sense that
it allows one to define and reason within several kinds of logics distinguishes
it from most other proof assistants. Examples of logics that have been defined
within \Isabelle's framework are classical first-order logic (FOL), constructive
type theory (CTT), or higher-order logic (HOL) which constitutes
the base logic in our development of monadic dynamic logic.

We will now introduce the foundations of \Isabelle which are the so called
meta-logic, its syntax and inference rules. We then introduce higher-order logic
as formalised in \Isabelle. Finally, we provide insight into basic proof methods
whose knowledge is necessary to comprehend or at least read printed \Isabelle
proofs. A full account of all facilities that were applied cannot be given in
this thesis; very readable introductions to \Isabelle and \IsabelleIsar can be
found in \cite{Nipkow03,IsabelleHOL}

But first, a note about terminology and the development of \Isabelle is in
order: Initially, communicating with \Isabelle meant sequentially applying ML
functions, since \Isabelle is written in this functional language. This user
interface has recently been discharged in favour of an independent proof and
theory language called \emph{Isar}, making proofs substantially more readable
(and maintainable). The combination of \Isabelle with Isar is named
\IsabelleIsar, which becomes Isabelle/Isar/HOL when referring to the
specific logic HOL, expressed in Isar. In the following, we will often use the
term \Isabelle for all these phrases, stating once and for all that the formal
proofs in this thesis are presented in \IsabelleIsar with HOL as the underlying
logic.


\section{The Meta-logic}
\label{sec:meta-logic}

\Isabelle lets the user define his own logics, so that he does not have to work
within a fixed logic that might not suit his needs. In doing so, one needs some
means to express the syntax of one's newly defined logic, to express inference
rules, and to impose side conditions on these rules. Take the following natural
deduction rule governing the introduction of the $\forall$-quantifier as an
example: 
\begin{equation}
  \Rule{P(x)}{\forall x.\,P(x)} \quad \text{($x$ not free in assumptions)}
\end{equation}
The annotation `$x$ not free in $\ldots$' is a very typical side condition,
while the horizontal bar expresses a possible logical inference from the
premisses (displayed above the bar) to the conclusion (below the bar).

Besides determining the basic syntax of all definable logics, it is the task of
the \emph{meta-logic} to enable the formulation of such `meta-logical'
constructs, \IE to formalise properties of concrete object-logics. Put shortly,
the meta-logic is an intuitionistic higher-order logic with polymorphic
functions in the style of ML or Haskell that possesses a universal quantifier,
implication and equality as its constants.


\subsection{Basic Syntax and Terminology}
\label{sec:meta-basic-syntax}

The meta-logic is syntactically based on the simply typed lambda calculus as
described in Section \ref{sec:adding-types} (although without product types).
The additional possibility to define polymorphic functions means that function
types may contain \emph{type variables}, \EG the identity function $\Id
\Map{\alpha}{\alpha}$ exists for \emph{every} type $\alpha$.  \emph{Type declarations} allow
the introduction of new base types, whereas \emph{type classes} may be seen as
collections of types that share some structure (a well-known example is the
class $\Type{ord}$, which the types with a notion of order among their elements
belong to). The latter concept comes close to Haskell's type classes, but is not
powerful enough to embrace Haskell's constructor classes as well. In particular,
the notion of a type constructor being an instance of a monad cannot be
specified in \Isabelle. A remark about how this problem has been resolved in the
implementation can be found in Section~\ref{sec:monads-isabelle}.

Some peculiarities of \Isabelle's syntax should be noted before proceeding: 
\begin{itemize}
\item The base type of truth values is named $\Type{prop}$.
\item Type annotations are denoted by two successive colons instead of one.
\item Function types may be built from existing types by means of the function
  type constructor $\Rightarrow$, such that $f \IsaMap{\sigma}{\tau}$ is \Isabelle's notation for
  $f\Map{\sigma}{\tau}$. The type constructor $\Rightarrow$ associates to the right.
\item The types of curried functions taking $n$ arguments, $f :: \sigma_1 \Rightarrow \cdots \Rightarrow \sigma_n \Rightarrow
  \sigma$ may be written in a list-like notation $f :: [\sigma_1,\ldots,\sigma_n] \Rightarrow \sigma$.
\item Type variables are written as Latin letters prefixed with an apostrophe
  ($'$), \EG $\ivar{a}$, $\ivar{b}$, $\ivar{b_1}$ are type variables. Inside
  normal text we will however not use this style.
\end{itemize}

The constants of the meta-logic are a universal quantifier, (denoted by the
symbol $\bigwedge$), implication ($\Longrightarrow$\footnote{note the difference between this symbol
  and the shorter one for the function type constructor $\Rightarrow$; both however
  associate to the right and there also is a list-like notation for repeated
  implication of the form $\llbracket\phi_1;\ldots;\phi_n\rrbracket \Longrightarrow \psi$}) and equality ($\equiv$). An interesting
property of higher-order logics that spring from the lambda calculus is the fact
that no variable binders other than $\lambda$ are needed: predicates are simply
interpreted as functions into truth values (\EG a predicate on the type
$\Type{nat}$ of natural numbers might be expressed as a function $P
\Map{\Type{nat}}{\Type{prop}}$), and quantifiers are interpreted as higher-order
functions from predicates to truth values. Thus, the type of the universal
quantifier is
\begin{equation}
  \label{eq:univ-quant-type}
  \bigwedge\nolimits_\alpha \IsaMap{(\alpha \Rightarrow \Type{prop})}{\Type{prop}}
\end{equation}
for each type $\alpha$; the polymorphism of \Isabelle is restricted in the same way
as in ML or Haskell in that it does not allow higher-order functions to take
polymorphic functions as arguments. This is made explicit here by indexing the
quantifier with the appropriate type under consideration.


\subsection{Defining Logics}
\label{sec:defining-logics}

Users are not expected to work within the meta-logic itself, but rather to
formalise their own logics by extending the meta-logic through the introduction
of new types and constants and through axioms capturing the properties of these
constants. An example is given in Section \ref{sec:higher-order-logic}, where
the formalisation of HOL within the meta-logic is described. The outline of such
a formalisation is as follows:
\begin{enumerate}
\item Introduce a new type for truth values, thereby distinguishing it from the
  type of truth values of the meta-logic. Furthermore introduce a predicate
  $\Fun{Trueprop}$ converting from object-level truth to meta-level truth; it
  has proved sensible to keep these two kinds of truth values apart. Other
  useful types may be added as well, of course.
\item Name and assign types to the constants that will serve as basic functions
  of the logic to be defined; examples include propositional connectives $\land$,
  $\longrightarrow$, etc., or even modal operators. It is possible to decorate constants with
  concrete syntax (by so called \emph{mixfix annotations}, cf. \cite{IsaRef04})
  that makes operations more readable than is possible with the minimalistic
  syntax of the lambda calculus. One way or the other, functions of the
  respective object-logic conventionally have higher precedence than those of
  the meta-logic.
\item Extend the meta-logic by further axioms that capture the properties of
  these constants and types. The basic idea is that axioms of the meta-logic are to be
  interpreted as rules in the object logic. For example, the typical
  rules for conjunction introduction and universal generalisation in first-order
  logic 
  \[ \Rule{P\quad Q}{P \land Q}\qquad \Rule{P\, x}{\forall x\bdot P\, x}\;\text{($x$ not
    free in assumptions)}\]
  might be formalised as 
  \[ \llbracket P; Q\rrbracket \Longrightarrow P \land Q\quad\text{and}\quad (\bigwedge x\bdot P\, x) \Longrightarrow \forall x\bdot P\, x \]
\end{enumerate}
Proofs from rules within the object-logic are then basically proofs from
corresponding axioms within the meta-logic.


\subsection{Meta-logic Rules}
\label{sec:meta-logic-rules}

To perform such proofs inside the meta-logic, a collection of meta-rules is
necessary. These rules are hard-wired into \Isabelle, which means they are
implemented as ML functions operating on meta-logic terms rather than being
terms of the meta-logic itself. A complete exposition of these rules
can be found in \cite[Section 2.4]{Paulson89}, which we do not repeat here,
since the meta-rules are virtually never applied in proofs inside object-logics.
Instead, we merely summarise the rules, giving an idea of the relative
compactness of the meta-logic. 

The meta-rules can roughly be put into three
categories:
\begin{enumerate}
\item Introduction and elimination rules for the constants $\bigwedge$, $\Longrightarrow$
and $\equiv$; 
\item Rules concerning lambda terms; put concretely, there is a rule for
$\alpha$-conversion, a rule for $\beta$-reduction admitting the conclusion $a[b/x]$ from
the premiss $(\LambdaTerm{x}{a})\ b$, and a rule of extensionality; 
\item  Finally,
there are basic rules for equality.
\end{enumerate}

\section{Higher-order Logic (HOL)}
\label{sec:higher-order-logic}

In this section we introduce the formalisation of the simply typed higher-order
logic HOL. The outstanding feature of higher-order logics is their
capability of expressing higher-order functions (in a sense similar to that of
functional programming languages), but also of expressing predicates and
quantification on arbitrarily typed terms. For example, one may state the
property of a set $S$ being infinite by expressing that there is an injective
function from $S$ into a proper subset $S' \subset S$:
\[
S\;\mathit{infinite} \quad\text{iff}\quad \exists S'\bnd S' \subset S\land\exists f :: S \Rightarrow S'\bnd f\
\mathit{injective}
\]
Because of the quantification on the function $f$ this statement is inherently
higher-order; it cannot even be expressed equivalently in first-order languages.
In HOL all functions are required to
be total; an extension incorporating concepts from domain theory that allows the
formulation of arbitrary computable functions is HOLCF
\cite{MuellerNipkowOheimbSlotosch}.  For in-depth descriptions of higher-order
logic and its implementation in \Isabelle, see \cite{Andrews00,IsabelleHOL}.


\subsection{Constants}
\label{sec:hol-constants}

HOL as implemented in \Isabelle extends the meta-logic by a number of constants
that are to be interpreted as the usual logical connectives, like conjunction,
universal quantification, or boolean case distinction (the familiar
\emph{if-then-else} construct). Differing from the notation used so far,
implication is denoted by a simple long arrow $\longrightarrow$. Some of the operations come
in two flavours, namely their functional form (as actual constants in the lambda
calculus of the meta-logic) and with some syntactical sugaring; Table
\ref{tab:hol-const-typ} lists the most important ones. The function $\Fun{The}$
is a \emph{definite description operator}; $\op{THE}~x\bdot P\ x$ is meant to be
interpreted as ``the $x$, such that $P\ x$ holds'' and will yield an arbitrary
value of the appropriate type if no such $x$ exists.  The interpretation of the
remaining functions and values is standard, but one should note that
quantification exists for arbitrary types, just as equality, \emph{if-then-else}
and \emph{let} do.
\begin{table}
  \centering \renewcommand{\arraystretch}{1.3}
  \begin{tabular}{|l|l|l|}\hline
    \emph{Constant}       & \emph{Term} & \emph{written as}\\\hline
    $\Fun{Not}\IsaMap{\Type{bool}}{\Type{bool}}$ & $\Fun{Not}\ P$ & $\lnot
      P$\\\hline
    $\op{True} :: \Type{bool}$ & &\\\hline
    $\op{False} :: \Type{bool}$ & &\\\hline
    $\Fun{If} \IsaMap{[\Type{bool}, 'a, 'a]}{'a}$ & $\Fun{If}\ b\ p\ q$ &
      $\If b\Then p \Else q$\\\hline
    $\Fun{The} \IsaMap{('a \Rightarrow \Type{bool})}{'a}$ & 
      $\Fun{The}\ P$ & $\ \op{THE}~ x\bdot P\ x$\\\hline
    $\Fun{All}\IsaMap{('a \Rightarrow \Type{bool})}{\Type{bool}}$ &
      $\Fun{All}\ P$ & $\forall x\bdot P\ x$\\\hline
    $\Fun{Ex}\IsaMap{('a \Rightarrow \Type{bool})}{\Type{bool}}$ &
      $\Fun{Ex}\ P$ & $\exists x\bdot P\ x$\\\hline
    $\Fun{Let}\IsaMap{['a, 'a \Rightarrow 'b]}{'b}$ & 
      $\Fun{Let}\ t\ \LambdaTerm{x}{e}$ & $\Let x = t \In e$\\\hline
    $= \IsaMap{['a, 'a]}{\Type{bool}}$ & 
      $a = b$ & \\\hline
    $\land, \lor, \longrightarrow\; \IsaMap{[\Type{bool}, \Type{bool}]}{\Type{bool}}$ &
      $P \land Q,\; \text{etc.}$ &\\\hline
  \end{tabular}
  \caption{Constants extending the meta-logic to HOL}
  \label{tab:hol-const-typ}
\end{table}

HOL inherits the ability to express functions as lambda terms from the
meta-logic by identifying HOL types and functions with the types and functions
of the meta-logic\footnote{this might seem an obvious choice, but some logics
  follow a different approach to make type systems possible that do not fit into
  the one provided by the meta-logic, cf.  \EG the formulations of
  Zermelo-Fraenkel set theory or CTT}.  This way, HOL also exploits \Isabelle's
built-in type checker, which is a great help in immediately refuting ill-typed
expressions. Nonetheless it has its own type of truth values, classically named
$\Type{bool}$. In fact, HOL is a classical logic (as opposed to a constructive
or intuitionistic logic) featuring the law of excluded middle (cf. rule
$\irule{True-or-False}$ in Table \ref{tab:axioms-hol}).

There is an interesting difference between variables in HOL and the more
syntactical variables encountered in the definition of logics `on paper', where
a rule of substitutivity of equality might be defined as follows
\begin{equation}
\label{eq:paper-rule-subst}
\Rule{a = b\quad \phi}{\phi[b/a]}
\end{equation}
In this rule, $\phi$ is a syntactical variable in the sense that it stands for an
arbitrary formula (\IE a term of type $\Type{bool}$ in HOL), probably containing
$a$ as a free variable -- otherwise  substituting $b$ for $a$ would be
pointless. To the contrary, in HOL there is no need for an explicit notion of
substitution, and the rule under consideration is expressed as
\begin{equation}
\label{eq:hol-rule-subst}
\Rule{a = b \quad \phi\ a}{\phi\ b}
\end{equation}
making $\phi\IsaMap{\sigma}{\Type{bool}}$ a function variable provided that $a, b : \sigma$. 
Here is a simple example to visualise the difference.
\begin{expl}
  Assuming some proof has reached a state such that $a = b$ and $f\,a = g\,x$
  have been proved. In this case, $\phi$ of \eqref{eq:paper-rule-subst} can be
  instantiated to $f\,a = g\,x$, whereas $\phi$ of \eqref{eq:hol-rule-subst} is
  $\LambdaTerm{y}{f\,y = g\,x}$. Applying rule \eqref{eq:hol-rule-subst} yields
  $(\LambdaTerm{y}{f\,y = g\,x})\ b$ which can be converted to $f\,b = g\,x$ by
  the $\beta$-rule of the meta-logic.
\end{expl}



\subsection{Definitions}
\label{sec:hol-definitions}

To avoid unnecessary redundancy, logics -- including HOL -- often only
axiomatise the properties of a minimal set of constants, with everything else
being defined in the form of abbreviations (the definition of implication
through negation and disjunction is a case in point, although in HOL implication
is the basic connective). It is here, where the constants of the meta-logic come
into play: we may use meta-equality to describe definitions, meta-implication to
express rules and the use of meta-quantification is a convenient way to capture
many common side conditions. Table \ref{tab:axioms-hol} shows the axiomatisation
of HOL as an extension of the meta-logic, where the usual connectives are still
missing; their definitions are presented in Table \ref{tab:defn-logic-op}.
Within the latter, the left column shows the logical constants with their types,
while their definition is presented in the right column.

\begin{rem}
  To ensure that this representation of higher-order logic is actually sensible,
  one would now go on and prove a kind of equivalence between a higher-order
  logic defined in the usual way (by axioms and rules with side conditions) and
  this extension of the meta-logic, showing that for every proof in the one
  system, there is always a corresponding proof in the other system. This
  meta-proof cannot be expressed within \Isabelle, though.
\end{rem}

\begin{table}
  \centering \renewcommand{\arraystretch}{1.3}
  \begin{tabular}{|l@{$\quad$}l|}\hline
    \irule{eq-reflection} & $(x = y) \Longrightarrow (x \equiv y)$ \\\hline
    \irule{refl}          & $(x  = x)$ \\\hline
    \irule{subst}         & $\llbracket s = t; P\, s\rrbracket \Longrightarrow P\, t$ \\\hline
    \irule{ext}           & $(\bigwedge x. f\,x = g\,x) \Longrightarrow \LambdaTerm{x}{f\,x} =
    \LambdaTerm{x}{g\,x}$ \\\hline
    \irule{the-eq-trivial} & $(\varepsilon x.\,x=a) = a$ \\\hline
    \irule{impI}          & $(P \Longrightarrow Q) \Longrightarrow P \longrightarrow Q$ \\\hline
    \irule{mp}            & $\llbracket P \longrightarrow Q; P\rrbracket \Longrightarrow Q$ \\\hline
    \irule{iff}           & $(P\longrightarrow Q) \longrightarrow (Q\longrightarrow P)\longrightarrow (P=Q)$\\\hline
    \irule{True-or-False} & $P = True \lor P = False$ \\\hline
  \end{tabular}
  \caption{Axiomatisation of HOL in \Isabelle}
  \label{tab:axioms-hol}
\end{table}

\begin{table}
  \centering \renewcommand{\arraystretch}{1.3}
  \begin{tabular}{|l|l@{$\quad\equiv\quad$}l|}\hline
    \multicolumn{2}{|l}{\emph{Constant}} & \emph{Definition} \\\hline
    $\op{True} :: \Type{bool}$ &
    $\op{True}$           & $(\LambdaTerm{x::\Type{bool}}{x}) =
    \LambdaTerm{x}{x}$\\\hline 
    $\Fun{All} :: ('a \Rightarrow \Type{bool}) \Rightarrow \Type{bool}$ &
    $\forall x.\, P\, x$      & $P = \LambdaTerm{x}{\op{True}}$\\\hline
    $\Fun{Ex} :: ('a \Rightarrow\Type{bool}) \Rightarrow \Type{bool}$ &
    $\exists x\bdot P\, x$    & $\forall b\bdot (\forall x\bdot P\, x \longrightarrow b) \longrightarrow b$\\\hline
    $\op{False} :: \Type{bool}$ &
    $\op{False}$          & $\forall b.\ b$\\\hline
    $\Fun{Not} \IsaMap{\Type{bool}}{\Type{bool}}$ &
    $\lnot P$            & $P \longrightarrow \op{False}$\\\hline
    $\land  \IsaMap{[\Type{bool}, \Type{bool}]}{\Type{bool}}$ &
    $P \land Q$          & $\forall R\bdot (P \longrightarrow Q \longrightarrow R) \longrightarrow R$\\\hline
    $\lor \IsaMap{[\Type{bool},\Type{bool}]}{\Type{bool}}$ &
    $P \lor Q$          & $\forall R\bdot (P\longrightarrow R)\longrightarrow(Q\longrightarrow R)\longrightarrow R$\\\hline
  \end{tabular}
  \caption{Definitions of some common logical constants in HOL}
  \label{tab:defn-logic-op}
\end{table}

\begin{expl}
  To make the definitions of Table \ref{tab:defn-logic-op} a little bit more
  convincing, we take a closer look at two of them:
  \begin{itemize}
  \item The most basic notion of HOL is equality, so it is tempting to define
    truth in terms of equality: $\op{True} \equiv (\LambdaTerm{x::\Type{bool}}{x}) =
    \LambdaTerm{x}{x}$. This term is entirely closed, \IE it neither contains
    free term variables nor free type variables, which is why this definition is
    used instead of the seemingly simpler $x=x$.
  \item Universal quantification is a predicate on predicates: if $\Fun{All}\,P$
    or equivalently $\forall x\bdot P\,x$ is true, this says that $P$ is a predicate
    that constantly yields true, no matter what argument it is applied to (of
    course, all arguments must have the appropriate type). So, one can define
    $(\forall x.\, P\, x) \equiv (P = \LambdaTerm{x}{\op{True}})$.
  \end{itemize}
\end{expl}



% \begin{equation}
% \small
% \renewcommand{\arraystretch}{2}
% \begin{array}{l|l|l|l}
%   \mathit{Constant} & \mathit{def. as} & \mathit{Rules} & \text{\itshape meta-logic
%     axioms} \\\hline
%   \longrightarrow    & & \Rule{{[P]\atop Q}}{P\longrightarrow Q}\hfill \text{(impI)} 
%            & \llbracket P \Longrightarrow Q\rrbracket \Longrightarrow P \longrightarrow Q\\
%          & & \Rule{P \longrightarrow Q\quad P}{Q}\hfill\text{(mp)} & \llbracket P\longrightarrow Q; P\rrbracket \Longrightarrow Q \\\hline
  
% \end{array}
% \end{equation}

\section{Proof Methods}
\label{sec:proof-methods}

Performing proofs from rules in an object-logic -- in examples this will always
be HOL -- means proving theorems in the meta-logic. Such proofs would be
incredibly tedious if only the meta-rules described in Section
\ref{sec:meta-logic-rules} had to be used. Fortunately, there is a powerful
proof method whose correctness is assured by the axiomatic properties of $\bigwedge$ and
$\Longrightarrow$: \emph{higher-order resolution}. As with first-order resolution, known from
logic programming in Prolog, this concept involves the \emph{unification} of
terms. As usual, if $\theta$ is a unifier of terms $t_1$ and $t_2$, \IE an assignment
of terms to variables, the simultaneous substitution of all variables mentioned
in $\theta$ by the according terms is written as $(t_1)\theta$ and $(t_2)\theta$, respectively.
Due to the fact that \Isabelle employs the lambda calculus as its formal basis,
it sometimes has to unify lambda abstractions that do not have a \emph{most
  general unifier} (mgu), which is in contrast to first-order unification, where
two terms either are not unifiable or have exactly one mgu (up to equivalence).
The effect of this problem mainly is that sometimes the user must assist
\Isabelle in finding a unifier by supplying instantiations of variables.

\begin{rem}
  \Isabelle distinguishes two kinds of variables that logically have the same
  meaning. On the one hand there are the usual variables with standard lexical
  syntax ($x$, $y$, $x_1$, $P$ are variables of this kind). On the other hand
  there are \emph{schematic variables} which may be used as variables for
  substitution during unification. These are prefixed with a question mark to
  emphasise their role as placeholders (\EG $?x$, $?P$). The usual way of
  proceeding is that theorems are stated solely with normal variables. After
  they have been proved, \Isabelle internally converts all free variables of the
  theorem into schematic variables. This is in accordance with intuition: in
  proving a theorem $T$, one would certainly not want $T$'s variables to be
  replaced by some concrete term; but one should be able to replace the free
  variables of already proved theorems, as they eventually represent arbitrary
  terms.
\end{rem}



\subsection{Higher-order Resolution}
\label{sec:nat-ded-resolution}

In what follows we will talk of the left-hand side of a meta-implication as the
premiss (or premisses, if the $\llbracket\ldots\rrbracket$ notation is used) and of the right-hand side
as the conclusion, to emphasise the role of meta-implication for object-logics.
Given two theorems $\llbracket P_1,\ldots,P_n\rrbracket \Longrightarrow P$ and $\llbracket Q_1,\ldots,Q_m\rrbracket \Longrightarrow Q$ in the meta-logic,
such that $(P_i \equiv Q)\theta$ holds for some $i\in\{1,\ldots,n\}$ and some unifier $\theta$,
resolution allows us to prove a new theorem that has $P$ as its conclusion and
all the $P_j$ and $Q_j$ except $P_i$ as premisses, but with $\theta$ applied to the
whole term
\begin{equation}
  \Rule{\llbracket P_1,\ldots,P_n\rrbracket \Longrightarrow P \quad \llbracket Q_1, \ldots, Q_m\rrbracket \Longrightarrow Q}
       {(\llbracket P_1,\ldots,P_{i-1},Q_1,\ldots,Q_m,P_{i+1},\ldots,P_n\rrbracket \Longrightarrow P)\theta}
\end{equation}
Apart from the substitution $\theta$, this rule is intuitively clear: if the $Q_j$
imply $Q$ and $Q \equiv P_i$, then the $Q_j$ are a suitable surrogate for $P_i$ as
premisses for the conclusion $P$. The involvement of substitution makes this
idea even more general by admitting terms that are only equal under a given
substitution $\theta$.  

A complication concerning the applicability of resolution arises when the
premisses of a meta-theorem contain a meta-implication or meta-quantification
themselves, as in the derived HOL rule (impI): $(A \Longrightarrow B) \Longrightarrow A \longrightarrow B$. The single
premiss of this meta-theorem will only be unifiable with the conclusion of
another meta-theorem if the latter consists of a variable or is of the form
$X\Longrightarrow Y$, but both forms seldom appear in theorems.  To circumvent this problem,
\Isabelle is able to \emph{lift} a rule into a context, which can be formalised
by the rule
\begin{equation}
  \Rule{\llbracket P_1,\ldots,P_n\rrbracket \Longrightarrow P}
       {\llbracket Q \Longrightarrow P_1,\ldots,Q\Longrightarrow P_n\rrbracket \Longrightarrow (Q \Longrightarrow P)}
\end{equation}
This transformation is done automatically during resolution if necessary. 

Although forward proof is also possible in \Isabelle -- mainly to derive new
theorems from existing ones in a rather direct manner -- theorems are usually
proved in a backward style: By applying rules backwards, a theorem is reduced
into simpler parts until the remaining propositions are trivially true (in
particular by reducing propositions to axioms, of course).  The ideas presented
so far can best be understood with the help of an example. 

\begin{expl}
  The backward proof a theorem $T$ within the object-logic always starts with
  the trivial meta-theorem $T \Longrightarrow T$. This theorem is then transformed
  by the meta-rules and resolution until  $T$ has been derived. The following
  are HOL rules, derivable from the axioms given in Table \ref{tab:axioms-hol}. 
  \begin{align*}
    &    (?A \Longrightarrow ?B) \Longrightarrow {}?A \longrightarrow {}?B && \quad \text{(impI)}\\
    & \llbracket?A; {}?B\rrbracket \Longrightarrow {}?A \land{} ?B    & & \quad \text{(conjI)}\\
    & \llbracket?A \land {}?B\rrbracket \Longrightarrow {}?A        && \quad \text{(conjunct1)}\\
    & \llbracket?A \land {}?B\rrbracket \Longrightarrow {}?B       && \quad \text{(conjunct2)}
  \end{align*}
  Here is  a proof of $A\land B \longrightarrow B \land A$ from these rules:
  \begin{flalign*}
    (.1) &&   (A\land B \longrightarrow B\land A) \Longrightarrow&\; (A\land B \longrightarrow B\land A)    \\
    (.2) &&      \llbracket A\land B \Longrightarrow B\land A\rrbracket \Longrightarrow&\; (A\land B \longrightarrow B\land A)   &
                \text{(impI)}\\
    (.3) &&     \llbracket A\land B \Longrightarrow B; A\land B \Longrightarrow A\rrbracket \Longrightarrow&\; (A\land B \longrightarrow B\land A) &
                \text{(conjI, lifted)}\\
    (.4) &&     \llbracket A\land B \Longrightarrow ?A\land B; A\land B \Longrightarrow A\rrbracket \Longrightarrow&\; (A\land B \longrightarrow B\land A)  &
                \text{(conjunct2, lifted)}\\
    (.5) &&     (A\land B \Longrightarrow A) \Longrightarrow&\; (A\land B \longrightarrow B\land A)   &
                \text{(assumption)}\\
    (.6) &&     (A\land B \Longrightarrow A \land{} ?B) \Longrightarrow&\; (A\land B \longrightarrow B\land A) &
                \text{(conjunct1, lifted)}\\
    (.7) &&     & \;(A\land B \longrightarrow B\land A) & \text{(assumption)}
  \end{flalign*}
  To derive (.2), the premiss of (.1) has been resolved with the conclusion of
  rule (impI), where $?A$ has been unified with $(A\land B)$ and $?B$ has been
  instantiated to $(B\land A)$. To arrive at (.3) lifting is necessary, because there
  is no rule that would otherwise match the premiss of (.2). Lifting rule
  (conjI) (to become $\llbracket?C \Longrightarrow {}?A; ?C \Longrightarrow {}?B\rrbracket \Longrightarrow (?C \Longrightarrow {}?A \land{} ?B)$) makes it
  possible to resolve it with the premiss of (.2). The step from (.3) to (.4) is
  justified by lifting rule (conjunct2) and then resolving with the first
  premiss of (.3). Note that at this point a new schematic variable $?A$ is
  introduced which is entirely independent from $A$. This introduction is due to
  the fact that (conjunct2) contains $?A$ in its premiss, but not in the
  conclusion. We arrive at (.5) by dismissing an assumption which is trivially
  true after unification of $?A$ with $A$. This type of proof step is called
  \emph{proof by assumption}. The remaining steps are analogous.
\end{expl}



\subsection{A Different Perspective}
\label{sec:diff-persp}


Another way to look at a proof of theorem $T$ that is a bit more natural is to
start with $T \Longrightarrow T$, but ignore the conclusion $T$ and simply look at the
premisses, regarding them as \emph{goals}, \IE statements that are yet to be
proved in order to finish the proof of $T$. Thus, the initial goal is the
theorem itself. Resolution of the theorem at hand with other theorems as
described above can then be imagined as the application of rules to the current
goal. For example, if the current goal is to show $A \longrightarrow B$ for some formulae $A$
and $B$ in the proof of $T$ (\IE internally the theorem $A \longrightarrow B \Longrightarrow T$ has been
derived), we may `apply the rule (impI)' to turn this goal into $A \Longrightarrow B$. Making
one further step of abstraction, this term can be taken as the goal $B$, to be
proved from the \emph{assumption} $A$. Lifting of rules into a context suddenly
takes the form of preservation of assumptions: In the above proof of $A\land B\longrightarrow B\land A$
the step from (.2) to (.3) preserves the assumption $A\land B$ for the two new
\emph{subgoals} $A\land B\Longrightarrow B$ and $A\land B \Longrightarrow A$.

One speaks of applying a \emph{rule} in \Isabelle parlance if it is applied in
this standard way. There are other ways of applying a rule that do not enlarge
the set of provable theorems, but that come in quite handy sometimes. Assume the
current subgoal is $\llbracket P_1;\ldots;P_n\rrbracket \Longrightarrow P$ and we try to apply the rule $\llbracket T_1;\ldots;T_k\rrbracket \Longrightarrow
T$, which is an already proved theorem.
\begin{itemize}
\item  The standard rule application unifies $P$ with $T$ giving a unifier $\theta$. It
  then replaces the subgoal by $k$ new subgoals $(\llbracket \llbracket P_1;\ldots;P_n\rrbracket \Longrightarrow T_1; \ldots;
  \llbracket P_1;\ldots;P_n\rrbracket \Longrightarrow T_k\rrbracket)\theta$. 
\item Applying a \emph{drule} (for destruction rule) is useful to modify a
  subgoal's assumptions. It unifies $T_1$ with some assumption -- which for
  simplicity we assume to be $P_1$ -- and yields the subgoals 
  \[(\llbracket \llbracket P_2;\ldots;P_n\rrbracket \Longrightarrow
  T_2; \ldots; \llbracket P_2;\ldots;P_n\rrbracket \Longrightarrow T_k; \llbracket P_2; \ldots; P_n; T\rrbracket \Longrightarrow P\rrbracket)\theta\] 
  The idea is that $T_1$ is among the current assumptions (it is unifiable with
  $P_1$ here) and can thus be proved trivially. It then remains to prove $T_2$
  to $T_k$, but if this can be done, it is reasonable to take $T$ as an
  assumption in proving $P$, since all of $T$'s premisses can be proved from the
  current assumptions. 

\item The application of an \emph{erule} (for elimination rule) lets $P$ be
  unified with $T$ and simultaneously unifies $T_1$ (called the \emph{major
    premiss} in this context) with one of the current assumptions (let it be
  $P_1$). It replaces the current subgoal with the new ones
  \[ (\llbracket \llbracket P_2;\ldots;P_n\rrbracket \Longrightarrow T_2; \ldots; \llbracket P_2;\ldots;P_n\rrbracket \Longrightarrow T_k\rrbracket)\theta \]
  This rule application is obviously quite similar to the standard way, but it
  deletes the assumption $P_1$ and it proves one subgoal immediately.
\end{itemize}

\subsection{Advanced Proof Methods}
\label{sec:adv-proof-meth}

For a proof assistant to be helpful in serious verification tasks, one may
expect it to come with more powerful proof methods than just the application of
axiomatically established rules in a backward proof. We now shortly present some
important principles supported by \Isabelle and which are regularly encountered
in proofs.

\begin{itemize}
\item \emph{Derived rules.} Every theorem that has been proved in \Isabelle can
  be given a name and subsequently be used as if it were a rule of the
  object-logic. The rules (conjI), (impI), etc. shown above are examples for
  derived rules: they represent valid modes of reasoning in HOL and
  extend the logic in a conservative way, \IE they do not enlarge the set of
  provable statements in HOL. In practice the largest part of rules applied in a
  proof will be derived rules of inference. A list of customary rules can be
  found in Appendix \ref{cha:freq-used-rules}.

\item \emph{The simplifier.} \Isabelle provides a powerful and extensible term
  rewriting (or simplification) tool. Term rewriting works by subsequently
  transforming terms with the help of \emph{rewrite rules} in a bottom-up
  fashion. The set of applicable rewrite rules is comprised of definitions and
  theorems. Adding the definition of Pierce's arrow $P\downarrow Q \equiv \lnot P \land \lnot Q$ to the set
  of rewrite rules lets the simplifier replace  occurrences of $\downarrow$ by the defining
  term; this can be useful if no theorems about $\downarrow$ are known yet, but for $\land$
  and $\lnot$ there are some. Certain theorems are also good candidates for term
  rewriting; given associativity and commutativity of
  addition, the simplifier is able to prove equations like $(a + b) + (c + d) =
  (a + (b + (d + c)))$ outright, relieving the user of several applications of
  these rules by hand.

  To avoid looping on so-called permutative rewrite rules in which the left-hand
  side of the equation is equal to the right-hand side up to a renaming of
  variables -- \EG the rule $a + b = b + a$ -- the simplifier performs
  \emph{ordered rewriting} so that terms are only rewritten by permutative rules
  if they become lexicographically smaller. Hence, $a + b$ may be rewritten to
  $b + a$, but not the other way round.

\item \emph{A classical tableau prover.} In contrast to the simplifier -- which
  can be employed as an intermediate proof step leaving a goal that is simpler
  to prove by hand, and which is able to manipulate arbitrary terms -- there
  also is a tool for proving logical formulae directly. This tool is known as
  the \texttt{blast} method and it is capable of proving theorems like $(\exists y\bdot
  \forall x\bdot P\ x\ y) \longrightarrow (\forall x\bdot \exists y\bdot P\ x\ y)$ without intervention from the
  user (this theorem could not even be altered by the simplifier in any way). It
  cannot modify theorems however, \EG to make the structure of the problem more
  apparent: if it fails to finish the proof, it fails completely.
\end{itemize}

\subsection{An Example Proof}
\label{sec:an-example-proof}

Concluding the presentation of \Isabelle, we provide a short example proof,
thereby explaining basic syntactic elements. 


\begin{isabellebody}%
\isanewline
\isamarkupfalse%
\isacommand{lemma}\ imp{\isacharunderscore}uncurry{\isacharcolon}\ {\isachardoublequote}P\ {\isasymlongrightarrow}\ {\isacharparenleft}Q\ {\isasymlongrightarrow}\ R{\isacharparenright}\ {\isasymLongrightarrow}\ {\isacharparenleft}P\ {\isasymand}\ Q{\isacharparenright}\ {\isasymlongrightarrow}\ R{\isachardoublequote}\isanewline
\isamarkupfalse%
\isacommand{apply}\ {\isacharparenleft}rule\ impI{\isacharparenright}\isanewline
\isamarkupfalse%
\isacommand{apply}\ {\isacharparenleft}erule\ conjE{\isacharparenright}\isanewline
\isamarkupfalse%
\isacommand{apply}\ {\isacharparenleft}drule\ mp{\isacharparenright}\isanewline
\isamarkupfalse%
\isacommand{apply}\ assumption\isanewline
\isamarkupfalse%
\isacommand{by}\ \ {\isacharparenleft}drule\ mp{\isacharparenright}\isanewline
\end{isabellebody}

Read as a rule of the object-logic HOL, $\irule{imp-uncurry}$ says that given
the implication $P \longrightarrow (Q \longrightarrow R)$, one may conclude $(P\land Q) \longrightarrow R$. These formulae are
well known to be equivalent, so we might even have proposed $(P \longrightarrow Q \longrightarrow R) = (P \land
Q \longrightarrow R)$ (omitting all unnecessary parentheses) which we have not done to keep
the example short. Let's walk through this proof
step by step: As has been said, the initial goal is the theorem (or lemma)
itself. Applying rule (impI) turns the goal into
\[
\llbracket P\longrightarrow Q\longrightarrow R; P\land Q\rrbracket \Longrightarrow R
\]
\IE it assumes $P\land Q$ and imposes the proof of $R$. The next step uses the
elimination rule (conjE) which is
\[
\llbracket{?P} \land {?Q}; \llbracket{?P}; {?Q}\rrbracket \Longrightarrow {?R}\rrbracket \Longrightarrow {?R}\qquad \text{(conjE)}
\]
This results in the subgoal
\begin{equation} \label{eq:prstep-conje}
\llbracket P\longrightarrow Q\longrightarrow R; P; Q\rrbracket \Longrightarrow R
\end{equation}
What happens is that ${?P} \land {?Q}$ is matched against $P \land Q$ and ${?R}$ is
matched against $R$. The only remaining subgoal is then to prove $\llbracket P; Q\rrbracket\Longrightarrow R$ from
the assumption $P\longrightarrow Q\longrightarrow R$ for which \eqref{eq:prstep-conje} is just a different
notation. As a final step of detailed analysis we show what subgoals are yielded
by applying rule (mp) destructively:
\[
\llbracket \llbracket P; Q\rrbracket \Longrightarrow P; \llbracket P; Q; Q \longrightarrow R\rrbracket \Longrightarrow R\rrbracket
\] 
The rest of the proof consists of proof by assumption and another application of
drule (mp). The \textbf{by} statement concludes a proof, possibly undertaking
further steps of proof by assumption if necessary.


\section{The Isar Proof Language}
\label{sec:isar-proof-language}

The proof style displayed in Section \ref{sec:an-example-proof} above --
occasionally termed the \emph{apply style} due to its excessive use of the
\isacommand{apply} method -- has two major drawbacks. The first one is that
proof scripts comprising a long sequence of \isacommand{apply}s are hard to
read, because there is no information about intermediate proof states shown.
The second one, which becomes evident in the presence of large numbers of
theories, is maintainability: if, for example, the simplifier by changing its
configuration becomes more powerful, an application of the \emph{simp} method which
previously resulted in a certain proof state might now result in quite a
different one. This often means that subsequent rules of the original proof
script are no longer applicable, so that the script has to be adjusted.

Another issue is that pure backward-oriented proofs are sometimes quite
unnatural to perform. This is especially true for proofs involving applications
of modus ponens. If at some point in a proof the goal $A$ remains, which one
wants to prove from the globally given facts $B$ and $B \longrightarrow A$, then an
application of rule (mp) results in the two new subgoals $?P \longrightarrow A$ and $?P$, thus
introducing a new unification variable $?P$. In this simple case the structure
of the goals containing the unification variable is very similar to the structure
of the given facts, but in practice their relation can be hard to guess, since
$?P$ may stand for any formula. This problem is of course closely related to the
reason why cut-freeness and the sub-formula property are desired properties of
logical calculi (see \cite{Andrews00}).

\subsection{Introducing Isar by Example}
The Isar proof language has been conceived as a formalism for writing proof
scripts that are both machine- and human-readable. Strictly speaking, one already
works within Isar when employing the apply style, since \isacommand{apply} is an
Isar command rather than one of basic \Isabelle. However, this mode of usage
closely resembles the original \Isabelle style in which ML functions were called
directly. Full Isar comes with several advanced features which are best
introduced with the help of a simple example. This is how a proof of the above
lemma $\irule{imp-uncurry}$ looks like in Isar:


%
\begin{isabellebody}%
\isanewline
\isamarkupfalse%
\isacommand{lemma}\ imp{\isacharunderscore}uncurry{\isadigit{2}}{\isacharcolon}\ {\isachardoublequote}P\ {\isasymlongrightarrow}\ {\isacharparenleft}Q\ {\isasymlongrightarrow}\ R{\isacharparenright}\ {\isasymLongrightarrow}\ {\isacharparenleft}P\ {\isasymand}\ Q{\isacharparenright}\ {\isasymlongrightarrow}\ R{\isachardoublequote}\isanewline
\isamarkupfalse%
\isacommand{proof}\isanewline
\ \ \isamarkupfalse%
\isacommand{assume}\ a{\isadigit{1}}{\isacharcolon}\ {\isachardoublequote}P\ {\isasymlongrightarrow}\ Q\ {\isasymlongrightarrow}\ R{\isachardoublequote}\isanewline
\ \ \isamarkupfalse%
\isacommand{assume}\ a{\isadigit{2}}{\isacharcolon}\ {\isachardoublequote}P\ {\isasymand}\ Q{\isachardoublequote}\isanewline
\ \ \isamarkupfalse%
\isacommand{show}\ {\isachardoublequote}R{\isachardoublequote}\isanewline
\ \ \isamarkupfalse%
\isacommand{proof}\ {\isacharminus}\isanewline
\ \ \ \ \isamarkupfalse%
\isacommand{from}\ a{\isadigit{2}}\ \isamarkupfalse%
\isacommand{have}\ P\ \isamarkupfalse%
\isacommand{by}\ {\isacharparenleft}rule\ conjunct{\isadigit{1}}{\isacharparenright}\isanewline
\ \ \ \ \isamarkupfalse%
\isacommand{with}\ a{\isadigit{1}}\ \isamarkupfalse%
\isacommand{have}\ qr{\isacharcolon}\ {\isachardoublequote}Q\ {\isasymlongrightarrow}\ R{\isachardoublequote}\ \isamarkupfalse%
\isacommand{by}\ {\isacharparenleft}rule\ mp{\isacharparenright}\isanewline
\ \ \ \ \isamarkupfalse%
\isacommand{from}\ a{\isadigit{2}}\ \isamarkupfalse%
\isacommand{have}\ Q\ \isamarkupfalse%
\isacommand{{\isachardot}{\isachardot}}\isanewline
\ \ \ \ \isamarkupfalse%
\isacommand{with}\ qr\ \isamarkupfalse%
\isacommand{show}\ {\isacharquery}thesis\ \isamarkupfalse%
\isacommand{{\isachardot}{\isachardot}}\isanewline
\ \ \isamarkupfalse%
\isacommand{qed}\isanewline
\isamarkupfalse%
\isacommand{qed}\isanewline
\end{isabellebody}%


\emph{Compound Isar proofs} are commenced by the keyword \isacommand{proof}. In
its pure form this statement tries to find a rule that can be applied to the
goal -- in the example, the implication introduction rule (impI) is selected.
This kind of implicit rule application, which is much the same as
\isacommand{apply}ing a rule in a backward-oriented proof, can be avoided by
appending a hyphen `$-$' or the rule selection can be made explicit by providing a
concrete rule. Applying (impI) here results in the \Isabelle proof state
\[
  \llbracket P \longrightarrow Q \longrightarrow R; P \land Q\rrbracket \Longrightarrow R
\]
which is exactly mirrored by the following two \isacommand{assume} commands
introducing the valid assumptions (which may be given a name for future
reference) in the proof script. Moreover, the succeeding \isacommand{show}
command precisely depicts the statement that remains to be shown.  In every
compound proof there occurs exactly one \isacommand{show}. To prove $R$ another
compound proof has to be initiated, this time without applying a backward rule.
From the given assumptions $\mathit{a1}$ and $\mathit{a2}$ it is very natural to
prove $R$ by forward reasoning: basically, two applications of modus ponens to
assumption $\mathit{a1}$ should yield the desired result. This is exactly what
we find in the proof script: first, we derive $P$ from $P \land Q$ by rule
(conjunct1) as an intermediate fact, then we may apply modus ponens to $a1$ to
obtain fact $\mathit{qr}$, \IE $Q \longrightarrow R$.  The same procedure can be executed once
more (this time on $\mathit{qr}$) to finally show the thesis. 

Several concepts
of Isar have been used to achieve this result. The \isacommand{by} command
represents \emph{basic proofs} which are finished immediately through an
application of the rule handed to it (\EG rule (conjunct1) or (mp)) and possibly
further steps of proof by assumption.  But how can a rule having itself some
premisses be used to prove a pending subgoal? For this purpose the
\isacommand{from} command is needed, which feeds facts into a proof so that
these are unified with the premisses of the applied rule. In the concrete
example, the fact $P \land Q$ is fed into the proof by rule (conjunct1) to obtain $P$. A
handy abbreviation is \isacommand{with}, which behaves like \isacommand{from},
but additionally feeds the most recent fact into the subsequent proof. For
example, to obtain $Q \longrightarrow R$ by (mp), one must feed the two premisses $P \longrightarrow Q \longrightarrow R$
and $P$ into the proof, where $P$ is the most recently established result.
Hence, \isacommand{with} $\mathit{a1}$ yields all that is required to finish the
proof by modus ponens.  Finally, \isacommand{qed} concludes a compound proof and
two dots `$..$' are shorthand for \isacommand{by} \textit{standard rules}, \IE a
basic proof established through the standard rule set which includes (mp),
(impI), (conjunct1) and many more. See Appendix \ref{cha:freq-used-rules} for
frequently used rules in HOL and refer to \cite{Nipkow03,Wenzel02} for further
details about the Isar proof language. Some more specialised features will also
be explained in Chapter \ref{cha:implementation} as required.




%%% Local Variables: 
%%% mode: latex
%%% TeX-master: "main"
%%% End: 

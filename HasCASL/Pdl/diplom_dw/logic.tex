
\chapter{Monadic Dynamic Logic}
\label{cha:logic}

In this chapter, the proof calculus of monadic dynamic logic is presented.
First, properties of monadic programs are introduced that will be needed later
on in order to develop the calculus; these include notions such as
discardability and side effect freeness of programs. After that, the modal
operators of dynamic logic are introduced in an axiomatic way and their meaning
in the example monads of Section \ref{sec:monads-cs} is explained. All
prerequisites gathered together, the monad-independent proof calculus for
dynamic logic is described in Section \ref{sec:monad-indep-calc}. Finally, an
extension of the calculus that is tailored towards the exception monad is
developed.

In what follows the type of truth values will be denoted by $\Omega$, and the entire
formalisation is suited for an intuitionistic as well as a classical framework.
$T$ will denote the type constructor mapping a type of \emph{values} into the
type of \emph{computations} or \emph{programs} over these values.
Formulae of dynamic logic will be taken to be terms of type $D\Omega$, where, for each
$A$, $D A$ is the subtype of $T A$ of all deterministically side effect free
programs, a notion depicted below. As a primary feature of the calculus, there
will be modal operators $\PDLBox{x \leteq p}{\Arg}$ and $\PDLDmd{x \leteq
  p}{\Arg}$ for each program $p$ that take a formula of dynamic logic $\phi$
(possibly containing $x$ as a free variable) to another formula which may state
properties of $x$ being the result of executing $p$. The modal operators thus
act as new variable binders; because we also allow program sequences to occur
inside the operators -- as in $\PDLBox{x\leteq p; y\leteq q}$ -- they may bind
several variables at once. An initial intuitive understanding of the box and
diamond operators can be most easily given in the nondeterminism monad, where
the formula $\PDLBox{x\leteq p}{(x=1)}$ should be interpreted as ``after
executing $p$ and binding the result to $x$, $(x=1)$ will hold \emph{for all}
possible outcomes of $p$''. On the other hand, $\PDLDmd{x\leteq p}{(x=1)}$
states that there will be \emph{some} result $x$ of $p$ such that $(x=1)$ is
true.


\section{Preliminaries}
\label{sec:dl-prelim}

The possibility for program sequences to occur inside the box and diamond
operators instead of single bindings should be regarded as a mere notational
convenience. That this does not add to the expressiveness of the operators can
be seen by translating multiple bindings into bindings of tuples, \EG the bound
variables $x$ and $y$ in $\PDLBox{x\leteq p; y\leteq q}$ can be packaged into the
single variable $z = (x,y)$ in $\PDLBox{z\leteq \DoStmt{x\leteq p; y\leteq q;
    \ret~(x,y)}}$.  Horizontal bars above variables will indicate that actually a
  non-empty program sequence is under consideration rather than a single
  binding. Let $\bar x = [x_1,\ldots,x_n]$ and $\bar p = [p_1,\ldots,p_n]$; then $\bar x
  \leteq \bar p$ will denote the program $\DoStmt{x_1\leteq p_1; \cdots; x_n\leteq
    p_n; \ret~(x_1,\ldots,x_n)}$ or, if it appears inside a do-statement or a box
  or diamond operator, just the
  binding sequence $x_1\leteq p_1;\cdots; x_n\leteq p_n$.


\subsection{Properties of Monadic Programs}
\label{sec:prop-monad-progr}

The property of a monadic program being deterministically side effect free
(abbreviated to \emph{dsef} in the following) relies on some simpler properties
that will now be defined. The main idea behind the introduction of a subtype $D
A$ of dsef programs is that these programs have properties allowing them to be
rearranged quite freely within a monadic program sequence. For example, if $p$
and $q$ are both dsef programs, the programs $\DoStmt{x\leteq p; y\leteq q; r}$
and $\DoStmt{y\leteq q; x\leteq p; r}$ will be equal for every program $r$
(possibly containing $x$ and $y$, in contrast to $p$ and $q$, which may not
mention them). This is an important fact when introducing connectives for
formulae of dynamic logic: intuitively, $\phi : D\Omega$ and $\phi \land \phi : D\Omega$ should be
regarded as equivalent formulae, but if $\phi$ has side effects or is
nondeterministic, this equivalence might break down.  Taking only terms of type
$D \Omega$ as formulae makes sure such equivalences are retained in the calculus.

A more elaborate account of the information provided in this section can be
found in \cite{SchroederMossakowski:PDL}, where virtually all lemmas and
propositions stated here were proved. To avoid overly repeating facts already
stated elsewhere, only the most important lemmas and some of their proofs are
given here.
% XXX refer to Isabelle section if proof will be presented there
Independently established proofs can be found in
Section~\ref{sec:spec-extens-except} covering  extensions specific to the
exception monad as well as in the chapters on application
(Chapter~\ref{cha:application}) and implementation
(Chapter~\ref{cha:implementation}) of the calculus.

\begin{defn}
  \label{defn:dis}
  Let $1$ be the unit type and $\unit$ the single element of this type. A
  program $p : T A$ is called \emph{discardable} if
  \[ \DoStmt{x\leteq p; \ret~\unit} = \ret~\unit \]
  \begin{itemize}
  \item In the state monad, $p$ is discardable if it terminates and does not
    alter the state; since its result is not used in the remainder of the
    program on the left-hand side (\IE in $\ret~\unit$), it might just as well
    be omitted altogether.
  \item In the nondeterminism monad, $p$ is discardable if it yields at least
    one result; in that case, both sides of the equation yield $\{\unit\}$.
  \item The concept of discardability reveals differences between the list monad
    and the nondeterminism monad: in the list monad, $p$ is discardable if it
    yields exactly one result, in which case both sides of the equation equal
    the singleton list $[\unit]$ -- which may well be distinguished from the
    list $[\unit, \unit]$ containing the same element twice.
  \end{itemize}
\end{defn}

\begin{defn}
  Let $p : T A$ be a program. $p$ is called \emph{stateless} if it is of the
  form $p = \ret~a$ for some $a : A$. Obviously, all stateless programs are
  discardable which follows immediately from the basic monad laws.
\end{defn}

The following lemma confirms the appropriateness of Definition \ref{defn:dis}
for indicating when a program may be discarded at the head of an arbitrary
program sequence:
\begin{lem}
  \label{thm:dis-general}
  Let $p: T A$ be discardable and $q : T B$ be an arbitrary program. Then
  \[ \DoStmt{p; q} = q \]
\end{lem}
Most proofs of the propositions in this section are by equational reasoning;
here is an example proof of the above lemma.
\begin{proof}
  \begin{align*}
    \DoStmt{p; q} &= \DoStmt{p; \ret~\unit; q}  && \text{(since
      $\DoStmt{\ret~\unit; q} = q$)}\\
                  &= \DoStmt{\ret~\unit; q} && \text{($p$ discardable)}\\
                  &= q
  \end{align*}
\end{proof}

While discardability allows one to omit certain programs altogether whose return
value is not used in the remainder, the following concept admits
statements about the behaviour of certain programs when they are executed
repeatedly:
\begin{defn}
  \label{defn:cp}
  Let $p : T A$ be a program. $p$ is called \emph{copyable} if the following
  equation holds:
  \[ \DoStmt{x\leteq p; y\leteq p; \ret~(x,y)} = \DoStmt{x\leteq p; \ret~(x,x)}
  \] 
\end{defn}
As with discardability, copyability entails a stronger form of program equality
which expresses the fact that copyable programs may be doubled (or cancelled,
taken the opposite way) without effect more directly:
\begin{prop}
  \label{thm:cp-general}
  Let $p : T A$ be a copyable and $r : T B$ be an arbitrary program possibly
  containing $y$ as a free variable. Then one has
  \[ \DoStmt{x\leteq p; y\leteq p; r} = \DoStmt{x\leteq p; r[x/y]} \]
\end{prop}


For various monads, the deterministically side effect free programs comprise the
copyable and discardable programs. That this type is not empty can easily be
seen by considering stateless programs of the form $\ret~a$, which are
discardable and copyable at any rate. These programs are also deterministically
side effect free in the general sense, which depends on one further concept.
\begin{defn}
  \label{defn:commutes}
  Let $p$ and $q$ be copyable and discardable programs with $x \notin FV(q)$ and
  $y \notin FV(p)$. Then $p$ \emph{commutes with} $q$ if the following three
  equivalent conditions hold:
\begin{align}
  \label{eq:commutes-1}
  \DoStmt{x\leteq p; y\leteq q; \ret~(x,y)} &\text{ is a copyable program}\\
  \label{eq:commutes-2}
   \DoStmt{x\leteq p; y\leteq q; \ret~(x,y)} &= \DoStmt{y\leteq q; x\leteq p;
    \ret~(x,y)}\\
  \label{eq:commutes-3}
  \DoStmt{x\leteq p; y\leteq q; r} &= \DoStmt{y\leteq q; x\leteq p; r}
\end{align}
\end{defn}

\begin{defn}
  \label{defn:dsef}
  A copyable and discardable program $p$ that commutes with \emph{all} copyable
  and discardable programs is called \emph{deterministically side effect free}
  (dsef).
\end{defn}

\begin{prop}
  Dsef programs are stable under sequential composition, \IE for every sequence
  $\bar x\leteq \bar p$ of dsef programs and every dsef program $q$, the program
  $\DoStmt{\bar x\leteq\bar p; q}$ is also dsef.
\end{prop}

\begin{rem}
\label{rem:isa-type-restr}
As a rather technical aside, \Isabelle imposes the restriction of quantifying
not over all programs of all possible types, but merely over all programs of a
fixed type.  Fortunately, a program already commutes with all discardable and
copyable programs if it commutes with all such programs of type $T\Omega$. Therefore,
the property of a program $p$ being dsef can also be expressed with more
stringent type constraints in \Isabelle.
\end{rem}
 
%% dsef hier einbauen, unten streichen. Dann: commutes-relation nur durch
%% primitive ausdruecken, [[\ldots]] kommt spaeter dazu. Getreu der Philosophie: erst
%% die Syntax, dann die Semantik\ldots oder so.

\subsection{Global Dynamic Judgements}
\label{sec:glob-dynam-judg}

Before introducing logical operators for dsef terms (viewed as formulae), we
clarify when such a formula is to be regarded as valid.  Contending the global
validity of a term of type $T\Omega$ (notation $\gbox \phi$)\footnote{note that the type
$T\Omega$ indicates that global validity is also defined for `formulae' with side
effects} amounts to saying that
\[
\phi = \DoStmt{a\leteq\phi; \ret~\top}
\]
\IE basically $\phi$ evaluates to truth ($\top$) if it yields any results at all. In
the state monad (resp. the exception monad), this equation also holds if $\phi$
is undefined (resp. throws an exception), while in the nondeterminism
monad it also holds if $\phi$ does not produce any results at all. For the
important special case when $\phi$ is discardable, $\gbox \phi$ reduces to $\phi = \ret~\top$.

Although global validity is a sufficiently strong concept to express when a term
of type $T\Omega$ is to be considered valid, there are monads (albeit rather exotic
ones such as the free Abelian group monad, see \cite[Section
3]{SchroederMossakowski:PDL}) for which it is too weak to give a semantics to
the box and diamond operators. For this to be possible in a most general manner,
the similar but more powerful notion of a \emph{global dynamic judgement} is
necessary: let $\Gdj{\bar x\leteq \bar p}{\phi}$ abbreviate
\[ \DoStmt{\bar x\leteq \bar p; \ret~(\bar x, \phi)} = \DoStmt{\bar x\leteq \bar p; \ret~(\bar
  x, \top)} \]
(note that $\phi : \Omega$ in $\Gdj{x\leteq p}{\phi}$, \IE $\phi$ is an actual formula,
whereas $\psi : T \Omega$ is a monadic term in $\gbox \psi$). 
\begin{rem} \label{rem:simple-monads}
  The monads that serve as examples in this thesis have been called \emph{simple}
  in \cite{SchroederMossakowski:PDL}; in simple monads the equivalence of the
  two statements $\gbox (\DoStmt{x\leteq p; \ret~\phi})$ and $\Gdj{x\leteq p}{\phi}$
  holds, such that one of these concepts would actually be sufficient. 
\end{rem}

The following lemma once more shows that a more general statement (in this case
about global dynamic judgements) drops out of an apparently primitive definition.
\begin{lem}
  \label{thm:gdj-general}
  If $\Gdj{\bar x\leteq \bar p}{\phi}$ holds, 
  \[ \DoStmt{\bar x\leteq \bar p; q[\phi/y]} = \DoStmt{\bar x\leteq \bar p; q[\top/y]}
  \]
  for each program $q$ containing $y : \Omega$ as a free variable.
\end{lem}
\begin{proof}
  Again, by a direct calculation: let $\pi_i$ denote the $i$-th projection
  function and let $\theta$ be the substitution $\{(\pi_1\cdot \pi_1)z/x_1, \ldots, (\pi_n\cdot
  \pi_1)z/x_n, \pi_2z/y\}$ which replaces $x_i$ and $y$ by their respective selection
  from the tuple $z= ((x_1,\ldots,x_n), y)$. Then
  \begin{align*}
       \DoStmt{\bar x\leteq \bar p; q[\phi/y]} &= 
       \DoStmt{\bar x\leteq \bar p; y\leteq \ret~\phi; q} \\
    &= \DoStmt{\bar x\leteq \bar p; y\leteq \ret~\phi; z\leteq \ret~(\bar x, y);
      (q)\theta} \\
    &= \DoStmt{z\leteq\DoStmt{\bar x\leteq \bar p; y\leteq \ret~\phi; \ret~(\bar x,
        y)}; (q)\theta}\\
    &= \DoStmt{z\leteq\DoStmt{\bar x\leteq\bar p; \ret~(\bar x, \phi)}; (q)\theta}\\
    &= \DoStmt{z\leteq\DoStmt{\bar x\leteq\bar p; \ret~(\bar x, \top)}; (q)\theta} &&(\star)\\
    &= \DoStmt{z\leteq\DoStmt{\bar x\leteq\bar p; y\leteq \ret~\top; \ret~(\bar x,
        y)}; (q)\theta}\\
    &= \ldots = \DoStmt{\bar x\leteq \bar p; q[\top/y]}
  \end{align*}
  where to arrive at $(\star)$ the assumption  $\Gdj{\bar x\leteq \bar p}{\phi}$ has
  been used.
\end{proof}

\begin{cor}
  \label{thm:gdj-gbox-equiv}
  One has $\Gdj{a\leteq \phi}{a}$ if and only if $\gbox \phi$. The implication from
  left to right is a direct consequence of Lemma \ref{thm:gdj-general},
  recalling that $\phi = \DoStmt{a \leteq \phi; \ret~a}$, whereas the implication from
  right to left is, again, a manipulation with the help of unit and
  associativity laws of monads.
\end{cor}

We will not devote ourselves to developing an entire calculus of global dynamic
judgements -- which indeed already is expressive enough to formulate a Hoare
calculus for partial correctness with it, as has been done in
\cite{SchroederMossakowski:Hoare} -- but rather make use of it to define the
modal operators of monadic dynamic logic. Global dynamic judgements are also
useful to formalise what it means for a program to terminate:
\begin{defn}
  \label{defn:termination}
  A program $p$ \emph{terminates} if
  \[ \Gdj{\bar x \leteq \bar q; p}\phi \quad \text{implies} \quad \Gdj{\bar x\leteq
    \bar q}\phi\]
  for each program sequence $\bar x\leteq \bar q$ and each formula $\phi : \Omega$.
\end{defn}
\begin{expl}
  \label{ex:term-except}
  Obviously, $\phi :\Omega$ in $\Gdj{\bar x\leteq \bar q; p}\phi$ cannot mention the result
  of $p$ since this result is not bound. To see how the above definition accords
  with the intuitive understanding of termination, consider the simplest
  possible exception monad where $T A = A + \{\bot\}$. In this setting, it is
  reasonable to talk of nontermination of a program $p$ if it throws an
  exception (\IE $p = \bot$). In this case $\Gdj{\bar x\leteq \bar q; p}\bot$ will be
  true for every program sequence $\bar x \leteq \bar q$ since $\DoStmt{\bar
    x\leteq \bar q; p; \ret~(\bar x, \bot)} = \bot = \DoStmt{x\leteq q;p; \ret~(\bar x,
      \top)}$ (recall the definition of binding in the exception monad). $\Gdj{\bar
        x\leteq \bar q}\bot$ will however be false for every  program
      sequence $\bar x\leteq \bar q$ not throwing any exceptions.
\end{expl}
\begin{rem}
  Reasoning about termination in the state monad (recall that here $T A$ takes
  the form $S \to S \times A$, \IE a function space of \emph{total} functions) only
  makes sense if either \emph{partial} functions are considered or the theory of
  \emph{complete partial orders} (cpos) with \emph{continuous functions} between
  them is employed. In a setting where programs are partial functions $f : S \rightharpoonup S
  \times A$, one finds that the above definition of termination precisely identifies
  the terminating programs with the total functions -- given an adapted
  definition of binding that takes the possibility of undefinedness of programs
  into account. Since in \IsabelleHOL every function is implicitly total, we
  also stick to this principle in the overall development, explicitly indicating
  when more structure is necessary, \EG in the definition of arbitrarily
  recursive definitions like that of a while-loop.
\end{rem}


The greater freedom in treatment that dsef programs are characterised by also
shows up when they appear in global dynamic judgements. Several properties such
as the equivalence of $\Gdj{\bar w\leteq \bar q; x\leteq p; y\leteq p; \bar
  z\leteq \bar r}\phi$ and $\Gdj{\bar w\leteq \bar q; x\leteq p; \bar z\leteq \bar
  r[x/y]}\phi[x/y]$ can be proved for a dsef program $p$. This leads to three
notational conventions that allow one to use dsef programs in places where
actual values are expected, and vice versa.  Put concretely, we allow
\begin{enumerate}
\item a dsef program $p : DA$ to occur in places where a variable $x : A$ is
  expected; the program $q[p/x]$ decodes into $\DoStmt{x\leteq p; q}$.
\item a formula $\psi : D\Omega$ to occur in places where a genuine formula $a: \Omega$ can
  appear in global dynamic judgements; the judgement $\Gdj{\bar x\leteq \bar
    p}\phi[\psi/a]$ decodes into $\Gdj{\bar x\leteq \bar p; a\leteq \psi}\phi$. Note that
  here the evaluation of $\psi$ takes place \emph{after} having evaluated $\bar
  x\leteq \bar p$, whereas in (1.) $p$ is evaluated \emph{before} $q$.
\item formulae of type $\Omega$ to be inserted in places where
  actually a formula of type $D\Omega$ is expected, since the former type can easily
  be cast to the latter through $\ret$. This is convenient if several stateless
  formulae are involved which, for instance, make statements about the data on
  which a program is supposed to work. 
\end{enumerate}
The specification of the tree-search algorithm in Section~\ref{sec:bfs} is an
example where this convention is employed. Compare also with
Remark~\ref{rem:cast-ret-isabelle} on how this convention is handled in
Isabelle.

\section{Logical Operators}
\label{sec:logic-operators}


\subsection{Primitive Connectives}
\label{sec:prim-conn}

The logical operators are defined in terms of already existing logical operators
for the type of truth values $\Omega$. So we assume that the background formalism at
least allows the formulation of the standard propositional connectives; this is
certainly the case for \IsabelleHOL which even allows the formulation of
higher-order functions and predicates. We will use the same symbols for both
actual formulae of type $\Omega$ as well as \emph{formulae of dynamic logic} of type
$D\Omega$; it will be clear from the context which of them is meant. Let
$\mathit{op}$ stand for conjunction $\land$, disjunction $\lor$, implication $\Rightarrow$ or
equivalence $\iff$ of two formulae of dynamic logic $\phi, \psi: D\Omega$ respectively. Then
these connectives are defined as
\begin{equation} \label{eq:defn-log-conn}
 \phi\ \mathit{op}\ \psi \quad\defeq\quad \DoStmt{a\leteq \phi; b\leteq \psi; \ret~(a\ \mathit{op}\ b)} 
\end{equation}
Negation is of course similarly defined  as
\[ \lnot \phi \quad \defeq \quad \DoStmt{a\leteq \phi; \ret~(\lnot a)} \]

First-order operators like a universal quantifier are not available for formulae
of dynamic logic; they may however appear in \emph{stateless formulae}, \EG of
the form $\ret~(\forall x\bdot P(x))$, if the underlying formalism allows their
formulation for formulae of type $\Omega$. An example thereof can be found in Chapter
\ref{cha:application} within the specification of a breadth-first search
algorithm.

Asserting the validity of a formula of dynamic logic can be done in two
equivalent ways, due to the existence of two different notations and their
relation to each other. The `global box' $\gbox$ basically serves the purpose of
asserting validity of a formula: $\gbox(\phi\land\psi)$ decodes into
$\gbox(\DoStmt{a\leteq \phi; b\leteq \psi; \ret~(a\land b)})$ according to the definition
of conjunction and is to be read as ``it is globally true that $\phi \land \psi$ holds''.
By Corollary \ref{thm:gdj-gbox-equiv}, an equivalent formulation is to say that
$\Gdj{a\leteq \phi; b\leteq \psi}(a\land b)$ holds. It is important to note that all
propositional tautologies carry over into the calculus of monadic dynamic logic:
$\phi \Rightarrow (\psi \Rightarrow \phi)$ is globally valid, since global validity amounts to $\Gdj{a\leteq
  \phi; b\leteq \psi; c\leteq \phi}(a \Rightarrow (b \Rightarrow c))$ being valid. The latter judgement is
valid because by Lemma \ref{thm:cp-general} it is equivalent to $\Gdj{a\leteq \phi;
  b\leteq \psi}(a \Rightarrow (b \Rightarrow a))$ in which $a \Rightarrow (b \Rightarrow a)$ is a tautology (in $\Omega$), thus
equal to $\top$.


\subsection{Boxes and Diamonds}
\label{sec:box-diamond}

The key feature of monadic dynamic logic is the existence of modal operators
that allow building \emph{formulae} (\IE terms of type $D\Omega$) stating that after
execution of a program some condition will necessarily or possibly hold. This is
in contrast to the global box $\gbox$ and the global dynamic judgements which, as
the name suggests, merely allow the formulation of \emph{global} statements
about program sequences and properties of their bound variables. The semantics
of the diamond and box operators $\PDLBox{x\leteq p}{\phi}$ and $\PDLDmd{x\leteq
  p}{\phi}$ is \emph{local} in the sense that the state in which $\phi$ is evaluated
may be modified by $p$, but the entire formula does not modify the state in
which itself is evaluated. Hence, it may appear as a sub-formula without
affecting the semantics of surrounding sub-formulae.

\begin{expl}
  The axiomatic introduction of the box and diamond operators given below does
  not quite point to an idea of what they intuitively express. We therefore give
  their intended interpretation for the monads described in
  Section~\ref{sec:monads-cs} as a motivation for their usefulness.
  \begin{itemize}
  \item In the state monad of total functions $\PDLBox{x\leteq p}{\phi}$ and
    $\PDLDmd{x\leteq p}{\phi}$ depend on the state. They denote the same
    formula which is true in a state $s$ if after execution of $p$ the result
    $x$ will satisfy $\phi$. If partial functions are involved $\PDLBox{x\leteq
      p}{\phi}$ is actually weaker than $\PDLDmd{x\leteq p}{\phi}$ in that the former
    is also true if $p$ is undefined.
  \item In the exception monad $\PDLBox{x\leteq p}{\phi}$ holds if $p$ throws an
    exception or yields a value satisfying $\phi$, whereas for $\PDLDmd{x\leteq
      p}{\phi}$ to hold it is additionally required that $p$ does not throw an
    exception.
  \item In the nondeterminism monad, where $p : T A$ is a set of elements of
    $A$, $\PDLBox{x\leteq p}{\phi}$ holds if all elements in $p$ satisfy $\phi$ (which
    also includes the case where $p = \emptyset$) and  $\PDLDmd{x\leteq p}{\phi}$ is true
    if and only if $p$ contains some value satisfying $\phi$.
  \item Finally, in the combination of the list monad and the state monad the
    modal operators depend on the state as well. Validity of $\PDLBox{x\leteq
      p}{\phi}$ (or $\PDLDmd{x\leteq p}{\phi}$) in a state $s$ means that all outcomes
    of $p$ satisfy $\phi$ (or at least one outcome satisfies $\phi$).
  \end{itemize}
\end{expl}


The following definition formalises the essential requirement that a monad must
satisfy in order to allow the interpretation of monadic dynamic logic. The
somehow dual operators $\PDLBox{x\leteq p}\Arg$ and $\PDLDmd{x\leteq p}\Arg$ are
introduced independently of each other in order to make their particular
interpretation possible in intuitionistic logics as well. In a classical setting,
one might define $\PDLDmd{x\leteq p}{\phi}$ as $\lnot \PDLBox{x\leteq p}\lnot \phi$, and in
fact this equivalence is shown to hold in \Isabelle later on.
\begin{defn}
  A monad \emph{admits dynamic logic} if there exist formulae $\PDLBox{\bar
    y\leteq \bar q}\phi$ and $\PDLDmd{\bar y\leteq \bar q}\phi$ for each program
  sequence $\bar y\leteq \bar q$ and each formula $\phi : D\Omega$ such that for each program
  sequence $\bar x\leteq \bar p = x_1\leteq p_1; \ldots; x_n\leteq p_n$ containing
  $x_i : \Omega \; (1\leq i\leq n)$ the following equivalences hold:
  \begin{align*}
    \Gdj{\bar x\leteq \bar p}(x_i \Rightarrow \PDLBox{\bar y\leteq\bar q}\phi)
      &\iff \Gdj{\bar x\leteq \bar p; \bar y\leteq\bar q}(x_i \Rightarrow \phi) \\
    \Gdj{\bar x\leteq\bar p}(\PDLDmd{\bar y\leteq\bar q}\phi \Rightarrow x_i)
      &\iff \Gdj{\bar x\leteq\bar p; \bar y\leteq\bar q}(\phi \Rightarrow x_i)
  \end{align*}
  The purpose of using the variable $x_i$ is  generality: one can express
  every formula $\psi$ in context of the other $x_j$ through it: simply put $x_i =
  x_n$ and $p_n = \ret~\psi$. Note also that the above equivalences make use of the
  notational convention of letting formulae of monadic logic appear where a
  formula of type $\Omega$ is expected: in decoded form the first equivalence reads
  as 
  \[  \Gdj{\bar x\leteq \bar p; a\leteq \PDLBox{\bar y\leteq\bar q}\phi}(x_i \Rightarrow a)
  \iff \Gdj{\bar x\leteq \bar p; \bar y\leteq\bar q; b\leteq \phi}(x_i \Rightarrow b) \]
  and similar for the second one.
\end{defn}

We state some basic properties that accompany the box and diamond operators.
\begin{prop}[Unique determination] \label{thm:unique-determ}
  One can turn the type of dsef programs $D\Omega$ into a partial order by setting $\phi
  \leq \chi$ if and only if $\phi \Rightarrow \chi$. Then $\PDLBox{\bar y\leteq\bar q}\phi$ is the
  greatest formula $\psi$ such that 
  $\Gdj{a\leteq \psi; \bar y\leteq\bar q}(a\Rightarrow\phi)$
  and $\PDLDmd{\bar y\leteq\bar q}\phi$ is the smallest formula $\psi$ such that 
  $\Gdj{a\leteq \psi; \bar y\leteq\bar q}(\phi\Rightarrow a)$.
\end{prop}
A proof of this proposition involves two steps: first, it has to be shown that
for each formula $\psi$ satisfying $\Gdj{a\leteq \psi; \bar y\leteq\bar q}(a\Rightarrow\phi)$ (or
$\Gdj{a\leteq \psi; \bar y\leteq\bar q}(\phi\Rightarrow a)$) one has $\psi \Rightarrow \PDLBox{\bar y\leteq\bar
  q}\phi$ (or $\PDLDmd{\bar y\leteq\bar q}\phi \Rightarrow \psi$). Second, it must be shown that
$\PDLBox{\bar y\leteq\bar q}\phi$ ($\PDLDmd{\bar y\leteq\bar q}\phi$) in fact satisfy
the judgements. Both parts of the proof follow more or less immediately from the
definition of the box and diamond operators.

\begin{prop}[Global validity of box formulas]
  Let $\bar y\leteq \bar q$ be an arbitrary program sequence and $\phi : D\Omega$ a
  formula. Then $\gbox(\PDLBox{\bar y\leteq\bar q}\phi)$ is equivalent to
  $\Gdj{\bar y\leteq \bar q}\phi$.
\end{prop}

The following equivalence allows us to reason about termination within the
calculus of monadic dynamic logic without having to fall back to reasoning about
global dynamic judgements.
\begin{prop}[Termination]
  A program $p$ terminates in the sense of Definition \ref{defn:termination} if
  and only if $\PDLDmd{p}\top$ holds.
\end{prop}


\subsubsection{Defining the Modal Operators}
In monads with additional structure that besides imposing some minor logical
well-be\-haved\-ness basically allows one to `read the current state' -- a
property which virtually all of the running example monads possess -- it is
possible to directly define the box operator\footnote{even in the intuitionistic
  case the diamond operator can then be defined in terms of the box operator,
  albeit in a rather contrived way that we will not present here}. This
definition is shown now as it enlightens the particular locality of the box
operator's semantics.

The general idea is that dsef programs can essentially be regarded as programs
that may read the `state', but not alter it, \IE there is an isomorphism between
the type $D A$ of dsef programs over $A$ and the function space $F \to A$, where
$F$ is the type of states (see below). With the help of this isomorphism, one
may describe the box operator $\PDLBox{x\leteq p}\phi$ as a function that maps the
current state to a global dynamic judgement (hence, a formula of type $\Omega$)
asserting that after setting this state and executing $p$, the formula $\phi$ will
be true. We need some definitions to make these ideas precise.
The notion of state has to be abstracted
from the  set of concrete state values $S$ in the state monad to a concept that
also makes sense in other monads.

\begin{defn}
  A \emph{state} is a terminating program $s : T1$ such that for each dsef
  program $p : DA$ there exists an element $a : A$ such that
  \[ \DoStmt{s; p} = \DoStmt{s; \ret~a} \]
  If for each terminating program $q$ one furthermore has
  \[ s             = \DoStmt{q; s} \]
  then $s$ is called a \emph{forcible state}. The subtype of $T1$ of all
  forcible states is denoted by $F$.
\end{defn}

In the state monad, a state as just defined would rather be thought of as an
update operation: the function $\mathit{update}\ s' = \LambdaTerm{s:S}{(s',
  \unit)}$ of Section~\ref{sec:monads-haskell} yields a state when it is applied
to an element of the  set of concrete states $S$. In the exception and
nondeterminism monads there is only the trivial state $\ret~\unit$, which in
both cases is forcible; the special kind of list monad we have described does
not have forcible states: its states take the form $s_c = \LambdaTerm{i :
  \Type{List}\ I}{[(c, \unit)]}$ for $c : \Type{List}\ I$, but for the program
$q = \LambdaTerm{i:\Type{List}\ I}{[(a, \unit), (b, \unit)]}$ one has
\[
  \DoStmt{q; s_c} = \LambdaTerm{i:\Type{List}\ I}{[(j,\unit), (j,\unit)]} \neq s_c
\]
The basic problem is that an element can occur multiple times in lists, in
contrast to sets so that forcibility is only available when the latter are used,
\EG in the nondeterminism monad.


For the definition of the box operator we need a further operation that allows
one to extract the state. It is determined by the property that accessing the
state with the help of it and then immediately executing this state has no effect
(since the state will be the same afterwards as beforehand):
\begin{defn}
  A program $d : DF$ is called a \emph{state discloser} if the term
  $\DoStmt{x\leteq d; x}$ is discardable.
\end{defn}

It is now possible to establish that $D A \cong (F \to A)$ by defining two
isomorphisms $\kappa_A \Map{D A}{(F \to A)}$ and its inverse $\kappa^{-1}_A \Map{(F \to A)}{D
  A}$ for each type $A$ (the index $A$ will be omitted in the following). While
to be able to define $\kappa$ one needs a Hilbert description operator, its inverse
$\kappa^{-1}_A$ can be defined purely by means already available. Let $d : DF$ be a
state discloser, then for each function $f \Map{F}{A}$ the program $\kappa^{-1}(f)$
accesses the current state and applies $f$ to it, \IE one has
\[
  \kappa^{-1}( f )\quad \defeq\quad  \DoStmt{s\leteq d; \ret~(f~s)}
\]
which is a dsef program of type $D A$, recalling that both $d$ and $\ret$ are
dsef programs. This mapping allows us to describe the box operator as a function
in $F \to \Omega$, \IE as a state dependent truth value, and then subsequently inject
it into $D A$: $\PDLBox{\bar y\leteq\bar q}{\phi}$ can be interpreted in $F\to\Omega$ as a
function that returns the global validity of $\phi$ after executing the state $s$
followed by $\bar y\leteq\bar q$. This is formalised by the following definition
of the box operator:
\[
\PDLBox{\bar y\leteq\bar q}\phi \quad\defeq\quad \kappa^{-1} (\LambdaTerm{s: F}{\Gdj{s; \bar
    y\leteq\bar q}\phi})
\]

\section{The Monad-independent Proof Calculus}
\label{sec:monad-indep-calc}

The entire proof calculus for monadic dynamic logic can be formalised by adding
the rules and axioms of Figure \ref{fig:mon-dyn-logic} to the set of
propositional tautologies in $D\Omega$. Certainly the inclusion of \emph{all}
tautologies is overkill which might be prevented by only including an
independent and complete set of axioms for propositional logic\footnote{complete
  in the sense that every tautology can be proved from these axioms together
  with modus ponens}, but here we are mainly concerned with rules and axioms for
the modal operators. The soundness of the calculus has been established in
\cite{SchroederMossakowski:PDL}, whereas its completeness is still an open
issue.

\begin{myfigure}
%  \centering
  \textbf{Rules:} 
  \begin{displaymath}\displaystyle
    \quad \NRule{nec}{\phi}{\PDLBox{\bar x\leteq\bar p}\phi}
    \mbox{~~\begin{tabular}{c}$\bar x$ not free\\ in assumptions\end{tabular}}
    \qquad
    \NRule{mp}{\phi\Rightarrow\psi;\quad\phi}{\psi}
  \end{displaymath}
  
  \textbf{Axioms:}
  \begin{displaymath}
    \begin{array}{lll}
      \textrm{(K1)} & \PDLBox{\bar x\leteq\bar p}(\phi\Rightarrow\psi)\Rightarrow
      \PDLBox{\bar x\leteq\bar p}\phi\Rightarrow\PDLBox{\bar x\leteq\bar p}\psi\\
      \textrm{(K2)} & \PDLBox{\bar x\leteq\bar p}(\phi\Rightarrow\psi)\Rightarrow
      \PDLDmd{\bar x\leteq\bar p}\phi\Rightarrow\PDLDmd{\bar x\leteq\bar p}\psi\\
      \textrm{(K3$\Box$)} & \ret{\phi}\Rightarrow\PDLBox{p}\ret{\phi}\\
      \textrm{(K3$\Diamond$)} & \PDLDmd{p}\ret{\phi}\Rightarrow\ret{\phi}\\
      \textrm{(K4)} & \PDLDmd{\bar x\leteq\bar p}(\phi\lor\psi)\Rightarrow
      (\PDLDmd{\bar x\leteq\bar p}\phi\lor\PDLDmd{\bar x\leteq\bar
        p}\psi)\\
      \textrm{(K5)} & (\PDLDmd{\bar x\leteq\bar p}\phi\Rightarrow\PDLBox{\bar x\leteq\bar
        p}\psi)
      \Rightarrow
      \PDLBox{\bar x\leteq\bar p}(\phi\Rightarrow\psi)\\
      \textrm{(seq$\Box$)} & \PDLBox{\bar x\leteq\bar p;y\leteq q}\phi \iff
      \PDLBox{\bar x\leteq\bar p}\PDLBox{y\leteq q}\phi\\
      \textrm{(seq$\Diamond$)} & \PDLDmd{\bar x\leteq\bar p; y\leteq q}\phi\iff
      \PDLDmd{\bar x\leteq\bar p}\,\PDLDmd{y\leteq q}\phi\\
      \textrm{(ctr$\Box$)} &
      \PDLBox{x\leteq p; y\leteq  q}\phi\Rightarrow
      \PDLBox{ y\leteq\DoStmt{ x\leteq p; q}}\phi
      & (x\notin FV(\phi))\\
      \textrm{(ctr$\Diamond$)} & \PDLDmd{ x\leteq p; y\leteq
        q}\phi \Leftarrow   \PDLDmd{ y\leteq\DoStmt{ x\leteq p; q}}\phi& (x\notin FV(\phi))\\
      \textrm{(ret$\Box$)} & \PDLBox{x\leteq \ret{a}}\phi\iff\phi[a/x]\\
      \textrm{(ret$\Diamond$)} & 
      \PDLDmd{x\leteq \ret{a}}\phi\iff\phi[a/x] \\
      \textrm{(dsef$\Box$)} & \PDLBox{x\leteq p}{P}\iff P[p/x]& (p \text{ is }
      \op{dsef}) \\
      \textrm{(dsef$\Diamond$)} & \PDLDmd{x\leteq p}{P} \iff  P[p/x] & (p \text{ is }
      \op{dsef} 
    \end{array}
  \end{displaymath}
  \mylinesep
  \caption{The generic proof calculus of monadic dynamic logic}
  \label{fig:mon-dyn-logic}
\end{myfigure}

The side condition `$\bar x$ not free in assumptions' in the necessitation rule
is a typical side condition analogous to the one for the universal quantifier in
first-order logic; the term \emph{assumptions} is to be understood as it is used
in natural deduction and does not refer to the \emph{premiss} of the rule, $\phi$.
The axioms K3$\Box$ and K3$\Diamond$ refer to stateless formulae that are mere
injections of formulae $\phi : \Omega$. The first one expresses the fact that stateless
formulae continue to hold after execution of programs (whereas the inverse is
not true due to possible nontermination of the program $p$), and the second one
expresses the fact that stateless formulae that hold after terminating
executions of $p$ also hold unconditionally. The sequencing axioms seq$\Box$ and
seq$\Diamond$ allow one to freely split and join boxes and diamonds. 

Essentially the $K$ axioms are the intuitionistic counterpart to the usual $K$
axiom of classical modal logic, which is called $K1$ here (see
\cite{Simpson94}). Further $K$ axioms are however necessary to be able to prove
intuitionistically valid formulae. This is mainly due to the fact that the box
and diamond operators are defined independently of each other. It will be seen
in Chapter~\ref{cha:implementation} that the implementation of the calculus in
Isabelle behaves classically, so that in it the classical equivalence of
$\PDLDmd{x\leteq p}{P}$ and $\lnot\PDLBox{x\leteq p}{\lnot P}$ can be shown.


Two further axioms that are needed in Chapter \ref{cha:implementation} can only
be proved in so called \emph{logically regular monads} (cf. \cite[Def.
5.14]{SchroederMossakowski:PDL}). Essentially, logical regularity means that
arbitrary formulae $c : \Omega$ implying some global dynamic judgement can be moved
into the scope of that judgement, as follows
\[
  c \Rightarrow \Gdj{x\leteq p}{\phi} \quad\mathit{implies}\quad 
  \Gdj{x\leteq p}{(c \Rightarrow \phi)}
\]
This restriction is only necessary in the intuitionistic case; if the underlying
logic is classical one can show that all monads are logically
regular. Even in the intuitionistic case all monads that are under consideration
here are logically regular. The axioms allow one to substitute equals for equals
inside boxes and diamonds:

\vspace{1.5ex} \noindent  \textbf{Axioms:}
  \begin{displaymath}
    \begin{array}{lll}
      \textrm{(eq$\Box$)} & p = q \Rightarrow \PDLBox{x\leteq p}{\phi} \Rightarrow \PDLBox{x\leteq q}{\phi}\\
      \textrm{(eq$\Diamond$)} & p = q \Rightarrow \PDLDmd{x\leteq p}{\phi} \Rightarrow \PDLDmd{x\leteq q}{\phi}
    \end{array}
  \end{displaymath}

\subsection{Hoare Calculi}
\label{sec:hoare-calculi}

The calculus of monadic dynamic logic can be applied in order to define a Hoare
logic for partial as well as one for total correctness of monadic programs. In
Hoare logics for partial correctness of imperative programs one has assertions
of the form $\PHTriple{\phi}{p}{\psi}$, which are to be understood as ``if the
precondition $\phi$ holds before execution of $p$, then the postcondition $\psi$ will
hold afterwards if $p$ terminates''. This idea also makes sense for monadic
programs, but in fact it is already incorporated in dynamic logic by formulae of
the form $\phi \Rightarrow \PDLBox{p}{\psi}$. Likewise, one can give meaning to Hoare assertions
for total correctness by adding the requirement that a program terminates. This
leads to the following definition.

\begin{defn}
  \label{defn:hoare-assns}
  A \emph{Hoare assertion for partial correctness} of monadic programs is a formula
  \[ \phi \Rightarrow \PDLBox{\bar x\leteq \bar p}{\psi} \qquad (\text{written as} \quad
  \PHTriple{\phi}{p}{\psi}) \] 

  A \emph{Hoare assertion for total correctness} also requires the termination of the
  program under consideration and hence takes the following form:
  \[ \phi \Rightarrow (\PDLBox{\bar x\leteq \bar p}{\psi} \land \PDLDmd{\bar x\leteq\bar p}{\top})
  \qquad (\text{written as} \quad 
  \THTriple{\phi}{p}{\psi}) \]
\end{defn}

Classical Hoare rules like a sequencing rule or a context weakening rule
\[
  \Rule{\displaystyle \datop{\THTriple{\phi}{\bar x\leteq\bar p}{\psi}} 
      {\THTriple{\psi}{\bar y\leteq\bar q}{\chi}}}
       {\THTriple{\phi}{\bar x\leteq\bar p; \bar y\leteq \bar q}{\chi}}
       \hspace*{15mm}
  \Rule{\begin{array}{c} \THTriple{\phi}{\bar x\leteq \bar p}{\psi} \\
             \phi' \Rightarrow \phi \\  \forall\bar x\bdot \psi \Rightarrow \psi' \end{array}}
       {\THTriple{\phi'}{\bar x\leteq\bar p}{\psi'}}
\]
(which of course also exist for partial correctness assertions) are easily
derived in the proof calculus of dynamic logic. In the next section we will  make
use of a Hoare logic definable in this way for specifying and 
proving correct a pattern match algorithm. While Hoare logic represents a
convenient way of reasoning about programs in the state monad (which naturally
comes quite close to reasoning about simple imperative programming languages),
\EG the queue monad used in the next chapter does not lend itself to an
axiomatisation simply by means of Hoare assertions about the basic queue
operations. Hence, proofs about the queue monad will be conducted in the
calculus of dynamic logic.

\section{Specific Extensions for the Exception Monad}
\label{sec:spec-extens-except}

We have mentioned in Example \ref{ex:term-except} that non-termination in the
(simple) exception monad means that an exception has been thrown. So, given an
operation $\raiseEx : E \to T A$ which raises an exception from the set $E$ of
exceptions, one has $\PDLBox{\raiseEx~e}{\bot}$ so that ``anything can be proved in
the presence of an exception''. This might be acceptable as long as exceptions
simply indicate some kind of failure and it does not matter much which error
eventually occurred.  In this case, partial correctness explicitly does
\emph{not} say anything about whether the program actually terminated and total
correctness excludes all situations in which an exception occurred.  But as soon
as exceptions are employed to deliberately manipulate the control flow and if
they may carry values (\EG in the monad for Java of
\cite{JacobsPoll00,HuismanJacobs00}) this turns out to be a serious lack of
expressiveness. An extension of the basic Hoare calculus described above has
been given in \cite{SchroederMossakowski:Java} which makes it possible to reason
about so called \emph{abnormal postconditions} required to hold if an exception
has been thrown (as opposed to the \emph{normal postcondition} which must be
satisfied in case of regular termination). This extension relies on the presence
of an operation to turn an exceptional state back into a normal one, which is,
of course, the well-known $\catchEx : T A \to T(A + E)$ operation. As indicated by
this signature, $\catchEx$ simply makes an exception visible rather than
additionally requiring a handler to cope with the exceptional situation, as in
Haskell's $\op{catch} : T A \to (E \to T A) \to T A$. The latter is easily definable
in terms of the former.  In \cite{SchroederMossakowski:Java} a categorical
definition of exception monads is given\footnote{in which $\catchEx$ is taken to
  be a natural transformation between $T$ and $T(\Arg + E)$ such that it
  equalises the strong monad morphisms $\catchEx_{\Arg + E}$ and $T\inl$}, from
which however one can derive all equations that may intuitively be expected to
hold, \EG
\begin{equation}
  \label{eq:except-eqns}
  \begin{split}
    & \catchEx\ \DoStmt{x\leteq p; q\ x} = \\
    & \quad\DoStmt{y\leteq \catchEx\ p; \Case\ y\ \Of\ \inl~a \to \catchEx\ (q~a) \Alt \inr~e \to
    \ret\ (\inr~e)}\\
    & \catchEx\ (\ret~x) = \ret~(\inl~x) \\
    & \catchEx\ (\raiseEx\ e) = \ret~(\inr~e)
  \end{split}
\end{equation}
where the first equation states how $\catchEx$ behaves under sequential
composition of programs (in particular the second program $q$ is only executed
if $p$ did not throw any exceptions), the second one states that $\ret$ does not
throw any exceptions and the third one expresses how $\catchEx$ interacts with
$\raiseEx$, namely that it precisely returns the exception thrown by this
operation. 

With these defining equations for $\catchEx$ available, one may reason in the
regular Hoare calculus by wrapping up all programs with a $\catchEx$ and doing a
case distinction about the return value of $\catchEx$ in the postcondition:
\[
\PHTriple{\phi}{y \leteq (\catchEx\ \xp)}{\Case\ y\ \Of\ \inl~x \to \psi \Alt \inr~e \to S~e}
\]
The abnormal postcondition $S : E \to D\Omega$ is a stateful predicate on exception
values and may not mention the normal return value $x$, whereas $\psi$ of the
normal postcondition may contain $x$ freely. This scheme can be given a more
convenient notation by explicitly distinguishing between normal and abnormal
postcondition and leaving the ubiquitous $\catchEx$ unmentioned:
\[
\EPHTriple{\phi}{\xp}{\psi}{S}
\]

It is now possible to derive a Hoare calculus to reason about exception monads,
including rules for sequential composition of programs, a rule for $\raiseEx$ as
well as one for $\catchEx$, etc.  Figure \ref{fig:exc-partial-hoare} lists all
rules that apply to arbitrary exception monads; in particular note rule (raise)
which shows how the problem of giving a reasonable postcondition for $\raiseEx$
has been resolved. \Eat{A two-step re-translation of these rules into dynamic logic
proceeds as described above and in Section \ref{sec:hoare-calculi}.}

\subsection{Parameterised Exceptions}
\label{sec:param-except}

As a concrete example, we will now describe how to translate the exception
handling mechanism of the Java programming language into the calculus described
here. It will then appear that one further extension has to be made, since in
Java even \code{return} statements terminate abnormally, resulting in exceptions
carrying values of an arbitrary type. The stipulation that \code{return} (and
\code{break} and \code{continue}) statements terminate abnormally is not
specific to the model of Java given here, but rather settled in the Java
language specification \cite{JoySteeleGoslingBracha00}.  To deal with this
situation a conversion function $\mbody$ is required that mediates between
slightly different monads. This is due to the fact that every concrete monad may only
carry exception values of a fixed type, as will be seen, whereas \emph{return
  exceptions} of different methods may have entirely unrelated types -- which is
naturally so, since methods may have different return types.

The fact that certain statements terminate abnormally suggests the following
data type be used as the type of exceptions -- ignoring for the time being the
class-hierarchy of exceptions rooting in class \code{Exception}, \IE all
run-time errors like \code{Array\-OutOf\-Bounds\-Exception} or
\code{IOException}. The main point to be made here is how to model the hidden
exceptions that do not show up as such within a real Java program. So let
\[
E\ a = \op{MRet}\ a \Alt \op{FallenOff} \Alt \op{Break} \Alt \op{Cont} \Alt \op{Error}
\]
where $\op{MRet}\ a$ represents a return exception carrying the value
which was the argument of the \code{return} statement that raised the exception.
$\op{FallenOff}$ will be raised by the yet to be defined $\mbody$ operation to
indicate that its argument illegally terminated normally, $\op{Break}$ and
$\op{Cont}$ are exceptions raised by \code{break} and \code{continue} statements
respectively, and $\op{Error}$ is an exception that slightly over-simplifyingly models
all other cases.
 
The monad in which the semantics of sequential Java is modelled best is the
state monad extended by exceptions and nontermination (where the latter is
treated similar to an exception by the binding operation)
\[
T\ a\ b = S \to S \times (b + E\ a) + 1
\]
such that $T\ a$ is an exception state monad for each each type $a$ in which
binding respects exceptions, \IE in $\DoStmt{x\gets p; q}$ the program $q$ is only
evaluated if $p$ did not raise an exception. In this monad one has $\catchEx :
T\ a\ b \to T\ a\ (b + E\ a)$ and $\raiseEx : E \to T\ a\ b$ for all types $a$ and
$b$. The type of $\catchEx$ already points out that it is not possible to switch
between monads of different exception types; this precludes the applicability of
this model in situations where \EG one method of Java-return type \code{int} is
called within another method of return type \code{boolean}. The following
example demonstrates the problem. 
\begin{expl} \label{ex:java-to-monad} Let $\op{mret}\ x$ abbreviate $\raiseEx\
  (\op{MRet}\ x)$, then the Java methods
\begin{alltt}
public static int f(int x) {
  if (g(x) < 0)
    return x + 1;
  else
    return x - 1;
}

public static boolean g(int x) {
  return x*x < 100;
}
\end{alltt}
might na\"\i vely be translated into the monadic model  to obtain
{
  \newcommand{\filla}{\hspace*{1.8cm}}
  \newcommand{\fillaa}{\hspace*{2.5cm}}
  \begin{flalign*}
    & 
    \begin{array}{l}
         f : \Type{Int} \to T\ \Type{Int}\ a \\
         f\ x = \DO\ \{\ r\leteq \catchEx\ (g~x);\\
         \filla \Case\ r\ \Of\\
         \fillaa \inl~(\op{MRet}~b) \to \If\ b\ \Then\ \op{mret}~(x+1)\ \Else\
         \op{mret}~(x-1)\\
         \fillaa \Arg \quad \to \raiseEx\ \op{Error}\ \}\\[2ex]
         g : \Type{Int} \to T\ \Omega\ a\\
         g\ x = \op{mret}\ (x \cdot x < 100)
    \end{array} & \mbox{}
  \end{flalign*}
}

But this results in a type error, since the program $\catchEx\ (g\ x)$ has type
$T\ \Omega\ (a + \op{Int})$ in $f$, which itself is a monadic computation in $T\
\op{Int}$. Thus, the two monadic computations are incompatible. Intuitively, it
should be possible to resolve this incompatibility, as the type of exceptions
$g$ may throw is not of importance to the exception type of $f$ (all calls to
methods are enclosed by $\catchEx$ and hence cannot propagate into $f$). In
fact, this can be achieved in a way that simultaneously avoids having to enclose
every method call by a $\catchEx$. The key to this solution is the observation
that every exception monad $T$ can be obtained by applying the \emph{exception
  monad transformer} (well known as \code{ErrorT} from the Haskell libraries) to
some existing monad $R$ such that $T$ is isomorphic to $R(\Arg + E)$. Basically,
this says that for every exception monad there is some underlying monad such
that they share the same structure, but the exception monad only lives on result
types enriched by some set $E$ of exceptions. In the case at hand, $R$ simply is
the state monad with non-termination, and binding in $R\ (\Arg+E) = S \to S \times
(\Arg+E) + 1$ means binding as defined for the state monad and not for the
exception monad.  The practical consequence of this relationship is that one can
also write programs in $R(\Arg+E)$ and convert them to $T$ via \code{ErrorT},
which is precisely what is done for $\mbody$. We refer to Appendix
\ref{cha:hask-impl-mbody}, p.~\pageref{page:mbody-def}, for a Haskell
implementation of $\mbody$ and the exception monad transformer. The pivotal
property of $\mbody$ from the viewpoint of the exception monad $T$ is that it
converts the exceptional state of a computation back into a normal one if a
return exception has been raised, but lets all other exceptions pass -- thus
making it \emph{polymorphic in its own exception type}. Additionally, in case of
normal termination of its argument, $\mbody$ will raise a $\op{FallenOff}$
exception.  Its type therefore is
\[
  \mbody : T\ a\ b \to T\ c\ a
\]

When translating Java methods into the monadic setting, one will thus enclose
the translation $m$ of every method body \code{m} of function \code{f} by
$\op{mbody}$ to obtain the translated function $f$. Conducted in this manner,
the translation of the above Java methods then is
{
\newcommand{\filla}{\hspace*{1.8cm}}
\newcommand{\fillaa}{\hspace*{2.5cm}}
\begin{flalign*}
  &
  \begin{array}{l}
    f : \Type{Int} \to T\ a\ \Type{Int}  \\
    f\ x = \op{mbody}\ (\\
    \filla \DO\ \{\ b\leteq g~x;\\
    \fillaa \If\ b\ \Then\ \op{mret}\ (x+1)\ \Else\ \op{mret}\ (x-1)\\
    \filla \}\ )\\[2ex]
    g : \Type{Int} \to T\ b\ \Omega\\
    g\ x = \op{mbody}\ (\ \op{mret}\ (x\cdot x < 100)\ )
  \end{array} & \mbox{}
\end{flalign*}
}
\end{expl}

Since every program $p$ obtained from a translation of a Java method into the
monadic setting will now contain an occurrence of $\mbody$, it is necessary at
this point to specify and prove a Hoare rule for this construct which captures
its decisive properties. Fortunately, one single rule suffices for this purpose
in the case of partial as well as total correctness assertions (and both rules
look alike so that only one of them is shown), noting that one will only want to
prove properties of programs that terminate abruptly with a return exception.
\[
    \NRule{mbody}
    {\EPHTriple{\phi}{x\leteq p}{\bot}{\LambdaTerm{e}{\Case\ e\ \Of\ \op{MRet}\ y \to \psi
          \Alt e \to S\ e}}}
    {\EPHTriple{\phi}{y\leteq \mbody\ p}{\psi}{S}}
\]

\subsubsection{Correctness of a pattern match algorithm}
\label{sec:corr-patt-match}

As an example of how to apply the extended calculus to realistic programs, we
will specify and prove the correctness of a pattern match algorithm which
searches for a given sub-pattern in a given base pattern.  The algorithm is
implemented in an exception monad with dynamic references and a while loop; the
existence of the latter implicitly presupposes additional structure on the
monad, see 
\cite[Section 7] {SchroederMossakowski:PDL} for details and Appendix \ref{cha:hask-impl-mbody}
for an implementation. One therefore has to axiomatise additional operations on
the monad (apart from $\ret$ and $\bindOp$); the corresponding specification is
shown in Figure~\ref{fig:spec-ex-ref-mon}.  A condensed version of this proof
already appeared in \cite{WalterEA05}, while here we provide the full picture.

{
\newcommand{\filla}{\hspace*{1cm}}
\newcommand{\fillaa}{\hspace*{2cm}}
\newcommand{\fillaaa}{\hspace*{3cm}}
\newcommand{\fillb}{\hspace*{0.5cm}}
\begin{flalign*}
  &
  \begin{array}{l}
     \op{pmatch} : \Type{List}\ a \to \Type{List}\ a \to T\ e\ \Type{Nat}\\
     \op{pmatch}\ \op{base}\ \op{pat} = \op{mbody}\ (\ \DO\ \{\\
     \filla r\leteq \op{new}\ 0;\\
     \filla s\leteq \op{new}\ 0;\\
     \filla \While\ (\ret~\top)\\
     \filla \filla (\DO\ \{\ u\leteq \rd{r};\\
     \filla \filla \filla v\leteq \rd{s};\\
     \filla \filla \filla \If\ u = \op{len}\ \op{pat}\\
     \fillaaa \fillb \Then\ \op{mret}\ v\\
     \fillaaa \fillb \Else\ \If\ v + u = \op{len}\ \op{base}\\
     \fillaaa \filla \Then\ \raiseEx\ \op{Error}\\
     \fillaaa \filla \Else\ \If\ \op{base}!!(v+u) = \op{pat}!!u\\
     \fillaaa \filla \fillb \Then\ r := u+1\\
     \fillaaa \filla \fillb \Else\ \DO\ \{\ s := v+1;\ r:= 0\} \\
     \filla \filla \})\\
     \filla \})
   \end{array} &\mbox{}
\end{flalign*}
}

    
This definition of $\op{pmatch}$ is almost identical to the Haskell
implementation to be found in Appendix \ref{cha:hask-impl-mbody}, with slight
modifications to retain the notation used so far. It introduces a type
constructor $\op{List}$ mapping each type $a$ to the type of lists over $a$, a
length function $\op{len} : \op{List}\ a \to \op{Nat}$ and an indexing function
$!! : \op{List}\ a \to \op{Nat} \to a$ operating on these lists in the usual way --
where the latter is undefined if the index exceeds the bounds of the list.
Further it requires a natural numbers type $\op{Nat}$ and makes use of
\emph{existential equality} when comparing elements of lists.  This means that a
comparison $v!!i = w!!j$ yields true if and only if both $v!!i$ and $w!!j$ are
defined and equal. An informal specification of this algorithm is as follows.
\begin{itemize}
\item $\op{pmatch}$ returns the first -- \IE least -- index $x$ such that the
  pattern $\op{pat}$ occurs in $\op{base}$ starting at index $x$.
\item If no such index exists, $\op{pmatch}$ will fail with an exception
  $\op{Error}$.
\end{itemize}


\begin{myfigure}
  \begin{flalign*}
    & \mathbf{Operations}\\
    & \op{read} : \op{Ref}\ a \to T\ b\ a &&  (\op{read}\ r \defeq \rd{r})\\
    & \op{write} : \op{Ref}\ a \to a \to T\ b\ 1 && (\op{write}\ r\ x \defeq {r
      := x})\\
    & \op{new} : a \to T\ b\ (\op{Ref}\ a)
  \end{flalign*}
  \begin{flalign*}
    & \mathbf{Axioms}\\
    & \AssertDsef{\op{read}} &&\text{(dsef-read)}\\
    & \ETHTriple{}{r := x}{x = \rd{r}}{\bot} && \text{(read-write)}\\
    & \ETHTriple{x = \rd{r} \land \lnot r=s}{s:=y}{x=\rd{r}}{\bot} 
        && \text{(read-write-other)}\\
    & \ETHTriple{}{r\leteq \op{new}\ x}{x=\rd{r}}{\bot}
        && \text{(read-new)}\\
    & \ETHTriple{x=\rd{r}\land\lnot r=s}{s\leteq \op{new}\ y}{x=\rd{r}}{\bot}
        && \text{(read-new-other)}\\
    & \ETHTriple{\phi}{r\leteq \op{new}\ x; p}{\top}{\top} \Rightarrow\\
    & \qquad \ETHTriple{\phi}{r\leteq \op{new}\ x; p; s\leteq \op{new}\ y}{\lnot r=s}{\top}
        && \text{(new-distinct)}
  \end{flalign*}
  \mylinesep
  \caption{Specification of the exception reference monad}
  \label{fig:spec-ex-ref-mon}
\end{myfigure}


% The generic Hoare calculus for total exception correctness

\begin{figure} 
\centering
\fbox{\parbox{0.95\textwidth}{
\begin{displaymath}
\NRule{seq}{\begin{array}{c}
    \ETHTriple{\phi}{\bar x\leteq \bar p}{\psi}{S}\\
    \ETHTriple{\psi}{\bar y\leteq \bar q}{\chi}{S}
\end{array}}{\ETHTriple{\phi}{\bar x\leteq \bar p;\bar y\leteq \bar q}{\chi}{S}}
\quad
\NRule{ctr}{
\begin{array}{c}
    \ETHTriple{\phi}{\dots; x\leteq  p; y\leteq q;\bar z\leteq\bar r}{\psi}{S}
\\
x\notin FV(\bar r)\cup FV(\psi)
\end{array}
}
{\ETHTriple{\phi}{\dots; y\leteq (\DoStmt{ x\leteq p; {q}});\bar z\leteq\bar r}{\psi}{S}}
\end{displaymath}

\begin{displaymath}
\NRule{conj}{\begin{array}{c}
    \ETHTriple{\phi}{\bar x\leteq \bar p}{\psi_1}{S_1}\\
    \ETHTriple{\phi}{\bar x\leteq \bar p}{\psi_2}{S_2}
\end{array}}{\ETHTriple{\phi}{\bar x\leteq \bar
    p}{\psi_1\land\psi_2}{S_1\land S_2}}
\qquad
\NRule{disj}{\begin{array}{c}
    \ETHTriple{\phi_1}{\bar x\leteq \bar p}{\psi}{S}\\
    \ETHTriple{\phi_2}{\bar x\leteq \bar p}{\psi}{S}
\end{array}}{\ETHTriple{\phi_1\lor\phi_2}{\bar x\leteq \bar
    p}{\psi}{S}}
\end{displaymath}

\begin{displaymath}
\NRule{wk}{\begin{array}{c}
    \ETHTriple{\phi}{\bar x\leteq \bar p}{\psi}{S}\\
    \phi'\Rightarrow\phi\quad\psi\Rightarrow\psi'\\
    \forall e\bnd S~e\Rightarrow S'~e 
\end{array}}{\ETHTriple{\phi'}{\bar x\leteq \bar p}{\psi'}{S'}}
\qquad
\NRule{stateless}{
\ETHTriple{\ret~\phi}{q}{\top}{\top}
}
{
\ETHTriple{\ret~\phi}{q}{\ret~\phi}{\LambdaTerm{e}{\ret~\phi}}
}
\end{displaymath}

\begin{displaymath}
\NRule{dsef}{p~\textrm{dsef}
}
{\ETHTriple{\phi}{x\leteq p}{\phi\land x=p}{\bot}}
%\NRule{Y}{\begin{array}{c}%
%\big[\ETHTriple{\phi}{x\leteq p\ y}{\psi}{S}\big]\\
%\vdots\\
%\ETHTriple{\phi}{x\leteq (F\ p)\ y}{\psi}{S}
%\end{array}}
%{
%\ETHTriple{\phi}{x\leteq Y(F)\ y}{\psi}{S}
%}
%
\qquad
\NRule{if}{\begin{array}{c}
    \ETHTriple{\phi\land b}{x\leteq p}{\psi}{S}\\
    \ETHTriple{\phi\land  \lnot b}{x\leteq q}{\psi}{S}
\end{array}}{\ETHTriple{\phi}{x\leteq \If\ b\ \Then\ p\ \Else\ q}{\psi}{S}}
\end{displaymath}

\begin{displaymath}
\NRule{catch}{
\ETHTriple{\phi}{\bar x\leteq\bar p}{\psi[\inl\ \bar x/y]}{\lambda
  e.\,\psi[\inr\ e/y]}
}
{
\ETHTriple{\phi}{y\leteq(\catchEx\ \bar x\leteq \bar p)}{\psi}{\bot}
}
\end{displaymath}

\begin{displaymath}
\NRule{raise}{}
{
\ETHTriple{\phi}{\raiseEx\ e_0}{\bot}{\LambdaTerm{e}{(\phi\land e=e_0)}}
}
\end{displaymath}
\begin{displaymath}
%\NRule{dis}{q~\textrm{discardable}}
%
%\ETHTriple{\phi}{q}{\phi}{\lambdaTerm{e}{{\phi}}}
%} Besser nicht...
%\qquad
\NRule{while}{
  \begin{array}{c}
    \ETHTriple{\phi \land b}{p}{\top}{\top}\\
    \EPHTriple{\phi \land b \land t = z}{p}{\phi \land b\land t < z}{\top}\\
    \EPHTriple{\phi \land b}{p}{\top}{S}
  \end{array}}
{
\ETHTriple{\phi}{\While \ b\ p}{\phi \land \lnot b}{S}
}
\quad
\end{displaymath}
}}
  \caption{The generic Hoare calculus for total exception correctness}
  \label{fig:exc-partial-hoare}
\end{figure}



%%% Local Variables: 
%%% mode: latex
%%% TeX-master: "main"
%%% End: 



The specification in Figure~\ref{fig:spec-ex-ref-mon} extends the axiomatisation
of the dynamic reference monad  given in
\cite{SchroederMossakowski:Hoare} by abnormal postconditions, which in most
cases are $\bot$, asserting that the corresponding operations do not raise
exceptions. An exception is the rule (new-distinct), which states that
the subsequent creation of references, with an arbitrary program $p$ (which may
raise exceptions) executed in between, produces distinct references.  We prove
total correctness of the algorithm generically, \IE without further assumptions
on the underlying monad other than the axioms of
Figure~\ref{fig:spec-ex-ref-mon} and the interpretability of a $\While$
construct. Figure \ref{fig:exc-partial-hoare} displays the generic Hoare
calculus for total exception correctness. The calculus for partial correctness
is essentially identical (where the square brackets are of course replaced by
curly brackets) except for rule (stateless) in which there is no need for a
premiss.

For the actual method body $p$, \IE the argument of $\mbody$ in function
$\op{pmatch}$, we claim that it terminates abnormally, raising either a return
exception carrying as its value an index $x$ that is the starting position of
the first occurrence of the pattern in the base string, or a failure exception
$\op{Error}$ indicating that there is no occurrence of the pattern in the base
string. Compare this to the specification \eqref{eq:pmmbody-spec} for
$\op{pmatch}$ itself, which actually \emph{returns} the index $x$ if found:
\begin{equation}
\label{eq:pmatchspec}
\begin{array}{ll}
[]~ p~ [\bot \parallel \lambda e.\; \Case&\ e\ \Of \\
 & \hspace{0.7em}\op{MRet}\ i \to \op{MPOS}\ i \land
    \forall j\bnd \op{MPOS}\ j\Rightarrow i\leq j\\
 &  \mid \op{Error} \to \lnot \exists i\bnd \op{MPOS}\ i\\
 &  \mid \Arg \to \bot]
\end{array}
\end{equation}
The abnormal postcondition above will be denoted by $\op{POST}$ below. Here,
$\op{MPOS}\ i$ states that the pattern is matched at position $i$ in the
base string:
\begin{equation*}
\op{MPOS}\ i \equiv \forall j.\; 0 \leq j < \mathit{len}~\op{pat} \Rightarrow
\op{base}!!(i+j) = \op{pat}!!j.
\end{equation*}
In order to apply the total exception while rule (while) of Figure
\ref{fig:exc-partial-hoare}, we need to provide a loop invariant $\op{INV}$ and
a termination measure $t$.  Putting
\begin{equation*}
\begin{array}{ll}
\op{INV} \equiv 
      & (\forall i.\; 0 \leq i < \rd{r} \Rightarrow 
       \op{base}!!(\rd{s}+i) = \op{pat}!!i)\ \land {}\\
      & \forall i.\; \op{MPOS}\ i \Rightarrow \rd{s} \leq i 
\end{array}
\end{equation*}
(which implies $0 \leq \rd{r} \leq \mathit{len}~\op{pat}$ and $0\leq
\rd{s} + \rd{r} \leq \mathit{len}~\op{base}$) guarantees that the
dsef term \(t = (\mathit{len}~\op{base} - \rd{s},
\mathit{len}~\op{pat} - \rd{r})\) always yields results of type
$\op{Nat} \times \op{Nat}$, on which we have the lexicographic ordering
as a well-founded relation.

Establishing the invariant upon entrance into the loop is easy, since
from the axioms given above, 
\begin{equation}\label{eq:init-seq}
\ETHTriple{}{r \leteq \op{new}\ 0; s \leteq \op{new}\ 0}
{\rd{s} = \rd{r} = 0 \land \lnot (r = s)}{\bot}
\end{equation}
can be derived by the rules (seq), (conj), (read-new-other) and (new-distinct).
Inside the loop, there are essentially four branches, arising from three
applications of the rule (if), so that the three premisses of the total
exception while rule are split into twelve proof goals (the two read operations
$u\leteq \rd{r}$ and $v\leteq \rd{s}$ are dealt with by rules (dsef) and (seq)).
The total exception while rule proof obligations, stated informally, are first
to prove termination of the program at hand, then to prove that the invariant is
maintained as well as that the termination measure decreases strictly, and
finally to prove that the abnormal postcondition can be established given the
loop invariant as a precondition. We now prove the goals for each branch, with
some of those having obvious proofs omitted. Furthermore, we will leave the pre-
and postcondition $\lnot (r = s)$ implicit, since it obviously prevails in the whole
proof thanks to rule (stateless).

\paragraph{(i) Returning an Index.} 
\begin{gather}
\label{eq:monkey1-1}
\ETHTriple{\op{INV} \land \rd{r} = u \land \rd{s} = v
  \land u = \op{len}~\op{pat}}{\op{mret}~v}{\top}{\top}\\[2ex]\nonumber
\{\op{INV} \land \rd{r} = u \land \rd{s} = v \land u = \op{len}~\op{pat} \land
(\op{len}~\op{base}-v, \op{len}~\op{pat}-u) = z\} \\
\label{eq:monkey1-2}
  {\op{mret}~v}\\\nonumber
  \{\op{INV} \land (\op{len}~\op{base} - \rd{s}, 
    \op{len}~\op{pat}-\rd{r}) < z \;\parallel\; \top\}\\[2ex]
\label{eq:monkey1-3}
\EPHTriple{\op{INV} \land \rd{r} = u \land \rd{s} = v \land u = \op{len}~\op{pat}}
  {\op{mret}~v}{\top}{\op{POST}}
\end{gather}
By the total and partial variants of rules (raise) and (wk), recalling that
$\op{mret}~v$ is just an abbreviation for $\raiseEx~(\op{MRet}~v)$ one easily
obtains \eqref{eq:monkey1-1} and \eqref{eq:monkey1-2}. It remains to show
\eqref{eq:monkey1-3}; now from its precondition we infer $\op{MPOS}\ v \land \forall i.\,
\op{MPOS}\ i \Rightarrow v \leq i$, a stateless formula.  Further we may derive
$\EPHTriple{}{\op{mret}~v}{\bot}{\lambda e.\; e = (\op{MRet}~v)}$ by (raise). Hence, by
(stateless), (conj), and (wk), we obtain
\begin{multline*}
\{\op{INV}  \land \rd{r} = u \land \rd{s} = v \land \rd{r}
= \mathit{len}~\op{pat}\}\ \op{mret}~v\\
 \{\bot \parallel \lambda e.\, e = \op{MRet}\ v
\land \op{MPOS}\ v \land \forall i.\, \op{MPOS}\ i \Rightarrow v\leq i\}
\end{multline*}
The formula in the abnormal postcondition implies $\op{POST}$, since the
kind of exception  is identified as $\op{MRet}$ and thus the formula can be
extended to the case construct of
$\op{POST}$. This means we are finished by another application
of (wk).

\paragraph{(ii) Failing to Find an Index.}
\begin{gather}
\label{eq:monkey2-1}
\ETHTriple{\op{INV} \land \rd{r} = u \land \rd{s} = v \land \lnot(u = \op{len}~\op{pat})
\land u+v = \op{len}~\op{base}}{\raiseEx~\op{Error}}{\top}{\top}\\[2ex]
\label{eq:monkey2-2}
\EPHTriple{\op{INV} \land (\op{len}~\op{base} - \rd{s}, \op{len}~\op{pat} - \rd{r}) = z
  \land \ldots}{ \raiseEx~\op{Error}}{\op{INV} \land \ldots}{\top}\\[2ex] \nonumber
\{\op{INV} \land \rd{r} = u \land \rd{s} = v \land 
 \lnot (u = \op{len}~\op{pat}) \land u+v = \op{len}~\op{base}\}\\
\label{eq:monkey2-3}
{\raiseEx~\op{Error}}\\\nonumber
\{\top \parallel \op{POST}\}
\end{gather}
Again, \eqref{eq:monkey2-1} and \eqref{eq:monkey2-2} -- which has not even been
written out in full; the pattern is as in (i) -- are immediate by rules
(raise) and (wk). To show \eqref{eq:monkey2-3} we note that by (raise) one has 
\begin{equation}
\label{eq:aux-raise-error}
\EPHTriple{}{\raiseEx~\op{Error}}{\bot}{\lambda e.\; e = \op{Error}}
\end{equation}
and letting $P \defeq \op{INV}[u/ \rd{r}, v/ \rd{s}] \land u + v =
\op{len}~\op{base} \land \lnot(u = \op{len}~\op{pat})$ by (stateless) one obtains
\begin{equation}\label{eq:aux-inv-stl}
\EPHTriple{P}{\raiseEx~\op{Error}}{P}{\lambda e.\, P}
\end{equation}
After strengthening the precondition of \eqref{eq:aux-raise-error} by (wk) one
may combine it with \eqref{eq:aux-inv-stl} by rule (conj). But the formula thus
obtained in the abnormal postcondition implies $\op{POST}$: informally this can
be seen because the exception type is $\op{Error}$; the substituted invariant
guarantees that no pattern has been found up to $v$ and none can appear later on
as the end of the pattern has been reached and $u$ is less than
$\op{len}~\op{pat}$; this means that no occurrence of $\op{pat}$ in $\op{base}$
exists. 

\paragraph{(iii) Proceeding With a Partial Match.}
In this branch the first and third goals are trivially proved, since assignment
terminates and does not raise exceptions. The second goal is
\begin{gather}
\nonumber
\{\op{INV} \land \rd{r} = u\land \rd{s} = v\land \lnot(u=\op{len}~\op{pat})\land
  \lnot(u+v = \op{len}~\op{base}) \\\nonumber
  {}\land (\op{len}~\op{base} - v,  z = \op{len}~\op{pat} - u)
  \land \op{base}!!(u+v) = \op{pat}!!u\}\\
\label{eq:monkey3-2}
  {r := u + 1}\\\nonumber
  \{\op{INV} \land (\op{len}~\op{base} - \rd{s}, \op{len}~\op{pat} - \rd{r})
  < z \;\parallel\; \top\}
\end{gather}
There are two parts to be shown here: on the one hand it has to be established
that the invariant holds in the postcondition, and on the other hand one must
show that the termination measure $t$ decreases. Both proofs hinge on rules
(read-write) and (read-write-other) from which one infers 
\begin{equation}
\label{eq:rd-wrt-carry}
\EPHTriple{\lnot(r = s) \land
  \rd{s} = v} {r := u+1}{\rd{s} = v \land \rd{r} = u+1}{\bot}
\end{equation}
so that in \eqref{eq:monkey3-2} the value of $\rd{s}$ carries over from the pre-
to the postcondition, while the value of $\rd{r}$ is increased exactly by one.
Taken together, this forces the measure $t$ to decrease strictly. Regarding the
invariant a similar point can be made: analogous to (ii) the invariant with $u$
and $v$ replacing $\rd{r}$ and $\rd{s}$ respectively carries over from the
precondition to the postcondition since it is stateless. Moreover, the facts
that the partial match may be extended, \IE one has $\op{base}!!(u+v) =
\op{pat}!!u$, and \eqref{eq:rd-wrt-carry} establish the invariant proper.


\paragraph{(iv) Starting a New Match.}
Again, the first and third goals are not shown, because the situation is
essentially the same as for (iii), the only difference being that two
assignments are executed instead of one.
\begin{gather}
\label{eq:monkey4-2}\nonumber
\{\op{INV} \land \rd{r} = u \land \rd{s} = v \land \lnot(u = \op{len}~\op{pat}) \land \lnot (v+u =
\op{len}~\op{base}) \\\nonumber
{} \land z = (\op{len}~\op{base} - v, \op{len}~\op{pat} - u) \land
   \lnot(\op{base}!!(u+v) = \op{pat}!!u)\}\\
{\DoStmt{s := v + 1; r := 0}}\\\nonumber
\{\op{INV} \land (\op{len}~\op{base} - \rd{s}, \op{len}~\op{pat} - \rd{r}) < z\}
\end{gather}
Once more the crucial fact can be obtained from (read-write) and
(read-write-other):
\begin{equation}
\label{eq:rd-wrt-carry2}
\EPHTriple{\lnot(r = s)} {s:= v+1; r := 0}{\rd{r} = 0 \land \rd{s} = v+1}{\bot}
\end{equation}
This forces the termination measure to decrease strictly, and enables one to
retain the invariant in the postcondition. Informally this is valid due to the
fact that $\lnot(\op{base}!!(u+v) = \op{pat}!!u)$, \IE the current partial match
cannot be completed. It is then legal to increase $s$ by one to search for
another match further on, validating the second conjunct of the invariant. By
setting $r$ to zero the first conjunct of the invariant becomes vacuously true.



Altogether, having arrived at proving Formula~(\ref{eq:pmatchspec}) by
composing (\ref{eq:init-seq}) with the conclusion of the total exception
while rule, we can then apply rule (mbody) to obtain the total
correctness of the whole algorithm:
\begin{multline}
\label{eq:pmmbody-spec}
[]~ i \leteq \mbody~p ~[\op{MPOS}\ i \land
    \forall j.\, \op{MPOS}\ j\Rightarrow i\leq j \parallel {} \\ 
\lambda e.\,\Case\ e\ \Of\ 
   \begin{array}[t]{l}
    \hspace{0.5em}\op{Error} \to \lnot \exists i.\, \op{MPOS}\ i \\
    \mid \Arg \to\bot].
   \end{array}
\end{multline}





%%% Local Variables: 
%%% mode: latex
%%% TeX-master: "main"
%%% End: 

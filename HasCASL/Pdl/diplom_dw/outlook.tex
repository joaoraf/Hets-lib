
\chapter{Conclusion and Outlook}
\label{cha:outlook}


In this thesis we have described a program logic for programs formulated in the
do-notation of monads. After having recalled that monads are an elegant and
effective means to model several kinds of computational effects like state,
input and output, exceptions, or nondeterminism we have depicted the
development of this \emph{monadic dynamic logic}. The prominent features of the
logic are that
\begin{enumerate}
  \item Programs with certain well-behavedness properties making them
    deterministically side effect free are taken as formulae
    of the logic
  \item Modal operators allow one to make statements of the form ``after
    execution of the program $p$, the formula $\phi$ will hold''
  \item The modal operators are entirely interpreted within the underlying monad
    (presupposing the monad satisfies certain additional conditions); no
    additional structure is required.
\end{enumerate}
The calculus has been extended by further axioms, rules and the $\mbody$
operation to evolve into a suitable logic for reasoning about abrupt termination
in Java. In this extension the correctness of a pattern match algorithm has been
verified. Back in the basic calculus we have then specified and proved correct an
implementation of a breadth-first search algorithm in the queue monad, which
represents a rather complex example on how to apply the general calculus to
realistic programs.  Finally, the calculus has been implemented on top of
higher-order logic in the proof assistant Isabelle.  In this formalisation
further monads like the reference monad and a monad for parser combinators have
been specified. To help automatise simple proof obligations, Isabelle's
simplifier has been extended to become able to prove tautologies of dynamic
logic automatically.


The implementation in Isabelle made it obvious that the formulation of the
calculus in Hilbert-style, \IE with several axioms and only the two proof rules
necessitation and modus ponens, makes proofs of rather simple theorems quite
expensive in terms of the required proof steps. The extension of the simplifier
to solve tautologies automatically is already a great help, but of course
tautologies do not constitute the most interesting part of the valid formulae of
dynamic logic. It has been pointed out that the major problem why we cannot
provide a natural deduction system for the calculus is the lack of an
appropriate rule for implication introduction. This also obviates the employment
of Isabelle's classical reasoner; it is thus an interesting question whether a
sequent or tableaux calculus can be found for the logic that allows for more
automation than has been achieved in this thesis.

It has turned out that proofs in monads where a Hoare calculus for total
correctness can be given -- most notably this applies to the state monad --
proofs as conducted in dynamic logic actually resembled the proof style for
Hoare logics. This is to say that proofs mostly proceeded in a sequential
fashion in which the modal operators were mainly indexed by the program fragment
to be verified; thus the only necessary modal expression was to state what will
hold after execution of the main program. It might therefore turn out to be
useful to formulate a Hoare calculus on top of the formalisation of dynamic
logic in Isabelle in which modal formulae do not appear in the precondition or
postcondition. The formulation of such a calculus for total correctness would
have the additional benefit of removing the duplicate proof obligations that
arose in dynamic logic due to the fact that in the latter termination and
correctness are expressed by two distinct formulae.

Finally, it would be nice to undermine the implementation provided in this
thesis with further foundations to make several axioms unnecessary. In
particular, the formalisation of global dynamic judgements would make it
possible for several monads to actually define the modal operators. Since this
formalisation is currently being worked out in a different diploma thesis this
should not constitute a major problem. Given a definition of the modal
operators, it would also be much more rewarding to actually define concrete
monads instead of just axiomatising their characteristic properties, because
then one could go on and actually establish these properties as theorems.

\section*{Acknowledgements}
First and foremost, I want to thank my family and Jenny for their love and
support especially in but of course not limited to the time during which I wrote
this thesis. Thanks also to my fellow students Martin K\"uhl and Tina Krau\ss er
for fruitful discussions and suggestions on how to improve this thesis.  Last
but not least thanks go to my supervisors Lutz Schr\"oder and Till Mossakowski
who always offered their expertise and advice when problems arose.



%%% Local Variables: 
%%% mode: latex
%%% TeX-master: "main"
%%% End: 
